% Nom de la recette à entrer entre les accolades {}
\section{Titre}

\begin{ingredients}
% Ici lister les ingrédients 
% Changer de ligne pour chaque ingrédient et commencer la ligne par : \item
% rajouter autant de ligne que d'ingrédient
\item 3 oeufs (4)
\item 150 g de farine (200 g)
\item 1 sachet de levure
\item 6 cl d’huile de tournesol (6 cl)
\item 12,5 cl de lait entier (16,5 cl)
\item 100 g de gruyère râpé (133 g)
\item 100 g d’oignons (133 g)
\item 200 g de lardons fumés (266 g)
\item 1 noisette de beurre demi-sel
\item 1 cuillerée à soupe d’huile de tournesol
\item 1 cuillerée à soupe de crème épaisse
\item 1 pincée de sel, 1 pincé de poivre.
\end{ingredients}
\begin{infos}
% Informations génériques
% Changer de ligne pour chaque et commencer par : \item
% Mettre une * si l'information n'est pas certaine 
\item Pour XX personnes*		% Nombre de personnes qu'on pourra nourrir ! :)
\item Préparation : XX min*		% Temps de préparation (sans la cuisson)
\item Cuisson : 45 min			% Temps de cuisson
\end{infos}
\begin{etapes}
% Ici les étapes à réaliser
% Une étape par ligne, chaque ligne commence par un \item
% Pour exemple les étapes pour faire un millas ;)
\item Préchauffez votre four à 180 °s C (thermostat 6).  
\item Emincez les oignons, faites-les revenir dans une poêle avec la noisette de beurre et la cuillerée d’huile, mettez la pincée de sel et de poivre. Lorsqu’ils blondissent, ajoutez les lardons et faites-les légèrement rissoler. Retirez-les du feu et versez-y la cuillerée à soupe de crème.
\item Pendant ce temps, dans un saladier, travaillez bien au fouet les œufs, la farine et la levure. Incorporez petit à petit l’huile et le lait préalablement chauffé. Ajoutez le gruyère râpé. Mélangez.
\item Incorporez le mélange oignons, lardons, et crème à la base.
\item Versez le tout dans un moule non graissé et mettez au four pendant 45 minutes.
\end{etapes}
\begin{conseils}
% Ici écrire les conseils concernant la recette 
\end{conseils}
