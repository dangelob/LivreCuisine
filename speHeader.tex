% https://tex.stackexchange.com/questions/88559/package-right-similar-to-tufte-book

\usepackage{ifthen}% for the \ifthenelse macro

% The Tufte-style running heads are defined similarly to the macros below.
% These macros avoid using Tufte-specific code, but may still be overkill for a
% particular document class. (For example, they detect if you're in twoside
% mode and use different running heads based on that.
\usepackage{fancyhdr}

%\pagestyle{fancy}

\makeatletter% so we can use macros with @ in their names

% Set the header/footer width to be the body text block plus the margin
% note area.
\newlength{\overhanglength}
\AtBeginDocument{%
  % Calculate the amount to extend the running heads
  \setlength{\overhanglength}{\marginparwidth}
  \addtolength{\overhanglength}{\marginparsep}

  % Set the running head offsets to the overhang length calculated above
  \ifthenelse{\NOT\boolean{@mparswitch}\AND\boolean{@twoside}}
    {\fancyhfoffset[RE,RO]{\overhanglength}}% asymmetric
    {\fancyhfoffset[LE,RO]{\overhanglength}}% symmetric
}

% The running heads/feet don't have rules
\renewcommand{\headrulewidth}{0pt}
\renewcommand{\footrulewidth}{0pt}

\fancyhf{} % clear any existing header and footer fields

% adjust the formatting code to suit your tastes here
\ifthenelse{\boolean{@twoside}}{%
  \fancyfoot[LE]{\thepage\quad\leftmark}%
  \fancyfoot[RO]{\rightmark\quad\thepage}%
}{%
  \fancyhead[RE,RO]{\rightmark\quad\thepage}%
}

\makeatother% restore the original meaning of @