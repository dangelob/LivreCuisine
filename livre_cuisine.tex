%     Livre de Recette des D'ANGELO     %
%            - version 0.1 -            %
%%%%%%%%%%%%%%%%%%%%%%%%%%%%%%%%%%%%%%%%%
% Compilation : pdflatex
% recipe_converter ~/Documents/PROJETS/LivreCuisine/txtRecettes/ ~/Documents/PROJETS/LivreCuisine/recettes/


\documentclass[A4paper,twoside, 12pt]{book}

%-- Packages 
\usepackage[utf8]{inputenc}
\usepackage[T1]{fontenc}
\usepackage{lmodern}% polices 

\usepackage[french]{babel}

\usepackage[]{geometry}%Gestion des marges % [showframe]
\geometry{
 inner=2.5cm, %left
 outer=9cm, %right
 top=2.2cm,
 bottom=2.2cm, 
 marginparwidth=7cm,
 marginparsep=.5cm,
 a4paper=true,
 twoside=true
}

\usepackage{marginnote} %Permet les notes dans la marge

\usepackage{tocloft} % gestion de table de matières personnalisées (recette au chocolat)

\usepackage{makeidx}

% pour les liens cliquables de la table des matières
\usepackage[colorlinks=true,linkcolor=blue]{hyperref}

\usepackage{graphicx} %insertion d'image

%\usepackage{fancyhdr} %En-tête et pied de pages


\usepackage{textcomp} % pour  \degree
\usepackage{enumitem} % Gestion fine des listes

 % Gestion des titres
\usepackage[]{titlesec}

\usepackage{environ} % gestion des environnements
\usepackage[strict]{changepage} % permet de savoir si on est sur une page paire ou impaire (+ marges des titres de section)

%--------------------------------------------------------%
%-----------------------LES COMMANDES--------------------%
%--------------------------------------------------------%
%\newcommand{\sommaire}{\shorttoc{Sommaire}{1}} %change la commande pour inserer une short table of content
\newcommand{\HRule}{\rule{\textwidth}{0.5mm}} % utilisation page de titre


\titleformat{\part}[display]
{\fontfamily{pag}\selectfont\Huge}
{\thepart}
{20pt}
{\Huge}

% Cette commande fait perdre les header et footer
\newcommand{\chapitre}[1]{
	\chapter*{#1}
	\addcontentsline{toc}{chapter}{#1}
	\clearpage
	% \markboth{#1}{#1} 
}

%titlesec
\titleformat{\chapter}[display]
{\fontfamily{pag}\selectfont\huge}
{\chaptertitlename\ \thechapter}
{20pt}
{\Huge}
\titleformat{\section}
{\centering\fontfamily{pag}\selectfont\Large}
{\thesection}
{1em}
{}



%--------------------------------------------------------%
%----------EN-TÊTE ET PIED DE PAGE----------------%
%--------------------------------------------------------%
% https://tex.stackexchange.com/questions/88559/package-right-similar-to-tufte-book

\usepackage{ifthen}% for the \ifthenelse macro

% The Tufte-style running heads are defined similarly to the macros below.
% These macros avoid using Tufte-specific code, but may still be overkill for a
% particular document class. (For example, they detect if you're in twoside
% mode and use different running heads based on that.
\usepackage{fancyhdr}

%\pagestyle{fancy}

\makeatletter% so we can use macros with @ in their names

% Set the header/footer width to be the body text block plus the margin
% note area.
\newlength{\overhanglength}
\AtBeginDocument{%
  % Calculate the amount to extend the running heads
  \setlength{\overhanglength}{\marginparwidth}
  \addtolength{\overhanglength}{\marginparsep}

  % Set the running head offsets to the overhang length calculated above
  \ifthenelse{\NOT\boolean{@mparswitch}\AND\boolean{@twoside}}
    {\fancyhfoffset[RE,RO]{\overhanglength}}% asymmetric
    {\fancyhfoffset[LE,RO]{\overhanglength}}% symmetric
}

% The running heads/feet don't have rules
\renewcommand{\headrulewidth}{0pt}
\renewcommand{\footrulewidth}{0pt}

\fancyhf{} % clear any existing header and footer fields

% adjust the formatting code to suit your tastes here
\ifthenelse{\boolean{@twoside}}{%
  \fancyfoot[LE]{\thepage\quad\leftmark}%
  \fancyfoot[RO]{\rightmark\quad\thepage}%
}{%
  \fancyhead[RE,RO]{\rightmark\quad\thepage}%
}

\makeatother% restore the original meaning of @



%------------Table des matières-------------------%
% profondeur de la numérotation (les sections ne sont pas numérotée)(si on veut les sections avec un numéro :  \setcounter{secnumdepth}{1})
\setcounter{secnumdepth}{0}

%jusqu'à quelle profondeur sont inséré les titres dans le sommaire
\setcounter{tocdepth}{1}


%-------------------------------------------------------------------%
%---------------CONFIG ENVIRONMENTS----------------------%
%-------------------------------------------------------------------%



\NewEnviron{recette}[2]{
	\noindent
	\begin{minipage}[adjusting]{\textwidth}
		\checkoddpage
		\ifoddpage
		\begin{adjustwidth}{}{-6.5cm} % left and right, positive values to increase margin
			\section[#1]{#2}	
		\end{adjustwidth}
		\else
			\begin{adjustwidth}{-6.5cm}{} % left and right, positive values to increase margin
			\section[#1]{#2}	
		\end{adjustwidth}
	\fi	
			\BODY
	\end{minipage}
\vfill
}

\NewEnviron{ingredients}{
	\setlength\parskip{.3em}
	\checkoddpage
	\ifoddpage %si impaire
		\marginnote{
			\begin{flushright}
				\BODY
			\end{flushright}
		}
	\else
		\marginnote{
			\begin{flushleft}
				\BODY
			\end{flushleft}
		}
	\fi	
}




\NewEnviron{infos}{	
\bfseries
\checkoddpage

\ifoddpage %si impaire
\begin{flushleft}
	\BODY
\end{flushleft}
\else
\begin{flushright}
	\BODY
\end{flushright}
\fi
\normalfont

}
%\newcommand{\info}[]}{\textbf{#1}}

\NewEnviron{etapes}{
		\vspace{.5cm}
\begin{itemize}[label={}, leftmargin=0cm]
	\setlength{\itemsep}{5pt} % space between items
	\BODY
\end{itemize}

}

\NewEnviron{conseils}{
	\begin{center}
		\par\noindent\rule{.5\textwidth}{0.4pt} 
	\end{center}
	\BODY
	\vspace{1cm}
}

\makeindex

%------------Config List-------------------%
% Liste chocolat
\newcommand{\listchoconame}{Recettes au chocolat}
\newlistof{choco}{cho}{\listchoconame}

\newcommand{\choco}[1]{%
\refstepcounter{choco}
%\par\noindent\textbf{Answer \theanswer. #1}
\addcontentsline{cho}{choco}{\protect\numberline{\thechoco}#1}\par}

%\newcommand{\sectionbreak}{\clearpage} %clearpage before session
%%%%%%%%%%%%%%%%%%%%
%-----------Début du document----------%
%%%%%%%%%%%%%%%%%%%%
\begin{document}
		
\frontmatter
	\newgeometry{inner=2.5cm,
	outer=2.5cm,
	bottom=2.5cm, 
	marginparwidth=65.0pt,
	marginparsep=11.0pt,
	a4paper=true,
	twoside=true}

\begin{titlepage}

\begin{center}


% Upper part of the page
  

\textsc{\LARGE Qu'est ce qu'on mange ?}\\[1.0cm]

%\includegraphics[width=0.35\textwidth]{./images/cooklogo}\\[1.2cm] 

\large Pour {\small(tenter de)} répondre à cette question :\\[0.5cm]


% Title
\HRule \\[0.8cm]
{ \Huge \bfseries Recettes de famille}\\[0.4cm]

\HRule \\[1.5cm]

% Author and supervisor
\begin{minipage}{0.4\textwidth}
\begin{center}

\large
\emph{Auteurs:}\\
\textsc{Collectif}
\end{center}
\end{minipage}


\vfill

% Bottom of the page
{\large ---~\today~---}
%BY-NC-SA

\end{center}

\end{titlepage}			% Titre
\cleardoublepage
\tableofcontents 		% Insertion table des matières

\mainmatter
\pagestyle{fancy}
\part{Les Recettes}

\restoregeometry

\chapitre{Salades}
% Generated file 2019-02-17 16:33:30.783870286 +01:00
\begin{recette}{Salade de pâtes à l'italienne}{Salade de pâtes à l'italienne}

\begin{ingredients}
300 g de pâtes\par
1 melon\par
4 belles tomates\par
150 g de mozzarella ou de feta\par
4 tranches de jambon de parme\par
12 feuilles de basilic\par
8 olives noires dénoyautées\par
Le jus d'1 citron\par
6 cuillères à soupe d'huile d'olive\par
Ail\par
Sel, poivre\par
\end{ingredients}

\begin{infos}
4 personnes\\
Préparation : 20 min\\
\end{infos}

\begin{etapes}
\item Couper le melon en deux et détailler la chair en billes à l'aide d'une cuillère parisienne. Couper les tomates en morceaux, la mozzarella en dés et le jambon en lanières. Détailler les olives en petits morceaux.
\item Dans un saladier, mélanger le jus d'un citron, l'huile d'olive, l'ail écrasé et 6 feuilles de basilic ciselées. Saler et poivrer.
\item Ajouter les pâtes dans le saladier. Mélanger. Ajouter les billes de melon, les tomates, la mozzarella, le jambon et les olives. Parsemer du reste de basilic.
\end{etapes}

\begin{conseils}
Servir frais, le melon n'est pas indispensable.
\end{conseils}

\end{recette}
% Generated file 2018-12-02 20:50:19.115962795 +01:00
\begin{recette}{Salade de pâtes provençale}{Salade de pâtes provençale}

\begin{ingredients}
1 poivron rouge;1 poivron jaune\par
350g de pâtes\par
1 citron\par
1 gousse d'ail\par
2 boules de mozzarella\par
6 c. à soupe d'huile d'olive\par
2 brins de basilic\par
\end{ingredients}

\begin{infos}
Pour 4 personnes\\
Préparation : 20 min\\
\end{infos}

\begin{etapes}
\item Faire cuire les pâtes.
\item Epépiner et éminr les poivrons.
\item Rincer le citron, prélever le zeste, l'ébouillanter 1 mint le hacher.
\item Dans un saladier, mélanger les poivrons, la mozzarella coupée en dés, le zeste, l'huile et l'ail haché.
\item Ajouter les pâtes, parsemer de basilic.
\end{etapes}

\end{recette}
% Generated file 2019-02-17 16:33:30.972818566 +01:00
\begin{recette}{Salade de blé aux légumes poêlés}{Salade de blé aux légumes poêlés}

\begin{ingredients}
350 g de blé précuit type Ebly\par
1 aubergine\par
1 courgette\par
1 poivron rouge et 1 poivron jaune\par
1 oignon\par
1 gousse d'ail\par
2 brins de thym, 1 brin de basilic\par
8 c. à soupe d'huile d'olive\par
Sel, poivre\par
\end{ingredients}

\begin{infos}
Pour 6 personnes\\
Préparation : 25 min\\
\end{infos}

\begin{etapes}
\item Faire cuire le blé environ 10 min.
\item Eminr l'aubergine, les poivrons et la courgette.
\item Faire revenir poivrons et courgette 5 min avec 3 cuillères à soupe d'huile. Les retirer.
\item Faire sauter 7 min l'aubergine, l'oignon et l'ail haché dans 3 cuillères à soupe d'huile.
\item Remettre les autres légumes, ajouter le thym, saler et poivrer. Couvrir et laisser cuire 5 min.
\item Mélanger le blé avec le reste d'huile et les légumes. Parsemer de basilic ciselé et servir tiède ou frais.
\end{etapes}

\begin{conseils}
Très bon mais manque une petite note acide. Peut être un peu de jus de citron ?
Pour un plat complet ou en accompagnement, mettre un peu moins de blé en proportions.
\end{conseils}

\end{recette}
% Generated file 2018-12-02 20:50:19.449702486 +01:00
\begin{recette}{Salade de mâche aux champignons}{Salade de mâche aux champignons}

\begin{ingredients}
150 g de mâche\par
150 g de champignons de Paris\par
100 g de chèvre frais\par
1/2 citron\par
\textbf{ sauce :}\par
\begin{itemize}\par
\item[] 1 cuillère à soupe de jus de citron\par
\item[] 3 cuillères à soupe d'huile d'olive\par
\item[] 1 oeuf dur\par
\item[] 2 cuillère à café de câpres\par
\item[] 6 olives vertes\par
\end{itemize}\par
\end{ingredients}

\begin{infos}
4 personnes\\
Préparation : 15 min\\
\end{infos}

\begin{etapes}
\item \textbf{Sauce :} Fouetter une cuillère à soupe de jus de citron avec du sel et du poivre. Incorporer 3 cuillères à soupe d'huile d'olive. Ajouter un œuf dur, 2 cuillères à café de câpres et 6 olives vertes. Hacher le tout.
\item Couper la base des champignons. Les rouler entre vos mains sous un filet d'eau froide pour élimin le sable. Les essuyer dans un linge. Les éminr et les arroser de jus de citron.
\item Laver et essorer la mâche puis la répartir avec les champignons dans un saladier.
\item Émietter le fromage dessus et servir avec la sauce aux olives et aux câpres.
\end{etapes}

\end{recette}
% Generated file 2019-02-17 16:33:30.755126962 +01:00
\begin{recette}{La fontaine de légumes}{La fontaine de légumes}

\begin{ingredients}
1 barquette de tomates cerises\par
3 oignons nouveaux\par
400 g de pointes d'asperges vertes surgelées\par
2 gousses d'ail\par
200 g de comté\par
3 tranches de pain de mie\par
3 cuillère à soupe de câpres\par
4 oeufs + 1 jaune\par
10 cl d'huile\par
1 cuillère à soupe de moutarde\par
30 g de beurre\par
sel, poivre\par
\end{ingredients}

\begin{infos}
6 personnes\\
Préparation : 30 min\\
Cuisson : 10 min\\
\end{infos}

\begin{etapes}
\item Cuire les asperges à l'eau bouillante salée. Rafraîchir, égoutter et couper en tronçons. Plonger les oeufs dans une casserole d'eau bouillante et les cuire dur (10 min). Écaler les oeufs tièdes et les couper en tranches.
\item Couper le pain de mie en dés et les dorer à la poêle dans le beurre chaud. Nettoyer les oignons et les éminr. Rincer les tomates et les couper en deux. Détailler le comté en cubes.
\item Peler et hacher finement l'ail.
\item Dans un bol, préparer la mayonnaise avec le jaune d'oeuf, la moutarde, le sel et le poivre. Émulsionner avec l'huile en filet.
\item Ajouter l'ail haché et mélanger.
\item Dans un saladier profond, transparent, disposer la moitié des tomates, les oeufs en tranches, les câpres, les oignons, les dés de comté, le reste de tomates les tronçons d'asperges et termin par les croûtons.
\item Servir avec la mayonnaise à part.
\end{etapes}

\begin{conseils}
Le bon vin : sylvaner à 8--10°C
	
\end{conseils}

\end{recette}
% Generated file 2018-12-02 20:50:19.430404297 +01:00
\begin{recette}{Salade Romaine sauce roquefort}{Salade Romaine sauce roquefort}

\begin{ingredients}
1 grosse laitue romaine\par
120 g de roquefort ou St Agurs\par
150 g de lardons nature\par
2 grandes tranches de pain de mie sans croûte\par
10 cl de crème liquide\par
1 cuillère à soupe de vinaigre de xéres\par
2 cuillères à soupe d'huile d'olive\par
1 gousse d'ail, poivre\par
\end{ingredients}

\begin{infos}
6 personnes	\\
Préparation : 20 min\\
Cuisson : 5 min\\
\end{infos}

\begin{etapes}
\item Rincer puis essorer les feuilles de romaine. Écraser finement la moitié du roquefort à la fourchette et mélangez-y la crème et le vinaigre. Poivrer cette sauce.
\item Couper le reste du roquefort en morceaux. Taillez les tranches de pain de mie en dés. Dans une poêle avec la gousse d'ail écrasée et l'huile, laisser dorer les dés de pain de mie 1 à 2 min en les tournant. Réserver. Les remplacer par les lardons. les faire revenir 4 à 5 min à feu doux jusqu'à ce qu'ils colorent légèrement.
\item Répartisser les feuilles de romaine dans des coupelles, les arroser de sauce. Garnir avec les petits morceaux de roquefort, les croûtons et les lardons encore chauds. Servir immédiatement.
\end{etapes}

\end{recette}
% Generated file 2018-11-25 21:43:22.182695644 +01:00
\begin{recette}{Salade croquante de fenouil, pomme et pécorino}{Salade croquante de fenouil, pomme et pécorino}

\begin{ingredients}
2 gros bulbes de fenouil\par
2 belles pommes\par
130 g de pécorino\par
3 cuillères à soupe d'hule d'olive\par
60 g de pignons\par
sel\par
\end{ingredients}

\begin{infos}
4/6 personnes\\
Préparation : 20 min\\
Cuisson : 3 min\\
\end{infos}

\begin{etapes}
\item Faire dorer les pignons quelques minutes dans le four, à 180°C, puis les laisser refroidir.
\item Laver les fenouils. Couper les tiges vertes ainsi que la base. Émincer finement les 2 bulbes de façon à obtenir des petits morceaux. Lavez les pommes et les détailler de la même manière.
\item Retirer la croûte du pécorino et le couper en bâtonnets. Réunir les morceaux de fenouil, pomme et fromage dans un saladier.
\item Mélanger, dans un ramequin, 3 cuillères à soupe d'huile d'olive avec 2 cuillères à soupe de jus de citron. Saler un peu. Verser la sauce sur la salade et remuez bien.
\item Parsemer les pignons dorés sur le dessus et déguster.
\end{etapes}

\begin{conseils}
Cette salade fraîche et croquante peut se préparer à l'avance avec sa vinaigrette (le jus de citron évitera que le fenouil et la pomme ne noircissent). Dans ce cas, conservez-là au frais dans une boite hermétique et n'ajoutez que le fromage et les pignons qu'au dernier moment.
Choisissez plutôt des pommes rouges et ne les pelez pas. Leur peau apportera de la couleur à la salade. Si vous aimez les saveurs plus acidulées, vous pouvez aussi utiliser des pommes vertes.
\end{conseils}

\end{recette}
% Nom de la recette à entrer entre les accolades {}
\section{Salade de gésiers au pamplemousse}

\begin{ingredients}
% Ici lister les ingrédients 
% Changer de ligne pour chaque ingrédient et commencer la ligne par : \item
% rajouter autant de ligne que d'ingrédient
\item 2 poignées de mesclun (salade mélangée)
\item 200 g de gésiers de volaille confits
\item 1 pamplemousse rose
\item 1 pamplemousse jaune
\item 50 g de pignons de pin
\item 2-3 pincées de piment doux
\item 1 cuillère à soupe de vinaigre de cidre
\item 3 cuillères à soupe d'huile
\item Quelques brins de cerfeuil, sel, poivre
\end{ingredients}
\begin{infos}
% Informations génériques
% Changer de ligne pour chaque et commencer par : \item
% Mettre une * si l'information n'est pas certaine 
\item Pour 4 personnes		% Nombre de personnes qu'on pourra nourrir ! :)
\item Préparation : 20 min		% Temps de préparation (sans la cuisson)
%\item Cuisson : 3 min			% Temps de cuisson
\end{infos}
\begin{etapes}
% Ici les étapes à réaliser
% Une étape par ligne, chaque ligne commence par un \item
% Pour exemple les étapes pour faire un millas ;)
\item Laver la salade. Peler à vif les 2 pamplemousses. Découper la chair en morceaux.
\item Dans une poêle, faire dorer les pignons de pin dans une cuillère à soupe d'huile bien chaude et les réserver.
\item Dans la même poêle, faites revenir les gésiers.
\item Émulsionner deux cuillères à soupe d'huile et le vinaigre avec le piment doux.
\item Mélanger l'ensemble des ingrédients délicatement en rectifiant l'assaisonnement si nécessaire. Parsemer de cerfeuil avant de servir.
\end{etapes}
\begin{conseils}
% Ici écrire les conseils concernant la recette 
\end{conseils}

% Generated file 2018-12-02 20:50:19.480726363 +01:00
\begin{recette}{Salade multivitamin}{Salade multivitamin}

\begin{ingredients}
1 fenouil\par
1 pamplemousse rose\par
1 avocat\par
3 sucrines (coeur de laitue\par
1/2 botte de radis\par
1 grosse poignée de roquette\par
1 petit citron\par
4 cuillères à soupe d'huile d'olive\par
sel, poivre\par
\end{ingredients}

\begin{infos}
4 personnes\\
Préparation : 15 min\\
\end{infos}

\begin{etapes}
\item Faites une vinaigrette avec le jus de citron, l'huile d'olive, le sel et le poivre. Réserver
\item Prélever les quartiers de pamplemousse.
\item Laver tous les légumes. Éminr le fenouil et les radis.
\item Dans un saladier, disposer la roquette, les feuilles de sucrine, le fenouil, les radis, et les quartiers de pamplemousse.
\item Au dernier moment, ajouter l'avocat coupé en quartiers et assaissonner avec la vinaigrette.
\item Pour un repas complet, ajouter une escalope de poulet grillé émine ou 300 g de crevettes roses décortiquées.
\end{etapes}

\end{recette}
	\chapitre{Cakes}
% Generated file 2018-11-25 21:43:22.171781952 +01:00
\begin{recette}{Cake à la courgette, lardons et fromage de chèvre}{Cake à la courgette, lardons et fromage de chèvre}

\begin{ingredients}
150 g de farine\par
1 sachet de levure\par
3 oeufs\par
12,5 cl de lait\par
2 courgettes\par
1 bûche de chèvre\par
100 g de comté\par
100 g de lardons\par
4 cuillères à soupe d'huile\par
sel, poivre, origan (ou herbes de Provence à défaut)\par
\end{ingredients}

\begin{infos}
Pour 6 personnes\\
Préparation : 30 min\\
Cuisson : 45 min\\
\end{infos}

\begin{etapes}
\item Préchauffer le four à 180
\item Découper les courgettes en rondelles et les passer au four à micro-ondes durant 4 mn pour les précuire. Les mettre ensuite dans une grande poêle anti-adhésive et continuer la cuisson jusqu'à ce qu'elles commencent à accrocher à la poêle (2 à 3 min).
\item Ajouter les lardons, l'origan, poivrer et laisser dorer.
\item Dans un grands saladier, battre les 3 oeufs et incorporer la farine, la levure, une pincée de sel, l'huile et le lait préalablement chauffé.
\item Ajouter ensuite le comté râpé, les courgettes, les lardons et la brique de chèvre découpée en petits morceaux.
\item Verser l'ensemble dans un moule à cake puis déposer sur le dessus quelques lamelles supplémentaires de comté pour gratiner.
\item Faire cuire au four 45 min.
\end{etapes}

\end{recette}
% Generated file 2018-11-25 21:43:22.237440476 +01:00
\begin{recette}{Cake au thon}{Cake au thon}

\begin{ingredients}
3 oeufs\par
25 cl de lait\par
150 g de farine\par
100 g de gruyère rapé\par
1 sachet de levure\par
190 g de thon égoutté\par
2 cuillères à soupe d'huile\par
sel, poivre\par
\end{ingredients}

\begin{infos}
Pour 6 personnes\\
Préparation : 15 min\\
Cuisson : 45 min\\
\end{infos}

\begin{etapes}
\item Préchauffer le four à 180° C.
\item Mélanger les oeufs et la farine.
\item Ajouter la levure, puis l'huile, le lait, le thon émietté et enfin le gruyère.
\item Saler, poivrer.
\item Enfourner 45 minutes.
\end{etapes}

\begin{conseils}
Petit cake : 2 oeufs, 17 cl de lait, 100 g de farine, 67 g de gruyère, 2/3 sachet de levure, une boîte de 132 g de thon (ou 127 g), 1 cuillère 1/3 à soupe d'huile.
\end{conseils}

\end{recette}
% Generated file 2018-11-25 21:43:22.177651292 +01:00
\begin{recette}{Cake aux tomates séchées, feta et olives}{Cake aux tomates séchées, feta et olives}

\begin{ingredients}
200 g de farine\par
4 oeufs\par
1 sachet de levure\par
10 cl de lait\par
2 cuillères à soupe d'huile d'olive\par
15 tomates séchées\par
200 de feta\par
10 olives vertes et 5 noires dénoyautées\par
5 feuilles de basilic\par
Poivre\par
\end{ingredients}

\begin{infos}
Pour 6 personnes\\
Préparation : 20 min\\
Cuisson : 40 min\\
\end{infos}

\begin{etapes}
\item Préchauffer le four à 180° C
\item Mélanger ensemble la farine, les oeufs et la levure.
\item Ajouter l'huile d'olive, et le lait froid.
\item Bien mélanger, en soulevant avec la cuillère, afin d'aérer la pâte.
\item Ajouter les tomates séchées coupée en morceaux, la feta, les olives vertes (en petits morceaux) et le basilic ciselé.
\item poivrer.
\item Verser la préparation dans un moule à cake.
\item Faire cuire pendant 40 min. Vérifier la cuisson avec la lame d'un couteau, elle doit ressortir sèche.
\end{etapes}

\end{recette}
% Generated file 2018-12-02 20:50:19.444803845 +01:00
\begin{recette}{Clafoutis léger aux courgettes et fromage de chèvre}{Clafoutis léger aux courgettes et fromage de chèvre}

\begin{ingredients}
150 ml de lait\par
40 g de farine\par
2 gros oeufs\par
150 g de fromage de chèvre frais\par
2 petites courgettes\par
huile d'olive\par
basilic et thym\par
sel \& poivre\par
\end{ingredients}

\begin{infos}
Pour 4 personnes\\
Préparation : 25 min\\
Cuisson : 35 min\\
\end{infos}

\begin{etapes}
\item Préchauffer le four à 200° C (thermostat 6-7).
\item Laver les courgettes et les couper en fines rondelles sans les éplucher.
\item Les faire revenir dans une poêle anti-adhésive avec un peu d'huile d'olive jusqu'à ce qu'elles aient bien rendu leur eau.
\item Battre les œufs en omelette dans un saladier. Ajouter le chèvre frais, puis petit à petit la farine.
\item Termin en incorporant le lait et les herbes. Saler et poivrer. Une fois la pâte homogène, y ajouter les rondelles de courgettes.
\item Verser le tout dans un moule à manqué en silicone de 20 cm de diamètre.
\item Faire cuire 35 à 40 min.
\end{etapes}

\begin{conseils}
Quantité d'oeufs peut être à revoir.
Essayer avec du chèvre plus fort (bûche ou crottins coupés en morceaux).
\end{conseils}

\end{recette}	
% Generated file 2019-02-17 16:33:30.922306398 +01:00
\begin{recette}{Cake Alsacien façon flammenküche}{Cake Alsacien façon flammenküche}

\begin{ingredients}
3 oeufs (4)\par
150 g de farine (200 g)\par
1 sachet de levure\par
6 cl d’huile de tournesol (6 cl)\par
12,5 cl de lait entier (16,5 cl)\par
100 g de gruyère râpé (133 g)\par
100 g d’oignons (133 g)\par
200 g de lardons fumés (266 g)\par
1 noisette de beurre demi-sel\par
1 cuillerée à soupe d’huile de tournesol\par
1 cuillerée à soupe de crème épaisse\par
1 pincée de sel, 1 pincé de poivre.\par
\end{ingredients}

\begin{infos}
Pour XX personnes\\
Préparation : XX min\\
Cuisson : 45 min\\
\end{infos}

\begin{etapes}
\item Préchauffez votre four à 180 °s C (thermostat 6).
\item Eminz les oignons, faites-les revenir dans une poêle avec la noisette de beurre et la cuillerée d’huile, mettez la pincée de sel et de poivre. Lorsqu’ils blondissent, ajoutez les lardons et faites-les légèrement rissoler. Retirez-les du feu et versez-y la cuillerée à soupe de crème.
\item Pendant ce temps, dans un saladier, travaillez bien au fouet les œufs, la farine et la levure. Incorporez petit à petit l’huile et le lait préalablement chauffé. Ajoutez le gruyère râpé. Mélangez.
\item Incorporez le mélange oignons, lardons, et crème à la base.
\item Versez le tout dans un moule non graissé et mettez au four pendant 45 min.
\end{etapes}

\end{recette}	
	\chapitre{Tartes}
% Generated file 2019-02-17 16:33:30.951147191 +01:00
\begin{recette}{Tarte au saumon et poireaux}{Tarte au saumon et poireaux}

\begin{ingredients}
1 pâte brisée\par
400 g de pavé de saumon\par
4 beaux poireaux\par
3 oeufs\par
1 petit pot de crème fraîche\par
huile d'olive\par
\end{ingredients}

\begin{infos}
Pour 6 personnes\\
Préparation : 25 min\\
Cuisson : 35 min\\
\end{infos}

\begin{etapes}
\item Couper les blancs de poireaux en deux, les laver et les détailler en rondelles. Les mettre à cuire dans une poêle avec une cuillère d'huile. Saler et poivrer légèrement et laisser dorer.
\item Préchauffer le four à 200° C.
\item Pendant que les poireaux cuisent, couper le saumon en gros dés. Tapisser un moule à tarte avec la pâte brisée et la piquer.
\item Répartir les dés de saumon dessus et les recouvrir avec les poireaux cuits.
\item Battre les oeufs avec la crème dans un saladier. Saler, poivrer et verser sur la tarte.
\item Mettre au four et faire cuire 30 à 40 min.
\end{etapes}

\begin{conseils}
Four Claire Cuisson 35 min, surveiller la couleur.
\end{conseils}

\end{recette}
% Generated file 2018-12-02 20:50:18.823656367 +01:00
\begin{recette}{Tarte aux aubergines, tomates et chèvre}{Tarte aux aubergines, tomates et chèvre}

\begin{ingredients}
1 belle aubergine\par
2 à 3 tomates\par
1 petite bûche de chèvre\par
1 rouleau de pâte feuilletée\par
Sel, poivre, herbes de Provence\par
Huile d'olive\par
\end{ingredients}

\begin{infos}
Pour 6 personnes\\
Préparation : 20 min\\
Cuisson : 25 min\\
\end{infos}

\begin{etapes}
\item Préchaufer le four à 200° C. Laver les aubergines et les couper en rondelles assez fines. Les déposer sur une plaque recouverte de papier sulfurisé et les badigeonner d'huile d'olive à l'aide d'un pinceau.
\item Enfourner pour environ 10 min. (surveiller).
\item Pendant ce temps, foncer un moule à tarte et piquer le fond avec une fourchette. Laver les tomates et les découper en fines rondelles.
\item Couper le fromage en tranches fines.
\item Lorsque les aubergines sont dorées, les retourner, badigeonner éventuellement d'un peu d'huile et les remettre au four environ 5 min.
\item Badigeonner le fond de tarte d'un peu de moutarde. Répartir les tranches d'aubergines, puis les recouvrir avec les rondelles de tomates.
\item Saler légèrement et poivrer.
\item Parsemer de fromage de chèvre, saupoudrer d'herbes de Provence et faire cuire 25 min.
\end{etapes}

\end{recette}
% Generated file 2018-11-25 21:43:22.380128872 +01:00
\begin{recette}{Tarte aux poireaux}{Tarte aux poireaux}

\begin{ingredients}
1 pâte brisée\par
2 à 3 poireaux\par
100 g de lardons\par
1 petite cuillère de farine\par
4 oeufs\par
200 g de crème fraîche\par
poivre\par
\end{ingredients}

\begin{infos}
Pour 6 personnes\\
Préparation : 20 min\\
Cuisson : 40 min\\
\end{infos}

\begin{etapes}
\item Préchauffer le four th 6/7.
\item Laver et émincer les poireaux. Faire revenir les lardons dans une pôele. Lorsqu'ils commencent à fondre, ajouter les poireaux. Quand ils commencent à colorer, lier avec une petite cuillère de farine. Laisser revenir à feu doux.
\item Battre dans un bol les oeufs, la crème fraîche et du poivre.
\item Chemiser le moule à tarte, ajouter la fondue de poireaux. Verser l'appareil oeufs et crème par dessus.
\item Enfourner et faire cuire 40 minutes environ.
\end{etapes}

\end{recette}	
% Generated file 2019-02-17 16:33:30.991951418 +01:00
\begin{recette}{Tarte chèvre-piperade}{Tarte chèvre-piperade}

\begin{ingredients}
1 pâte brisée\par
300 g de poivrons en petits dés (surgelés)\par
200 g de chair de tomates pelées\par
2 oeufs\par
1 belle gousse d'ail\par
1 petit oignon\par
50 g de pancetta\par
10 cl de crème\par
1/2 bûche de chèvre\par
1 cuillère à café de piment d'Espelette\par
quelques feuilles de basilic frais\par
2 cuillères à soupe d'huile d'olive\par
\end{ingredients}

\begin{infos}
Pour 6 personnes\\
Préparation : 45 min\\
Cuisson : 35 min\\
\end{infos}

\begin{etapes}
\item Préchauffer le four à 200° C.
\item Faire revenir l'oignon et l'ail dans 2 cuillères à soupe d'huile d'olive.
\item Ajouter les dés de poivron, la tomate et le piment d'Espelette, et laisser mijoter 15 min environ jusqu'à ce que les poivrons deviennent tendres. Réserver.
\item Dans une autre poêle, faire revenir, à sec et à feu très vif, la pancetta coupée en petis morceaux, jusqu'à ce qu'elle soit bien dorée.
\item Egoutter pour ôter l'excès de gras et réserver.
\item Mélanger les légumes et la pancetta. Tapisser un moule à tarte avec la pâte brisée et la piquer. Répartir les légumes dessus.
\item Battre les oeufs avec la crème dans un saladier. Saler légèrement et verser sur la tarte. Garnir de rondelles de chèvre et de feuilles de basilic.
\item Baisser la température du four à 180° C et enfourner pour 30 à 40 min. Servir tiède ou froid.
\end{etapes}

\end{recette}
% Generated file 2018-12-02 20:50:19.307534695 +01:00
\begin{recette}{Tarte fine mascarpone jambon poireaux chèvre}{Tarte fine mascarpone jambon poireaux chèvre}

\begin{ingredients}
1 pâte brisée ou feuilletée\par
5 petits poireaux\par
3 tranches de jambon cru\par
1 bûche de chèvre\par
3 à 4 cuillères à soupe de mascarpone\par
Herbes de Provence\par
Paprika\par
Sel, poivre\par
\end{ingredients}

\begin{infos}
Pour 4 personnes\\
Préparation : 40 min\\
Cuisson : 20 min\\
\end{infos}

\begin{etapes}
\item Étaler la pâte à plat sur une plaque de cuisson.
\item Retourner le rebord sur 1 cm. Piquer la pâte à la fourchette et la précuire 10 min à 220° C.
\item Pendant ce temps laver et éminr les poireaux puis les faire revenir à feu vif dans une poêle anti adhésive, avec un filet d'huile d'olive, jusqu'à ce qu'ils soient bien dorés. Ajouter 1 à 2 verre(s) d'eau. Couvrir et cuire pendant 15 min à feu doux.
\item Découvrir et laisser réduire jusqu'à évaporation de l'eau. Saler et poivrer.
\item Étaler le mascarpone sur la pâte précuite. Répartir le jambon cru coupé en fines lanières. Ajouter le mélange de poireaux. Répartir la bûche de chèvre tranchée.
\item Parsemer de paprika et d'herbes de Provence.
\item Cuire 20 min à 190° C. Servir tiède.
\end{etapes}

\begin{conseils}
Possibilité de remplacer le mascarpone par de la crème.
\end{conseils}

\end{recette}
% Generated file 2019-02-17 16:33:30.937013298 +01:00
\begin{recette}{Tarte au chou romanesco et gouda au cumin}{Tarte au chou romanesco et gouda au cumin}

\begin{ingredients}
1 pâte brisée parfumée avec 1 cuillère à café de cuminn poudre\par
1 gros chou romanesco\par
200 g de gouda au cumin\par
2 oeufs\par
20 cl de crème\par
sel, poivre\par
\end{ingredients}

\begin{infos}
Pour 6 personnes	\\
Préparation : 20 min\\
Cuisson : 30 min\\
\end{infos}

\begin{etapes}
\item Détailler le chou romanesco sans abîmer les « minchoux ». Cuire 3 mn à la cocotte mine dans le panier vapeur.
\item Préchauffer le four à 180°C. Foncer le moule à tarte.
\item Battre ensemble les oeufs et la crème, ajouter le gouda râpé. Saler et poivrer. Verser sur le fond de tarte. Disposer harmonieusement par-dessus le chou cuit.
\item Enfourner pour 30 min
\end{etapes}

\end{recette}
	\chapitre{Soupes}
% Generated file 2019-02-17 16:33:30.880460774 +01:00
\begin{recette}{Soupe au chou vert}{Soupe au chou vert}

\begin{ingredients}
1 petit chou vert\par
4 carottes\par
2 navets\par
5 pommes de terre\par
3 ou 4 saucisses fumées\par
4 cubes de Kub Or\par
Thym, laurier, Poivre\par
\end{ingredients}

\begin{infos}
Pour 6 personnes\\
Préparation : 20 min\\
Cuisson : 35 min\\
\end{infos}

\begin{etapes}
\item Blanchir le chou 5~min
\item Laver, éplucher et couper tous les légumes en morceaux, puis les mettre dans un autocuiseur. Remplir d'eau au moins jusqu'à la moitié.
\item Ajouter les saucisses, le laurier, le thym, les cubes de Kub Or. Poivrer.
\item Fermer la cocotte et faire cuire 30 minà partir du sifflement.
\end{etapes}

\end{recette}
% Generated file 2018-12-02 20:50:18.957775640 +01:00
\begin{recette}{Soupe de navets, lard croustillant et croûtons}{Soupe de navets, lard croustillant et croûtons}

\begin{ingredients}
800 g de navets (nouveaux)\par
260 g de pommes de terre\par
1 oignon\par
1/2 l de bouillon de volaille\par
10 à 15 cl de lait\par
10 cl de crème fraîche épaisse\par
sel, poivre\par
4 fines tranches de pancetta\par
2 tranches de pain de campagne (rassis)\par
1 noisette de beurre\par
\end{ingredients}

\begin{infos}
Pour 4 personnes\\
Préparation : 25 min\\
Cuisson : 35 min\\
\end{infos}

\begin{etapes}
\item Peler les navets et les pommes de terre, les couper en morceaux.
\item Éminr l'oignon finement.
\item Faire revenir l'oignon dans un peu de beurre, à feu doux, sans le faire colorer.
\item Ajouter les pommes de terre et les navets, puis mouiller avec le bouillon de volaille. Faire cuire à feu doux et à couvert pendant 30 à 35 min jusqu'à ce que les légumes soient tendres.
\item Mixer le tout, ajouter le lait (la quantité peut varier en fonction de la consistance désirée), la crème fraîche, rectifier en sel et poivre, réserver au chaud.
\item Découper le pain de campagne en petits dés et la pancetta en morceaux. Faire chauffer une poêle, y déposer le pain (sans huile), laisser dessécher 3 minà feu vif, en remuant constamment.
\item Ajouter alors la pancetta, la faire dorer à feu vif pour qu'elle devienne bien croustillante, tout en continuant à remuer pour que les croûtons s'imprègnent de la graisse de cuisson.
\item Servir la soupe dans des petits bols ou des assiettes creuses et répartir les croûtons et la pancetta sur le dessus.
\end{etapes}

\end{recette}
% Generated file 2019-02-17 16:33:30.793746253 +01:00
\begin{recette}{Velouté de courgettes au boursin}{Velouté de courgettes au boursin}

\begin{ingredients}
6 courgettes\par
Boursin cuisine ail et fines herbes\par
2 bouillon cubes de volaille\par
\end{ingredients}

\begin{infos}
Pour 6 personnes\\
Préparation : 15 min\\
Cuisson : 5 min\\
\end{infos}

\begin{etapes}
\item Laver les courgettes, les couper grossièrement en tronçons et les mettre dans la cocotte.
\item Recouvrir tout juste d’eau, ajouter les bouillon cubes.
\item Laisser cuire 5 minà partir du sifflement de la soupape.
\item Enlever un peu d'eau de cuisson, réserver, et passer les courgettes au mixeur à soupe tout en rajoutant 2 belles cuillères à soupe de boursin.
\item Rectifier l'assaisonnement et/ou rajouter du liquide si besoin est.
\end{etapes}

\end{recette}
% Generated file 2019-02-17 16:33:30.832244403 +01:00
\begin{recette}{Carabaccia}{Carabaccia}

\begin{ingredients}
1 kg d'oignons (rouges et jaunes mélangés)\par
400 g de céleri branche\par
500 g de carottes\par
500 g de petits pois surgelés\par
3/4 de verre d'huile d'olive\par
500 ml d'eau\par
origan\par
sel, poivre\par
parmesan\par
par personne : 1 oeuf, 1 tranche de pain type campagne épaisse\par
\end{ingredients}

\begin{infos}
Pour 6 personnes	\\
Préparation : 60 min\\
Cuisson : 60 min\\
\end{infos}

\begin{etapes}
\item Faire cuire les petits pois surgelés et réserver. Mixer la moitié des petits pois pour les réduire en purée.
\item Peler les oignons et les carottes. Couper oignons, carottes et céleri en petits dés.
\item Dans une grande marmite, verser les oignons et l’huile d’olive. Faire revenir pendant quelques minutes, puis ajouter  la purée de petits pois, les carottes et le céleri.
\item Ajouter l’eau, de l’origan, et laisser mijoter pendant une heure environ à feu doux.
\item Ajouter le reste de petits pois pour les réchauffer. Placer une tranche de pain dans chaque assiette.
\item Pocher les oeufs.
\item Verser du bouillon sur le pain et couvrir de légumes. Ajouter l'oeuf. Saler, poivrer, et parsemer de parmesan râpé. Déguster sans attendre.
\end{etapes}

\end{recette}
% Generated file 2018-11-25 21:43:22.286662712 +01:00
\begin{recette}{Velouté de topinambours au chèvre frais}{Velouté de topinambours au chèvre frais}

\begin{ingredients}
800 g de topinambours\par
1 petit oignon\par
bouillon de légumes\par
200 ml de lait\par
beurre\par
sel, poivre\par
50 g de chèvre frais\par
crème fraîche (selon l'envie)\par
1 bonne poignée de noisettes\par
\end{ingredients}

\begin{infos}
Pour 4 personnes\\
Préparation : 40 min\\
Cuisson : 15 min\\
\end{infos}

\begin{etapes}
\item Laver et peler les topinambours, les couper en petits morceaux. Émincer l'oignon.
\item Dans une casserole, faire chauffer une noix de beurre. Y faire revenir l'oignon et le topinambour pendant 1 à 2 minutes sur feux doux. Mouiller avec le bouillon de légumes (recouvrir) et laisser mijoter pendant environ 15 minutes sur feux doux, pour que le topinambour soit cuit (piquer au couteau, il ne doit pas rester sur la lame). Rajouter un peu de bouillon pendant la cuisson si besoin.
\item Pendant ce temps, concasser les noisettes et les torréfier à sec dans une poêle.
\item Réserver le bouillon. Mixer le topinambour, ajouter le lait, puis progressivement un peu de bouillon jusqu'à obtenir la consistance désirée. Ajouter le chèvre, la crème et bien mixer. Rectifier l'assaisonnement au besoin.
\item Servir parsemé de noisettes concassées avec éventuellement un filet d'huile de noisette.
\end{etapes}

\begin{conseils}
Ne pas hésiter à préparer ce plat à l'avance, comme tous les plats mijotés c'est encore meilleur réchauffé.
\end{conseils}

\end{recette}
% Nom de la recette à entrer entre les accolades {}
\section{Soupe de potiron}

\begin{ingredients}
% Ici lister les ingrédients 
% Changer de ligne pour chaque ingrédient et commencer la ligne par : \item
% rajouter autant de ligne que d'ingrédient
\item 1,5 kg de potiron
\item 2 grosses pommes de terre (bintje)
\item 1 blanc de poireau
\item 1 oignon
\item 1,2 de bouillon de volaille (2 tablettes de concentré)
\item 25 g de beurre
\item 200 g de crème
\item 1 pincée de noix de muscade râpée
\item 1 morceau de sucre
\item 1 bouquet garni
\item 10 brins de ciboulette
\item sel
\end{ingredients}
\begin{infos}
% Informations génériques
% Changer de ligne pour chaque et commencer par : \item
% Mettre une * si l'information n'est pas certaine 
\item Pour 6 personnes		% Nombre de personnes qu'on pourra nourrir ! :)
\item Préparation : 20 min		% Temps de préparation (sans la cuisson)
\item Cuisson : 40 min			% Temps de cuisson
\end{infos}
\begin{etapes}
% Ici les étapes à réaliser
% Une étape par ligne, chaque ligne commence par un \item
% Pour exemple les étapes pour faire un millas ;)
\item Éplucher et épépiner le quartier de potiron. Couper la chair en morceaux. Fendre le blanc de poireau.Le rincer et l'émincer finement. Éplucher les pommes de terre et les couper en gros dés. Peler et hacher l’oignon.  
\item Chauffer le beurre dans un faitout. Faites-y revenir l’oignon et le poireau 5 minutes à feu très doux sans laisser colorer. Mouiller avec le bouillon. Ajouter le potiron, les pommes de terre, le bouquet garni et le sucre. Saler légèrement. Laisser cuire 30 minutes.
\item Retirer  alors le bouquet garni. Passez la soupe au robot-mixer ou au moulin à légumes (grille fine).
\item Reverser la soupe dans le faitout propre. Rectifier l’assaisonnement , parfumer de noix de muscade râpée et porter à nouveau à ébullition.
\item Au moment de servir ajouter la crème et la ciboulette ciselée.
\end{etapes}
\begin{conseils}
% Ici écrire les conseils concernant la recette 
\end{conseils}

% Generated file 2019-02-17 16:33:30.818870213 +01:00
\begin{recette}{Soupe au lard et aux poids cassés}{Soupe au lard et aux poids cassés}

\begin{ingredients}
175 g de pois cassés\par
1 branche de thym\par
100 g de lard fumé\par
1 l d'eau\par
1 verre de lait\par
20 g de margarine\par
sel, poivre\par
\end{ingredients}

\begin{infos}
Pour XX personnes\\
Préparation : XX min\\
Cuisson : 45 min\\
\end{infos}

\begin{etapes}
\item Faire fondre la margarine dans la cocotte-mine. Ajouter les lardon
\item Dès que ceux-ci sont devenus transparents, ajouter les pois cassés, l'eau, une bonne pincée de poivre, un peu de sel (attention les lardons sont déjà salés) et le thym.
\item Couvrir et laisser cuire doucement 25 min à partir du moment ou la soupape chuchote.
\item Laisser échapper la pression, batter le potage avec un fouet tout en ajoutant le lait petit à petit. Il est également possible de le passer au mixer en ayant au préalable retiré les lardons.
\item chauffer à nouveau 2 min et servir
\end{etapes}

\end{recette}
% Generated file 2018-11-25 21:43:22.235044316 +01:00
\begin{recette}{Soupe de légumes aux lentilles corail}{Soupe de légumes aux lentilles corail}

\begin{ingredients}
2 pommes de terre\par
3 petites carottes\par
1 poireau ou 1 gros oignon\par
1 poivron rouge ou vert\par
3 dl de lentilles corail\par
470g de tomates pelées et leur jus (une boîte)\par
1 L d’eau\par
1 bouillon cube aux légumes\par
1 cuillerée à café de cumin ou de curry\par
sel\par
1 feuille de laurier\par
\end{ingredients}

\begin{infos}
Pour 4 personnes\\
Préparation : 40 min\\
Cuisson : 30 min\\
\end{infos}

\begin{etapes}
\item Éplucher et couper les légumes en dés. Mettre dans une marmite avec le reste des ingrédients (ajouter le sel en fin de cuisson).
\item Laisser cuire pendant une bonne demi-heure après ébullition, ou jusqu’à ce que les légumes soient tendres.
\item Servir bien chaud, avec éventuellement un peu de crème fraîche.
\end{etapes}

\end{recette}
% Generated file 2019-02-17 16:33:30.945603416 +01:00
\begin{recette}{Minestrone d'hiver}{Minestrone d'hiver}

\begin{ingredients}
3 carottes\par
2 branches de céleri\par
2 navets\par
2 pommes de terre\par
1 poireau\par
1 gros oignon\par
2 gousses d'ail\par
1 petite boîte de tomates concassées\par
250 g de mélange spécial minestrone (épeautre, haricots secs, lentilles)\par
petits pois (quantité ?)\par
1 bouquet garni (thym, laurier)\par
sel, poivre\par
petites pâtes (facultatif)\par
\end{ingredients}

\begin{infos}
Pour 6 personnes\\
Préparation : trempage 12h et 30 min\\
Cuisson : mini 1h30\\
\end{infos}

\begin{etapes}
\item La veille faire tremper le mélange minestrone pendant 12h dans une grande quantité d'eau.
\item Faire blondir l'oignon émincé et l'ail hachés dans de l'huile d'olive. Ajouter le poireau émincé finement, laisser fondre un peu. Ajouter les carottes, les navets, le céleri et les pommes de terre préalablemment coupés en dés puis mélanger. Après quelques instants, ajouter le mélange minestrone, les tomates, le bouquet garni.
\item Couvrir d'eau et porter à ébullition. Laisser cuire à feu doux à couvert au moins 1h30.
\item 15 min avant la fin, ajouter les petits pois. En fin de cuisson assaisonner et éventuellement ajouter des petites pâtes (environ 1 poignée pour 2 personnes).
\end{etapes}

\begin{conseils}
Généralement on utilise un mélange minestrone bio "Moulin des moines".
Idées de variantes : mettre du chou frisé - ajouter du pesto à la fin (ou du basilic)
\end{conseils}

\end{recette}

	\chapitre{Pâtes}
% Generated file 2018-11-25 21:43:22.289207880 +01:00
\begin{recette}{Pâtes Carbonara (façon Claire T.)}{Pâtes Carbonara (façon Claire T.)}

\begin{ingredients}
200 g de lardons\par
20 cl  de crème (grand modele)\par
3 oeufs\par
50g de beurre\par
100 g de fromage rapé\par
Piment de cayenne moulu\par
Poivre\par
\end{ingredients}

\begin{infos}
Pour 4 personnes\\
Préparation : 15 min\\
\end{infos}

\begin{etapes}
\item Faire revenir les lardons à la poêle.
\item Quand ils sont dorés, ajouter la crème et un peu de piment de cayenne. Laisser réduire.
\item Pendant ce temps, mettre dans le plat de présentation les oeufs battus, le beurre, le fromage rapé et du poivre.
\item Mélanger à la fourchette.
\item Lorsque les pâtes sont cuites, les verser dans le plat et ajouter le mélange crème/lardons par dessus.
\item Bien mélanger et servir.
\end{etapes}

\end{recette}	
	\chapitre{Viandes**}
\section[\normalsize{R\^oti de porc au miel}]{\LARGE{\textsc{R\^oti de porc au miel}}}


\begin{itemize}
\item Pour 6 personnes
\item Préparation : 15 min	
\item Cuisson : 50 min
\end{itemize}
\subsection*{\textsc{Ingrédients~:}}

\begin{itemize}
\item 1 rôti de porc\index{porc} d’environ 1 kg
\item 2 cuill\`ere \`a soupe de miel\index{miel} liquide
\item	30 g de beurre
\item	15 cl de Noilly Prat
\item 1 brin de thym
\item Noix de muscade
\item Sel, poivre
\end{itemize}


\subsection*{\textsc{Marche \`a suivre~:}}

\begin{enumerate}
\item Pr\'echauffez le four sur th.6 (180° C). Enduisez le r\^oti avec le beurre ramolli, salez-le, poivrez-le, puis parsemez-le de thym effeuill\'e et d’un peu de muscade fra\^ichement  r\^ap\'ee.

\item D\'eposez le r\^oti dans un plat, enfournez et cuisez pendant 35 min environ en l’arrosant régulièrement de son jus. A mi-cuisson, ajoutez le Noilly Prat.

\item Ouvrez le four, nappez le r\^oti avec le miel et continuez de cuire 15 min en l’arrosant deux fois avec son jus. Présentez-le d\'eficel\'e et coup\'e en tranches.
\end{enumerate}


\subsection*{\textsc{Conseil~:}}

Faites cuire votre r\^oti dans un plat compatible avec une cuisson au four juste assez grand pour le contenir ; vous \'eviterez ainsi au jus de br\^uler.

Le bon vin : un \emph{gewurztraminer}.

% Generated file 2018-11-25 21:43:22.326860944 +01:00
\begin{recette}{Rôti de porc aux oignons miel}{Rôti de porc aux oignons miel}

\begin{ingredients}
800 g de rôti de porc dans l'échine\par
1 kg d'oignons de différente taille\par
1 verre de vin blanc sec\par
huile\par
sel, poivre\par
\end{ingredients}

\begin{infos}
Pour 4 à 6 personnes\\
Préparation :30 min\\
Cuisson : 30 min\\
\end{infos}

\begin{etapes}
\item Éplucher les oignons, couper les plus gros en quatre ou en huit, en émincer quelques uns et couper les petits en deux ou les laisser entiers.
\item Mettre l'huile à chauffer dans une cocotte-minute. Faire dorer le rôti sur chaque face. Réserver.
\item Faire revenir les oignons jusqu'à qu'ils blondissent. Saler, poivrer légèrement.
\item Disposer le rôti sur les oignons, verser le vin blanc. Fermer la cocotte et laisser cuire 20 minutes à partir du sifflement de la soupape.
\item Servir avec un peu de riz éventuellement, ou seulement avec les oignons.
\end{etapes}

\end{recette}
% Generated file 2019-02-17 16:33:30.964629060 +01:00
\begin{recette}{Rôti de porc aux pruneaux}{Rôti de porc aux pruneaux}

\begin{ingredients}
1 rôti de porc (800 g env.)\par
1 ou 2 oignon\par
200 à 300 g de pruneaux\par
600 g de pommes de terre\par
1 petite courge butternut\par
1 petit verre de vin blanc\par
2 c. à soupe d'huile\par
2 branches de thym\par
1 feuille de laurier\par
sel, poivre\par
\end{ingredients}

\begin{infos}
Pour 4 personnes\\
Préparation : 15 min\\
Cuisson : 15 min\\
\end{infos}

\begin{etapes}
\item Faire chauffer l'huile dans une cocotte, y faire dorer le rôti sur toutes ses faces et réserver. Émincer l'oignon et le faire revenir dans la cocotte. Remettre le rôti, ajouter le vin blanc et un verre d'eau, le thym, le laurier, saler et poivrer.
\item Couvrir et laisser cuire à feu doux 1h. Ajouter de l'eau en cours de cuisson si nécessaire.
\item Laver et éplucher les pommes de terre, les couper en deux si elles sont grosses. Les disposer autour du rôti 30 à 40 min avant la fin de la cuisson. Laver et éplucher la courge, la couper en deux. Ôter les graines et la couper en gros morceaux. L'ajouter dans la cocotte avec les pruneaux 20 min avant la fin de la cuisson.
\item Servir le rôti découpé en tranches avec sa garniture.
\end{etapes}

\begin{conseils}
Ajuster la quantité de pruneaux selon les goûts
\end{conseils}

\end{recette}
% Generated file 2018-11-25 21:43:22.190154115 +01:00
\begin{recette}{Lapin aux pruneaux}{Lapin aux pruneaux}

\begin{ingredients}
1 beau lapin\index{lapin} coupé en 8 morceaux\par
50 cl litre de vin rouge\par
1 oignon piqué de 2 clous de girofle\par
1 bouquet garni\par
16 pruneaux dénoyautés\par
1 cuillerée à soupe d’huile\par
1 gousse d’ail\par
1 cuillerée à soupe de gelée de groseille\par
sel, poivre\par
1 cuillerée à soupe de crème\par
1 cuillerée à soupe rase de Maizena\par
80 g de margarine\par
1 petite boîte de champignons de Paris\par
\end{ingredients}

\begin{infos}
Pour 6 personnes\\
Préparation : 30 min\\
Cuisson : 50 min\\
\end{infos}

\begin{etapes}
\item La veille, faites mariner le lapin au frais avec la gousse d’ail, le vin rouge, l’huile, l’oignon et le bouquet garni.
\item Faites chauffer la margarine dans la cocotte-minute SEB : mettez-y à dorer les morceaux de lapin bien égouttés, versez par-dessus la marinade et la Ma\"\i zena mélangée dans très peu d’eau froide, tourner.
\item Puis ajouter les champignons, salez, poivrez et faites cuire 20 minutes à feu doux à partir du moment où la soupape chuchote.
\item Ajoutez alors les pruneaux, faites cuire 5 minutes sous pression.
\item Mélangez la crème avec la gelée de groseille. Ouvrez la cocotte-minute, mettez les morceaux de lapin dans le plat de service, versez dans la sauce le mélange crème-gelée et nappez le lapin.
\end{etapes}

\begin{conseils}
Une purée maison sera la bienvenue comme garniture.
\end{conseils}

\end{recette}
% Generated file 2018-11-25 21:43:22.256930225 +01:00
\begin{recette}{Boeuf sauté au brocoli et aux oignons}{Boeuf sauté au brocoli et aux oignons}

\begin{ingredients}
200 g de rumsteak\par
1 petit brocoli\par
2 gros oignons\par
sauce soja\par
sauce soja sucrée\par
1 cm de gingembre émincé\par
5 épices\par
Huile\par
\end{ingredients}

\begin{infos}
Pour 2 personnes\\
Préparation : 35 min\\
Cuisson : 5 --- 10 min\\
\end{infos}

\begin{etapes}
\item Émincer le bœuf, le mettre dans un plat creux et arroser de sauce soja. Laisser mariner au moins 30 min.
\item Pendant ce temps, rincer et découper le brocoli, émincer les oignons et le gingembre.
\item Saisir rapidement en plusieurs fois la viande dans un wok avec une cuillère d'huile, réserver.
\item Faire revenir le brocoli avec les oignons et le gingembre dans le wok. Ajouter un peu de sauce soja sucrée et un peu d'eau, laisser cuire 5 à 8 minutes.
\item Assaisonner avec de la sauce soja et les 5 épices, remettre la viande et bien remuer, servir aussitôt.
\end{etapes}

\begin{conseils}
Quantités assez généreuses pour deux. Si on veut accompagner avec du riz, réduire d'un quart voire un tiers.
\end{conseils}

\end{recette}
% Generated file 2019-02-17 16:33:31.027182437 +01:00
\begin{recette}{Chop Suey de porc}{Chop Suey de porc}

\begin{ingredients}
300 g de filet de porc\par
200 g d'oignons\par
200 g de pousses de soja\par
200 g de pousses de bambou en conserve\par
6 champignons parfumés\par
Quelques brins de coriandre\par
40 g de vermicelles de soja\par
7 cuillères à soupe d'huile\par
1 cuillère à soupe de vin chinois (ou xerès)\par
2 cuillères à soupe de sauce soja\par
sel\par
\end{ingredients}

\begin{infos}
Pour 4 personnes\\
Préparation : 30 min\\
Cuisson : ?? min\\
\end{infos}

\begin{etapes}
\item Faire gonfler dans de l'eau chaude les champignons. Peler et hacher les oignons, couper le porc en lanières.
\item Verser 4 cuillères à soupe d'huile dans un wok. Quand l'huile est chaude, faire revenir les oignons 5 min. Augmenter le feu et ajouter la viande. Saler légèrement et faire cuire 5 min en remuant. Réserver.
\item Nettoyer et rincer le soja, égoutter et rincer le bambou, rincer et ciseler la coriandre.
\item Faire cuire les vermicelles comme indiqué sur le paquet. Rincer à l'eau froide.
\item Éponger les champignons et les couper en lamelles en élimint les pieds. Verser l'huile restante dans le wok, faire chauffer et mettre les vermicelles, le soja, le bambou et les champignons. Saler légèrement, cuire 4 à 5 min en remuant sans arrêt.
\item Ajouter la viande pour la faire réchauffer, puis le vin chinois et la sauce soja. Bien mélanger. Parsemer de coriandre et servir.
\end{etapes}

\end{recette}
% Generated file 2019-02-17 16:33:30.956471001 +01:00
\begin{recette}{Rosbeef}{Rosbeef}

\begin{ingredients}
1 rosbeef\par
Quelques gousses d'ail\par
1 oignon\par
Huile d'olive\par
Sel, poivre\par
\end{ingredients}

\begin{infos}
Pour 4 personnes\\
Préparation : 15 min\\
Cuisson : 20 min\\
\end{infos}

\begin{etapes}
\item Sortir le rosbeef quelques heures à l'avance pour qu'il soit à température ambiante.
\item Préchauffer le four à 240° C.
\item Piquer le rosbeef avec de l'ail. Le badigeonner d'huile d'olive et poivrer. Couper un oignon en lanières, les disposer au fond d'un plat et mettre le rosbeef par dessus.
\item Laisser cuire 15 à 20 min par livre de viande. Découper la viande, saler.
\item Déglacer le fond du plat et ajouter le jus de découpe pour faire une sauce. Servir.
\end{etapes}

\end{recette}
% Generated file 2018-12-02 20:50:19.178078525 +01:00
\begin{recette}{Stir fry de porc aux nouilles}{Stir fry de porc aux nouilles}

\begin{ingredients}
30ml d'huile\par
500g de filet de porc en fines lamelles\par
250g de poireau en rondelles de 6 mm\par
2 gousses d'ail écrasées\par
2,5 cm de gigembre pelé et finement hâché\par
200g de nouilles Chinoises\par
250g de champignons Huitres ou Chinois\par
200g de chou chinois\par
40ml de sauce Soja\par
15ml de miel liquide\par
10ml d'huile de sésame\par
\end{ingredients}

\begin{infos}
Pour 4 personnes\\
Préparation : 35 min\\
Cuisson : 10 min\\
\end{infos}

\begin{etapes}
\item Faire chauffer l'huile dans un wok ou une grande poêle.
\item Ajouter le porc, faire cuire pendant 3 min en remuant sans cesse.
\item Ajouter les poireaux et cuire pendant 2 min. Retirer la viande et les poireaux du feu, réserver.
\item Essuyer le wok avec un Sopalin. Faire chauffer le reste de l'huile. Ajouter le gingembre et l'ail et sauter 1 mine.
\item Pendant ce temps faire cuire les nouilles suivant les instructions sur le paquet.
\item Ajouter les champignons et le chou et continuer la cuisson pendant 3 min.
\item Remettre le porc et les poireaux dans la poêle. Remuer.
\item Ajouter la sauce soja ou haricots noirs, le miel et 50ml d'eau. Cuire pendant 2 min ou jusqu'à ce que la sauce soit très chaude.
\item Égoutter les nouilles. Arroser de l'huile de sésame et bien remuer. Servir de suite avec le porc.
\end{etapes}

\end{recette}
% Generated file 2018-11-25 21:43:22.219693105 +01:00
\begin{recette}{Bœuf en cocotte aux carottes}{Bœuf en cocotte aux carottes}

\begin{ingredients}
1 kg de viande de bœuf (joue, paleron, gîte, macreuse...)\par
1,5 kg d carottes\par
1 L de vin rouge\par
4 ou 5 échalotes\par
1 oignon\par
1 bouquet garni (thym et laurier)\par
4 clous de girofle\par
quelques grains de poivre\par
2 cubes de bouillon de boeuf\par
2 cuillères à soupe de farine\par
\end{ingredients}

\begin{infos}
Pour 6 personnes		\\
Préparation : 30 min + repos 12 h		\\
Cuisson : 3 h			\\
\end{infos}

\begin{etapes}
\item La veille, détailler la viande en gros cubes de 2cm de côté environ. Couper l'oignon en 2 et y planter les clous de girofles. Placer le tout dans un grand saladier avec le bouquet garni et les grains de poivre. Recouvrir avec le vin rouge. Placer au frais au moins 12 heures.
\item Le jour même, émincer les échalotes finement. Egoutter la viande, la mettre dans un autre récipient. L'éponger grossièrement avec du papier absorbant.
\item Dans une cocotte en fonte faire revenir les cubes de boeuf dans un peu d'huile. Attendre quelques minutes puis avec une écumoire, retirer la viande. Récupérer le vin rendu par la viande et l'ajouter à celui de la marinade. Remettre 2 cuillères à soupe d'huile dans la cocotte et la viande. Faire cuire les morceaux de tous les côtés, saler et poivrer. Retirer de nouveau les cubes de boeuf, réserver.
\item Ajouter une cuillère d'huile puis les échalotes dans la cocotte. Les faire rissoler jusqu'à ce qu'elles deviennent translucides puis saupoudrer avec la farine. Faire cuire ainsi quelques minutes.
\item Mouiller avec le vin de la marinade en prenant soin de le passer à travers un chinois. Ajouter les cubes de bouillon. Remettre la viande dans la cocotte. Laisser mijoter à couvert à feu doux.
\item Peler et couper les carottes en fines rondelles. Les ajouter dans la cocotte environ une heure avant la fin de la cuisson.
\item Si besoin, lier la sauce à l'aide de maïzena.
\end{etapes}

\begin{conseils}
Ne pas hésiter à préparer ce plat à l'avance, comme tous les plats mijotés c'est encore meilleur réchauffé.
\end{conseils}

\end{recette}
% Generated file 2018-11-25 21:43:22.247290845 +01:00
\begin{recette}{Boulettes de viande}{Boulettes de viande}

\begin{ingredients}
400 g de viande hachée\par
2 œufs (ou 3 si pas de pain)\par
2 tartines de pain\par
persil, ail\par
chapelure\par
sel, poivre\par
\end{ingredients}

\begin{infos}
Pour XX personnes\\
Préparation : XX min\\
Cuisson : XX min\\
\end{infos}

\begin{etapes}
\item Mélanger le tout et façonner des boulettes
\item Faire cuire à la poêle
\end{etapes}

\end{recette}
% Generated file 2019-02-17 16:33:30.780355459 +01:00
\begin{recette}{Rôti de porc braisé aux fenouils}{Rôti de porc braisé aux fenouils}

\begin{ingredients}
1 rôti de porc d’environ 1kg\par
4 bulbes de fenouil\par
1 oignon\par
2 feuilles de laurier\par
15 cl de vin blanc sec\par
2 cuil. à soupe d’huile\par
1 cuil. à soupe d’herbes de Provence\par
1 cuil. à café de poivre mignonnettes, sel.\par
\end{ingredients}

\begin{infos}
4 personnes\\
Prep. 10 min\\
Cuisson 35 min\\
\end{infos}

\begin{etapes}
\item Nettoyez les bulbes de fenouil et coupez-les en quartiers.
\item Dans un autocuiseur, faites dorer le rôti de porc avec l’huile chaude et réservez-le. A sa place, faites revenir rapidement les quartiers de fenouil et l’oignon émincé.
\item Remettez le rôti dans l’autocuiseur au centre des légumes et parsemez-le avec les herbes de Provence et le poivre mignonnette. Versez le vin blanc sur les fenouils, ajouter le laurier, salez.
\item Fermez l’autocuiseur et comptez 30 min de cuisson à partir de la rotation de la soupape. Placez le rôti dans un plat avec les légumes et servez vite.
\end{etapes}

\end{recette}
	\chapitre{Volailles**}
\section[\normalsize{Poulet \`a la Normande}]{\LARGE{\textsc{Poulet \`a la Normande}}}



\subsection*{\textsc{Ingr\'edients~:}}
\begin{itemize}
\item 1 poulet
\item 1 bouteille de cidre doux
\item	400 cl crème fraîche
\item	calvados
\end{itemize}


\subsection*{\textsc{Marche \`a suivre~:}}
\begin{enumerate}
\item D\'ecouper le poulet en morceaux.
\item Faire revenir ces morceaux dans l’huile.
Bien les dorer.
\item Flamber au calvados (attention de ne pas se mettre sous la hotte).
\item Ajouter le cidre.
\item Laisser cuire.
\item Enlever le poulet et laisser r\'eduire la sauce jusqu’\`a ce que la consistance soit celle d’une cr\`eme.
\item Ajouter la cr\`eme fra\^iche.
\item Servir avec des pommes cuites au four. 
\end{enumerate}
\section[\normalsize{Aiguillettes de dinde au citron}]{Aiguillettes de dinde au citron}

\begin{ingredients}
\item 600 g d’escalopes de dinde \index{dinde}
\item 400 g de tagliatelles
\item 300 g d’asperges vertes \index{asperge}
\item 40 g de beurre
\item 15 cl de cr\`eme fra\^iche
\item 1 citron
\item sel, poivre moulu
\end{ingredients}
\begin{infos}
\item Pour 4 personnes
\item Préparation : 15 min
\item Cuisson : 15 min
\end{infos}
\begin{etapes}
\item Raccourcissez la tige des asperges et faite-les cuire environ 15 min dans de l’eau bouillante sal\'ee.
\item Rincez et \'epongez le citron. Pr\'elevez le zeste au couteau \'econome. Plongez-le 1 min puis d\'etaillez-le en fines lani\`eres.
\item Coupez les escalopes en aiguillettes. Faites-les dorer \`a la poêle 7 min avec 20 g de beurre chaud. Ajoutez le jus de citron, la cr\`eme fra\^iche et les lani\`eres de zestes. Salez, poivrez et poursuivez la cuisson 3 min.
\item Faites cuire les tagliatelles dans beaucoup d’eau bouillante sal\'ee, selon les indications not\'ees sur l’emballage. Égouttez-les rapidement puis m\'elangez-les avec le reste de beurre.
\item R\'epartissez les tagliatelles et les asperges sur quatre assiettes pr\'echauff\'ees. Ajouter les aiguillettes, nappez-les de sauce. D\'ecorez \'eventuellement de fines herbes et servez aussitôt.
\end{etapes}
\begin{conseils}
Le zeste du citron ne sera pas amer si vous \'evitez de pr\'elever la peau blanche qui se trouve en dessous.
\end{conseils}
% Generated file 2018-11-25 21:43:22.314319789 +01:00
\begin{recette}{Poularde au vin jaune et aux morilles}{Poularde au vin jaune et aux morilles}

\begin{ingredients}
1 poularde de 2 kg coupée en morceaux\par
50 g de morielles séchées\par
25 cl de vin jaune\par
25 g de beurre\par
50 cl de crème liquide\par
1 cuillère à soupe de farine\par
Sel, poivre\par
\end{ingredients}

\begin{infos}
Pour 6 personnes\\
Préparation : 60 min\\
Cuisson : 50 min\\
\end{infos}

\begin{etapes}
\item Laisser tremper les morilles 30 à 60 minutes dans une jatte d'eau tiède pour les réhydrater.
\item Pendant ce temps, préchauffer le four à 180° C. Saler, poivrer et fariner les morceaux de poularde. Les faire dorer dans une cocotte allant au four avec le beurre.
\item Couvrir, cuire 25 min. au four.
\item Retirer la poularde de la cocotte, jeter la graisse. Verser le vin jaune, le faire bouillir et réduire 3 minutes. Ajouter la crème.
\item Les morilles égouttées, replacer la poularde et cuire 20 minutes à feu doux sans couvrir.
\item Rectifier l'assaisonnement en fin de cuisson, servir chaud.
\end{etapes}

\begin{conseils}
Peut nécessiter un temps de cuisson plus long. Pour corser la sauce, mélanger l'eau de trempage des morilles filtrée avec la crème et verser dans le vin réduit.
\end{conseils}

\end{recette}
% Generated file 2019-02-17 16:33:30.983558076 +01:00
\begin{recette}{Poulet à la moutarde, à l'estragon et aux champignons}{Poulet à la moutarde, à l'estragon et aux champignons}

\begin{ingredients}
2 blancs de poulet sans la peau\par
200 g de champignons de Paris émins\par
1 cube de bouillon de volaille\par
1 verre de vin blanc\par
2 cuillères à café de moutarde\par
2 cuillères à café de crème fraîche allégée\par
2 cuillères à café d'estragon\par
2 cuillères à café d'huile d'olive\par
2 échalotes émine\par
sel, poivre\par
\end{ingredients}

\begin{infos}
Pour 2 personnes\\
Préparation : 40 min\\
Cuisson : 20 min\\
\end{infos}

\begin{etapes}
\item Faire revenir les échalotes dans l'huile d'olive 3 minans faire roussir.
\item Ajouter les champignons et laisser cuire 2 min
\item Ajouter le bouillon de volaille dissout dans 1/2 verre d'eau. Cuire 10 min
\item Ajouter le vin blanc, laisser réduire.
\item Faire revenir les blancs de poulet dans une poêle anti-adhésive jusqu'à ce qu'ils soient dorés. Les ajouter aux champignons et cuire 10 min
\item A la fin, enlever le poulet, ajouter la moutarde, la crème fraîche et l'estragon. Éminr les blancs de poulet en tranches et servir avec les champignons.
\end{etapes}

\end{recette}
% Generated file 2018-12-02 20:50:19.188554474 +01:00
\begin{recette}{Poulet à l'ananas}{Poulet à l'ananas}

\begin{ingredients}
4 ou 5 blancs de poulet\par
1 boîte d'ananas en morceaux\par
sauce soja\par
1 cube de bouillon de volaille\par
un peu de maïzena\par
sucre en poudre\par
huile\par
épices (ail, gingembre, 5 épices)\par
\end{ingredients}

\begin{infos}
Pour 4 personnes\\
Préparation : 30 min\\
Cuisson : ?? min\\
\end{infos}

\begin{etapes}
\item Coupez les blancs de poulet en fines lamelles.
\item Ajouter 3 cuillères à soupe de sauce soja, le jus de l'ananas, 1 cuillère de maïzena, des épices, laisser mariner.
\item Faire fondre le bouillon cube dans un peu d'eau.
\item Dans une poêle anti-adhésive, verser quelques cuillères à soupe d'huile et faire chauffer. Quand celle ci est bien chaude, y faire revenir les ananas.
\item Ajouter le bouillon afin que celui ci recouvre entièrement les ananas. Réserver.
\item Faire dorer le poulet (sans la marinade) dans la poêle, ajouter l'ananas et la marinade, bien mélanger. Sucrer la sauce en goûtant au fur et à mesure.
\item Laissez mijoter le tout jusqu'à complète cuisson. Rectifier l'assaisonnement si nécessaire.
\end{etapes}

\end{recette}
% Generated file 2018-11-25 21:43:22.199955352 +01:00
\begin{recette}{Poulet au curry et lait de coco}{Poulet au curry et lait de coco}

\begin{ingredients}
4 blancs de poulet \index{poulet}\par
1 oignon \index{oignon}\par
4 gousses d'ail\par
une boîte de lait de coco\index{lait de coco}\par
curry en poudre\par
1 c. à soupe de gingembre haché (facultatif)\par
le jus d'1 citron vert\par
huile\par
sel, poivre\par
\end{ingredients}

\begin{infos}
Pour 4 personnes\\
Préparation : 45 min\\
Cuisson : ?? min\\
\end{infos}

\begin{etapes}
\item Préparer les ingrédients : couper les blancs de poulet en lanières, émincer l'oignon, éplucher les gousses d'ail (les hacher si on ne dispose pas d'un presse ail).
\item Faire chauffer un peu d'huile dans le wok. Faire revenir la viande en plusieurs fois ; elle doit être saisie et presque cuite. Réserver.
\item Remettre un peu d'huile et faire revenir l'oignon. Lorsqu'il est est tendre, ajouter le poulet et les gousses d'ail hachées. Bien mélanger, saler, poivrer.
\item Ajouter environ quatre cuillères à café de curry (en mettre selon son goût), le gingembre et mélanger, puis verser les 3/4 du jus de citron vert environ (ne pas tout mettre afin de pouvoir rectifier l'assaisonnement si nécessaire). Lorsque la viande est cuite, ajouter le lait de coco (ne pas tout mettre non
\item plus).
\item Goûter et rectifier si besoin en ajoutant du jus de citron, du lait de coco ou du curry. Laisser mijoter un peu et servir bien chaud avec du riz cuit à la vapeur.
\end{etapes}

\end{recette}
% Generated file 2018-12-02 20:50:19.099886223 +01:00
\begin{recette}{Poulet basquaise (façon Claire T.)}{Poulet basquaise (façon Claire T.)}

\begin{ingredients}
6 morceaux de poulet \index{poulet}\par
1 kg de tomates \index{tomate}\par
700 g de poivrons (verts et rouges) \index{poivron}\par
3 oignons émins \index{oignon}\par
3 gousses d'ail\par
1 verre de vin blanc\par
1 bouquet garni,\par
huile d'olive,\par
poivre, sel\par
\end{ingredients}

\begin{infos}
Pour 6 personnes\\
Préparation : 35 min\\
Cuisson : 55 min\\
\end{infos}

\begin{etapes}
\item Dans une cocotte, faire dorer dans l'huile d'olive les morceaux de poulet salés et poivrés. Réserver.
\item Faire chauffer 4 cuillères à soupe d'huile, y faire dorer les oignons, l'ail pressé, les poivrons taillés en lanières. Laisser cuire 5 min
\item Laver, éplucher et couper les tomates en morceaux, les ajouter à la cocotte, sel, poivre. Couvrir et laisser mijoter 20 min
\item Ajouter le poulet aux légumes, ajouter le bouquet garni et le vin blanc, couvrir et laisser cuire à feu très doux 35 min
\end{etapes}

\begin{conseils}
Attention à la cuisson, ça accroche vite...
\end{conseils}

\end{recette}	
	\chapitre{Poissons}
% Generated file 2019-02-17 16:33:31.029863384 +01:00
\begin{recette}{Papillote de saumon et julienne de légumes au boursin}{Papillote de saumon et julienne de légumes au boursin}

\begin{ingredients}
2 pavés de saumon\par
250 g de julienne de légumes surgelée\par
boursin cuisine ail et fines herbes\par
\end{ingredients}

\begin{infos}
Pour 2 personnes\\
Préparation : 15 min\\
Cuisson : 15 min\\
\end{infos}

\begin{etapes}
\item Si le saumon n'est pas congelé, faire décongeler la juliene de légumes.
\item Préchauffer le four à 180° C.
\item Découper 2 grandes feuilles de papier sulfurisé. Ôter la peau du saumon.
\item Répartir environ 1/4 de la julienne sur chaque feuille, déposer un pavé de saumon par dessus, puis étaler un peu de boursin dessus.
\item Mélanger 2 à 3 cuillères à soupe de boursin avec le reste de julienne, puis la répartir sur et autour du saumon.
\item Fermer les papillottes, et les faire cuire au four 15 min(30 si les produits sont congelés).
\end{etapes}

\end{recette}
% Generated file 2019-02-17 16:33:30.980864797 +01:00
\begin{recette}{Saumon en papillotes tout simple}{Saumon en papillotes tout simple}

\begin{ingredients}
4 pavés de saumon\par
huile d'olive\par
thym, romarin (ou herbes de Provence)\par
fleur de sel, poivre\par
\end{ingredients}

\begin{infos}
Pour 4 personnes\\
Préparation : 10min\\
Cuisson : 15 min\\
\end{infos}

\begin{etapes}
\item Préchauffer le four à 180°C.
\item Placer les pavés sur 4 feuilles de papier sulfurisé ou d'aluminm, ou dans des papillottes en silicone.
\item Les arroser d'un petit filet d'huile d'olive.
\item Saupoudrer de thym et de romarin (briser le romarin en petits morceaux au préalable).
\item Saler, poivrer, refermer les papillottes.
\item Faire cuire 10 à 15 mi., le saumon ne doit pas être trop sec.
\end{etapes}

\end{recette}	
	\chapitre{Oeufs}

	\chapitre{Pains}
% Generated file 2018-11-25 21:43:22.298520339 +01:00
\begin{recette}{Petits pains aux lardons}{Petits pains aux lardons}

\begin{ingredients}
500 g de farine\par
1/2 cuillère à café de sel\par
25 g de levure de boulanger\par
25 cl de lait tiède\par
50 g de beurre\par
1 oeuf\par
lardons allumettes\par
\end{ingredients}

\begin{infos}
Pour XX personnes\\
Préparation : XX min*	\\
Cuisson : 15 -- 20 min\\
\end{infos}

\begin{etapes}
\item Mélanger la farine, la levure et le sel
\item Ajouter le reste des ingrédients et pétrir
\item Laisser reposer (combien de temps ?)
\item Façonner les petits pains en faisant des boules de 40 g de pâte environ.
\item Cuire à 200 °C pendant 15 à 20 minute
\end{etapes}

\begin{conseils}
Pour faire le grand raisin, prendre 1 kg de farine pour 1 cube de levure et 2 boîtes de lardon.
\end{conseils}

\end{recette}
	\chapitre{Gratins}
	\chapitre{Plat unique ? (couscous choucroute)}
% Generated file 2019-02-17 16:33:30.931494136 +01:00
\begin{recette}{Tartiflette (façon Claire T.)}{Tartiflette (façon Claire T.)}

\begin{ingredients}
1 kg de pommes de terre à chair ferme\par
1 reblochon fermier ou fruité\par
200g de lardons\par
1 gros oignon\par
2 cuillères à soupe de crème fraîche\par
vin blanc de Savoie (Apremont)\par
1 gousse d'ail\par
\end{ingredients}

\begin{infos}
Pour 6 personnes\\
Préparation : 50 min\\
Cuisson : 10 min\\
\end{infos}

\begin{etapes}
\item Faire cuire les pommes de terre dans de l'eau (départ à froid) pendant 20 à 25 min. Egoutter et éplucher.
\item Préchauffer le four à 200° C.
\item Eminr finement l'oignon. Faire fondre les lardons dans une poêle, puis ajouter les oignons (avant que les lardons ne colorent).
\item Laisser revenir à feu doux.
\item Pendant ce temps, couper les pommes de terre en rondelles dans un saladier. Frotter un plat à gratin avec une gousse d'ail et le beurrer généreusement.
\item Gratter le reblochon des deux côtés, ôter la pastille de caséine et le couper en deux dans l'épaisseur.
\item Quand les lardons et les oignons ont bien fondu (attention à ne pas faire trop colorer), ajouter un peu de vin blanc et faire réduire.
\item Les ajouter aux pommes de terre. Ajouter la crème et mélanger.
\item Verser ce mélange dans le plat à gratin, arroser de vin blanc et poser les deux moitiés de reblochon par dessus, croûte vers le haut.
\item Enfourner une dizaine de min, jusqu'à ce que le fromage ait bien fondue.
\end{etapes}

\end{recette}
% Generated file 2019-02-17 16:33:30.925267585 +01:00
\begin{recette}{Chili con carne}{Chili con carne}

\begin{ingredients}
1 oignon\par
1 poivron rouge\par
500 g de viande hachée\par
3 boîtes de 400 g de haricots rouges\par
2 boîtes de 400 g de tomates en dés\par
épices mexicaines\par
cumin\par
huile d'olive\par
\end{ingredients}

\begin{infos}
Pour 6 personnes\\
Préparation : 30 min\\
Cuisson : 30 min\\
\end{infos}

\begin{etapes}
\item Faire revenir l'oignon et le poivron émins dans un peu d'huile.
\item Ajouter la viande hachée et faire dorer.
\item Ajouter les haricots rouges égouttés, le mélange d'épices et le cumin. Laisser cuire quelques minutes.
\item Ajouter les tomates, bien mélanger et laisser revenir à feu doux 30 mines. Rectifier l'assaisonnement si nécessaire.
\end{etapes}

\end{recette}
% Generated file 2019-02-17 16:33:30.899765004 +01:00
\begin{recette}{Hamburgers}{Hamburgers}

\begin{ingredients}
4 grands pains à hamburgers\par
4 steaks hachés (500 g de viande hachée)\par
1 oeuf\par
1 cuillère à soupe de chapelure\par
1 oignon frais\par
2 cuillères à soupe de persil\par
1 cuillère à soupe de parmesan\par
Sel, poivre\par
4 feuilles de salade\par
1 ou 2 tomates\par
1 oignon rouge\par
8 tranches de cheddar\par
Sauce :\par
1 yaourt nature\par
2 cuillères à soupe de ketchup\par
1 à 2 cuillères à soupe de moutarde\par
\end{ingredients}

\begin{infos}
Pour 4 personnes\\
Préparation : 1h + 30 min\\
Cuisson : 10 min\\
\end{infos}

\begin{etapes}
\item Mettre les steaks dans un saladier, les écraser rapidement à la fourchette. Ajouter la chapelure, l'oeuf, l'oignon et le persil hachés finement, le parmesan, sel et poivre, et mélanger.
\item Façonner 4 boules un peu aplaties, les rouler dans un peu de farine et les déposer dans une assiette. Filmer et laisser reposer au frais 1h.
\item Laver la salade et les tomates, éplucher l'oignon, préparer la sauce. Aplatir les steaks à un diamètre à peine inférieur à celui des pains et les faire cuire à feu vif.
\item Préchauffer le four à 230° C.
\item Trancher les tomates, couper la salade en larges lanières, couper l'oignon en rondelles.
\item Toaster légèrement la moitié inférieure des pains et les déposer sur une plaque de cuisson. Mettre une cuillère de sauce, un peu de salade, 2 rondelles de tomates et une d'oignon et une tranche de cheddar sur chaque moitié de pain.
\item Déposer un steaks sur la moitié inférieure, et recouvrir avec l'autre moitié. Presser un peu les hamburgers, les piquer avec une brochette et enfourner quelques min le temps que le fromage fonde.
\end{etapes}

\begin{conseils}
Possibilité de rajouter un peu de coriandre ciselée dans la viande, ou encore 1/4 de cuillère à café de coriandre moulue et 1/4 de cuillère à café de cumin
\end{conseils}

\end{recette}
% Generated file 2018-12-02 20:50:19.238774748 +01:00
\begin{recette}{Riz aux saucisses}{Riz aux saucisses}

\begin{ingredients}
1 bel oignon\par
2 saucisses de toulouse\par
3 doses de riz\par
2 doses d'eau\par
4 doses de vin rouge\par
fromage rapé ou parmesan\par
sel, poivre\par
\end{ingredients}

\begin{infos}
Pour 3 personnes\\
Préparation : 15 min\\
Cuisson : 15 min\\
\end{infos}

\begin{etapes}
\item Faire revenir l'oignon émin dans un peu d'huile sans faire colorer.
\item Ajouter les saucisses coupées en rondelles et faire dorer. * Verser le riz, remuer jusqu'à ce que les grains commencent à devenir translucides.
\item Ajouter l'eau et le vin, saler, poivrer et couvrir.
\item Laisser cuire jusqu'à ce que le liquide soit pratiquement absorbé (10 à 15 min).
\item Ajouter le fromage rapé, mélanger et finir la cuisson encore quelques min.
\end{etapes}

\end{recette}		
	\chapitre{Non classés}
% Generated file 2019-02-17 16:33:30.978205966 +01:00
\begin{recette}{Poêlée de petit épeautre aux échalotes et pesto de roquette}{Poêlée de petit épeautre aux échalotes et pesto de roquette}

\begin{ingredients}
Pour le pesto :\par
60 g de roquette\par
40 g de pignons (ou noisettes, noix, amandes...)\par
20 g de parmesan\par
3 cuillères à soupe d'huile d'olive\par
1 cuillère à soupe de jus de citron\par
Pour le petit épeautre :\par
200 g de petit épeautre\par
8 à 12 échalotes (selon la taille)\par
3 belles poignées de roquette (facultatif)\par
huile d'olive\par
\end{ingredients}

\begin{infos}
Pour 4 personnes*	\\
Préparation : 20 min\\
Cuisson : 75 min\\
\end{infos}

\begin{etapes}
\item Mesurer le volume de petit épeautre. Si possible, le faire tremper pendant environ douze heures. Égoutter et rincer.
\item Verser le petit épeautre dans une grande casserole avec deux fois son volume d'eau froide. Porter à frémissements. Ajouter des aromates (romarin, laurier, bouillon cube...). Couvrir et laisser cuire à feu doux 45 minsi trempage) à 1 heure. Couper le feu et laisser gonfler 5 à 10 min. Égoutter.
\item Pendant ce temps, préparer le pesto en mixant ensemble tous les ingrédients. Réserver.
\item Peler les échalotes. Les couper en deux ou en quatre (si elles sont grosses) dans la longueur.
\item Faire chauffer 2 cuillères à soupe d’huile dans une sauteuse et faire suer les échalotes sans trop les défaire. Ajouter le petit-épeautre et le pesto. Mélanger pendant 1 minAjouter la roquette restante, mélanger et servir.
\end{etapes}

\end{recette}
% Generated file 2019-02-17 16:33:30.737434868 +01:00
\begin{recette}{Bouchées au thon}{Bouchées au thon}

\begin{ingredients}
1 boîte de thon (112g égoutté)\par
1 oeuf\par
20 g de gruyère\par
20 g de crème liquide\par
15 g de concentré de tomate\par
1/2 oignon\par
1 c. à s. de moutarde (moyenne)\par
sel, poivre\par
\end{ingredients}

\begin{infos}
Pour ~15 bouchées\\
Préparation : 5 min\\
Cuisson : 20 min\\
\end{infos}

\begin{etapes}
\item Hacher l'oignon, mélanger avec le reste des ingrédients.
\item Verser dans les alvéoles d'un moule à petits fours et cuire 20 min au four à 180°C. Servir froid.
\end{etapes}

\begin{conseils}
Choisir un moule avec des toutes petites alvéoles.
\end{conseils}

\end{recette}
	

\part{Les desserts}
	\chapitre{Tartes}
\section[\normalsize{Tarte mousseuse aux framboises}]{Tarte mousseuse aux framboises}

\begin{ingredients}
\item Pour la p\^ate :
\begin{itemize}
\item	250 g de farine
\item	10 g de levure de boulanger
\item	1 oeuf
\item	40 g de sucre
\item	10 cl de lait
\item	un quart de cuil. à caf\'e de sel
\item	100 g de beurre ramolli
\end{itemize}
\item Pour la garniture :
\begin{itemize}
\item	300 g de framboises \index{framboises}
\item	100 g de sucre
\item	3 oeufs
\item	1 cuil. à soupe de sucre glace
\item	125 g d’amandes en poudre
\end{itemize}
\end{ingredients}
\begin{infos}
\item Pour 6 personnes
\item Préparation : 45 min
\item Repos : 1 h
\item Cuisson : 40 min
\end{infos}
\begin{etapes}
\item  Dans une terrine, d\'elayez la levure dans 5 cl de lait ti\`ede et une pinc\'ee de sucre. Ajoutez 60 g de farine et m\'elangez pour obtenir une sorte de bouillie. Couvrez d’un linge et laissez lever 15 min dans un endroit ti\`ede.
\item  Versez le reste de farine sur le levain, ajoutez le reste de lait ti\`ede, le sel, le sucre et l’oeuf. P\'etrissez la p\^ate à la main ou au robot, jusqu’à ce qu’elle soit homog\`ene. Incorporez le beurre et travaillez jusqu’à ce que la p\^ate se d\'etache et ne colle plus aux mains (ou 5 min au robot).
\item Laissez reposer sous un linge 45 min dans un endroit ti\`ede. La p\^ate doit doubler de volume.
\item  P\'etrissez la p\^ate 30 secondes pour la faire retomber. Beurrez un moule rectangulaire et garnissez-le avec la p\^ate, en appuyant pour faire adh\'erer.
\item  Allumez le four th. 7 (210° C). Laissez la p\^ate au ti\`ede pendant que vous pr\'eparez la garniture.
\item  S\'eparez les blancs des jaunes d’oeufs. Fouettez les jaunes avec 50 g de sucre jusqu’à ce que le m\'elange blanchisse. Ajoutez la poudre d’amandes, m\'elangez. Battez les blancs en neige ferme avec le reste du sucre et incorporez d\'elicatement cette mousse dans la masse pr\'ec\'edente.
\end{etapes}
\begin{etapes}
\item  Etalez cette pr\'eparation sur le fond de tarte, puis r\'epartissez les framboises à la surface. Elles vont s’enfoncer dans la cr\`eme au cours de la cuisson. Enfournez la tarte à mi-hauteur et faites cuire pendant environ 40 min.
\item  Laissez refroidir 5 à 10 min dans le moule, pus d\'emoulez la tarte sur une grille. Poudrez de sucre glace avant de d\'eguster. 
\end{etapes}
\begin{conseils}
\end{conseils}


% Generated file 2018-12-02 20:50:18.915801277 +01:00
\begin{recette}{Tarte aux pommes alsacienne (façon Claire T.)}{Tarte aux pommes alsacienne (façon Claire T.)}

\begin{ingredients}
1 pâte brisée\par
5 ou 6 belles pommes\par
Pour la garniture :\par
2 jaunes d'oeufs\par
1 petit pot de crème\par
Environ 75g de sucre\par
1 c. à café de cannelle\par
\end{ingredients}

\begin{infos}
Pour 6 personnes\\
Préparation : 45 min\\
Cuisson : ?? min\\
\end{infos}

\begin{etapes}
\item Étaler la pâte et la disposer dans un moule beurré. Couper les pommes en tranches pas trop fines, les disposer en cercles concentriques.
\item Préparer la garniture : dans un saladier, battre les oeufs à la fourchette puis ajouter la crème, le sucre et la cannelle. Bien mélanger et verser sur les pommes.
\item Mettre au four.
\end{etapes}

\end{recette}
	\chapitre{G\^ateaux}
% Generated file 2018-11-25 21:43:22.244903399 +01:00
\begin{recette}{Amandines à la poire}{Amandines à la poire}

\begin{ingredients}
2 poires William ou des poires au sirop \index{poire}\par
100 g chocolat noir \index{chocolat}\par
1 boite de 375 g de lait concentré sucré\par
4 oeufs\par
60 g d’amandes en poudre \index{amandes}\par
\end{ingredients}

\begin{infos}
Pour 6 personnes\\
Préparation : 15 min\\
Cuisson : 25 min\\
\end{infos}

\begin{etapes}
\item Préchauffer le four th. 5 180° C.
\item Laver et peler les poires.
\item À l’aide d’un couteau, hacher le chocolat en pépites.
\item Dans un saladier, mélanger le lait concentré sucré, les œufs, la poudre d’amandes et les pépites de chocolat.
\item Mixer les poires et ajouter-les à la préparation.
\item Répartir la préparation dans des ramequins.
\item Faites cuire 25 min.
\end{etapes}

\begin{conseils}
\end{conseils}

\end{recette}
% Generated file 2019-02-17 16:33:30.806495388 +01:00
\begin{recette}{Gâteau poires chocolat}{Gâteau poires chocolat}

\begin{ingredients}
150 g de farine\par
125 g de beurre\par
125 g de sucre\par
4 oeufs\par
150 g de pépites de chocolat\par
100 g de poudre d'amandes ou noisettes\par
1/2 paquet de levure chimique\par
1 grosse boîte de poires au sirop (ou 4-5 poires fraiches)\par
\end{ingredients}

\begin{infos}
Pour 6 personnes\\
Préparation : 15 min\\
Cuisson : 50 min\\
\end{infos}

\begin{etapes}
\item Faire fondre le beurre, puis dans un saladier mélanger la farine, la levure, le sucre, le beurre fondu, les oeufs et la poudre d'amande (ou noisettes).
\item Ajouter les poires coupées en dés et les pépites de chocolat. Mélanger délicatement pour ne pas broyer les poires.
\item Verser dans un moule à manqué beurré (25 cm de diamètre environ). Mettre dans le four préchauffé, à hauteur moyenne et faire cuire 10 minutes à 160°C, puis 40 minutes à 180°C. Vérifier la cuisson avec la pointe d'un couteau planté dans le gâteau : la lame doit ressortir sèche. Si ce n'est pas le cas, prolonger la cuisson 10 minutes de plus.
\end{etapes}

\begin{conseils}
Faire cuire tout en bas du four.
\end{conseils}

\end{recette}
% Generated file 2018-12-02 20:50:19.497513642 +01:00
\begin{recette}{Brownies (façon Claire T.)}{Brownies (façon Claire T.)}

\begin{ingredients}
2 oeufs\par
50 g de sucre\par
50 g de farine\par
150 g de chocolat\par
50 g de beurre\par
noix (env. 50 g)\par
\end{ingredients}

\begin{infos}
Pour 6 personnes\\
Préparation : 15 min\\
Cuisson : 15 min\\
\end{infos}

\begin{etapes}
\item Préchauffer le four à 200° C.
\item Dans une terrine, battre les oeufs et le sucre.
\item Ajouter la farine, mélanger.
\item Faire fondre le chocolat avec le beurre.
\item Ajouter à la préparation et bien mélanger.
\item Ajouter les noix en morceaux.
\item Faire cuire 15 à 20 min.
\end{etapes}

\begin{conseils}
En divisant les quantités par deux on obtient 6 minbrownies (dans des moules à muffins).
\end{conseils}

\end{recette}
% Generated file 2018-12-02 20:50:19.320801139 +01:00
\begin{recette}{Fondant au chocolat (façon Claire T.)}{Fondant au chocolat (façon Claire T.)}

\begin{ingredients}
200 g de chocolat\par
150 g de beurre\par
5 oeufs\par
1 cuillère à soupe de farine bombée\par
180 g de sucre\par
\end{ingredients}

\begin{infos}
Pour 6 personnes\\
Préparation : 15 min\\
Cuisson : 20 min\\
\end{infos}

\begin{etapes}
\item Préchauffer le four à 190° C.
\item Faire fondre le chocolat et le beurre au bain marie. Ajouter au sucre et laisser refroidir un peu.
\item Ajouter les oeufs un à un en remuant bien à chaque fois. Verser la farine, lisser le mélange.
\item Verser dans un moule à manqué beurré et faire cuire 20 min au four.
\end{etapes}

\begin{conseils}
Ne pas attendre pour mélanger les oeufs au risque de les voire cuire dans la préparation.
À la fin de la cuisson le gâteau doit avoir un aspec un peu tremblottant au centre.
\end{conseils}

\end{recette}
% Generated file 2018-11-25 21:43:22.259276832 +01:00
\begin{recette}{Gâteau au chocolat (façon Claire T.)}{Gâteau au chocolat (façon Claire T.)}

\begin{ingredients}
5 oeufs\par
150 g de sucre\par
250 g de chocolat\par
100 g d'amandes en poudre\par
200 g de beurre\par
3 c. à soupe de fécule de pomme de terre\par
2 sachets de sucre vanillé\par
1/2 c. à café de levure\par
Pour le glaçage :\par
125 g de chocolat à croquer\par
50 g de beurre\par
100 g de sucre glace\par
\end{ingredients}

\begin{infos}
Pour 6 personnes\\
Préparation : 45 min\\
Cuisson : 40 min\\
\end{infos}

\begin{etapes}
\item Fouetter les jaunes d'oeufs avec les sucres jusqu'à ce qu'ils blanchissent et fassent ruban.
\item Faire fondre le chocolat avec deux cuillères à soupe d'eau au bain-marie non bouillant. L'ajouter aux jaunes, mélanger.
\item Ajouter la poudre d'amandes tamisée avec la levure et la fécule, le beurre très ramolli mais non fondu, puis les blancs d'oeufs battus en neige avec une pincée de sel.
\item Verser dans un moule de 24 cm de diamètre, beurré et saupoudré de sucre semoule, en ne remplissant qu'aux trois quarts et mettre au four th 6-7. Démouler tiède.
\item Pour le glaçage, faire fondre le chocolat avec une cuillère à soupe d'eau et le beurre au bain-marie. Ajouter le sucre glace tamisé par cuillères puis enduire le gâteau.
\end{etapes}

\end{recette}
% Generated file 2018-12-02 20:50:19.120028376 +01:00
\begin{recette}{Muffins aux pépites de chocolat (façon Claire T.)}{Muffins aux pépites de chocolat (façon Claire T.)}

\begin{ingredients}
300 g de farine\par
1 sachet de levure\par
100 g de sucre\par
1 pincée de sel\par
150 g de chocolat à pâtisserie\par
25 cl de lait\par
2 oeufs\par
75 g de beurre fondu\par
1 cuillère à soupe de sirop d'érable\par
\end{ingredients}

\begin{infos}
Pour 4 personnes\\
Préparation : 40 min\\
Cuisson : 20 min\\
\end{infos}

\begin{etapes}
\item Préchauffer le four à 180° C (th. 6).
\item Couper chaque carré de chocolat en 4 en se servant d'un grand couteau.
\item Dans un grand bol, mélanger la farine, la levure, le sucre, la pincée de sel et le chocolat.
\item Dans un pichet, casser et battre légèrement les oeufs. Ajouter le lait, le sirop d'érable et le beurre fondu.
\item Verser ce liquide sur le mélange sec et mélanger juste assez pour que la farine ne soit plus visible : la pâte doit être grumeleuse.
\item Verser dans les moules à muffins à l'aide d'une grande cuillère et enfourner pour 20 min.
\end{etapes}

\end{recette}
% Generated file 2018-12-02 20:50:19.403002807 +01:00
\begin{recette}{Scones}{Scones}

\begin{ingredients}
250 g de farine\par
40 g de sucre\par
50 g de beurre\par
1 jaune d'oeuf\par
150 ml de lait\par
50 g de raisins secs\par
1 pincée de sel\par
1 sachet de levure chimique\par
\end{ingredients}

\begin{infos}
Pour 4 personnes\\
Préparation : 15 min\\
Cuisson : 15 min\\
\end{infos}

\begin{etapes}
\item Faire préchauffer le four à 220° C.
\item Dans un bol, mélanger la farine, le sucre, le sachet de levure et la pincée de sel. Rajouter au mélange le beurre très mou coupé en petites lamelles.
\item Dans un autre bol, mélanger ensemble le lait avec le jaune d'oeuf.
\item Rajouter progressivement ce mélange à la farine, rajouter un peu de farine si la pâte colle trop.
\item Rajouter les raisins. Etaler la pâte sur 2cm d'épaisseur, découper des cercles à l'emporte pièce ou avec un verre, les déposer sur une plaque de cuisson anti-adhésive. Badigeonner de lait avec un pinceau.
\item Mettre au four pendant 15 min
\end{etapes}

\end{recette}
% Generated file 2018-12-02 20:50:19.302917438 +01:00
\begin{recette}{Gâteau au yaourt}{Gâteau au yaourt}

\begin{ingredients}
1 pot de yaourt nature\par
2 pot de sucre\par
3 pot de farine\par
1 pot d'huile\par
3 oeufs\par
1 sachet de levure chimique\par
15 g de beurre\par
\end{ingredients}

\begin{infos}
Pour 4 personnes\\
Préparation : 15 min\\
Cuisson : 35 min\\
\end{infos}

\begin{etapes}
\item Préchauffer le four à 180°C.
\item Verser le yaourt dans un saladier puis rincer le pot pour pouvoir l'utiliser comme doseur.
\item Ajouter le sucre, les oeufs et mélanger le tout pour obtenir un mélange mousseux.
\item Ajouter la farine et la levure. Mélanger et ajouter l'huile.
\item Beurrer le moule, verser la préparation et laisser cuire 35 min.
\end{etapes}

\begin{conseils}
Il est possible de remplacer le pot d'huile par un pot de beurre ou de crème fraîche et le pot de farine par un pot d'amandes en poudre.
\end{conseils}

\end{recette}
% Generated file 2019-02-17 16:33:30.856021469 +01:00
\begin{recette}{Gâteau meringué}{Gâteau meringué}

\begin{ingredients}
300 g de sucre semoule\par
240 g de farine\par
200 g de chocolat à croquer\par
250 ml de lait\par
3 oeufs\par
50 g de beurre mou\par
1 cuillère à soupe de levure chimique\par
1 cuillère à café de vanille en poudre\par
\end{ingredients}

\begin{infos}
Pour 8 personnes\\
Préparation : 30 min\\
Cuisson : 50-60 min\\
\end{infos}

\begin{etapes}
\item Préchauffer le four à thermostat 5 (200°C) et beurrer le moule (rectangulaire 33x22 cm).
\item Casser les oeufs en séparant les blancs des jaunes; mettez les blancs dans un saladier et les jaunes dans une tasse.
\item Tamiser la farine avec la levure.
\item Casser le chocolat en petits morceaux, faites fondre ceux-ci dans une grande jatte posée au dessus d'une casserole remplie d'eau frémissante.
\item Passer une terrine à l'eau chaude, l'essuyer, y mettre le beurre et le travailler avec une cuillère pour le rendre crémeux.
\item Y verser ensuite 250 g de sucre en pluie en continuant de tourner.
\item Ajouter les jaunes d'oeufs un à un, en mélangeant entre chaque addition.
\item Incorporer alors le chocolat et la vanille.
\item Verser un tiers du lait, mélanger, ajouter un tiers de la farine, mélanger et continuer ainsi jusqu'à épuisement des ingrédients.
\item Ajouter 2 pincées de sel aux blancs d'oeufs et les battre en neige ; lorsqu'ils commencent à prendre, verser progressivement le sucre restant tout en continuant à battre jusqu'à obtenir la consistance d'une meringue.
\item Incorporer ces blancs battus à la pâte en soulevant la masse de bas en haut en tournant à l'aide d'une spatule.
\item Remplir le moule de pâte et faire cuire 1~h au four. À la fin de la cuisson, le gâteau doit être recouvert d'une croûte min et satinée. Laisser tiédir pendant 5 min, puis démouler le gâteau et laisser le refroidir complètement avant de le servir.
\end{etapes}

\end{recette}
% Generated file 2019-02-17 16:33:30.997255733 +01:00
\begin{recette}{Gâteau renversé à l'ananas}{Gâteau renversé à l'ananas}

\begin{ingredients}
3 oeufs\par
120 g de sucre\par
100 g de beurre\par
150 g de farine\par
1/2 sachet de levure\par
1 boîte d'ananas en tranches\par
rhum\par
60 g de sucre pour le caramel\par
\end{ingredients}

\begin{infos}
Pour ~15 bouchées\\
Préparation : 5 min\\
Cuisson : 20 min\\
\end{infos}

\begin{etapes}
\item Travailler le beurre pour le rendre crémeux, ajouter le sucre et mélanger. Ajouter les oeufs un par un puis la farine mélangée à la levure et parfumer avec un peu de rhum.
\item Beurrer un moule à manqué, préchauffer le four à 180°C.
\item Préparer le caramel en faisant cuire le sucre légèrement humidifié avec du jus d'ananas dans une petite poêle. Le liquéfier si besoin une fois cuit avec du jus d'ananas, le verser dans le moule.
\item Répartir les tranches d'ananas sur le caramel et mettre le restant soit sur les côtés du moule soit en morceaux dans la pâte. Garder le jus.
\item Verser la pâte dans le moule et cuire environ 30 minutes. À la sortie du four, démouler le gâteau et l'imbiber doucement de jus d'ananas. Laisser refroidir.
\end{etapes}

\begin{conseils}
\end{conseils}

\end{recette}	
	\chapitre{Flans}
% Millas Bordelais											% <-- x1
% % Millas Bordelais											% <-- x1
% % Millas Bordelais											% <-- x1
% \include{./recettes/millasbordelais}								% <-- x1
\section[\normalsize{Millas Bordelais}]{\LARGE{\textsc{Millas Bordelais}}}		% <-- x2


\begin{itemize}
\item Pour 6 personnes*
\item Préparation : 15 min*
\item Cuisson : 30 min
\end{itemize}

\subsection*{\textsc{Ingr\'edients~:}}

\begin{itemize}
\item 50 cl de lait vanill\'e
\item 3 oeufs
\item 6 cs de sucre
\item 6 cs de farine
\item 50 g de beurre
\end{itemize}


\subsection*{\textsc{Marche \`a suivre~:}}

\begin{enumerate}
\item Mettre les jaunes et le sucre.

\item Battre en cr\`eme.

\item Ajouter le beurre fondu, la farine.

\item D\'elayer avec le lait chaud.

\item Battre les blancs en neige, les incorporer.

\item Beurrer le moule.

\item Mettre au four 30 min th.6-7.
\end{enumerate}
\subsection*{\textsc{Conseil~:}}

								% <-- x1
\section[\normalsize{Millas Bordelais}]{\LARGE{\textsc{Millas Bordelais}}}		% <-- x2


\begin{itemize}
\item Pour 6 personnes*
\item Préparation : 15 min*
\item Cuisson : 30 min
\end{itemize}

\subsection*{\textsc{Ingr\'edients~:}}

\begin{itemize}
\item 50 cl de lait vanill\'e
\item 3 oeufs
\item 6 cs de sucre
\item 6 cs de farine
\item 50 g de beurre
\end{itemize}


\subsection*{\textsc{Marche \`a suivre~:}}

\begin{enumerate}
\item Mettre les jaunes et le sucre.

\item Battre en cr\`eme.

\item Ajouter le beurre fondu, la farine.

\item D\'elayer avec le lait chaud.

\item Battre les blancs en neige, les incorporer.

\item Beurrer le moule.

\item Mettre au four 30 min th.6-7.
\end{enumerate}
\subsection*{\textsc{Conseil~:}}

								% <-- x1
\section[\normalsize{Millas Bordelais}]{\LARGE{\textsc{Millas Bordelais}}}		% <-- x2


\begin{itemize}
\item Pour 6 personnes*
\item Préparation : 15 min*
\item Cuisson : 30 min
\end{itemize}

\subsection*{\textsc{Ingr\'edients~:}}

\begin{itemize}
\item 50 cl de lait vanill\'e
\item 3 oeufs
\item 6 cs de sucre
\item 6 cs de farine
\item 50 g de beurre
\end{itemize}


\subsection*{\textsc{Marche \`a suivre~:}}

\begin{enumerate}
\item Mettre les jaunes et le sucre.

\item Battre en cr\`eme.

\item Ajouter le beurre fondu, la farine.

\item D\'elayer avec le lait chaud.

\item Battre les blancs en neige, les incorporer.

\item Beurrer le moule.

\item Mettre au four 30 min th.6-7.
\end{enumerate}
\subsection*{\textsc{Conseil~:}}


% Generated file 2018-12-02 20:50:19.351307809 +01:00
\begin{recette}{Flan aux pruneaux}{Flan aux pruneaux}

\begin{ingredients}
1 litre de lait\par
10 cuillère à soupe de maïzena\par
150 g de sucre\par
vanille\par
pruneaux (facultatif)\par
\end{ingredients}

\begin{infos}
Pour 6 personnes\\
Préparation : 15 min	\\
\end{infos}

\begin{etapes}
\item Délayer la maïzena dans un peu de lait.
\item Faire chauffer le reste du lait sucré.
\item À ébullition verser la maïzena, et laisser cuire quelques secondes.
\item Ajouter la vanille.
\item Beurrer un plat, disposer les pruneaux dénoyautés et verser la crème.
\item Mettre au four 30 min (thermostat 5)
\end{etapes}

\end{recette}
	\chapitre{Brioches et viennoiseries}
% Kugelhof											% <-- x1
% % Kugelhof											% <-- x1
% % Kugelhof											% <-- x1
% \include{./recettes/kugelhof}								% <-- x1
\section[\normalsize{Kugelhof}]{\LARGE{\textsc{Kugelhof}}}		% <-- x2


\begin{itemize}
\item Pour 6 personnes
\item Préparation : 20 min*
\item Cuisson : 45 min
\end{itemize}

\subsection*{\textsc{Ingr\'edients~:}}

\begin{itemize}
\item 333 g de farine
\item 50 g de sucre
\item 100 g de beurre
\item 6 g de sel
\item 2 oeufs
\item 13 cl de lait
\item 1 paquet de levure de boulanger
\item 50 g  de raisins secs
\item 20 g d’amandes effil\'ees 
\end{itemize}


\subsection*{\textsc{Marche \`a suivre~:}}

\begin{enumerate}
\item Dans une terrine, verser la farine, le sucre, le sel , les oeufs, la levure et le lait ti\`ede. 

\item M\'elanger et p\'etrir pendant 10 minutes . La p\^ate doit devenir \'elastique et se d\'etacher des bords de la terrine. 

\item Ajouter le beurre mou.

\item Laisser reposer dans un endroit ti\`ede quelques heures. La p\^ate doit doubler de volume. 

\item Casser la p\^ate et ajouter les raisins tremp\'es dans l ’eau.

\item Mettre dans un moule bien beurr\'e et garni d’amandes.

\item Laisser la p\^ate monter. Elle doit doubler de volume.

\item Mettre dans un four pr\'echauff\'e : 200° C pendant 45 minutes environ.
\end{enumerate}
\subsection*{\textsc{Conseil~:}}

								% <-- x1
\section[\normalsize{Kugelhof}]{\LARGE{\textsc{Kugelhof}}}		% <-- x2


\begin{itemize}
\item Pour 6 personnes
\item Préparation : 20 min*
\item Cuisson : 45 min
\end{itemize}

\subsection*{\textsc{Ingr\'edients~:}}

\begin{itemize}
\item 333 g de farine
\item 50 g de sucre
\item 100 g de beurre
\item 6 g de sel
\item 2 oeufs
\item 13 cl de lait
\item 1 paquet de levure de boulanger
\item 50 g  de raisins secs
\item 20 g d’amandes effil\'ees 
\end{itemize}


\subsection*{\textsc{Marche \`a suivre~:}}

\begin{enumerate}
\item Dans une terrine, verser la farine, le sucre, le sel , les oeufs, la levure et le lait ti\`ede. 

\item M\'elanger et p\'etrir pendant 10 minutes . La p\^ate doit devenir \'elastique et se d\'etacher des bords de la terrine. 

\item Ajouter le beurre mou.

\item Laisser reposer dans un endroit ti\`ede quelques heures. La p\^ate doit doubler de volume. 

\item Casser la p\^ate et ajouter les raisins tremp\'es dans l ’eau.

\item Mettre dans un moule bien beurr\'e et garni d’amandes.

\item Laisser la p\^ate monter. Elle doit doubler de volume.

\item Mettre dans un four pr\'echauff\'e : 200° C pendant 45 minutes environ.
\end{enumerate}
\subsection*{\textsc{Conseil~:}}

								% <-- x1
\section[\normalsize{Kugelhof}]{\LARGE{\textsc{Kugelhof}}}		% <-- x2


\begin{itemize}
\item Pour 6 personnes
\item Préparation : 20 min*
\item Cuisson : 45 min
\end{itemize}

\subsection*{\textsc{Ingr\'edients~:}}

\begin{itemize}
\item 333 g de farine
\item 50 g de sucre
\item 100 g de beurre
\item 6 g de sel
\item 2 oeufs
\item 13 cl de lait
\item 1 paquet de levure de boulanger
\item 50 g  de raisins secs
\item 20 g d’amandes effil\'ees 
\end{itemize}


\subsection*{\textsc{Marche \`a suivre~:}}

\begin{enumerate}
\item Dans une terrine, verser la farine, le sucre, le sel , les oeufs, la levure et le lait ti\`ede. 

\item M\'elanger et p\'etrir pendant 10 minutes . La p\^ate doit devenir \'elastique et se d\'etacher des bords de la terrine. 

\item Ajouter le beurre mou.

\item Laisser reposer dans un endroit ti\`ede quelques heures. La p\^ate doit doubler de volume. 

\item Casser la p\^ate et ajouter les raisins tremp\'es dans l ’eau.

\item Mettre dans un moule bien beurr\'e et garni d’amandes.

\item Laisser la p\^ate monter. Elle doit doubler de volume.

\item Mettre dans un four pr\'echauff\'e : 200° C pendant 45 minutes environ.
\end{enumerate}
\subsection*{\textsc{Conseil~:}}


% Generated file 2018-12-02 20:50:19.124032396 +01:00
\begin{recette}{Galette au sucre}{Galette au sucre}

\begin{ingredients}
250 g de farine\par
1 cuillère à soupe de sucre (~10 g)\par
2 oeufs\par
1/2 verre de lait tiède\par
75 g de beurre\par
20 g de levure de boulanger  (ou 1 sachet)\par
\end{ingredients}

\begin{infos}
Pour 6 personnes\\
Préparation : 30 min\\
Cuisson : 20 min\\
\end{infos}

\begin{etapes}
\item Préchauffez votre four à 180 °C (thermostat 6).
\item Mettre la farine, le sucre, les oeufs, la levure, le lait et le sel. Pétrir jusqu'à ce que la pâte se détache des parois.
\item Ajouter le beurre mou. Mélanger et laisser lever (combien de temps ?) dans un endroit tiède.
\item Mettre la pâte dans un moule à tarte beurré. Saupoudrer de sucre et parsemer de noisettes de beurre.
\item Faire cuire au four 20 min à 180°C
\end{etapes}

\begin{conseils}
Le batteur K fonctionne bien
\end{conseils}

\end{recette}
% Generated file 2019-02-17 16:33:30.797046664 +01:00
\begin{recette}{Pain d'épices (façon Claire T.)}{Pain d'épices (façon Claire T.)}

\begin{ingredients}
300 g de farine\par
1/3 l de lait\par
100 g de cassonnade\par
100 g de miel\par
1 yaourt\par
2 c. à café rases de bicarbonate de soude\par
1 cuillères à café de cannelle\par
1 cuillères à café de 4 épices\par
1 cuillère à café de gingembre\par
\end{ingredients}

\begin{infos}
Pour 8 personnes\\
Préparation : 20 min\\
Cuisson : 50 min\\
\end{infos}

\begin{etapes}
\item Faire préchauffer le four à 160° C.
\item Faire fondre à feu très doux le sucre, le lait et le
\item miel.
\item Dans une terrine, mélanger la farine, les épices et le
\item yaourt. Verser le lait.
\item Ajouter le bicarbonate de soude délayé dans 2 cuill
\item ères à soupe d'eau chaude.
\item Mettre au four 25 min, puis 25 min à 200° C.
\end{etapes}

\begin{conseils}
Si pas de bicarbonate de soude, remplacer par 1/2 sachet de levure.
Ne pas hésiter à rajouter des épices si besoin.
\end{conseils}

\end{recette}
% Generated file 2019-02-17 16:33:30.885859396 +01:00
\begin{recette}{Briochettes à la purée d'amandes et au citron}{Briochettes à la purée d'amandes et au citron}

\begin{ingredients}
2 oeufs (1)\par
1 grosse c à s de purée d'amandes (2)\par
1 yaourt nature (3)\par
lait (1 + 2 + 3 + lait = 425 ml)\par
le zeste d'un citron\par
80 g de sucre (moitié blanc, moitié roux)\par
1 cuillère à café de sel\par
500 g de farine\par
1 sachet de levure Briochin\par
\end{ingredients}

\begin{infos}
Pour 6 personnes\\
Préparation : 3h + 1 nuit\\
Cuisson : 15 min\\
\end{infos}

\begin{etapes}
\item Mélanger tous les ingrédient soit à la main soit au robot ou à la MAP
\item Pétrir une bonne dizaine de mines (à la main)
\item Quand le pétrissage est fini, laisser lever 30 mines puis filmer et mettre au réfrigérateur pour la nuit.
\item Le lendemain, verser la pâte sur le plan de travail fariné et la découper au couteau en 12 portions.
\item Prélever dans chaque portion un morceau de la taille d'une grosse bille pour la tête. Façonner les parts en boules, les répartir dans les empreintes (à briochettes ou à muffins) légèrement beurrées et les inciser en croix sur le dessus à l'aide de ciseaux. * Déposer la petite boule dans le creux formé en appuyant légèrement (durée totale : environ 30 min
\item Laisser lever 1h à 1h30.
\item Préchauffer le four à 180\ C.
\item Dorer les briochettes à l'oeuf ou au lait, et enfourner pour 12 à 15 min.
\end{etapes}

\begin{conseils}
Pâte trop humide au départ, il a fallu rajouter plusieurs c à s de farine pendant le pétrissage. Diminr la quantité de liquide ?
Si utilisation de la MAP : Mettre les ingrédients dans l'ordre dans la cuve de la MAP, lancer le programme pâte levée. Vérifier l'aspect de la pâte, rajouter de la farine si besoin.
\end{conseils}

\end{recette}
	\chapitre{Mousses}
	\chapitre{Cr\`emes}
% Generated file 2019-02-17 16:33:30.822080050 +01:00
\begin{recette}{Crème mic-mac}{Crème mic-mac}

\begin{ingredients}
5 œufs\par
125 g de chocolat dessert\par
50 g de farine\par
50 g de beurre\par
1 paquet de sucre vanillé\par
150 g de sucre\par
1 l de lait\par
\end{ingredients}

\begin{infos}
Pour 6 personnes\\
Préparation : 15 min\\
\end{infos}

\begin{etapes}
\item Faire bouillir le lait.
\item Faire fondre le chocolat brisé dans 2 cs d’eau.
\item Mélanger le sucre, la farine, 3 jaunes et 2 œufs entiers.
\item Délayer avec le lait chaud.
\item Faire chauffer jusqu’aux 1ers bouillons.
\item Ajouter le beurre hors du feu.
\item 1ère moitié ajouter le sucre vanillé
\item 2ème moitié le chocolat fondu.
\item Verser en même temps les 2 crèmes.
\item Servir froid.
\end{etapes}

\begin{conseils}
Remplacer le lait demi-écrémé par du lait entier pour obtenir encore plus d'onctuosité.
\end{conseils}

\end{recette}
% Generated file 2018-12-02 20:50:19.018375139 +01:00
\begin{recette}{Crème renversée}{Crème renversée}

\begin{ingredients}
1/2 litre de lait\par
3 oeufs\par
100 g de sucre\par
une gousse de vanille\par
1 pincée de sel\par
\end{ingredients}

\begin{infos}
Pour 6 personnes\\
Préparation : 15 min	\\
\end{infos}

\begin{etapes}
\item % Ici les étapes à réaliser
\item % Une étape par ligne, chaque ligne commence par un
\item % Pour exemple les étapes pour faire un millas ;)
\item Faire bouillir le lait avec le sucre, le sel et la gousse de vanille.
\item Hors du feu ajouter le lait peu à peu aux oeufs battus en fouettant sans arrêt.
\item Verser dans un moule à soufflé caramélisé.
\item Verser 2 cm d'eau dans la cocotte mine. Y déposer le moule à soufflé et le couvrir d'une assiette.
\item Fermer la cocotte et placer là sur un feu vif. Dès le chuchotement, réduire le feu et laisser cuire 7 min. Aussitôt la cuisson termin, ôter la soupape. Quand la vapeur s'est échappée, ouvrir la cocotte mine.
\end{etapes}

\begin{conseils}
Servir très froid.
\end{conseils}

\end{recette}
	\chapitre{Verrines}
% Generated file 2018-12-02 20:50:19.254403738 +01:00
\begin{recette}{Tiramisu aux framboises en verrines}{Tiramisu aux framboises en verrines}

\begin{ingredients}
2 oeufs\par
250 g de mascarpone\par
50 g de sucre roux\par
1 sachet de sucre vanillé\par
100 g de framboises\par
1/4 de zeste de citron vert haché\par
environ 10 biscuits rose de Reims\par
amandes effilées\par
Pour le coulis :\par
150 g de framboises\par
50 g de sucre\par
\end{ingredients}

\begin{infos}
Pour 4 personnes\\
Préparation : 30 + 180 min\\
\end{infos}

\begin{etapes}
\item Séparer le blanc des jaunes d’oeufs.
\item Mélanger les jaunes avec le sucre et le sucre vanillé.
\item Ajouter le mascarpone au fouet, puis le zeste de citron vert.
\item Monter les blancs en neige et les incorporer délicatement à la spatule au mélange précédent.
\item Préparer le coulis en mixant 150 g de framboises avec 50 g de sucre.
\item Tapisser les verrines de biscuits. Recouvrir de coulis et de quelques framboises et étaler une couche de crème.
\item Alterner biscuits, coulis, framboises et crème. Termin par une couche de crème.
\item Garder 4 framboises pour la déco.
\item Filmer les verrines et réfrigérer au moins 3 heures.
\item Au moment de servir, saupoudrer d’amandes effilées et déposer une framboise au centre.
\end{etapes}

\end{recette}	
	\chapitre{Biscuits}
% Generated file 2018-11-25 21:43:22.366614193 +01:00
\begin{recette}{Biscuits de noël à la cannelle}{Biscuits de noël à la cannelle}

\begin{ingredients}
270 g de beurre\par
500 g de farine\par
2 + 1 oeufs\par
150 g de poudre d'amandes\par
15 g de cannelle\par
1 zeste de citron rapé\par
250 g de sucre\par
\end{ingredients}

\begin{infos}
Pour 1 grande boite ($\approx$ 120 pièces)\\
Préparation : 30 min + 12~h de repos\\
Cuisson : 10 min / fournée  ($\approx$ 40 min)\\
\end{infos}

\begin{etapes}
\item Dans une terrine, pétrir le beurre ramolli avec la farine du bout des doigts.
\item Ajouter 2 oeufs, la poudre d'amandes, la cannelle, le zeste de citron et le sucre.
\item Travailler jusqu'à obtenir une pâte homogène.
\item Couvrir d'un torchon et laisser reposer une nuit dans un endroit frais.
\item Etaler la pâte (pour une épaisseur comprise entre 3 et 10~mm selon les goûts), découper à l'emporte pièces les biscuits et les déposer sur une plaque de cuisson.
\item Dorer à l'oeuf battu et faire cuire au four à 200° C
\item Cuisson 10 min environ selon le four
\end{etapes}

\begin{conseils}
Il est possible d'en faire une version sans cannelle.
\end{conseils}

\end{recette}
% Generated file 2018-11-25 21:43:22.371560450 +01:00
\begin{recette}{Cookies (façon Claire T.)}{Cookies (façon Claire T.)}

\begin{ingredients}
510 g de farine\par
200 g de beurre\par
170 g de sucre roux\par
170 g de sucre semoule\par
340 g de chocolat\par
2 oeufs\par
1 c. à café de bicarbonate de soude\par
1 c. à café d'extrait de vanille\par
\end{ingredients}

\begin{infos}
Pour 12 personnes\\
Préparation : 30 min\\
Cuisson : 10 min\\
\end{infos}

\begin{etapes}
\item Préchauffer le four à 180° C.
\item Mélanger les 2 sucres et le beurre ramolli jusqu'à
\item obtention d'un aspect crémeux.
\item Ajouter les oeufs, la vanille, 1 pincée de sel et mélanger
\item le tout. Ajouter le bicarbonate de soude et la farine au fur et à mesure, puis le chocolat
\item brisé en morceaux.
\item Sur une plaque de cuisson disposer du papier sulfurisé et y déposer des tas de pâte
\item d'environ une cuillère à soupe (bien espacer).
\item Faire cuire 8 à 12 minutes environ.
\end{etapes}

\end{recette}
% Generated file 2018-11-25 21:43:22.351830918 +01:00
\begin{recette}{Damiers vanille et chocolat (Wiss un schwarzi butter bredele)}{Damiers vanille et chocolat (Wiss un schwarzi butter bredele)}

\begin{ingredients}
250 g de farine\par
150 g de beurre pommade\par
100 g de sucre\par
1 sachet de sucre vanillé\par
1 cuillère à café de levure chimique\par
1 cuillère à soupe d'eau de vie (ou d'eau)\par
30 g de cacao non sucré en poudre\par
\end{ingredients}

\begin{infos}
Pour environ 75 pièces		\\
Préparation : 1h + 1h de repos	\\
Cuisson : 20 min		\\
\end{infos}

\begin{etapes}
\item Dans un bol mélanger la farine et la levure chimique, rajouter le sucre et le sucre vanillé. Incorporer l’eau de vie et le beurre pommade et pétrir le tout pour former une boule de pâte. Ajouter un peu d'eau si nécessaire.
\item Partager la pâte en deux, et incorporer le cacao dans une moitié.
\item Rouler chaque partie en forme de petits boudins de taille égale, d'environ 1 cm d'épaisseur. Disposer un boudin vanille et un boudin chocolat côte à côte puis déposer par dessus deux autres boudins en inversant les couleurs de façon à faire un damier carré. Appuyer légèrement de tous les côtés pour les souder.
\item Laisser durcir au frais pendant 1h.
\item Préchauffer le four à 160°C.
\item Découper la pâte en tranches d'un bon centimètre avec un couteau. Déposer les damiers sur une plaque en les espaçant bien et faire cuire 8 à 10 min.
\item Laisser refroidir sur une grille et ranger dans une boîte en métal.
\end{etapes}

\begin{conseils}
On peut aussi façonner les biscuits en spirales : étaler les pâtes sur 2 mm d'épaisseur, les découper à la même taille, les superposer en appuyant un peu et rouler en cylindre dans le sens de la longueur. Après repos au frais, couper des tranches de la même façon que pour les damiers.
\end{conseils}

\end{recette}
% Generated file 2018-11-25 21:43:22.329324781 +01:00
\begin{recette}{Croissants à la vanille (Vanillekipferl)}{Croissants à la vanille (Vanillekipferl)}

\begin{ingredients}
250 g de farine\par
180 g de beurre\par
100 g de poudre d'amandes\par
65 g de sucre\par
1 cuillère à café d'extrait de vanille\par
1 pincée de sel\par
sucre vanillé (2 à 3 sachets) et sucre glace pour la finition\par
\end{ingredients}

\begin{infos}
Pour 40 à 50 pièces\\
Préparation : 30 min + 1h de repos\\
Cuisson : 20 à 30 min	\\
\end{infos}

\begin{etapes}
\item Pétrir à la main la farine, le beurre coupé en dés, le sucre, les amandes, la vanille et le sel de manière à obtenir une pâte homogène. Ajouter un petit peu d'eau si la pâte est trop friable.
\item Rouler en boule et mettre au réfrigérateur pendant 1 heure au moins.
\item Préchauffer le four à 170°C.
\item Prélever de petits morceaux de pâte, les rouler en boudin en faisant les extrémités plus fines et les façonner en forme de petits croissants. Les déposer sur une plaque en les espaçant un peu.
\item Faire cuire 10 à 12 min jusqu'à ce qu'ils soient très légèrement dorés. Les déposer sur une grille à la sortie du four. Attention ils cassent facilement !
\item Mélanger dans une assiette creuse du sucre glace et du sucre vanillé (environ 2-3 cuillères à soupe pour un sachet). Passer les croissants encore tièdes dans ce mélange, seulement sur le dessus ou entièrement selon qu'on les aime plus ou moins sucrés.
\item Finir de laisser refroidir et ranger dans une boîte en métal.
\end{etapes}

\end{recette}
	\chapitre{Fruits}
% Generated file 2018-12-02 20:50:19.412057143 +01:00
\begin{recette}{Pêches Ascona}{Pêches Ascona}

\begin{ingredients}
4 jaunes d’oeufs\par
2 grosses cuillerées à soupe de farine\par
50g de sucre semoule\par
1 sachet de sucre vanillé\par
50 cl de lait\par
4 blancs d’œufs\par
1 boîte de pêches au sirop\par
amandes effilées\par
\end{ingredients}

\begin{infos}
Pour 6 personnes\\
Préparation : 15 min\\
Cuisson : 5 min\\
\end{infos}

\begin{etapes}
\item Travailler les jaunes d’oeufs dans une casserole avec la farine, le sucre et le sucre vanillé.
\item Mouiller peu à peu de lait bouillant et faites épaissir sur feu doux.
\item Laisser tiédir, puis incorporez les blancs battus en neige très ferme.
\item Verser dans un compotier et laisser refroidir complètement.
\item Garnissez le dessus de demi pêches.
\item Faites fondre la gelée de framboise sur le feu très doux avec le sirop de la boîte de pêches.
\item Laissez tiédir avant d’en napper les pêches.
\item Mettez au frais.
\item Faites griller une poignée d’amandes effilées.
\item Parsemez-en le dessert au moment de servir.
\end{etapes}

\end{recette}
	\chapitre{Glaces}
	\chapitre{Non classés}
% Generated file 2019-02-17 16:33:30.825398171 +01:00
\begin{recette}{Teurgoule}{Teurgoule}

\begin{ingredients}
2 litres de lait\par
130 g de riz\index{riz} rond\par
150 g de sucre + vanille + cannelle\par
\end{ingredients}

\begin{infos}
Pour XX personnes*	\\
Préparation : 15 min\\
Cuisson : 4 min\\
\end{infos}

\begin{etapes}
\item Beurrer le plat.
\item Saupoudrer de cannelle.
\item Mettre tous les ingrédients dans le plat.
\item Mettre 4 heures au four th.3-4 = 200°s C.
\end{etapes}

\end{recette}
% Nom de la recette à entrer entre les accolades {}
\section{Cigares banane-choco}

% Informations génériques
% Changer de ligne pour chaque et commencer par : \item
% Mettre une * si l'information n'est pas certaine 
\begin{itemize}
\item Pour 4 personnes*			% Nombre de personnes qu'on pourra nourrir ! :)
\item Préparation : 15 min*		% Temps de préparation (sans la cuisson)
\item Cuisson : 10 min			% Temps de cuisson
\end{itemize}

\subsection*{\textsc{Ingrédients~:}}

% Ici lister les ingrédients 
% Changer de ligne pour chaque ingrédient et commencer la ligne par : \item
% rajouter autant de ligne que d'ingrédient
\begin{itemize}
\item 3 feuilles de brick
\item 2 bananes
\item 12 carrés de chocolat noir
\item 25 g de beurre
\end{itemize}


\subsection*{\textsc{Marche à suivre~:}}

% Ici les étapes à réaliser
% Une étape par ligne, chaque ligne commence par un \item
% Pour exemple les étapes pour faire un millas ;)
\begin{enumerate}
\item Faites fondre le beurre au micro-onde à puissance moyenne. 

\item Allumez le four à 210° C, th 7.

\item Épluchez les bananes et coupez-les en trois tronçons, puis chacun en deux dans la longueur.

\item Coupez les feuilles de brick en quatre triangles. 

\item Beurrez-les au pinceau.

\item Déposez la partie la plus large vers vous. Posez un tronçon de banane et un carré de chocolat dessus. Rabattez les côtés, enroulez. Formez ainsi 12 petits cigares.
Déposez-les sur une plaque recouverte de papier cuisson. 

\item Cuisez 10 min au four.
\end{enumerate}

\subsection*{\textsc{Conseil~:}}
% Ici écrire les conseils concernant la recette 
Servez les cigares tout chauds avec une boule de glace à la vanille.

Vin : banyuls ou maury

% Generated file 2018-12-02 20:50:19.462947258 +01:00
\begin{recette}{Oeufs à la neige}{Oeufs à la neige}

\begin{ingredients}
un demi litre de lait\par
75 g de sucre\par
4 oeufs\par
\end{ingredients}

\begin{infos}
Pour XX personnes\\
Préparation : XX min\\
Cuisson : XX min\\
\end{infos}

\begin{etapes}
\item Faire chauffer le lait.
\item Pendant ce temps battre les blancs en meringue (25 g de sucre).
\item Faire pocher 1 à 2 min de chaque côté. On peut tremper une cuillère dans de l'eau bouillante pour prélever des blancs et lisser avec une autre cuillère ou spatule mouillée.
\item Égoutter.
\item Disposer sur un plat creux.
\item Faire la crème anglaise.
\item Verser cette crème dans le plat ou sont disposés les blancs. Ceux-ci surnagent.
\item On peut verser sur les blancs des amandes grillées ou du caramel brun roux.
\end{etapes}

\begin{conseils}
Variante : ÎLE FLOTTANTE
Faire un sirop de caramel dans un moule à charlotte (50g de sucre).
Faire cuire la meringue dans le moule enduit du sirop.
Cuire à four modéré au bain marie (20 à 30 min
Démouler l’île sur un plat à crème.
Verser s’y la crème anglaise.
\end{conseils}

\end{recette}
% Generated file 2019-02-17 16:33:31.024261347 +01:00
\begin{recette}{Bugnes}{Bugnes}

\begin{ingredients}
1 kg de farine\par
1 sachet de levure chimique\par
10 g de sel\par
200 g de sucre\par
1 sachet de sucre vanillé ou 2 c. à soupe d'eau de fleur d'oranger\par
200 g de beurre\par
8 œufs\par
sucre glace\par
grande friture\par
\end{ingredients}

\begin{infos}
Pour une grande corbeille\\
Préparation : 20 min + 30 min de repos\\
Cuisson : 3 à 4 min par fournée\\
\end{infos}

\begin{etapes}
\item Disposer la farine en fontaine ; au centre mettre le sel, le sucre, le sucre vanillé ou la fleur d'oranger, la levure, le beurre juste ramolli et les œufs entiers. Mélanger du bout des doigts, puis pétrir avec les mains jusqu'à ce que le mélange soit parfait. Mettre en boule, couvrir d'un torchon, laisser reposer 30 min.
\item Détacher de la masse des morceaux gros comme une orange. Les étendre au rouleau sur quelque millimètres d'épaisseur.Tailler des bandes de 4 cm de large, les couper en longs losanges puis fendre le milieu pour y passer une des pointes (tirer jusqu'au bout pour la retourner complètement). Préparer toutes les bugnes avant d'entreprendre la cuisson, car elles cuisent très vite et il faut les surveiller pour qu'elles ne brunissent pas.
\item Faire cuire dans la grande friture chaude non fumante, égoutter sur du papier absorbant et servir saupoudré de sucre glace.
\end{etapes}

\begin{conseils}
Les bugnes se conservent bien dans une boîte en métal.
\end{conseils}

\end{recette}
% Generated file 2018-11-25 21:43:22.311592906 +01:00
\begin{recette}{Falafels}{Falafels}

\begin{ingredients}
500g de pois chiches secs ou fèves sèches\par
6 gousses d'ail\par
1/2 bouquet de persil plat\par
1/2 bouquet de coriandre (en tout pour les 2: 60g)\par
1/2 oignon\par
1 cuillère à café de bicarbonate de soude\par
1 cuillère à soupe de sésame doré\par
2 cuillères à café de coriandre en poudre\par
2 cuillères à café de cumin en poudre\par
1/2 cuillère à café de piment en poudre\par
sel, poivre\par
huile pour friture\par
\end{ingredients}

\begin{infos}
Pour 50 falafels\\
Préparation : 45 min + 24h\\
Cuisson : 20 min\\
\end{infos}

\begin{etapes}
\item La veille, mettre les pois chiches à tremper pendant 12 à 24h.
\item Égoutter et sécher soigneusement les pois chiches.
\item Les verser dans le bol d'un mixeur. Ajouter les herbes lavées et bien séchées et l'oignon coupé en morceaux.
\item Mixer par à coups pour obtenir une pâte qui colle un peu, mais pas une purée. Elle doit s'amalgamer si on la tasse.
\item Mettre la pâte dans un bol puis ajouter les épices : coriandre, cumin, piment, sel, bicarbonate de soude, et le sésame. Bien mélanger.
\item Former des boulettes applaties d'environ 5 cm de diamètre.
\item Faire cuire dans l'huile bien chaude jusqu'à obtenir une couleur bien dorée à brune.
\end{etapes}

\begin{conseils}
Attention à bien sécher tous les ingrédients et ne surtout pas rajouter d'eau, sinon les falafels risquent de se déliter à la cuisson.
On peut les congeler une fois cuits et les faire réchauffer au four ensuite.
\end{conseils}

\end{recette}


\part{Les p\^ates et "Morceaux de Recettes"}
	\chapitre{P\^ate à ...}
% Generated file 2019-02-17 16:33:30.842486227 +01:00
\begin{recette}{Pâte à quiche}{Pâte à quiche}

\begin{ingredients}
220 g de farine\par
100 g de beurre\par
1 oeuf\par
Eau\par
Sel\par
\end{ingredients}

\begin{infos}
Pour 6 personnes\\
Préparation : 10 min\\
\end{infos}

\begin{etapes}
\item Verser la farine dans une terrine.
\item Ajouter le beurre mou coupé en morceaux et mélanger du bout des doigts jusqu'à ce que la farine ait absorbé tout le beurre (on obtient un sable grossier).
\item Ajouter l'oeuf et un peu d'eau (1 ou 2 cuillères à soupe) et travailler rapidement jusqu'à former une boule de pâte souple et non collante.
\item Filmer et laisser reposer au réfrigérateur une demie heure au moins.
\end{etapes}

\end{recette}
% Generated file 2019-02-17 16:33:30.902590606 +01:00
\begin{recette}{Pâte brisée (façon Claire T.)}{Pâte brisée (façon Claire T.)}

\begin{ingredients}
250 g de farine\par
1 oeuf\par
100 à 125 g de beurre\par
Eau\par
Sel\par
\end{ingredients}

\begin{infos}
Pour 6 personnes\\
Préparation : 10 min\\
\end{infos}

\begin{etapes}
\item Versez la farine dans une terrine et faites un puit.
\item Ajoutez l'oeuf, une pincée de sel et le beurre coupé en morceaux.
\item Mélangez en ajoutant de l'eau peu à peu jusqu'à obtention d'une boule de pâte souple et pas trop collante.
\item Laissez reposer avant utilisation.
\end{etapes}

\end{recette}
% Generated file 2018-11-25 21:43:22.321971355 +01:00
\begin{recette}{Pancakes}{Pancakes}

\begin{ingredients}
250 g de farine\par
30 g de sucre\par
2 oeufs\par
1 sachet de levure traditionnelle\par
65 g de beurre\par
1 pincée de sel\par
30 cl de lait\par
\end{ingredients}

\begin{infos}
Pour 4 personnes\\
Préparation : 10 min + 1h\\
\end{infos}

\begin{etapes}
\item Mettre la farine, la levure, le sel et le sucre dans un saladier. Rajouter ensuite les oeufs entiers et mélanger.
\item Ajouter ensuite le beurre fondu, puis délayer progressivement le mélange avec le lait afin d'éviter les
\item grumeaux.
\item Laisser reposer au minimum 1h au réfrigérateur avant de faire cuire.
\end{etapes}

\end{recette}
% Generated file 2018-11-25 21:43:22.331850612 +01:00
\begin{recette}{Pâte à crêpes (façon Claire T.)}{Pâte à crêpes (façon Claire T.)}

\begin{ingredients}
250 g de farine\par
1/2 l de lait\par
3 oeufs\par
1 cuillère d'huile\par
1 pincée de sel\par
1 ou 2 cuillères d'eau\par
\end{ingredients}

\begin{infos}
Pour 2 --- 4 personnes\\
Préparation : 15 min\\
\end{infos}

\begin{etapes}
\item Mettre la farine dans une terrine. Faire un puits, et y casser les oeufs entiers.
\item Ajouter l'huile, le sel et un peu de lait ; travailler énergiquement la pâte avec une cuillère (en bois, de préférence), pour la rendre légère.
\item Mouiller progressivement avec le lait, jusqu'à ce que la pâte devienne homogène. On peut, à ce moment là, ajouter de l'extrait de fleur d'oranger, un peu de jus de citron, de la vanille, etc...
\item Ajouter ensuite 1 à 2 cuillères d'eau. Puis, passer dans une passoire 1 à 2 fois la pâte pour enlever les grumeaux.
\item Laisser reposer la pâte pendant 1 h, recouverte d'une serviette ou d'un chiffon propre.
\end{etapes}

\end{recette}
% Generated file 2019-02-17 16:33:30.883188329 +01:00
\begin{recette}{Pâte à  gaufres}{Pâte à  gaufres}

\begin{ingredients}
450 g de farine\par
120 g beurre\par
75 g de sucre\par
3 oeufs\par
3/4 de litre de lait tiède\par
1 sachet de levure\par
\end{ingredients}

\begin{infos}
Pour X personnes\\
Préparation : 15 min\\
\end{infos}

\begin{etapes}
\item % Ici les étapes à réaliser
\item % Une étape par ligne, chaque ligne commence par un
\item % Pour exemple les étapes pour faire un millas ;)
\item Mélanger la farine, le sucre, la levure, les oeufs et le lait tiède.
\item Ajouter le beurre fondu.
\item Laisser lever dans un endroit tiède.
\end{etapes}

\end{recette}
	\chapitre{P\^at\'e}
	\chapitre{Sauces}
% Generated file 2019-02-17 16:33:30.790510800 +01:00
\begin{recette}{Pesto rouge}{Pesto rouge}

\begin{ingredients}
25 g de pignons\par
50 g de tomates séchées\par
25 g de parmesan rapé\par
1 gousse d’ail\par
7,5 cl d’huile d’olive\par
\end{ingredients}

\begin{infos}
Pour 6 personnes\\
Préparation : 15 min\\
\end{infos}

\begin{etapes}
\item Faire griller les pignons dans une poêle à sec, sans
\item matière grasse.
\item Couper les tomates et la gousse d’ail en morceaux.
\item Passer tous les ingrédients au mixer.
\end{etapes}

\end{recette}

\part{Quelle recette aujourd'hui ?}
%	\chapitre{Rapide et simple}
%	\chapitre{Aujourd'hui on reçoit}
%	\chapitre{Recette d'hiver}
%	\chapitre{Recette d'\'et\'e}

\addcontentsline{toc}{chapitre}{\listchoconame}
\listofchoco

\printindex

\backmatter

\end{document}