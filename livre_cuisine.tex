%     Livre de Recette des D'ANGELO     %
%            - version 0.1 -            %
%%%%%%%%%%%%%%%%%%%%%%%%%%%%%%%%%%%%%%%%%
% Compilation : pdflatex

\documentclass[A4paper,twoside, 12pt]{book}
%--------------------------------------------------------%
%-------------------LES PACKAGES-------------------------%
%--------------------------------------------------------%
%pdflatex
\usepackage[utf8]{inputenc}
\usepackage[T1]{fontenc}
\usepackage{lmodern}% polices 
%\usepackage{newcent}% polices 

\usepackage{geometry}%Gestion des marges
\geometry{
a4paper,
left=25mm, top=25mm}

%\usepackage{minitoc}
% Permet d'avoir un sommaire ET un table des matières
%\usepackage[tight]{shorttoc} 
\usepackage{tocloft} % gestion de table de matières personnalisées (recette au chocolat)

\usepackage{makeidx}

% pour les liens cliquables de la table des matières
\usepackage[colorlinks=true,linkcolor=blue]{hyperref}

\usepackage{graphicx} %insertion d'image

\usepackage{fancyhdr} %En-tête et pied de pages
\usepackage{fancybox} % Des boites sympa

\usepackage{textcomp} % pour  \degree
\usepackage{enumitem} % Gestion fine des listes
%--------------------------------------------------------%
%-----------------------LES COMMANDES--------------------%
%--------------------------------------------------------%
%\newcommand{\sommaire}{\shorttoc{Sommaire}{1}} %change la commande pour inserer une short table of content
\newcommand{\HRule}{\rule{\linewidth}{0.5mm}} % utilisation page de titre

%%------indentation des titres de section
\newlength{\sectiontitleindent}
\setlength{\sectiontitleindent}{-1cm}	% attribution valeur
%%------indentation des titres de subsection
\newlength{\subsectiontitleindent}
\setlength{\subsectiontitleindent}{-.5cm}	% attribution valeur

%%------définition police de section
\newcommand{\sectionfont}{%
\fontencoding{\encodingdefault}%
\fontfamily{pag}%
\fontseries{m}%
\fontshape{sc}%
\selectfont}
%%------définition police de subsection
\newcommand{\subsectionfont}{%
\fontencoding{\encodingdefault}%
\fontfamily{pag}%
\fontseries{m}%
\fontshape{n}%
\selectfont}

	%%------Mise en forme des titres de sections
\makeatletter %Permet de bidouiller avec l'@
%--SECTION
\renewcommand{\section}{%
\@startsection%
{section}%
{1}%
{\sectiontitleindent}%
{-3.5ex plus -1ex minus -.2ex}%
{2.3ex plus .2ex}%
{\sectionfont\LARGE}}
%%--SUBSECTION
\renewcommand{\subsection}{%
\@startsection%
{subsection}%
{2}%
{\subsectiontitleindent}%
{-3.5ex plus -1ex minus -.2ex}%
{2.3ex plus .2ex}%
{\subsectionfont\large}}
\makeatother %fin de la permission pour l'@

%--------------------------------------------------------%
%----------EN-TÊTE ET PIED DE PAGE----------------%
%--------------------------------------------------------%
\pagestyle{fancy}
\fancyhf{}
%%---en-tête---%%
\fancyhead[LE]{\leftmark}
\fancyhead[RO]{\leftmark}
\renewcommand{\headrulewidth}{.5pt} %Utilisé pour la page de titre
%%---pied de page---%%
%\fancyfoot[LE,RO]{\thepage}			%pages sur l'extérieur
\fancyfoot[CE,CO]{---~\thepage~---}		%pages au centre

%------------Infos Titres-------------------%
%\title{\textsc{Recettes Familliales}}
%\author{Collectif}
%\date{---~Novembre 2011~---} 

%------------Table des matières-------------------%
% profondeur de la numérotation (les sections ne sont pas numérotée)(si on veut les sections avec un numéro :  \setcounter{secnumdepth}{1})
\setcounter{secnumdepth}{0}

%jusqu'à quelle profondeur sont inséré les titres dans le sommaire
\setcounter{tocdepth}{1}

%Configuration de la minitoc
%\mtcsettitle{minitoc}{}
%\mtcsetrules{minitoc}{off}
%\mtcsetfeature{minitoc}{before}{\vspace{-20pt}}% à ajuster
%\mtcsetpagenumbers{minitoc}{off}
%\setcounter{minitocdepth}{1}
%\dominitoc

%------------Config Boite-------------------%

%--------------------------------------------------------%
%---------------CONFIG ENVIRONMENTS----------------------%
%--------------------------------------------------------%
% Mise en forme partie infos
\newenvironment{infos}%
{\begin{minipage}[t][][t]{.35\textwidth}
\begin{flushright}
\small\upshape
\begin{itemize}[label=]\setlength{\itemsep}{.1cm}\setlength{\parskip}{0cm}}
{\end{itemize}
\normalsize\upshape
\end{flushright}
\end{minipage}}

% Mise en forme partie ingrédient
\newenvironment{ingredients}
{\vspace{.5cm}
\begin{minipage}[t][][t]{.65\textwidth}
\subsection{Ingrédients~:}
\small\slshape 
\begin{itemize}[label={--}]\setlength{\itemsep}{0.1cm}\setlength{\parskip}{0cm}}%
{\end{itemize}
\normalsize\upshape
\end{minipage}}

% Mise en forme partie marche à suivre
\newenvironment{etapes}%
{\begin{minipage}[t][][t]{\textwidth}
\vspace{.8cm}
\subsection{Marche à suivre~:}
\normalsize\upshape 
\begin{enumerate}}%
{\end{enumerate}
\normalsize\upshape
\end{minipage}}

% Mise en forme partie conseil
\newenvironment{conseils}%
{\begin{minipage}[t][][t]{\textwidth}
\vspace{.8cm}
\subsection{Conseils~:}
\normalsize\upshape 
\begin{flushleft}}%
{\end{flushleft}
\normalsize\upshape
\end{minipage}}

\makeindex

%------------Config List-------------------%
% Liste chocolat
\newcommand{\listchoconame}{Recettes au chocolat}
\newlistof{choco}{cho}{\listchoconame}

\newcommand{\choco}[1]{%
\refstepcounter{choco}
%\par\noindent\textbf{Answer \theanswer. #1}
\addcontentsline{cho}{choco}{\protect\numberline{\thechoco}#1}\par}



%%%%%%%%%%%%%%%%%%%%
%-----------Début du document----------%
%%%%%%%%%%%%%%%%%%%%
\begin{document}
\frontmatter
	\newgeometry{inner=2.5cm,
	outer=2.5cm,
	bottom=2.5cm, 
	marginparwidth=65.0pt,
	marginparsep=11.0pt,
	a4paper=true,
	twoside=true}

\begin{titlepage}

\begin{center}


% Upper part of the page
  

\textsc{\LARGE Qu'est ce qu'on mange ?}\\[1.0cm]

%\includegraphics[width=0.35\textwidth]{./images/cooklogo}\\[1.2cm] 

\large Pour {\small(tenter de)} répondre à cette question :\\[0.5cm]


% Title
\HRule \\[0.8cm]
{ \Huge \bfseries Recettes de famille}\\[0.4cm]

\HRule \\[1.5cm]

% Author and supervisor
\begin{minipage}{0.4\textwidth}
\begin{center}

\large
\emph{Auteurs:}\\
\textsc{Collectif}
\end{center}
\end{minipage}


\vfill

% Bottom of the page
{\large ---~\today~---}
%BY-NC-SA

\end{center}

\end{titlepage}			% Titre
\tableofcontents 		% Insertion table des matières

\mainmatter


\part{Des plats salés}
	\chapter{Salades}
%\begin{minipage}{15cm}
%%\minitoc
%\end{minipage}
%\newpage

\section{Salade de pâtes à l'italienne}		% <-- x2

\begin{ingredients}
\item 300 g de p\^ates
\item 1 melon
\item 4 belles tomates
\item 150 g de mozzarella ou de feta
\item 4 tranches de jambon de parme
\item 12 feuilles de basilic
\item 8 olives noires d\'enoyaut\'ees
\item Le jus d'1 citron
\item 6 cuill\`eres \`a soupe d'huile d'olive
\item Sel, poivre
\end{ingredients}
\begin{infos}
\item Pour 4 personnes
\item Préparation : 20 min
\end{infos}
\begin{etapes}
\item Faire cuire les p\^ates \textit{al dente}.
\item Couper le melon en deux et d\'etailler la chair en billes \`a l'aide d'une cuill\`ere parisienne.
\item Couper les tomates en morceaux, la mozzarella en d\'es et le jambon en lani\`eres. D\'etailler les olives en petits morceaux.
\item Dans un saladier, m\'elanger le jus du citron et l'huile d'olive. Saler et poivrer. Ajouter 6 feuilles de basilic cisel\'ees.
\item Egoutter les p\^ates. Les refroidir puis les verser dans le saladier. M\'elanger. Ajouter les billes de melon, les tomates, la mozzarella, le jambon et les olives.
\item Parsemer du reste de basilic.
\end{etapes}
\begin{conseils}
Servir frais, le melon n'est pas indispensable.
\end{conseils}


% Salade de p\^ates provençale											% <-- x1
% % Salade de p\^ates provençale											% <-- x1
% % Salade de p\^ates provençale											% <-- x1
% \include{./recettes/Claire/saladedepatesprovencale}								% <-- x1
\section[\normalsize{Salade de p\^ates provençale}]{\LARGE{\textsc{Salade de p\^ates provençale}}}		% <-- x2


\begin{itemize}
\item Pour 4 personnes
\item Préparation : 20 min*
\end{itemize}

\subsection*{\textsc{Ingr\'edients~:}}

\begin{itemize}
\item 1 poivron rouge;1 poivron jaune
\item 350g de p\^ates
\item 1 citron
\item 1 gousse d'ail
\item 2 boules de mozzarella
\item 6 c. \`a soupe d'huile d'olive
\item 2 brins de basilic
\end{itemize}


\subsection*{\textsc{Marche \`a suivre~:}}

\begin{enumerate}

\item Faire cuire les p\^ates.

\item Ep\'epiner et \'emincer les poivrons.
 
\item Rincer le citron, pr\'elever le zeste, l'\'ebouillanter 1 min et le hacher.

\item Dans un saladier, m\'elanger les poivrons, la mozzarella coup\'ee en d\'es, le zeste, l'huile et l'ail hach\'e.

\item Ajouter les p\^ates, parsemer de basilic.

\end{enumerate}
\subsection*{\textsc{Conseil~:}}

								% <-- x1
\section[\normalsize{Salade de p\^ates provençale}]{\LARGE{\textsc{Salade de p\^ates provençale}}}		% <-- x2


\begin{itemize}
\item Pour 4 personnes
\item Préparation : 20 min*
\end{itemize}

\subsection*{\textsc{Ingr\'edients~:}}

\begin{itemize}
\item 1 poivron rouge;1 poivron jaune
\item 350g de p\^ates
\item 1 citron
\item 1 gousse d'ail
\item 2 boules de mozzarella
\item 6 c. \`a soupe d'huile d'olive
\item 2 brins de basilic
\end{itemize}


\subsection*{\textsc{Marche \`a suivre~:}}

\begin{enumerate}

\item Faire cuire les p\^ates.

\item Ep\'epiner et \'emincer les poivrons.
 
\item Rincer le citron, pr\'elever le zeste, l'\'ebouillanter 1 min et le hacher.

\item Dans un saladier, m\'elanger les poivrons, la mozzarella coup\'ee en d\'es, le zeste, l'huile et l'ail hach\'e.

\item Ajouter les p\^ates, parsemer de basilic.

\end{enumerate}
\subsection*{\textsc{Conseil~:}}

								% <-- x1
\section[\normalsize{Salade de p\^ates provençale}]{\LARGE{\textsc{Salade de p\^ates provençale}}}		% <-- x2


\begin{itemize}
\item Pour 4 personnes
\item Préparation : 20 min*
\end{itemize}

\subsection*{\textsc{Ingr\'edients~:}}

\begin{itemize}
\item 1 poivron rouge;1 poivron jaune
\item 350g de p\^ates
\item 1 citron
\item 1 gousse d'ail
\item 2 boules de mozzarella
\item 6 c. \`a soupe d'huile d'olive
\item 2 brins de basilic
\end{itemize}


\subsection*{\textsc{Marche \`a suivre~:}}

\begin{enumerate}

\item Faire cuire les p\^ates.

\item Ep\'epiner et \'emincer les poivrons.
 
\item Rincer le citron, pr\'elever le zeste, l'\'ebouillanter 1 min et le hacher.

\item Dans un saladier, m\'elanger les poivrons, la mozzarella coup\'ee en d\'es, le zeste, l'huile et l'ail hach\'e.

\item Ajouter les p\^ates, parsemer de basilic.

\end{enumerate}
\subsection*{\textsc{Conseil~:}}


\begin{recette}{Salade de bl\'e aux l\'egumes po\^el\'es}{Salade de bl\'e aux l\'egumes po\^el\'es}

\begin{ingredients}
 350 g de bl\'e pr\'ecuit type Ebly\par
1 aubergine\par
1 courgette\par
1 poivron rouge et 1 poivron jaune\par
1 oignon\par
1 gousse d'ail\par
2 brins de thym, 1 brin de basilic\par
8 c. \`a soupe d'huile d'olive\par
Sel, poivre
\end{ingredients}
\begin{infos}
Pour 6 personnes \\
Préparation : 25 min*
\end{infos}
\begin{etapes}
\item Faire cuire le bl\'e environ 10 minutes.
\item Emincer l'aubergine, les poivrons et la courgette. 
\item Faire revenir poivrons et courgette 5 minutes avec 3 cuill\`eres \`a soupe d'huile. Les retirer.
\item Faire sauter 7 minutes l'aubergine, l'oignon et l'ail hach\'e dans 3 cuill\`eres \`a soupe d'huile. 
\item Remettre les autres l\'egumes, ajouter le thym, saler et poivrer. Couvrir et laisser cuire 5 minutes.
\item M\'elanger le bl\'e avec le reste d'huile et les l\'egumes. Parsemer de basilic cisel\'e et servir ti\`ede ou frais.
\end{etapes}
\begin{conseils}
Tr\`es bon mais manque une petite note acide. Peut \^etre un peu de jus de citron ?
Pour un plat complet ou en accompagnement, mettre un peu moins de bl\'e en proportions.
\end{conseils}
\end{recette}
% Generated file 2018-12-02 20:50:19.449702486 +01:00
\begin{recette}{Salade de mâche aux champignons}{Salade de mâche aux champignons}

\begin{ingredients}
150 g de mâche\par
150 g de champignons de Paris\par
100 g de chèvre frais\par
1/2 citron\par
\textbf{ sauce :}\par
\begin{itemize}\par
\item[] 1 cuillère à soupe de jus de citron\par
\item[] 3 cuillères à soupe d'huile d'olive\par
\item[] 1 oeuf dur\par
\item[] 2 cuillère à café de câpres\par
\item[] 6 olives vertes\par
\end{itemize}\par
\end{ingredients}

\begin{infos}
4 personnes\\
Préparation : 15 min\\
\end{infos}

\begin{etapes}
\item \textbf{Sauce :} Fouetter une cuillère à soupe de jus de citron avec du sel et du poivre. Incorporer 3 cuillères à soupe d'huile d'olive. Ajouter un œuf dur, 2 cuillères à café de câpres et 6 olives vertes. Hacher le tout.
\item Couper la base des champignons. Les rouler entre vos mains sous un filet d'eau froide pour élimin le sable. Les essuyer dans un linge. Les éminr et les arroser de jus de citron.
\item Laver et essorer la mâche puis la répartir avec les champignons dans un saladier.
\item Émietter le fromage dessus et servir avec la sauce aux olives et aux câpres.
\end{etapes}

\end{recette}
	\chapter{Cakes}
%\begin{minipage}{15cm}
%\minitoc
%\end{minipage}
%\newpage

% Cake à la courgette, lardons et fromage de chèvre							% <-- x1
% % Cake à la courgette, lardons et fromage de chèvre							% <-- x1
% % Cake à la courgette, lardons et fromage de chèvre							% <-- x1
% \include{./recettes/Claire/cakecourgetteslardonsfromagedechevre}								% <-- x1
\section[\normalsize{Cake \`a la courgette, lardons et fromage de ch\`evre}]{\LARGE{\textsc{Cake \`a la courgette, lardons et fromage de ch\`evre}}}		% <-- x2


\begin{itemize}
\item Pour 6 personnes
\item Préparation : 30 min
\item Cuisson : 45 min
\end{itemize}

\subsection*{\textsc{Ingr\'edients~:}}

\begin{itemize}
\item 150 g de farine
\item 1 sachet de levure
\item 3 oeufs
\item 12,5 cl de lait
\item 2 courgettes
\item 1 bûche de ch\`evre
\item 100 g de comt\'e
\item 100 g de lardons
\item 4 cuill\`eres \`a soupe d'huile
\item sel, poivre, origan (ou herbes de Provence \`a d\'efaut)
\end{itemize}


\subsection*{\textsc{Marche \`a suivre~:}}

\begin{enumerate}
\item Pr\'echauffer le four \`a 180° C.

\item D\'ecouper les courgettes en rondelles et les passer au four \`a micro-ondes durant 4 mn pour les pr\'ecuire. Les mettre ensuite dans une grande po\^ele anti-adh\'esive et continuer la cuisson jusqu'\`a ce qu'elles commencent \`a accrocher \`a la po\^ele (2 \`a 3 mn). 

\item Ajouter les lardons, l'origan, poivrer et laisser dorer.

\item Dans un grands saladier, battre les 3 oeufs et incorporer la farine, la levure, une pinc\'ee de sel, l'huile et le lait pr\'ealablement chauff\'e.

\item Ajouter ensuite le comt\'e r\^ap\'e, les courgettes, les lardons et la brique de ch\`evre d\'ecoup\'ee en petits morceaux. 

\item Verser l'ensemble dans un moule \`a cake puis d\'eposer sur le dessus quelques lamelles suppl\'ementaires de comt\'e pour gratiner.

\item Faire cuire au four 45 min.

\end{enumerate}
\subsection*{\textsc{Conseil~:}}

								% <-- x1
\section[\normalsize{Cake \`a la courgette, lardons et fromage de ch\`evre}]{\LARGE{\textsc{Cake \`a la courgette, lardons et fromage de ch\`evre}}}		% <-- x2


\begin{itemize}
\item Pour 6 personnes
\item Préparation : 30 min
\item Cuisson : 45 min
\end{itemize}

\subsection*{\textsc{Ingr\'edients~:}}

\begin{itemize}
\item 150 g de farine
\item 1 sachet de levure
\item 3 oeufs
\item 12,5 cl de lait
\item 2 courgettes
\item 1 bûche de ch\`evre
\item 100 g de comt\'e
\item 100 g de lardons
\item 4 cuill\`eres \`a soupe d'huile
\item sel, poivre, origan (ou herbes de Provence \`a d\'efaut)
\end{itemize}


\subsection*{\textsc{Marche \`a suivre~:}}

\begin{enumerate}
\item Pr\'echauffer le four \`a 180° C.

\item D\'ecouper les courgettes en rondelles et les passer au four \`a micro-ondes durant 4 mn pour les pr\'ecuire. Les mettre ensuite dans une grande po\^ele anti-adh\'esive et continuer la cuisson jusqu'\`a ce qu'elles commencent \`a accrocher \`a la po\^ele (2 \`a 3 mn). 

\item Ajouter les lardons, l'origan, poivrer et laisser dorer.

\item Dans un grands saladier, battre les 3 oeufs et incorporer la farine, la levure, une pinc\'ee de sel, l'huile et le lait pr\'ealablement chauff\'e.

\item Ajouter ensuite le comt\'e r\^ap\'e, les courgettes, les lardons et la brique de ch\`evre d\'ecoup\'ee en petits morceaux. 

\item Verser l'ensemble dans un moule \`a cake puis d\'eposer sur le dessus quelques lamelles suppl\'ementaires de comt\'e pour gratiner.

\item Faire cuire au four 45 min.

\end{enumerate}
\subsection*{\textsc{Conseil~:}}

								% <-- x1
\section[\normalsize{Cake \`a la courgette, lardons et fromage de ch\`evre}]{\LARGE{\textsc{Cake \`a la courgette, lardons et fromage de ch\`evre}}}		% <-- x2


\begin{itemize}
\item Pour 6 personnes
\item Préparation : 30 min
\item Cuisson : 45 min
\end{itemize}

\subsection*{\textsc{Ingr\'edients~:}}

\begin{itemize}
\item 150 g de farine
\item 1 sachet de levure
\item 3 oeufs
\item 12,5 cl de lait
\item 2 courgettes
\item 1 bûche de ch\`evre
\item 100 g de comt\'e
\item 100 g de lardons
\item 4 cuill\`eres \`a soupe d'huile
\item sel, poivre, origan (ou herbes de Provence \`a d\'efaut)
\end{itemize}


\subsection*{\textsc{Marche \`a suivre~:}}

\begin{enumerate}
\item Pr\'echauffer le four \`a 180° C.

\item D\'ecouper les courgettes en rondelles et les passer au four \`a micro-ondes durant 4 mn pour les pr\'ecuire. Les mettre ensuite dans une grande po\^ele anti-adh\'esive et continuer la cuisson jusqu'\`a ce qu'elles commencent \`a accrocher \`a la po\^ele (2 \`a 3 mn). 

\item Ajouter les lardons, l'origan, poivrer et laisser dorer.

\item Dans un grands saladier, battre les 3 oeufs et incorporer la farine, la levure, une pinc\'ee de sel, l'huile et le lait pr\'ealablement chauff\'e.

\item Ajouter ensuite le comt\'e r\^ap\'e, les courgettes, les lardons et la brique de ch\`evre d\'ecoup\'ee en petits morceaux. 

\item Verser l'ensemble dans un moule \`a cake puis d\'eposer sur le dessus quelques lamelles suppl\'ementaires de comt\'e pour gratiner.

\item Faire cuire au four 45 min.

\end{enumerate}
\subsection*{\textsc{Conseil~:}}


\section[\normalsize{Cake au thon}]{Cake au thon}

\begin{ingredients}
\item 3 oeufs
\item 25 cl de lait
\item 150 g de farine
\item 100 g de gruy\`ere rap\'e
\item 1 sachet de levure
\item 190 g de thon \'egoutt\'e
\item 2 cuill\`eres \`a soupe d'huile
\item sel, poivre 
\end{ingredients}
\begin{infos}
\item Pour 6 personnes
\item Préparation : 15 min
\item Cuisson : 45 min
\end{infos}
\begin{etapes}
\item Pr\'echauffer le four à 180° C.
\item M\'elanger les oeufs et la farine.
\item Ajouter la levure, puis l'huile, le lait, le thon \'emiett\'e et enfin le gruy\`ere.
\item Saler, poivrer.
\item Enfourner 45 minutes. 
\end{etapes}
\begin{conseils}
Petit cake : 2 oeufs, 17 cl de lait, 100 g de farine, 67 g de gruy\`ere, 2/3 sachet de levure, une bo\^ite de 132 g de thon (ou 127 g), 1 cuill\`ere 1/3 \`a soupe d'huile.
\end{conseils}

\section[\normalsize{Cake aux tomates s\'ech\'ees, feta et olives}]{Cake aux tomates s\'ech\'ees, feta et olives}

\begin{ingredients}
\item 200 g de farine
\item 4 oeufs
\item 1 sachet de levure
\item 10 cl de lait
\item 2 cuill\`eres \`a soupe d'huile d'olive
\item 15 tomates s\'ech\'ees
\item 200 de feta
\item 10 olives vertes et 5 noires d\'enoyaut\'ees
\item 5 feuilles de basilic
\item Poivre
\end{ingredients}
\begin{infos}
\item Pour 6 personnes
\item Préparation : 20 min
\item Cuisson : 40 min
\end{infos}
\begin{etapes}
\item Pr\'echauffer le four \`a 180° C
\item M\'elanger ensemble la farine, les oeufs et la levure.
\item Ajouter l'huile d'olive, et le lait froid.
\item Bien m\'elanger, en soulevant avec la cuill\`ere, afin d'a\'erer la p\^ate.
\item Ajouter les tomates s\'ech\'ees coup\'ee en morceaux, la feta, les olives vertes (en petits morceaux) et le basilic cisel\'e. 
\item poivrer.
\item Verser la pr\'eparation dans un moule \`a cake.
\item Faire cuire pendant 40 min. V\'erifier la cuisson avec la lame d'un couteau, elle doit ressortir s\`eche.
\end{etapes}
\begin{conseils}
\end{conseils}
%Clafoutis léger aux courgettes et fromage de chèvre							% <-- x1
% %Clafoutis léger aux courgettes et fromage de chèvre							% <-- x1
% %Clafoutis léger aux courgettes et fromage de chèvre							% <-- x1
% \include{./recettes/Claire/clafoutiscourgettesfromagedechevre}					% <-- x1
\section[\normalsize{Clafoutis l\'eger aux courgettes et fromage de ch\`evre}]{\LARGE{\textsc{Clafoutis l\'eger aux courgettes et fromage de ch\`evre}}}		% <-- x2


\begin{itemize}
\item Pour 4 personnes
\item Préparation : 25 min
\item Cuisson : 35 min
\end{itemize}

\subsection*{\textsc{Ingr\'edients~:}}

\begin{itemize}
\item 150 ml de lait
\item 40 g de farine
\item 2 gros oeufs
\item 150 g de fromage de ch\`evre frais
\item 2 petites courgettes
\item huile d'olive 
\item basilic et thym
\item sel \& poivre 
\end{itemize}


\subsection*{\textsc{Marche \`a suivre~:}}

\begin{enumerate}
\item Pr\'echauffer le four \`a 200° C (thermostat 6-7).

\item Laver les courgettes et les couper en fines rondelles sans les \'eplucher.

\item Les faire revenir dans une po\^ele anti-adh\'esive avec un peu d'huile d'olive jusqu'\`a ce qu'elles aient bien rendu leur eau.

\item Battre les œufs en omelette dans un saladier. Ajouter le ch\`evre frais, puis petit \`a petit la farine.

\item Terminer en incorporant le lait et les herbes. Saler et poivrer. Une fois la p\^ate homog\`ene, y ajouter les rondelles de courgettes.

\item Verser le tout dans un moule \`a manqu\'e en silicone de 20 cm de diam\`etre.

\item Faire cuire 35 \`a 40 minutes. 
\end{enumerate}
\subsection*{\textsc{Conseil~:}}

Quantit\'e d'oeufs peut \^etre \`a revoir.
Essayer avec du ch\`evre plus fort (bûche ou crottins coup\'es en morceaux).
					% <-- x1
\section[\normalsize{Clafoutis l\'eger aux courgettes et fromage de ch\`evre}]{\LARGE{\textsc{Clafoutis l\'eger aux courgettes et fromage de ch\`evre}}}		% <-- x2


\begin{itemize}
\item Pour 4 personnes
\item Préparation : 25 min
\item Cuisson : 35 min
\end{itemize}

\subsection*{\textsc{Ingr\'edients~:}}

\begin{itemize}
\item 150 ml de lait
\item 40 g de farine
\item 2 gros oeufs
\item 150 g de fromage de ch\`evre frais
\item 2 petites courgettes
\item huile d'olive 
\item basilic et thym
\item sel \& poivre 
\end{itemize}


\subsection*{\textsc{Marche \`a suivre~:}}

\begin{enumerate}
\item Pr\'echauffer le four \`a 200° C (thermostat 6-7).

\item Laver les courgettes et les couper en fines rondelles sans les \'eplucher.

\item Les faire revenir dans une po\^ele anti-adh\'esive avec un peu d'huile d'olive jusqu'\`a ce qu'elles aient bien rendu leur eau.

\item Battre les œufs en omelette dans un saladier. Ajouter le ch\`evre frais, puis petit \`a petit la farine.

\item Terminer en incorporant le lait et les herbes. Saler et poivrer. Une fois la p\^ate homog\`ene, y ajouter les rondelles de courgettes.

\item Verser le tout dans un moule \`a manqu\'e en silicone de 20 cm de diam\`etre.

\item Faire cuire 35 \`a 40 minutes. 
\end{enumerate}
\subsection*{\textsc{Conseil~:}}

Quantit\'e d'oeufs peut \^etre \`a revoir.
Essayer avec du ch\`evre plus fort (bûche ou crottins coup\'es en morceaux).
					% <-- x1
\section[\normalsize{Clafoutis l\'eger aux courgettes et fromage de ch\`evre}]{\LARGE{\textsc{Clafoutis l\'eger aux courgettes et fromage de ch\`evre}}}		% <-- x2


\begin{itemize}
\item Pour 4 personnes
\item Préparation : 25 min
\item Cuisson : 35 min
\end{itemize}

\subsection*{\textsc{Ingr\'edients~:}}

\begin{itemize}
\item 150 ml de lait
\item 40 g de farine
\item 2 gros oeufs
\item 150 g de fromage de ch\`evre frais
\item 2 petites courgettes
\item huile d'olive 
\item basilic et thym
\item sel \& poivre 
\end{itemize}


\subsection*{\textsc{Marche \`a suivre~:}}

\begin{enumerate}
\item Pr\'echauffer le four \`a 200° C (thermostat 6-7).

\item Laver les courgettes et les couper en fines rondelles sans les \'eplucher.

\item Les faire revenir dans une po\^ele anti-adh\'esive avec un peu d'huile d'olive jusqu'\`a ce qu'elles aient bien rendu leur eau.

\item Battre les œufs en omelette dans un saladier. Ajouter le ch\`evre frais, puis petit \`a petit la farine.

\item Terminer en incorporant le lait et les herbes. Saler et poivrer. Une fois la p\^ate homog\`ene, y ajouter les rondelles de courgettes.

\item Verser le tout dans un moule \`a manqu\'e en silicone de 20 cm de diam\`etre.

\item Faire cuire 35 \`a 40 minutes. 
\end{enumerate}
\subsection*{\textsc{Conseil~:}}

Quantit\'e d'oeufs peut \^etre \`a revoir.
Essayer avec du ch\`evre plus fort (bûche ou crottins coup\'es en morceaux).
	
% Generated file 2019-02-17 16:33:30.922306398 +01:00
\begin{recette}{Cake Alsacien façon flammenküche}{Cake Alsacien façon flammenküche}

\begin{ingredients}
3 oeufs (4)\par
150 g de farine (200 g)\par
1 sachet de levure\par
6 cl d’huile de tournesol (6 cl)\par
12,5 cl de lait entier (16,5 cl)\par
100 g de gruyère râpé (133 g)\par
100 g d’oignons (133 g)\par
200 g de lardons fumés (266 g)\par
1 noisette de beurre demi-sel\par
1 cuillerée à soupe d’huile de tournesol\par
1 cuillerée à soupe de crème épaisse\par
1 pincée de sel, 1 pincé de poivre.\par
\end{ingredients}

\begin{infos}
Pour XX personnes\\
Préparation : XX min\\
Cuisson : 45 min\\
\end{infos}

\begin{etapes}
\item Préchauffez votre four à 180 °s C (thermostat 6).
\item Eminz les oignons, faites-les revenir dans une poêle avec la noisette de beurre et la cuillerée d’huile, mettez la pincée de sel et de poivre. Lorsqu’ils blondissent, ajoutez les lardons et faites-les légèrement rissoler. Retirez-les du feu et versez-y la cuillerée à soupe de crème.
\item Pendant ce temps, dans un saladier, travaillez bien au fouet les œufs, la farine et la levure. Incorporez petit à petit l’huile et le lait préalablement chauffé. Ajoutez le gruyère râpé. Mélangez.
\item Incorporez le mélange oignons, lardons, et crème à la base.
\item Versez le tout dans un moule non graissé et mettez au four pendant 45 min.
\end{etapes}

\end{recette}	
	\chapter{Tartes}
% Tarte au saumon et poireaux										% <-- x1
% % Tarte au saumon et poireaux										% <-- x1
% % Tarte au saumon et poireaux										% <-- x1
% \include{./recettes/Claire/tarteausaumonetpoireaux}								% <-- x1
\section[\normalsize{Tarte au saumon et poireaux}]{\LARGE{\textsc{Tarte au saumon et poireaux}}}		% <-- x2


\begin{itemize}
\item Pour 6 personnes
\item Préparation : 25 min
\item Cuisson : 35 min
\end{itemize}

\subsection*{\textsc{Ingr\'edients~:}}

\begin{itemize}
\item 1 p\^ate bris\'ee
\item 400 g de pav\'e de saumon
\item 4 beaux poireaux
\item 3 oeufs
\item 1 petit pot de cr\`eme fra\^iche
\item huile d'olive
\end{itemize}


\subsection*{\textsc{Marche \`a suivre~:}}

\begin{enumerate}
\item Couper les blancs de poireaux en deux, les laver et les d\'etailler en rondelles. Les mettre \`a cuire dans une po\^ele avec une cuill\`ere d'huile. Saler et poivrer l\'eg\`erement et laisser dorer.

\item Pr\'echauffer le four \`a 200° C.

\item Pendant que les poireaux cuisent, couper le saumon en gros d\'es. Tapisser un moule \`a tarte avec la p\^ate bris\'ee et la piquer. 

\item R\'epartir les d\'es de saumon dessus et les recouvrir avec les poireaux cuits.

\item Battre les oeufs avec la cr\`eme dans un saladier. Saler, poivrer et verser sur la tarte.

\item Mettre au four et faire cuire 30 \`a 40 minutes.
\end{enumerate}
\subsection*{\textsc{Conseil~:}}

*Four Claire Cuisson 35 minutes, surveiller la couleur.
								% <-- x1
\section[\normalsize{Tarte au saumon et poireaux}]{\LARGE{\textsc{Tarte au saumon et poireaux}}}		% <-- x2


\begin{itemize}
\item Pour 6 personnes
\item Préparation : 25 min
\item Cuisson : 35 min
\end{itemize}

\subsection*{\textsc{Ingr\'edients~:}}

\begin{itemize}
\item 1 p\^ate bris\'ee
\item 400 g de pav\'e de saumon
\item 4 beaux poireaux
\item 3 oeufs
\item 1 petit pot de cr\`eme fra\^iche
\item huile d'olive
\end{itemize}


\subsection*{\textsc{Marche \`a suivre~:}}

\begin{enumerate}
\item Couper les blancs de poireaux en deux, les laver et les d\'etailler en rondelles. Les mettre \`a cuire dans une po\^ele avec une cuill\`ere d'huile. Saler et poivrer l\'eg\`erement et laisser dorer.

\item Pr\'echauffer le four \`a 200° C.

\item Pendant que les poireaux cuisent, couper le saumon en gros d\'es. Tapisser un moule \`a tarte avec la p\^ate bris\'ee et la piquer. 

\item R\'epartir les d\'es de saumon dessus et les recouvrir avec les poireaux cuits.

\item Battre les oeufs avec la cr\`eme dans un saladier. Saler, poivrer et verser sur la tarte.

\item Mettre au four et faire cuire 30 \`a 40 minutes.
\end{enumerate}
\subsection*{\textsc{Conseil~:}}

*Four Claire Cuisson 35 minutes, surveiller la couleur.
								% <-- x1
\section[\normalsize{Tarte au saumon et poireaux}]{\LARGE{\textsc{Tarte au saumon et poireaux}}}		% <-- x2


\begin{itemize}
\item Pour 6 personnes
\item Préparation : 25 min
\item Cuisson : 35 min
\end{itemize}

\subsection*{\textsc{Ingr\'edients~:}}

\begin{itemize}
\item 1 p\^ate bris\'ee
\item 400 g de pav\'e de saumon
\item 4 beaux poireaux
\item 3 oeufs
\item 1 petit pot de cr\`eme fra\^iche
\item huile d'olive
\end{itemize}


\subsection*{\textsc{Marche \`a suivre~:}}

\begin{enumerate}
\item Couper les blancs de poireaux en deux, les laver et les d\'etailler en rondelles. Les mettre \`a cuire dans une po\^ele avec une cuill\`ere d'huile. Saler et poivrer l\'eg\`erement et laisser dorer.

\item Pr\'echauffer le four \`a 200° C.

\item Pendant que les poireaux cuisent, couper le saumon en gros d\'es. Tapisser un moule \`a tarte avec la p\^ate bris\'ee et la piquer. 

\item R\'epartir les d\'es de saumon dessus et les recouvrir avec les poireaux cuits.

\item Battre les oeufs avec la cr\`eme dans un saladier. Saler, poivrer et verser sur la tarte.

\item Mettre au four et faire cuire 30 \`a 40 minutes.
\end{enumerate}
\subsection*{\textsc{Conseil~:}}

*Four Claire Cuisson 35 minutes, surveiller la couleur.

\section[\normalsize{Tarte aux aubergines, tomates et ch\`evre}]{Tarte aux aubergines, tomates et ch\`evre}

\begin{ingredients}
\item 1 belle aubergine
\item 2 \`a 3 tomates
\item 1 petite bûche de ch\`evre
\item 1 rouleau de p\^ate feuillet\'ee
\item Sel, poivre, herbes de Provence
\item Huile d'olive
\end{ingredients}
\begin{infos}
\item Pour 6 personnes
\item Préparation : 20 min
\item Cuisson : 25 min
\end{infos}
\begin{etapes}
\item Pr\'echaufer le four \`a 200° C. Laver les aubergines et les couper en rondelles assez fines. Les d\'eposer sur une plaque recouverte de papier sulfuris\'e et les badigeonner d'huile d'olive \`a l'aide d'un pinceau. 
\item Enfourner pour environ 10 minutes. (surveiller).
\item Pendant ce temps, foncer un moule \`a tarte et piquer le fond avec une fourchette. Laver les tomates et les d\'ecouper en fines rondelles. 
\item Couper le fromage en tranches fines.
\item Lorsque les aubergines sont dor\'ees, les retourner, badigeonner \'eventuellement d'un peu d'huile et les remettre au four environ 5 minutes.
\item Badigeonner le fond de tarte d'un peu de moutarde. R\'epartir les tranches d'aubergines, puis les recouvrir avec les rondelles de tomates. 
\item Saler l\'eg\`erement et poivrer. 
\item Parsemer de fromage de ch\`evre, saupoudrer d'herbes de Provence et faire cuire 25 minutes.
\end{etapes}
\begin{conseils}
\end{conseils}
% Tarte aux poireaux										% <-- x1
% % Tarte aux poireaux										% <-- x1
% % Tarte aux poireaux										% <-- x1
% \include{./recettes/Claire/tarteauxpoireaux}								% <-- x1
\section[\normalsize{Tarte aux poireaux}]{\LARGE{\textsc{Tarte aux poireaux}}}		% <-- x2


\begin{itemize}
\item Pour 6 personnes
\item Préparation : 20 min
\item Cuisson : 40 min
\end{itemize}

\subsection*{\textsc{Ingr\'edients~:}}

\begin{itemize}
\item 1 p\^ate bris\'ee
\item 2 \`a 3 poireaux
\item 100 g de lardons
\item 1 petite cuill\`ere de farine
\item 4 oeufs
\item 200 g de cr\`eme fra\^iche
\item poivre 
\end{itemize}


\subsection*{\textsc{Marche \`a suivre~:}}

\begin{enumerate}
\item Pr\'echauffer le four th 6/7.

\item Laver et \'emincer les poireaux. Faire revenir les lardons dans une pôele. Lorsqu'ils commencent \`a fondre, ajouter les poireaux. Quand ils commencent \`a colorer, lier avec une petite cuill\`ere de farine. Laisser revenir \`a feu doux.

\item Battre dans un bol les oeufs, la cr\`eme fra\^iche et du poivre.

\item Chemiser le moule \`a tarte, ajouter la fondue de poireaux. Verser l'appareil oeufs et cr\`eme par dessus. 

\item Enfourner et faire cuire 40 minutes environ.
\end{enumerate}
\subsection*{\textsc{Conseil~:}}

								% <-- x1
\section[\normalsize{Tarte aux poireaux}]{\LARGE{\textsc{Tarte aux poireaux}}}		% <-- x2


\begin{itemize}
\item Pour 6 personnes
\item Préparation : 20 min
\item Cuisson : 40 min
\end{itemize}

\subsection*{\textsc{Ingr\'edients~:}}

\begin{itemize}
\item 1 p\^ate bris\'ee
\item 2 \`a 3 poireaux
\item 100 g de lardons
\item 1 petite cuill\`ere de farine
\item 4 oeufs
\item 200 g de cr\`eme fra\^iche
\item poivre 
\end{itemize}


\subsection*{\textsc{Marche \`a suivre~:}}

\begin{enumerate}
\item Pr\'echauffer le four th 6/7.

\item Laver et \'emincer les poireaux. Faire revenir les lardons dans une pôele. Lorsqu'ils commencent \`a fondre, ajouter les poireaux. Quand ils commencent \`a colorer, lier avec une petite cuill\`ere de farine. Laisser revenir \`a feu doux.

\item Battre dans un bol les oeufs, la cr\`eme fra\^iche et du poivre.

\item Chemiser le moule \`a tarte, ajouter la fondue de poireaux. Verser l'appareil oeufs et cr\`eme par dessus. 

\item Enfourner et faire cuire 40 minutes environ.
\end{enumerate}
\subsection*{\textsc{Conseil~:}}

								% <-- x1
\section[\normalsize{Tarte aux poireaux}]{\LARGE{\textsc{Tarte aux poireaux}}}		% <-- x2


\begin{itemize}
\item Pour 6 personnes
\item Préparation : 20 min
\item Cuisson : 40 min
\end{itemize}

\subsection*{\textsc{Ingr\'edients~:}}

\begin{itemize}
\item 1 p\^ate bris\'ee
\item 2 \`a 3 poireaux
\item 100 g de lardons
\item 1 petite cuill\`ere de farine
\item 4 oeufs
\item 200 g de cr\`eme fra\^iche
\item poivre 
\end{itemize}


\subsection*{\textsc{Marche \`a suivre~:}}

\begin{enumerate}
\item Pr\'echauffer le four th 6/7.

\item Laver et \'emincer les poireaux. Faire revenir les lardons dans une pôele. Lorsqu'ils commencent \`a fondre, ajouter les poireaux. Quand ils commencent \`a colorer, lier avec une petite cuill\`ere de farine. Laisser revenir \`a feu doux.

\item Battre dans un bol les oeufs, la cr\`eme fra\^iche et du poivre.

\item Chemiser le moule \`a tarte, ajouter la fondue de poireaux. Verser l'appareil oeufs et cr\`eme par dessus. 

\item Enfourner et faire cuire 40 minutes environ.
\end{enumerate}
\subsection*{\textsc{Conseil~:}}

	
\section[\normalsize{Tarte ch\`evre-piperade}]{Tarte ch\`evre-piperade}

\begin{ingredients}
\item 1 p\^ate bris\'ee
\item 300 g de poivrons en petits d\'es (surgel\'es)
\item 200 g de chair de tomates pel\'ees
\item 2 oeufs
\item 1 belle gousse d'ail
\item 1 petit oignon
\item 50 g de pancetta
\item 10 cl de cr\`eme
\item 1/2 bûche de ch\`evre
\item 1 cuill\`ere \`a caf\'e de piment d'Espelette 
\item quelques feuilles de basilic frais
\item 2 cuill\`eres \`a soupe d'huile d'olive
\end{ingredients}
\begin{infos}
\item Pour 6 personnes
\item Préparation : 45 min
\item Cuisson : 35 min
\end{infos}
\begin{etapes}
\item Pr\'echauffer le four \`a 200° C. 
\item Faire revenir l'oignon et l'ail dans 2 cuill\`eres \`a soupe d'huile d'olive. 
\item Ajouter les d\'es de poivron, la tomate et le piment d'Espelette, et laisser mijoter 15 minutes environ jusqu'\`a ce que les poivrons deviennent tendres. R\'eserver. 
\item Dans une autre po\^ele, faire revenir, \`a sec et \`a feu tr\`es vif, la pancetta coup\'ee en petis morceaux, jusqu'\`a ce qu'elle soit bien dor\'ee. 
\item Egoutter pour ôter l'exc\`es de gras et r\'eserver.
\item M\'elanger les l\'egumes et la pancetta. Tapisser un moule \`a tarte avec la p\^ate bris\'ee et la piquer. R\'epartir les l\'egumes dessus.
\item Battre les oeufs avec la cr\`eme dans un saladier. Saler l\'eg\`erement et verser sur la tarte. Garnir de rondelles de ch\`evre et de feuilles de basilic.
\item Baisser la temp\'erature du four \`a 180° C et enfourner pour 30 \`a 40 minutes. Servir ti\`ede ou froid.
\end{etapes}
\begin{conseils}
\end{conseils}

\section[\normalsize{Tarte fine mascarpone jambon poireaux ch\`evre}]{Tarte fine mascarpone jambon poireaux ch\`evre}

\begin{ingredients}
\item 1 p\^ate bris\'ee ou feuillet\'ee
\item 5 petits poireaux
\item 3 tranches de jambon cru
\item 1 bûche de ch\`evre
\item 3 \`a 4 cuill\`eres \`a soupe de mascarpone
\item Herbes de Provence
\item Paprika
\item Sel, poivre
\end{ingredients}
\begin{infos}
\item Pour 4 personnes
\item Préparation : 40 min
\item Cuisson : 20 min
\end{infos}
\begin{etapes}
\item \'etaler la p\^ate \`a plat sur une plaque de cuisson.
\item Retourner le rebord sur 1 cm. Piquer la p\^ate \`a la fourchette et la pr\'ecuire 10 minutes \`a 220° C.
\item Pendant ce temps laver et \'emincer les poireaux puis les faire revenir \`a feu vif dans une po\^ele anti adh\'esive, avec un filet d'huile d'olive, jusqu'\`a ce qu'ils soient bien dor\'es. Ajouter 1 \`a 2 verre(s) d'eau. Couvrir et cuire pendant 15 minutes \`a feu doux.
\item D\'ecouvrir et laisser r\'eduire jusqu'\`a \'evaporation de l'eau. Saler et poivrer.
\item \'etaler le mascarpone sur la p\^ate pr\'ecuite. R\'epartir le jambon cru coup\'e en fines lani\`eres. Ajouter le m\'elange de poireaux. R\'epartir la bûche de ch\`evre tranch\'ee. 
\item Parsemer de paprika et d'herbes de Provence.
\item Cuire 20 minutes \`a 190° C. Servir ti\`ede.
\end{etapes}
\begin{conseils}
Possibilit\'e de remplacer le mascarpone par de la cr\`eme.
\end{conseils}

% Nom de la recette à entrer entre les accolades {}
\section{Tarte au chou romanesco et gouda au cumin}

\begin{ingredients}
\item 1 pâte brisée parfumée avec 1 cuillère à café de cumin en poudre
\item 1 gros chou romanesco
\item 200 g de gouda au cumin
\item 2 oeufs
\item 20 cl de crème
\item sel, poivre
\end{ingredients}
\begin{infos}
\item Pour 6 personnes		% Nombre de personnes qu'on pourra nourrir ! :)
\item Préparation : 20 min		% Temps de préparation (sans la cuisson)
\item Cuisson : 30 min			% Temps de cuisson
\end{infos}
\begin{etapes}
\item Détailler le chou romanesco sans abîmer les « mini choux ». Cuire 3 mn à la cocotte minute dans le panier vapeur.
\item Préchauffer le four à 180°C. Foncer le moule à tarte.
\item Battre ensemble les oeufs et la crème, ajouter le gouda râpé. Saler et poivrer. Verser sur le fond de tarte. Disposer harmonieusement par-dessus le chou cuit.
\item Enfourner pour 30 min.
\end{etapes}
\begin{conseils}
\end{conseils}
	\chapter{Pizzas}
	\chapter{Pâtes}
% Pâtes Carbonara											% <-- x1
% % Pâtes Carbonara											% <-- x1
% % Pâtes Carbonara											% <-- x1
% \include{./recettes/Claire/patescarbonarafaconclaire}								% <-- x1
\section[\normalsize{P\^ates Carbonara (fa\c con Claire T.)}]{\LARGE{\textsc{P\^ates Carbonara (fa\c con Claire T.)}}}		% <-- x2


\begin{itemize}
\item Pour 4 personnes
\item Préparation : 15 min
\end{itemize}

\subsection*{\textsc{Ingr\'edients~:}}

\begin{itemize}
\item 200 g de lardons
\item 20 cl  de cr\`eme (grand modele)
\item 3 oeufs
\item 50g de beurre
\item 100 g de fromage rap\'e
\item Piment de cayenne moulu
\item Poivre
\end{itemize}


\subsection*{\textsc{Marche \`a suivre~:}}

\begin{enumerate}
\item Faire revenir les lardons \`a la po\^ele. 
\item Quand ils sont dor\'es, ajouter la cr\`eme et un peu de piment de cayenne. Laisser r\'eduire.
\item Pendant ce temps, mettre dans le plat de pr\'esentation les oeufs battus, le beurre, le fromage rap\'e et du poivre. 
\item M\'elanger \`a la fourchette.
\item Lorsque les p\^ates sont cuites, les verser dans le plat et ajouter le m\'elange cr\`eme/lardons par dessus. 
\item Bien m\'elanger et servir.
\end{enumerate}
\subsection*{\textsc{Conseil~:}}
								% <-- x1
\section[\normalsize{P\^ates Carbonara (fa\c con Claire T.)}]{\LARGE{\textsc{P\^ates Carbonara (fa\c con Claire T.)}}}		% <-- x2


\begin{itemize}
\item Pour 4 personnes
\item Préparation : 15 min
\end{itemize}

\subsection*{\textsc{Ingr\'edients~:}}

\begin{itemize}
\item 200 g de lardons
\item 20 cl  de cr\`eme (grand modele)
\item 3 oeufs
\item 50g de beurre
\item 100 g de fromage rap\'e
\item Piment de cayenne moulu
\item Poivre
\end{itemize}


\subsection*{\textsc{Marche \`a suivre~:}}

\begin{enumerate}
\item Faire revenir les lardons \`a la po\^ele. 
\item Quand ils sont dor\'es, ajouter la cr\`eme et un peu de piment de cayenne. Laisser r\'eduire.
\item Pendant ce temps, mettre dans le plat de pr\'esentation les oeufs battus, le beurre, le fromage rap\'e et du poivre. 
\item M\'elanger \`a la fourchette.
\item Lorsque les p\^ates sont cuites, les verser dans le plat et ajouter le m\'elange cr\`eme/lardons par dessus. 
\item Bien m\'elanger et servir.
\end{enumerate}
\subsection*{\textsc{Conseil~:}}
								% <-- x1
\section[\normalsize{P\^ates Carbonara (fa\c con Claire T.)}]{\LARGE{\textsc{P\^ates Carbonara (fa\c con Claire T.)}}}		% <-- x2


\begin{itemize}
\item Pour 4 personnes
\item Préparation : 15 min
\end{itemize}

\subsection*{\textsc{Ingr\'edients~:}}

\begin{itemize}
\item 200 g de lardons
\item 20 cl  de cr\`eme (grand modele)
\item 3 oeufs
\item 50g de beurre
\item 100 g de fromage rap\'e
\item Piment de cayenne moulu
\item Poivre
\end{itemize}


\subsection*{\textsc{Marche \`a suivre~:}}

\begin{enumerate}
\item Faire revenir les lardons \`a la po\^ele. 
\item Quand ils sont dor\'es, ajouter la cr\`eme et un peu de piment de cayenne. Laisser r\'eduire.
\item Pendant ce temps, mettre dans le plat de pr\'esentation les oeufs battus, le beurre, le fromage rap\'e et du poivre. 
\item M\'elanger \`a la fourchette.
\item Lorsque les p\^ates sont cuites, les verser dans le plat et ajouter le m\'elange cr\`eme/lardons par dessus. 
\item Bien m\'elanger et servir.
\end{enumerate}
\subsection*{\textsc{Conseil~:}}
	
	\chapter{Viandes**}
\section[\normalsize{R\^oti de porc au miel}]{\LARGE{\textsc{R\^oti de porc au miel}}}


\begin{itemize}
\item Pour 6 personnes
\item Préparation : 15 min	
\item Cuisson : 50 min
\end{itemize}
\subsection*{\textsc{Ingrédients~:}}

\begin{itemize}
\item 1 rôti de porc\index{porc} d’environ 1 kg
\item 2 cuill\`ere \`a soupe de miel\index{miel} liquide
\item	30 g de beurre
\item	15 cl de Noilly Prat
\item 1 brin de thym
\item Noix de muscade
\item Sel, poivre
\end{itemize}


\subsection*{\textsc{Marche \`a suivre~:}}

\begin{enumerate}
\item Pr\'echauffez le four sur th.6 (180° C). Enduisez le r\^oti avec le beurre ramolli, salez-le, poivrez-le, puis parsemez-le de thym effeuill\'e et d’un peu de muscade fra\^ichement  r\^ap\'ee.

\item D\'eposez le r\^oti dans un plat, enfournez et cuisez pendant 35 min environ en l’arrosant régulièrement de son jus. A mi-cuisson, ajoutez le Noilly Prat.

\item Ouvrez le four, nappez le r\^oti avec le miel et continuez de cuire 15 min en l’arrosant deux fois avec son jus. Présentez-le d\'eficel\'e et coup\'e en tranches.
\end{enumerate}


\subsection*{\textsc{Conseil~:}}

Faites cuire votre r\^oti dans un plat compatible avec une cuisson au four juste assez grand pour le contenir ; vous \'eviterez ainsi au jus de br\^uler.

Le bon vin : un \emph{gewurztraminer}.

% Generated file 2018-11-25 21:43:22.326860944 +01:00
\begin{recette}{Rôti de porc aux oignons miel}{Rôti de porc aux oignons miel}

\begin{ingredients}
800 g de rôti de porc dans l'échine\par
1 kg d'oignons de différente taille\par
1 verre de vin blanc sec\par
huile\par
sel, poivre\par
\end{ingredients}

\begin{infos}
Pour 4 à 6 personnes\\
Préparation :30 min\\
Cuisson : 30 min\\
\end{infos}

\begin{etapes}
\item Éplucher les oignons, couper les plus gros en quatre ou en huit, en émincer quelques uns et couper les petits en deux ou les laisser entiers.
\item Mettre l'huile à chauffer dans une cocotte-minute. Faire dorer le rôti sur chaque face. Réserver.
\item Faire revenir les oignons jusqu'à qu'ils blondissent. Saler, poivrer légèrement.
\item Disposer le rôti sur les oignons, verser le vin blanc. Fermer la cocotte et laisser cuire 20 minutes à partir du sifflement de la soupape.
\item Servir avec un peu de riz éventuellement, ou seulement avec les oignons.
\end{etapes}

\end{recette}
% Generated file 2018-11-25 21:43:22.190154115 +01:00
\begin{recette}{Lapin aux pruneaux}{Lapin aux pruneaux}

\begin{ingredients}
1 beau lapin\index{lapin} coupé en 8 morceaux\par
50 cl litre de vin rouge\par
1 oignon piqué de 2 clous de girofle\par
1 bouquet garni\par
16 pruneaux dénoyautés\par
1 cuillerée à soupe d’huile\par
1 gousse d’ail\par
1 cuillerée à soupe de gelée de groseille\par
sel, poivre\par
1 cuillerée à soupe de crème\par
1 cuillerée à soupe rase de Maizena\par
80 g de margarine\par
1 petite boîte de champignons de Paris\par
\end{ingredients}

\begin{infos}
Pour 6 personnes\\
Préparation : 30 min\\
Cuisson : 50 min\\
\end{infos}

\begin{etapes}
\item La veille, faites mariner le lapin au frais avec la gousse d’ail, le vin rouge, l’huile, l’oignon et le bouquet garni.
\item Faites chauffer la margarine dans la cocotte-minute SEB : mettez-y à dorer les morceaux de lapin bien égouttés, versez par-dessus la marinade et la Ma\"\i zena mélangée dans très peu d’eau froide, tourner.
\item Puis ajouter les champignons, salez, poivrez et faites cuire 20 minutes à feu doux à partir du moment où la soupape chuchote.
\item Ajoutez alors les pruneaux, faites cuire 5 minutes sous pression.
\item Mélangez la crème avec la gelée de groseille. Ouvrez la cocotte-minute, mettez les morceaux de lapin dans le plat de service, versez dans la sauce le mélange crème-gelée et nappez le lapin.
\end{etapes}

\begin{conseils}
Une purée maison sera la bienvenue comme garniture.
\end{conseils}

\end{recette}
% Boeuf sauté au brocoli et aux oignons										% <-- x1
% % Boeuf sauté au brocoli et aux oignons										% <-- x1
% % Boeuf sauté au brocoli et aux oignons										% <-- x1
% \include{./recettes/Claire/boeufsauteaubrocolietauxoignons}						% <-- x1
\section[\normalsize{Boeuf saut\'e au brocoli et aux oignons}]{\LARGE{\textsc{Boeuf saut\'e au brocoli et aux oignons}}}		% <-- x2


\begin{itemize}
\item Pour 2 personnes
\item Préparation : 35 min
\item Cuisson : 5 --- 10 min
\end{itemize}

\subsection*{\textsc{Ingr\'edients~:}}

\begin{itemize}
\item 200 g de rumsteak
\item 1 petit brocoli
\item 2 gros oignons
\item sauce soja
\item sauce soja sucr\'ee
\item 1 cm de gingembre \'eminc\'e
\item 5 \'epices
\item Huile
\end{itemize}


\subsection*{\textsc{Marche \`a suivre~:}}

\begin{enumerate}
\item Emincer le boeuf, le mettre dans un plat creux et arroser de sauce soja. Laisser mariner au moins 30 min.
\item Pendant ce temps, rincer et d\'ecouper le brocoli, \'emincer les oignons et le gingembre.
\item Saisir rapidement en plusieurs fois la viande dans un wok avec une cuill\`ere d'huile, r\'eserver. 
\item Faire revenir le brocoli avec les oignons et le gingembre dans le wok. Ajouter un peu de sauce soja sucr\'ee et un peu d'eau, laisser cuire 5 \`a 8 minutes.
\item Assaisonner avec de la sauce soja et les 5 \'epices, remettre la viande et bien remuer, servir aussitôt.
\end{enumerate}
\subsection*{\textsc{Conseil~:}}
Quantit\'es assez g\'en\'ereuses pour deux. Si on veut accompagner avec du riz, r\'eduire d'un quart voire un tiers.
						% <-- x1
\section[\normalsize{Boeuf saut\'e au brocoli et aux oignons}]{\LARGE{\textsc{Boeuf saut\'e au brocoli et aux oignons}}}		% <-- x2


\begin{itemize}
\item Pour 2 personnes
\item Préparation : 35 min
\item Cuisson : 5 --- 10 min
\end{itemize}

\subsection*{\textsc{Ingr\'edients~:}}

\begin{itemize}
\item 200 g de rumsteak
\item 1 petit brocoli
\item 2 gros oignons
\item sauce soja
\item sauce soja sucr\'ee
\item 1 cm de gingembre \'eminc\'e
\item 5 \'epices
\item Huile
\end{itemize}


\subsection*{\textsc{Marche \`a suivre~:}}

\begin{enumerate}
\item Emincer le boeuf, le mettre dans un plat creux et arroser de sauce soja. Laisser mariner au moins 30 min.
\item Pendant ce temps, rincer et d\'ecouper le brocoli, \'emincer les oignons et le gingembre.
\item Saisir rapidement en plusieurs fois la viande dans un wok avec une cuill\`ere d'huile, r\'eserver. 
\item Faire revenir le brocoli avec les oignons et le gingembre dans le wok. Ajouter un peu de sauce soja sucr\'ee et un peu d'eau, laisser cuire 5 \`a 8 minutes.
\item Assaisonner avec de la sauce soja et les 5 \'epices, remettre la viande et bien remuer, servir aussitôt.
\end{enumerate}
\subsection*{\textsc{Conseil~:}}
Quantit\'es assez g\'en\'ereuses pour deux. Si on veut accompagner avec du riz, r\'eduire d'un quart voire un tiers.
						% <-- x1
\section[\normalsize{Boeuf saut\'e au brocoli et aux oignons}]{\LARGE{\textsc{Boeuf saut\'e au brocoli et aux oignons}}}		% <-- x2


\begin{itemize}
\item Pour 2 personnes
\item Préparation : 35 min
\item Cuisson : 5 --- 10 min
\end{itemize}

\subsection*{\textsc{Ingr\'edients~:}}

\begin{itemize}
\item 200 g de rumsteak
\item 1 petit brocoli
\item 2 gros oignons
\item sauce soja
\item sauce soja sucr\'ee
\item 1 cm de gingembre \'eminc\'e
\item 5 \'epices
\item Huile
\end{itemize}


\subsection*{\textsc{Marche \`a suivre~:}}

\begin{enumerate}
\item Emincer le boeuf, le mettre dans un plat creux et arroser de sauce soja. Laisser mariner au moins 30 min.
\item Pendant ce temps, rincer et d\'ecouper le brocoli, \'emincer les oignons et le gingembre.
\item Saisir rapidement en plusieurs fois la viande dans un wok avec une cuill\`ere d'huile, r\'eserver. 
\item Faire revenir le brocoli avec les oignons et le gingembre dans le wok. Ajouter un peu de sauce soja sucr\'ee et un peu d'eau, laisser cuire 5 \`a 8 minutes.
\item Assaisonner avec de la sauce soja et les 5 \'epices, remettre la viande et bien remuer, servir aussitôt.
\end{enumerate}
\subsection*{\textsc{Conseil~:}}
Quantit\'es assez g\'en\'ereuses pour deux. Si on veut accompagner avec du riz, r\'eduire d'un quart voire un tiers.

%Chop Suey de porc											% <-- x1
% %Chop Suey de porc											% <-- x1
% %Chop Suey de porc											% <-- x1
% \include{./recettes/Claire/chopsueydeporc}								% <-- x1
\section[\normalsize{Chop Suey de porc}]{\LARGE{\textsc{Chop Suey de porc}}}		% <-- x2


\begin{itemize}
\item Pour 4 personnes
\item Préparation : 30 min
\item Cuisson : ?? min
\end{itemize}

\subsection*{\textsc{Ingr\'edients~:}}

\begin{itemize}
\item 300 g de filet de porc
\item 200 g d'oignons
\item 200 g de pousses de soja
\item 200 g de pousses de bambou en conserve
\item 6 champignons parfum\'es
\item Quelques brins de coriandre
\item 40 g de vermicelles de soja
\item 7 cuill\`eres \`a soupe d'huile
\item 1 cuill\`ere \`a soupe de vin chinois (ou xer\`es)
\item 2 cuill\`eres \`a soupe de sauce soja
\item sel
\end{itemize}


\subsection*{\textsc{Marche \`a suivre~:}}

\begin{enumerate}
\item Faire gonfler dans de l'eau chaude les champignons.
\item Peler et hacher les oignons, couper le porc en lani\`eres. Verser 4 cuill\`eres \`a soupe d'huile dans un wok. Quand l'huile est chaude, 
\item faire revenir les oignons 5 minutes. Augmenter le feu et ajouter la viande. Saler l\'eg\`erement et faire cuire 5 minutes en remuant. R\'eserver.
\item Nettoyer et rincer le soja, \'egoutter et rincer le bambou, rincer et ciseler la coriandre.
\item Faire cuire les vermicelles comme indiqu\'e sur le paquet. Rincer \`a l'eau froide.
\item Eponger les champignons et les couper en lamelles en \'eliminant les pieds. Verser l'huile restante dans le wok, faire chauffer et
\item mettre les vermicelles, le soja, le bambou et les champignons. Saler l\'eg\`erement, cuire 4 \`a 5 minutes en remuant sans arr\^et.
\item Ajouter la viande pour la faire r\'echauffer, puis le vin chinois et la sauce soja. Bien m\'elanger. Parsemer de coriandre et servir.
\end{enumerate}
\subsection*{\textsc{Conseil~:}}

								% <-- x1
\section[\normalsize{Chop Suey de porc}]{\LARGE{\textsc{Chop Suey de porc}}}		% <-- x2


\begin{itemize}
\item Pour 4 personnes
\item Préparation : 30 min
\item Cuisson : ?? min
\end{itemize}

\subsection*{\textsc{Ingr\'edients~:}}

\begin{itemize}
\item 300 g de filet de porc
\item 200 g d'oignons
\item 200 g de pousses de soja
\item 200 g de pousses de bambou en conserve
\item 6 champignons parfum\'es
\item Quelques brins de coriandre
\item 40 g de vermicelles de soja
\item 7 cuill\`eres \`a soupe d'huile
\item 1 cuill\`ere \`a soupe de vin chinois (ou xer\`es)
\item 2 cuill\`eres \`a soupe de sauce soja
\item sel
\end{itemize}


\subsection*{\textsc{Marche \`a suivre~:}}

\begin{enumerate}
\item Faire gonfler dans de l'eau chaude les champignons.
\item Peler et hacher les oignons, couper le porc en lani\`eres. Verser 4 cuill\`eres \`a soupe d'huile dans un wok. Quand l'huile est chaude, 
\item faire revenir les oignons 5 minutes. Augmenter le feu et ajouter la viande. Saler l\'eg\`erement et faire cuire 5 minutes en remuant. R\'eserver.
\item Nettoyer et rincer le soja, \'egoutter et rincer le bambou, rincer et ciseler la coriandre.
\item Faire cuire les vermicelles comme indiqu\'e sur le paquet. Rincer \`a l'eau froide.
\item Eponger les champignons et les couper en lamelles en \'eliminant les pieds. Verser l'huile restante dans le wok, faire chauffer et
\item mettre les vermicelles, le soja, le bambou et les champignons. Saler l\'eg\`erement, cuire 4 \`a 5 minutes en remuant sans arr\^et.
\item Ajouter la viande pour la faire r\'echauffer, puis le vin chinois et la sauce soja. Bien m\'elanger. Parsemer de coriandre et servir.
\end{enumerate}
\subsection*{\textsc{Conseil~:}}

								% <-- x1
\section[\normalsize{Chop Suey de porc}]{\LARGE{\textsc{Chop Suey de porc}}}		% <-- x2


\begin{itemize}
\item Pour 4 personnes
\item Préparation : 30 min
\item Cuisson : ?? min
\end{itemize}

\subsection*{\textsc{Ingr\'edients~:}}

\begin{itemize}
\item 300 g de filet de porc
\item 200 g d'oignons
\item 200 g de pousses de soja
\item 200 g de pousses de bambou en conserve
\item 6 champignons parfum\'es
\item Quelques brins de coriandre
\item 40 g de vermicelles de soja
\item 7 cuill\`eres \`a soupe d'huile
\item 1 cuill\`ere \`a soupe de vin chinois (ou xer\`es)
\item 2 cuill\`eres \`a soupe de sauce soja
\item sel
\end{itemize}


\subsection*{\textsc{Marche \`a suivre~:}}

\begin{enumerate}
\item Faire gonfler dans de l'eau chaude les champignons.
\item Peler et hacher les oignons, couper le porc en lani\`eres. Verser 4 cuill\`eres \`a soupe d'huile dans un wok. Quand l'huile est chaude, 
\item faire revenir les oignons 5 minutes. Augmenter le feu et ajouter la viande. Saler l\'eg\`erement et faire cuire 5 minutes en remuant. R\'eserver.
\item Nettoyer et rincer le soja, \'egoutter et rincer le bambou, rincer et ciseler la coriandre.
\item Faire cuire les vermicelles comme indiqu\'e sur le paquet. Rincer \`a l'eau froide.
\item Eponger les champignons et les couper en lamelles en \'eliminant les pieds. Verser l'huile restante dans le wok, faire chauffer et
\item mettre les vermicelles, le soja, le bambou et les champignons. Saler l\'eg\`erement, cuire 4 \`a 5 minutes en remuant sans arr\^et.
\item Ajouter la viande pour la faire r\'echauffer, puis le vin chinois et la sauce soja. Bien m\'elanger. Parsemer de coriandre et servir.
\end{enumerate}
\subsection*{\textsc{Conseil~:}}


% Rosbeef											% <-- x1
% % Rosbeef											% <-- x1
% % Rosbeef											% <-- x1
% \include{./recettes/Claire/rosbeef}								% <-- x1
\section[\normalsize{Rosbeef}]{\LARGE{\textsc{Rosbeef}}}		% <-- x2


\begin{itemize}
\item Pour 4 personnes
\item Préparation : 15 min
\item Cuisson : 20 min
\end{itemize}

\subsection*{\textsc{Ingr\'edients~:}}

\begin{itemize}
\item 1 rosbeef
\item Quelques gousses d'ail
\item 1 oignon
\item Huile d'olive
\item Sel, poivre 
\end{itemize}


\subsection*{\textsc{Marche \`a suivre~:}}

\begin{enumerate}
\item Sortir le rosbeef quelques heures \`a l'avance pour qu'il soit \`a temp\'erature ambiante.
\item Pr\'echauffer le four \`a 240° C.
\item Piquer le rosbeef avec de l'ail. Le badigeonner d'huile d'olive et poivrer. Couper un oignon en lani\`eres, les disposer au fond d'un plat et mettre le rosbeef par dessus.
\item Laisser cuire 15 \`a 20 minutes par livre de viande. D\'ecouper la viande, saler. 
\item D\'eglacer le fond du plat et ajouter le jus de d\'ecoupe pour faire une sauce. Servir.
\end{enumerate}
\subsection*{\textsc{Conseil~:}}
								% <-- x1
\section[\normalsize{Rosbeef}]{\LARGE{\textsc{Rosbeef}}}		% <-- x2


\begin{itemize}
\item Pour 4 personnes
\item Préparation : 15 min
\item Cuisson : 20 min
\end{itemize}

\subsection*{\textsc{Ingr\'edients~:}}

\begin{itemize}
\item 1 rosbeef
\item Quelques gousses d'ail
\item 1 oignon
\item Huile d'olive
\item Sel, poivre 
\end{itemize}


\subsection*{\textsc{Marche \`a suivre~:}}

\begin{enumerate}
\item Sortir le rosbeef quelques heures \`a l'avance pour qu'il soit \`a temp\'erature ambiante.
\item Pr\'echauffer le four \`a 240° C.
\item Piquer le rosbeef avec de l'ail. Le badigeonner d'huile d'olive et poivrer. Couper un oignon en lani\`eres, les disposer au fond d'un plat et mettre le rosbeef par dessus.
\item Laisser cuire 15 \`a 20 minutes par livre de viande. D\'ecouper la viande, saler. 
\item D\'eglacer le fond du plat et ajouter le jus de d\'ecoupe pour faire une sauce. Servir.
\end{enumerate}
\subsection*{\textsc{Conseil~:}}
								% <-- x1
\section[\normalsize{Rosbeef}]{\LARGE{\textsc{Rosbeef}}}		% <-- x2


\begin{itemize}
\item Pour 4 personnes
\item Préparation : 15 min
\item Cuisson : 20 min
\end{itemize}

\subsection*{\textsc{Ingr\'edients~:}}

\begin{itemize}
\item 1 rosbeef
\item Quelques gousses d'ail
\item 1 oignon
\item Huile d'olive
\item Sel, poivre 
\end{itemize}


\subsection*{\textsc{Marche \`a suivre~:}}

\begin{enumerate}
\item Sortir le rosbeef quelques heures \`a l'avance pour qu'il soit \`a temp\'erature ambiante.
\item Pr\'echauffer le four \`a 240° C.
\item Piquer le rosbeef avec de l'ail. Le badigeonner d'huile d'olive et poivrer. Couper un oignon en lani\`eres, les disposer au fond d'un plat et mettre le rosbeef par dessus.
\item Laisser cuire 15 \`a 20 minutes par livre de viande. D\'ecouper la viande, saler. 
\item D\'eglacer le fond du plat et ajouter le jus de d\'ecoupe pour faire une sauce. Servir.
\end{enumerate}
\subsection*{\textsc{Conseil~:}}

%Stir fry de porc aux nouilles											% <-- x1
% %Stir fry de porc aux nouilles											% <-- x1
% %Stir fry de porc aux nouilles											% <-- x1
% \include{./recettes/Claire/Stirfrydeporcauxnouilles}								% <-- x1
\section[\normalsize{Stir fry de porc aux nouilles}]{\LARGE{\textsc{Stir fry de porc aux nouilles}}}		% <-- x2


\begin{itemize}
\item Pour 4 personnes
\item Préparation : 35 min
\item Cuisson : 10 min
\end{itemize}

\subsection*{\textsc{Ingr\'edients~:}}

\begin{itemize}
\item 30ml d'huile
\item 500g de filet de porc en fines lamelles 
\item 250g de poireau en rondelles de 6 mm 
\item 2 gousses d'ail \'ecras\'ees 
\item 2,5 cm de gigembre pel\'e et finement h\^ach\'e 
\item 200g de nouilles Chinoises 
\item 250g de champignons Huitres ou Chinois 
\item 200g de chou chinois 
\item 40ml de sauce Soja
\item 15ml de miel liquide 
\item 10ml d'huile de s\'esame 
\end{itemize}


\subsection*{\textsc{Marche \`a suivre~:}}

\begin{enumerate}
\item Faire chauffer l'huile dans un wok ou une grande poele. 
\item Ajouter le porc, faire cuire pendant 3 minutes en remuant sans cesse. 
\item Ajouter les poireaux et cuire pendant 2 minutes. Retirer la viande et les poireaux du feu, r\'eserver.
\item Essuyer le wok avec un Sopalin. Faire chauffer le reste de l'huile. Ajouter le gingembre et l'ail et sauter 1 minute.
\item Pendant ce temps faire cuire les nouilles suivant les instructions sur le paquet.
\item Ajouter les champignons et le chou et continuer la cuisson pendant 3 minutes. 
\item Remettre le porc et les poireaux dans la poele. Remuer. 
\item Ajouter la sauce soja ou haricots noirs, le miel et 50ml d'eau. Cuire pendant 2 minutes ou jusqu'\`a ce que la sauce soit tr\`es chaude.
\item Egoutter les nouilles. Arroser de l'huile de s\'esame et bien remuer. Servir de suite avec le porc.

\end{enumerate}
\subsection*{\textsc{Conseil~:}}
								% <-- x1
\section[\normalsize{Stir fry de porc aux nouilles}]{\LARGE{\textsc{Stir fry de porc aux nouilles}}}		% <-- x2


\begin{itemize}
\item Pour 4 personnes
\item Préparation : 35 min
\item Cuisson : 10 min
\end{itemize}

\subsection*{\textsc{Ingr\'edients~:}}

\begin{itemize}
\item 30ml d'huile
\item 500g de filet de porc en fines lamelles 
\item 250g de poireau en rondelles de 6 mm 
\item 2 gousses d'ail \'ecras\'ees 
\item 2,5 cm de gigembre pel\'e et finement h\^ach\'e 
\item 200g de nouilles Chinoises 
\item 250g de champignons Huitres ou Chinois 
\item 200g de chou chinois 
\item 40ml de sauce Soja
\item 15ml de miel liquide 
\item 10ml d'huile de s\'esame 
\end{itemize}


\subsection*{\textsc{Marche \`a suivre~:}}

\begin{enumerate}
\item Faire chauffer l'huile dans un wok ou une grande poele. 
\item Ajouter le porc, faire cuire pendant 3 minutes en remuant sans cesse. 
\item Ajouter les poireaux et cuire pendant 2 minutes. Retirer la viande et les poireaux du feu, r\'eserver.
\item Essuyer le wok avec un Sopalin. Faire chauffer le reste de l'huile. Ajouter le gingembre et l'ail et sauter 1 minute.
\item Pendant ce temps faire cuire les nouilles suivant les instructions sur le paquet.
\item Ajouter les champignons et le chou et continuer la cuisson pendant 3 minutes. 
\item Remettre le porc et les poireaux dans la poele. Remuer. 
\item Ajouter la sauce soja ou haricots noirs, le miel et 50ml d'eau. Cuire pendant 2 minutes ou jusqu'\`a ce que la sauce soit tr\`es chaude.
\item Egoutter les nouilles. Arroser de l'huile de s\'esame et bien remuer. Servir de suite avec le porc.

\end{enumerate}
\subsection*{\textsc{Conseil~:}}
								% <-- x1
\section[\normalsize{Stir fry de porc aux nouilles}]{\LARGE{\textsc{Stir fry de porc aux nouilles}}}		% <-- x2


\begin{itemize}
\item Pour 4 personnes
\item Préparation : 35 min
\item Cuisson : 10 min
\end{itemize}

\subsection*{\textsc{Ingr\'edients~:}}

\begin{itemize}
\item 30ml d'huile
\item 500g de filet de porc en fines lamelles 
\item 250g de poireau en rondelles de 6 mm 
\item 2 gousses d'ail \'ecras\'ees 
\item 2,5 cm de gigembre pel\'e et finement h\^ach\'e 
\item 200g de nouilles Chinoises 
\item 250g de champignons Huitres ou Chinois 
\item 200g de chou chinois 
\item 40ml de sauce Soja
\item 15ml de miel liquide 
\item 10ml d'huile de s\'esame 
\end{itemize}


\subsection*{\textsc{Marche \`a suivre~:}}

\begin{enumerate}
\item Faire chauffer l'huile dans un wok ou une grande poele. 
\item Ajouter le porc, faire cuire pendant 3 minutes en remuant sans cesse. 
\item Ajouter les poireaux et cuire pendant 2 minutes. Retirer la viande et les poireaux du feu, r\'eserver.
\item Essuyer le wok avec un Sopalin. Faire chauffer le reste de l'huile. Ajouter le gingembre et l'ail et sauter 1 minute.
\item Pendant ce temps faire cuire les nouilles suivant les instructions sur le paquet.
\item Ajouter les champignons et le chou et continuer la cuisson pendant 3 minutes. 
\item Remettre le porc et les poireaux dans la poele. Remuer. 
\item Ajouter la sauce soja ou haricots noirs, le miel et 50ml d'eau. Cuire pendant 2 minutes ou jusqu'\`a ce que la sauce soit tr\`es chaude.
\item Egoutter les nouilles. Arroser de l'huile de s\'esame et bien remuer. Servir de suite avec le porc.

\end{enumerate}
\subsection*{\textsc{Conseil~:}}

\section{Bœuf en cocotte aux carottes}

\begin{ingredients}
\item 1 kg de viande de boeuf (joue, paleron, gîte, macreuse...)
\item 1,5 kg d carottes
\item 1 L de vin rouge
\item 4 ou 5 échalotes
\item 1 oignon
\item 1 bouquet garni (thym et laurier)
\item 4 clous de girofle
\item quelques grains de poivre
\item 2 cubes de bouillon de boeuf
\item 2 cuillères à soupe de farine
\end{ingredients}
\begin{infos}
\item Pour 6 personnes		% Nombre de personnes qu'on pourra nourrir ! :)
\item Préparation : 30 min + repos 12 h		% Temps de préparation (sans la cuisson)
\item Cuisson : 3 h			% Temps de cuisson
\end{infos}
\begin{etapes}
\item La veille, détailler la viande en gros cubes de 2cm de côté environ. Couper l'oignon en 2 et y planter les clous de girofles. Placer le tout dans un grand saladier avec le bouquet garni et les grains de poivre. Recouvrir avec le vin rouge. Placer au frais au moins 12 heures.
\item Le jour même, émincer les échalotes finement. Egoutter la viande, la mettre dans un autre récipient. L'éponger grossièrement avec du papier absorbant. 
\item Dans une cocotte en fonte faire revenir les cubes de boeuf dans un peu d'huile. Attendre quelques minutes puis avec une écumoire, retirer la viande. Récupérer le vin rendu par la viande et l'ajouter à celui de la marinade. Remettre 2 cuillères à soupe d'huile dans la cocotte et la viande. Faire cuire les morceaux de tous les côtés, saler et poivrer. Retirer de nouveau les cubes de boeuf, réserver. 
\item Ajouter une cuillère d'huile puis les échalotes dans la cocotte. Les faire rissoler jusqu'à ce qu'elles deviennent translucides puis saupoudrer avec la farine. Faire cuire ainsi quelques minutes. 
\item Mouiller avec le vin de la marinade en prenant soin de le passer à travers un chinois. Ajouter les cubes de bouillon. Remettre la viande dans la cocotte. Laisser mijoter à couvert à feu doux.
\item Peler et couper les carottes en fines rondelles. Les ajouter dans la cocotte environ une heure avant la fin de la cuisson.
\item Si besoin, lier la sauce à l'aide de maïzena.
\end{etapes}
\begin{conseils}
Ne pas hésiter à préparer ce plat à l'avance, comme tous les plats mijotés c'est encore meilleur réchauffé.
\end{conseils}

% Generated file 2018-11-25 21:43:22.247290845 +01:00
\begin{recette}{Boulettes de viande}{Boulettes de viande}

\begin{ingredients}
400 g de viande hachée\par
2 œufs (ou 3 si pas de pain)\par
2 tartines de pain\par
persil, ail\par
chapelure\par
sel, poivre\par
\end{ingredients}

\begin{infos}
Pour XX personnes\\
Préparation : XX min\\
Cuisson : XX min\\
\end{infos}

\begin{etapes}
\item Mélanger le tout et façonner des boulettes
\item Faire cuire à la poêle
\end{etapes}

\end{recette}
	\chapter{Volailles**}
\section[\normalsize{Poulet \`a la Normande}]{\LARGE{\textsc{Poulet \`a la Normande}}}



\subsection*{\textsc{Ingr\'edients~:}}
\begin{itemize}
\item 1 poulet
\item 1 bouteille de cidre doux
\item	400 cl crème fraîche
\item	calvados
\end{itemize}


\subsection*{\textsc{Marche \`a suivre~:}}
\begin{enumerate}
\item D\'ecouper le poulet en morceaux.
\item Faire revenir ces morceaux dans l’huile.
Bien les dorer.
\item Flamber au calvados (attention de ne pas se mettre sous la hotte).
\item Ajouter le cidre.
\item Laisser cuire.
\item Enlever le poulet et laisser r\'eduire la sauce jusqu’\`a ce que la consistance soit celle d’une cr\`eme.
\item Ajouter la cr\`eme fra\^iche.
\item Servir avec des pommes cuites au four. 
\end{enumerate}
\section[\normalsize{Aiguillettes de dinde au citron}]{Aiguillettes de dinde au citron}

\begin{ingredients}
\item 600 g d’escalopes de dinde \index{dinde}
\item 400 g de tagliatelles
\item 300 g d’asperges vertes \index{asperge}
\item 40 g de beurre
\item 15 cl de cr\`eme fra\^iche
\item 1 citron
\item sel, poivre moulu
\end{ingredients}
\begin{infos}
\item Pour 4 personnes
\item Préparation : 15 min
\item Cuisson : 15 min
\end{infos}
\begin{etapes}
\item Raccourcissez la tige des asperges et faite-les cuire environ 15 min dans de l’eau bouillante sal\'ee.
\item Rincez et \'epongez le citron. Pr\'elevez le zeste au couteau \'econome. Plongez-le 1 min puis d\'etaillez-le en fines lani\`eres.
\item Coupez les escalopes en aiguillettes. Faites-les dorer \`a la poêle 7 min avec 20 g de beurre chaud. Ajoutez le jus de citron, la cr\`eme fra\^iche et les lani\`eres de zestes. Salez, poivrez et poursuivez la cuisson 3 min.
\item Faites cuire les tagliatelles dans beaucoup d’eau bouillante sal\'ee, selon les indications not\'ees sur l’emballage. Égouttez-les rapidement puis m\'elangez-les avec le reste de beurre.
\item R\'epartissez les tagliatelles et les asperges sur quatre assiettes pr\'echauff\'ees. Ajouter les aiguillettes, nappez-les de sauce. D\'ecorez \'eventuellement de fines herbes et servez aussitôt.
\end{etapes}
\begin{conseils}
Le zeste du citron ne sera pas amer si vous \'evitez de pr\'elever la peau blanche qui se trouve en dessous.
\end{conseils}
% Poularde au vin jaune et aux morilles							% <-- x1
% % Poularde au vin jaune et aux morilles							% <-- x1
% % Poularde au vin jaune et aux morilles							% <-- x1
% \include{./recettes/Claire/poulardeauvinjauneetauxmorilles}				% <-- x1
\section[\normalsize{Poularde au vin jaune et aux morilles}]{\LARGE{\textsc{Poularde au vin jaune et aux morilles}}}		% <-- x2


\begin{itemize}
\item Pour 6 personnes
\item Préparation : 60 min*
\item Cuisson : 50 min
\end{itemize}

\subsection*{\textsc{Ingr\'edients~:}}

\begin{itemize}
\item 1 poularde de 2 kg coup\'ee en morceaux
\item 50 g de morielles s\'ech\'ees
\item 25 cl de vin jaune
\item 25 g de beurre
\item 50 cl de cr\`eme liquide
\item 1 cuill\`ere \`a soupe de farine
\item Sel, poivre
\end{itemize}


\subsection*{\textsc{Marche \`a suivre~:}}

\begin{enumerate}
\item Laisser tremper les morilles 30 \`a 60 minutes dans une jatte d'eau ti\`ede pour les r\'ehydrater.
\item Pendant ce temps, pr\'echauffer le four \`a 180° C. Saler, poivrer et fariner les morceaux de poularde. Les faire dorer dans une cocotte allant au four avec le beurre. 
\item Couvrir, cuire 25 min. au four.
\item Retirer la poularde de la cocotte, jeter la graisse. Verser le vin jaune, le faire bouillir et r\'eduire 3 minutes. Ajouter la cr\`eme.
\item Les morilles \'egoutt\'ees, replacer la poularde et cuire 20 minutes \`a feu doux sans couvrir. 
\item Rectifier l'assaisonnement en fin de cuisson, servir chaud. 
\end{enumerate}
\subsection*{\textsc{Conseil~:}}

Peut n\'ecessiter un temps de cuisson plus long. Pour corser la sauce, m\'elanger l'eau de trempage des morilles filtr\'ee avec la cr\`eme et verser dans le vin r\'eduit.				% <-- x1
\section[\normalsize{Poularde au vin jaune et aux morilles}]{\LARGE{\textsc{Poularde au vin jaune et aux morilles}}}		% <-- x2


\begin{itemize}
\item Pour 6 personnes
\item Préparation : 60 min*
\item Cuisson : 50 min
\end{itemize}

\subsection*{\textsc{Ingr\'edients~:}}

\begin{itemize}
\item 1 poularde de 2 kg coup\'ee en morceaux
\item 50 g de morielles s\'ech\'ees
\item 25 cl de vin jaune
\item 25 g de beurre
\item 50 cl de cr\`eme liquide
\item 1 cuill\`ere \`a soupe de farine
\item Sel, poivre
\end{itemize}


\subsection*{\textsc{Marche \`a suivre~:}}

\begin{enumerate}
\item Laisser tremper les morilles 30 \`a 60 minutes dans une jatte d'eau ti\`ede pour les r\'ehydrater.
\item Pendant ce temps, pr\'echauffer le four \`a 180° C. Saler, poivrer et fariner les morceaux de poularde. Les faire dorer dans une cocotte allant au four avec le beurre. 
\item Couvrir, cuire 25 min. au four.
\item Retirer la poularde de la cocotte, jeter la graisse. Verser le vin jaune, le faire bouillir et r\'eduire 3 minutes. Ajouter la cr\`eme.
\item Les morilles \'egoutt\'ees, replacer la poularde et cuire 20 minutes \`a feu doux sans couvrir. 
\item Rectifier l'assaisonnement en fin de cuisson, servir chaud. 
\end{enumerate}
\subsection*{\textsc{Conseil~:}}

Peut n\'ecessiter un temps de cuisson plus long. Pour corser la sauce, m\'elanger l'eau de trempage des morilles filtr\'ee avec la cr\`eme et verser dans le vin r\'eduit.				% <-- x1
\section[\normalsize{Poularde au vin jaune et aux morilles}]{\LARGE{\textsc{Poularde au vin jaune et aux morilles}}}		% <-- x2


\begin{itemize}
\item Pour 6 personnes
\item Préparation : 60 min*
\item Cuisson : 50 min
\end{itemize}

\subsection*{\textsc{Ingr\'edients~:}}

\begin{itemize}
\item 1 poularde de 2 kg coup\'ee en morceaux
\item 50 g de morielles s\'ech\'ees
\item 25 cl de vin jaune
\item 25 g de beurre
\item 50 cl de cr\`eme liquide
\item 1 cuill\`ere \`a soupe de farine
\item Sel, poivre
\end{itemize}


\subsection*{\textsc{Marche \`a suivre~:}}

\begin{enumerate}
\item Laisser tremper les morilles 30 \`a 60 minutes dans une jatte d'eau ti\`ede pour les r\'ehydrater.
\item Pendant ce temps, pr\'echauffer le four \`a 180° C. Saler, poivrer et fariner les morceaux de poularde. Les faire dorer dans une cocotte allant au four avec le beurre. 
\item Couvrir, cuire 25 min. au four.
\item Retirer la poularde de la cocotte, jeter la graisse. Verser le vin jaune, le faire bouillir et r\'eduire 3 minutes. Ajouter la cr\`eme.
\item Les morilles \'egoutt\'ees, replacer la poularde et cuire 20 minutes \`a feu doux sans couvrir. 
\item Rectifier l'assaisonnement en fin de cuisson, servir chaud. 
\end{enumerate}
\subsection*{\textsc{Conseil~:}}

Peut n\'ecessiter un temps de cuisson plus long. Pour corser la sauce, m\'elanger l'eau de trempage des morilles filtr\'ee avec la cr\`eme et verser dans le vin r\'eduit.
\section[\normalsize{Poulet à la moutarde, à l'estragon et aux champignons}]{Poulet à la moutarde, à l'estragon et aux champignons}

\begin{ingredients}
\item 2 blancs de poulet sans la peau
\item 200 g de champignons de Paris \'eminc\'es
\item 1 cube de bouillon de volaille
\item 1 verre de vin blanc
\item 2 cuill\`eres \`a caf\'e de moutarde
\item 2 cuill\`eres \`a caf\'e de cr\`eme fra\^iche all\'eg\'ee
\item 2 cuill\`eres \`a caf\'e d'estragon
\item 2 cuill\`eres \`a caf\'e d'huile d'olive
\item 2 \'echalotes \'eminc\'ee
\item sel, poivre
\end{ingredients}
\begin{infos}
\item Pour 2 personnes
\item Préparation : 40 min
\item Cuisson : 20 min
\end{infos}
\begin{etapes}
\item Faire revenir les \'echalotes dans l'huile d'olive 3 mn sans faire roussir. 
\item Ajouter les champignons et laisser cuire 2 mn. 
\item Ajouter le bouillon de volaille dissout dans 1/2 verre d'eau. Cuire 10 mn. 
\item Ajouter le vin blanc, laisser r\'eduire.
\item Faire revenir les blancs de poulet dans une po\^ele anti-adh\'esive jusqu'\`a ce qu'ils soient dor\'es. Les ajouter aux champignons et cuire 10 mn.
\item A la fin, enlever le poulet, ajouter la moutarde, la cr\`eme fra\^iche et l'estragon. Emincer les blancs de poulet en tranches et servir avec les champignons.
\end{etapes}

% Poulet à l'ananas											% <-- x1
% % Poulet à l'ananas											% <-- x1
% % Poulet à l'ananas											% <-- x1
% \include{./recettes/Claire/pouletalananas}								% <-- x1
\section[\normalsize{Poulet \`a l'ananas}]{\LARGE{\textsc{Poulet \`a l'ananas}}}		% <-- x2


\begin{itemize}
\item Pour 4 personnes
\item Préparation : 30 min
\item Cuisson : ?? min
\end{itemize}

\subsection*{\textsc{Ingr\'edients~:}}

\begin{itemize}
\item 4 ou 5 blancs de poulet
\item 1 bo\^ite d'ananas en morceaux
\item sauce soja
\item 1 cube de bouillon de volaille
\item un peu de maïzena
\item sucre en poudre
\item huile
\item \'epices (ail, gingembre, 5 \'epices) 
\end{itemize}


\subsection*{\textsc{Marche \`a suivre~:}}

\begin{enumerate}
\item Coupez les blancs de poulet en fines lamelles. 
\item Ajouter 3 cuill\`eres \`a soupe de sauce soja, le jus de l'ananas, 1 cuill\`ere de maïzena, des \'epices, laisser mariner.
\item Faire fondre le bouillon cube dans un peu d'eau.
\item Dans une po\^ele anti-adh\'esive, verser quelques cuill\`eres \`a soupe d'huile et faire chauffer. Quand celle ci est bien chaude, y faire revenir les ananas. 
\item Ajouter le bouillon afin que celui ci recouvre enti\`erement les ananas. R\'eserver.
\item Faire dorer le poulet (sans la marinade) dans la po\^ele, ajouter l'ananas et la marinade, bien m\'elanger. Sucrer la sauce en goûtant au fur et \`a mesure.
\item Laissez mijoter le tout jusqu'\`a compl\`ete cuisson. Rectifier l'assaisonnement si n\'ecessaire.
\end{enumerate}
\subsection*{\textsc{Conseil~:}}

								% <-- x1
\section[\normalsize{Poulet \`a l'ananas}]{\LARGE{\textsc{Poulet \`a l'ananas}}}		% <-- x2


\begin{itemize}
\item Pour 4 personnes
\item Préparation : 30 min
\item Cuisson : ?? min
\end{itemize}

\subsection*{\textsc{Ingr\'edients~:}}

\begin{itemize}
\item 4 ou 5 blancs de poulet
\item 1 bo\^ite d'ananas en morceaux
\item sauce soja
\item 1 cube de bouillon de volaille
\item un peu de maïzena
\item sucre en poudre
\item huile
\item \'epices (ail, gingembre, 5 \'epices) 
\end{itemize}


\subsection*{\textsc{Marche \`a suivre~:}}

\begin{enumerate}
\item Coupez les blancs de poulet en fines lamelles. 
\item Ajouter 3 cuill\`eres \`a soupe de sauce soja, le jus de l'ananas, 1 cuill\`ere de maïzena, des \'epices, laisser mariner.
\item Faire fondre le bouillon cube dans un peu d'eau.
\item Dans une po\^ele anti-adh\'esive, verser quelques cuill\`eres \`a soupe d'huile et faire chauffer. Quand celle ci est bien chaude, y faire revenir les ananas. 
\item Ajouter le bouillon afin que celui ci recouvre enti\`erement les ananas. R\'eserver.
\item Faire dorer le poulet (sans la marinade) dans la po\^ele, ajouter l'ananas et la marinade, bien m\'elanger. Sucrer la sauce en goûtant au fur et \`a mesure.
\item Laissez mijoter le tout jusqu'\`a compl\`ete cuisson. Rectifier l'assaisonnement si n\'ecessaire.
\end{enumerate}
\subsection*{\textsc{Conseil~:}}

								% <-- x1
\section[\normalsize{Poulet \`a l'ananas}]{\LARGE{\textsc{Poulet \`a l'ananas}}}		% <-- x2


\begin{itemize}
\item Pour 4 personnes
\item Préparation : 30 min
\item Cuisson : ?? min
\end{itemize}

\subsection*{\textsc{Ingr\'edients~:}}

\begin{itemize}
\item 4 ou 5 blancs de poulet
\item 1 bo\^ite d'ananas en morceaux
\item sauce soja
\item 1 cube de bouillon de volaille
\item un peu de maïzena
\item sucre en poudre
\item huile
\item \'epices (ail, gingembre, 5 \'epices) 
\end{itemize}


\subsection*{\textsc{Marche \`a suivre~:}}

\begin{enumerate}
\item Coupez les blancs de poulet en fines lamelles. 
\item Ajouter 3 cuill\`eres \`a soupe de sauce soja, le jus de l'ananas, 1 cuill\`ere de maïzena, des \'epices, laisser mariner.
\item Faire fondre le bouillon cube dans un peu d'eau.
\item Dans une po\^ele anti-adh\'esive, verser quelques cuill\`eres \`a soupe d'huile et faire chauffer. Quand celle ci est bien chaude, y faire revenir les ananas. 
\item Ajouter le bouillon afin que celui ci recouvre enti\`erement les ananas. R\'eserver.
\item Faire dorer le poulet (sans la marinade) dans la po\^ele, ajouter l'ananas et la marinade, bien m\'elanger. Sucrer la sauce en goûtant au fur et \`a mesure.
\item Laissez mijoter le tout jusqu'\`a compl\`ete cuisson. Rectifier l'assaisonnement si n\'ecessaire.
\end{enumerate}
\subsection*{\textsc{Conseil~:}}


% Poulet au curry et lait de coco											% <-- x1
% % Poulet au curry et lait de coco											% <-- x1
% % Poulet au curry et lait de coco											% <-- x1
% \include{./recettes/Claire/pouletaucurryetlaitdecoco}								% <-- x1
\section[\normalsize{Poulet au curry et lait de coco}]{\LARGE{\textsc{Poulet au curry et lait de coco}}}		% <-- x2


\begin{itemize}
\item Pour 4 personnes
\item Préparation : 45 min
\item Cuisson : ?? min
\end{itemize}

\subsection*{\textsc{Ingr\'edients~:}}

\begin{itemize}
\item 4 blancs de poulet \index{poulet}
\item 1 oignon \index{oignon}
\item 4 gousses d'ail
\item une bo\^ite de lait de coco\index{lait de coco}
\item curry en poudre
\item 1 c. \`a soupe de gingembre hach\'e (facultatif)
\item le jus d'1 citron vert
\item huile
\item sel, poivre
\end{itemize}


\subsection*{\textsc{Marche \`a suivre~:}}

\begin{enumerate}
\item Pr\'eparer les ingr\'edients : couper les blancs de poulet en lani\`eres, \'emincer l'oignon, \'eplucher les gousses d'ail (les hacher si on ne dispose pas d'un presse ail).
\item Faire chauffer un peu d'huile dans le wok. Faire revenir la viande en plusieurs fois ; elle doit \^etre saisie et presque cuite. R\'eserver. 
\item Remettre un peu d'huile et faire revenir l'oignon. Lorsqu'il est est tendre, ajouter le poulet et les gousses d'ail hach\'ees. Bien m\'elanger, saler, poivrer.
\item Ajouter environ quatre cuill\`eres \`a caf\'e de curry (en mettre selon son goût), le gingembre et m\'elanger, puis verser les 3/4 du jus de citron vert environ (ne pas tout mettre afin de pouvoir rectifier l'assaisonnement si n\'ecessaire). Lorsque la viande est cuite, ajouter le lait de coco (ne pas tout mettre non 
plus).
\item Goûter et rectifier si besoin en ajoutant du jus de citron, du lait de coco ou du curry. Laisser mijoter un peu et servir bien chaud avec du riz cuit \`a la vapeur.
\end{enumerate}
\subsection*{\textsc{Conseil~:}}

								% <-- x1
\section[\normalsize{Poulet au curry et lait de coco}]{\LARGE{\textsc{Poulet au curry et lait de coco}}}		% <-- x2


\begin{itemize}
\item Pour 4 personnes
\item Préparation : 45 min
\item Cuisson : ?? min
\end{itemize}

\subsection*{\textsc{Ingr\'edients~:}}

\begin{itemize}
\item 4 blancs de poulet \index{poulet}
\item 1 oignon \index{oignon}
\item 4 gousses d'ail
\item une bo\^ite de lait de coco\index{lait de coco}
\item curry en poudre
\item 1 c. \`a soupe de gingembre hach\'e (facultatif)
\item le jus d'1 citron vert
\item huile
\item sel, poivre
\end{itemize}


\subsection*{\textsc{Marche \`a suivre~:}}

\begin{enumerate}
\item Pr\'eparer les ingr\'edients : couper les blancs de poulet en lani\`eres, \'emincer l'oignon, \'eplucher les gousses d'ail (les hacher si on ne dispose pas d'un presse ail).
\item Faire chauffer un peu d'huile dans le wok. Faire revenir la viande en plusieurs fois ; elle doit \^etre saisie et presque cuite. R\'eserver. 
\item Remettre un peu d'huile et faire revenir l'oignon. Lorsqu'il est est tendre, ajouter le poulet et les gousses d'ail hach\'ees. Bien m\'elanger, saler, poivrer.
\item Ajouter environ quatre cuill\`eres \`a caf\'e de curry (en mettre selon son goût), le gingembre et m\'elanger, puis verser les 3/4 du jus de citron vert environ (ne pas tout mettre afin de pouvoir rectifier l'assaisonnement si n\'ecessaire). Lorsque la viande est cuite, ajouter le lait de coco (ne pas tout mettre non 
plus).
\item Goûter et rectifier si besoin en ajoutant du jus de citron, du lait de coco ou du curry. Laisser mijoter un peu et servir bien chaud avec du riz cuit \`a la vapeur.
\end{enumerate}
\subsection*{\textsc{Conseil~:}}

								% <-- x1
\section[\normalsize{Poulet au curry et lait de coco}]{\LARGE{\textsc{Poulet au curry et lait de coco}}}		% <-- x2


\begin{itemize}
\item Pour 4 personnes
\item Préparation : 45 min
\item Cuisson : ?? min
\end{itemize}

\subsection*{\textsc{Ingr\'edients~:}}

\begin{itemize}
\item 4 blancs de poulet \index{poulet}
\item 1 oignon \index{oignon}
\item 4 gousses d'ail
\item une bo\^ite de lait de coco\index{lait de coco}
\item curry en poudre
\item 1 c. \`a soupe de gingembre hach\'e (facultatif)
\item le jus d'1 citron vert
\item huile
\item sel, poivre
\end{itemize}


\subsection*{\textsc{Marche \`a suivre~:}}

\begin{enumerate}
\item Pr\'eparer les ingr\'edients : couper les blancs de poulet en lani\`eres, \'emincer l'oignon, \'eplucher les gousses d'ail (les hacher si on ne dispose pas d'un presse ail).
\item Faire chauffer un peu d'huile dans le wok. Faire revenir la viande en plusieurs fois ; elle doit \^etre saisie et presque cuite. R\'eserver. 
\item Remettre un peu d'huile et faire revenir l'oignon. Lorsqu'il est est tendre, ajouter le poulet et les gousses d'ail hach\'ees. Bien m\'elanger, saler, poivrer.
\item Ajouter environ quatre cuill\`eres \`a caf\'e de curry (en mettre selon son goût), le gingembre et m\'elanger, puis verser les 3/4 du jus de citron vert environ (ne pas tout mettre afin de pouvoir rectifier l'assaisonnement si n\'ecessaire). Lorsque la viande est cuite, ajouter le lait de coco (ne pas tout mettre non 
plus).
\item Goûter et rectifier si besoin en ajoutant du jus de citron, du lait de coco ou du curry. Laisser mijoter un peu et servir bien chaud avec du riz cuit \`a la vapeur.
\end{enumerate}
\subsection*{\textsc{Conseil~:}}


% Poulet basquaise (façon Claire T.)											% <-- x1
% % Poulet basquaise (façon Claire T.)											% <-- x1
% % Poulet basquaise (façon Claire T.)											% <-- x1
% \include{./recettes/Claire/pouletbasquaisefaconclaire}						% <-- x1
\section[\normalsize{Poulet basquaise (fa\c con Claire T.)}]{\LARGE{\textsc{(fa\c con Claire T.)}}}		% <-- x2


\begin{itemize}
\item Pour 6 personnes
\item Préparation : 35 min
\item Cuisson : 55 min
\end{itemize}

\subsection*{\textsc{Ingr\'edients~:}}

\begin{itemize}
\item 6 morceaux de poulet
\item 1 kg de tomates
\item 700 g de poivrons (verts et rouges)
\item 3 oignons \'eminc\'es
\item 3 gousses d'ail
\item 1 verre de vin blanc
\item 1 bouquet garni, 
\item huile d'olive, 
\item poivre, sel
\end{itemize}


\subsection*{\textsc{Marche \`a suivre~:}}

\begin{enumerate}
\item Dans une cocotte, faire dorer dans l'huile d'olive les morceaux de poulet sal\'es et poivr\'es. R\'eserver.
\item Faire chauffer 4 cuill\`eres \`a soupe d'huile, y faire dorer les oignons, l'ail press\'e, les poivrons taill\'es en lani\`eres. Laisser cuire 5 min.
\item Laver, \'eplucher et couper les tomates en morceaux, les ajouter \`a la cocotte, sel, poivre. Couvrir et laisser mijoter 20 min.
\item Ajouter le poulet aux l\'egumes, ajouter le bouquet garni et le vin blanc, couvrir et laisser cuire \`a feu tr\`es doux 35 min. 
\end{enumerate}
\subsection*{\textsc{Conseil~:}}
Attention \`a la cuisson, ça accroche vite...
						% <-- x1
\section[\normalsize{Poulet basquaise (fa\c con Claire T.)}]{\LARGE{\textsc{(fa\c con Claire T.)}}}		% <-- x2


\begin{itemize}
\item Pour 6 personnes
\item Préparation : 35 min
\item Cuisson : 55 min
\end{itemize}

\subsection*{\textsc{Ingr\'edients~:}}

\begin{itemize}
\item 6 morceaux de poulet
\item 1 kg de tomates
\item 700 g de poivrons (verts et rouges)
\item 3 oignons \'eminc\'es
\item 3 gousses d'ail
\item 1 verre de vin blanc
\item 1 bouquet garni, 
\item huile d'olive, 
\item poivre, sel
\end{itemize}


\subsection*{\textsc{Marche \`a suivre~:}}

\begin{enumerate}
\item Dans une cocotte, faire dorer dans l'huile d'olive les morceaux de poulet sal\'es et poivr\'es. R\'eserver.
\item Faire chauffer 4 cuill\`eres \`a soupe d'huile, y faire dorer les oignons, l'ail press\'e, les poivrons taill\'es en lani\`eres. Laisser cuire 5 min.
\item Laver, \'eplucher et couper les tomates en morceaux, les ajouter \`a la cocotte, sel, poivre. Couvrir et laisser mijoter 20 min.
\item Ajouter le poulet aux l\'egumes, ajouter le bouquet garni et le vin blanc, couvrir et laisser cuire \`a feu tr\`es doux 35 min. 
\end{enumerate}
\subsection*{\textsc{Conseil~:}}
Attention \`a la cuisson, ça accroche vite...
						% <-- x1
\section[\normalsize{Poulet basquaise (fa\c con Claire T.)}]{\LARGE{\textsc{(fa\c con Claire T.)}}}		% <-- x2


\begin{itemize}
\item Pour 6 personnes
\item Préparation : 35 min
\item Cuisson : 55 min
\end{itemize}

\subsection*{\textsc{Ingr\'edients~:}}

\begin{itemize}
\item 6 morceaux de poulet
\item 1 kg de tomates
\item 700 g de poivrons (verts et rouges)
\item 3 oignons \'eminc\'es
\item 3 gousses d'ail
\item 1 verre de vin blanc
\item 1 bouquet garni, 
\item huile d'olive, 
\item poivre, sel
\end{itemize}


\subsection*{\textsc{Marche \`a suivre~:}}

\begin{enumerate}
\item Dans une cocotte, faire dorer dans l'huile d'olive les morceaux de poulet sal\'es et poivr\'es. R\'eserver.
\item Faire chauffer 4 cuill\`eres \`a soupe d'huile, y faire dorer les oignons, l'ail press\'e, les poivrons taill\'es en lani\`eres. Laisser cuire 5 min.
\item Laver, \'eplucher et couper les tomates en morceaux, les ajouter \`a la cocotte, sel, poivre. Couvrir et laisser mijoter 20 min.
\item Ajouter le poulet aux l\'egumes, ajouter le bouquet garni et le vin blanc, couvrir et laisser cuire \`a feu tr\`es doux 35 min. 
\end{enumerate}
\subsection*{\textsc{Conseil~:}}
Attention \`a la cuisson, ça accroche vite...
	
	\chapter{Poissons}
% Papillote de saumon et julienne de légumes au boursin				% <-- x1
% % Papillote de saumon et julienne de légumes au boursin				% <-- x1
% % Papillote de saumon et julienne de légumes au boursin				% <-- x1
% \include{./recettes/Claire/papillotesaumonetjulienndelegumesauboursin}		% <-- x1
\section[\normalsize{Papillote de saumon et julienne de l\'egumes au boursin}]{\LARGE{\textsc{Papillote de saumon et julienne de l\'egumes au boursin}}}		% <-- x2


\begin{itemize}
\item Pour 2 personnes
\item Préparation : 15 min*
\item Cuisson : 15 min
\end{itemize}

\subsection*{\textsc{Ingr\'edients~:}}

\begin{itemize}
\item 2 pav\'es de saumon
\item 250 g de julienne de l\'egumes surgel\'ee
\item boursin cuisine ail et fines herbes
\end{itemize}


\subsection*{\textsc{Marche \`a suivre~:}}

\begin{enumerate}
\item Si le saumon n'est pas congel\'e, faire d\'econgeler la juliene de l\'egumes. 
\item Pr\'echauffer le four \`a 180° C.
\item D\'ecouper 2 grandes feuilles de papier sulfuris\'e. Ôter la peau du saumon. 
\item R\'epartir environ 1/4 de la julienne sur chaque feuille, d\'eposer un pav\'e de saumon par dessus, puis \'etaler un peu de boursin dessus.
\item M\'elanger 2 \`a 3 cuill\`eres \`a soupe de boursin avec le reste de juliuenne, puis la r\'epartir sur et autour du saumon.
\item Fermer les papillottes, et les faire cuire au four 15 minutes (30 si les produits sont congel\'es).
\end{enumerate}
\subsection*{\textsc{Conseil~:}}

		% <-- x1
\section[\normalsize{Papillote de saumon et julienne de l\'egumes au boursin}]{\LARGE{\textsc{Papillote de saumon et julienne de l\'egumes au boursin}}}		% <-- x2


\begin{itemize}
\item Pour 2 personnes
\item Préparation : 15 min*
\item Cuisson : 15 min
\end{itemize}

\subsection*{\textsc{Ingr\'edients~:}}

\begin{itemize}
\item 2 pav\'es de saumon
\item 250 g de julienne de l\'egumes surgel\'ee
\item boursin cuisine ail et fines herbes
\end{itemize}


\subsection*{\textsc{Marche \`a suivre~:}}

\begin{enumerate}
\item Si le saumon n'est pas congel\'e, faire d\'econgeler la juliene de l\'egumes. 
\item Pr\'echauffer le four \`a 180° C.
\item D\'ecouper 2 grandes feuilles de papier sulfuris\'e. Ôter la peau du saumon. 
\item R\'epartir environ 1/4 de la julienne sur chaque feuille, d\'eposer un pav\'e de saumon par dessus, puis \'etaler un peu de boursin dessus.
\item M\'elanger 2 \`a 3 cuill\`eres \`a soupe de boursin avec le reste de juliuenne, puis la r\'epartir sur et autour du saumon.
\item Fermer les papillottes, et les faire cuire au four 15 minutes (30 si les produits sont congel\'es).
\end{enumerate}
\subsection*{\textsc{Conseil~:}}

		% <-- x1
\section[\normalsize{Papillote de saumon et julienne de l\'egumes au boursin}]{\LARGE{\textsc{Papillote de saumon et julienne de l\'egumes au boursin}}}		% <-- x2


\begin{itemize}
\item Pour 2 personnes
\item Préparation : 15 min*
\item Cuisson : 15 min
\end{itemize}

\subsection*{\textsc{Ingr\'edients~:}}

\begin{itemize}
\item 2 pav\'es de saumon
\item 250 g de julienne de l\'egumes surgel\'ee
\item boursin cuisine ail et fines herbes
\end{itemize}


\subsection*{\textsc{Marche \`a suivre~:}}

\begin{enumerate}
\item Si le saumon n'est pas congel\'e, faire d\'econgeler la juliene de l\'egumes. 
\item Pr\'echauffer le four \`a 180° C.
\item D\'ecouper 2 grandes feuilles de papier sulfuris\'e. Ôter la peau du saumon. 
\item R\'epartir environ 1/4 de la julienne sur chaque feuille, d\'eposer un pav\'e de saumon par dessus, puis \'etaler un peu de boursin dessus.
\item M\'elanger 2 \`a 3 cuill\`eres \`a soupe de boursin avec le reste de juliuenne, puis la r\'epartir sur et autour du saumon.
\item Fermer les papillottes, et les faire cuire au four 15 minutes (30 si les produits sont congel\'es).
\end{enumerate}
\subsection*{\textsc{Conseil~:}}

	
	\chapter{Oeufs}
	\chapter{Soupes}
\section[\normalsize{Soupe au chou vert}]{Soupe au chou vert}


\begin{ingredients}
\item 1 petit chou vert
\item 4 carottes
\item 2 navets
\item 5 pommes de terre
\item 3 ou 4 saucisses fum\'ees
\item 4 cubes de Kub Or
\item Thym, laurier, Poivre
\end{ingredients}
\begin{infos}
\item Pour 6 personnes
\item Préparation : 20 min
\item Cuisson : 35 min
\end{infos}

\begin{etapes}
\item Laver, \'eplucher et couper tous les l\'egumes en morceaux.
\item Blanchir le chou 5~min.
\item Les mettre dans un autocuiseur, remplir d'eau au moins 
jusqu'\`a la moiti\'e. 
\item Ajouter les saucisses, le laurier, le thym, les cubes de 
Kub Or. Poivrer.
\item Fermer la cocotte et faire cuire 30 minutes \`a partir du 
sifflement.
\end{etapes}
\begin{conseils}
\end{conseils}

\section[\normalsize{Soupe de navets, lard croustillant et croûtons}]{Soupe de navets, lard croustillant et croûtons}

\begin{ingredients}
\item 800 g de navets (nouveaux)
\item 260 g de pommes de terre
\item 1 oignon
\item 1/2 l de bouillon de volaille
\item 10 \`a 15 cl de lait
\item 10 cl de cr\`eme fra\^iche \'epaisse
\item sel, poivre
\item 4 fines tranches de pancetta
\item 2 tranches de pain de campagne (rassis)
\item 1 noisette de beurre
\end{ingredients}
\begin{infos}
\item Pour 4 personnes
\item Préparation : 25 min
\item Cuisson : 35 min
\end{infos}
\begin{etapes}
\item Peler les navets et les pommes de terre, les couper en morceaux. 
\item Émincer l'oignon finement. 
\item Faire revenir l'oignon dans un peu de beurre, \`a feu doux, sans le faire colorer.
\item Ajouter les pommes de terre et les navets, puis mouiller avec le bouillon de volaille. Faire cuire \`a feu doux et \`a couvert pendant 30 \`a 35 minutes, jusqu'\`a ce que les l\'egumes soient tendres.
\item Mixer le tout, ajouter le lait (la quantit\'e peut varier en fonction de la consistance d\'esir\'ee), la cr\`eme fra\^iche, rectifier en sel et poivre, r\'eserver au chaud.
\item D\'ecouper le pain de campagne en petits d\'es et la pancetta en morceaux. Faire chauffer une po\^ele, y d\'eposer le pain (sans huile), laisser dess\'echer 3 minutes \`a feu vif, en remuant constamment. 
\item Ajouter alors la pancetta, la faire dorer \`a feu vif pour qu'elle devienne bien croustillante, tout en continuant \`a remuer pour que les croûtons s'impr\`egnent de la graisse de cuisson.
\item Servir la soupe dans des petits bols ou des assiettes creuses et r\'epartir les croûtons et la pancetta sur le dessus.
\end{etapes}
\begin{conseils}
\end{conseils}

\section[\normalsize{Velout\'e de courgettes au boursin}]{Velouté de courgettes au boursin}

\begin{ingredients}
\item 6 courgettes
\item Boursin cuisine ail et fines herbes
\item 2 bouillon cubes de volaille
\end{ingredients}
\begin{infos}
\item Pour 6 personnes
\item Préparation : 15 min
\item Cuisson : 5 min
\end{infos}
\begin{etapes}
\item Laver les courgettes, les couper grossi\`erement en 
tronçons et les mettre dans la cocotte. 
\item Recouvrir tout juste d’eau, ajouter les bouillon cubes. 
\item Laisser cuire 5 minutes \`a partir du sifflement de la 
soupape.
\item Enlever un peu d'eau de cuisson, r\'eserver, et passer les 
courgettes au mixeur \`a soupe tout en rajoutant 2 belles cuill
\`eres \`a soupe de boursin. 
\item Rectifier l'assaisonnement et/ou rajouter du liquide si 
besoin est.
\end{etapes}
\begin{conseils}
\end{conseils}


% Nom de la recette à entrer entre les accolades {}
\section{Carabaccia}

\begin{ingredients}
\item 1 kg d'oignons (rouges et jaunes mélangés)
\item 400 g de céleri branche
\item 500 g de carottes
\item 500 g de petits pois surgelés
\item 3/4 de verre d'huile d'olive
\item 500 ml d'eau
\item origan
\item sel, poivre
\item parmesan
\item par personne : 1 oeuf, 1 tranche de pain type campagne épaisse
\end{ingredients}
\begin{infos}
\item Pour 6 personnes		% Nombre de personnes qu'on pourra nourrir ! :)
\item Préparation : 60 min		% Temps de préparation (sans la cuisson)
\item Cuisson : 60 min			% Temps de cuisson
\end{infos}
\begin{etapes}
\item Faire cuire les petits pois surgelés et réserver. Mixer la moitié des petits pois pour les réduire en purée.
\item Peler les oignons et les carottes. Couper oignons, carottes et céleri en petits dés.
\item Dans une grande marmite, verser les oignons et l’huile d’olive. Faire revenir pendant quelques minutes, puis ajouter  la purée de petits pois, les carottes et le céleri.
\item Ajouter l’eau, de l’origan, et laisser mijoter pendant une heure environ à feu doux.
\item Ajouter le reste de petits pois pour les réchauffer. Placer une tranche de pain dans chaque assiette.
\item Pocher les oeufs.
\item Verser du bouillon sur le pain et couvrir de légumes. Ajouter l'oeuf. Saler, poivrer, et parsemer de parmesan râpé. Déguster sans attendre.
\end{etapes}
\begin{conseils}
\end{conseils}

% Nom de la recette à entrer entre les accolades {}
\section{Velouté de topinambours au chèvre frais}

\begin{ingredients}
\item 800 g de topinambours
\item 1 petit oignon
\item bouillon de légumes
\item 200 ml de lait
\item beurre
\item sel, poivre
\item 50 g de chèvre frais
\item crème fraîche (selon l'envie)
\item 1 bonne poignée de noisettes
\end{ingredients}
\begin{infos}
\item Pour 4 personnes		% Nombre de personnes qu'on pourra nourrir ! :)
\item Préparation : 40 min		% Temps de préparation (sans la cuisson)
\item Cuisson : 15 min			% Temps de cuisson
\end{infos}
\begin{etapes}
\item Laver et peler les topinambours, les couper en petits morceaux. Émincer l'oignon.
\item Dans une casserole, faire chauffer une noix de beurre. Y faire revenir l'oignon et le topinambour pendant 1 à 2 minutes sur feux doux. Mouiller avec le bouillon de légumes (recouvrir) et laisser mijoter pendant environ 15 minutes sur feux doux, pour que le topinambour soit cuit (piquer au couteau, il ne doit pas rester sur la lame). Rajouter un peu de bouillon pendant la cuisson si besoin.
\item Pendant ce temps, concasser les noisettes et les torréfier à sec dans une poêle.
\item Réserver le bouillon. Mixer le topinambour, ajouter le lait, puis progressivement un peu de bouillon jusqu'à obtenir la consistance désirée. Ajouter le chèvre, la crème et bien mixer. Rectifier l'assaisonnement au besoin.
\item Servir parsemé de noisettes concassées avec éventuellement un filet d'huile de noisette.
\end{etapes}
\begin{conseils}
Ne pas hésiter à préparer ce plat à l'avance, comme tous les plats mijotés c'est encore meilleur réchauffé.
\end{conseils}





% Nom de la recette à entrer entre les accolades {}
\section{Soupe de potiron}

\begin{ingredients}
% Ici lister les ingrédients 
% Changer de ligne pour chaque ingrédient et commencer la ligne par : \item
% rajouter autant de ligne que d'ingrédient
\item 1,5 kg de potiron
\item 2 grosses pommes de terre (bintje)
\item 1 blanc de poireau
\item 1 oignon
\item 1,2 de bouillon de volaille (2 tablettes de concentré)
\item 25 g de beurre
\item 200 g de crème
\item 1 pincée de noix de muscade râpée
\item 1 morceau de sucre
\item 1 bouquet garni
\item 10 brins de ciboulette
\item sel
\end{ingredients}
\begin{infos}
% Informations génériques
% Changer de ligne pour chaque et commencer par : \item
% Mettre une * si l'information n'est pas certaine 
\item Pour 6 personnes		% Nombre de personnes qu'on pourra nourrir ! :)
\item Préparation : 20 min		% Temps de préparation (sans la cuisson)
\item Cuisson : 40 min			% Temps de cuisson
\end{infos}
\begin{etapes}
% Ici les étapes à réaliser
% Une étape par ligne, chaque ligne commence par un \item
% Pour exemple les étapes pour faire un millas ;)
\item Éplucher et épépiner le quartier de potiron. Couper la chair en morceaux. Fendre le blanc de poireau.Le rincer et l'émincer finement. Éplucher les pommes de terre et les couper en gros dés. Peler et hacher l’oignon.  
\item Chauffer le beurre dans un faitout. Faites-y revenir l’oignon et le poireau 5 minutes à feu très doux sans laisser colorer. Mouiller avec le bouillon. Ajouter le potiron, les pommes de terre, le bouquet garni et le sucre. Saler légèrement. Laisser cuire 30 minutes.
\item Retirer  alors le bouquet garni. Passez la soupe au robot-mixer ou au moulin à légumes (grille fine).
\item Reverser la soupe dans le faitout propre. Rectifier l’assaisonnement , parfumer de noix de muscade râpée et porter à nouveau à ébullition.
\item Au moment de servir ajouter la crème et la ciboulette ciselée.
\end{etapes}
\begin{conseils}
% Ici écrire les conseils concernant la recette 
\end{conseils}

% Generated file 2019-02-17 16:33:30.818870213 +01:00
\begin{recette}{Soupe au lard et aux poids cassés}{Soupe au lard et aux poids cassés}

\begin{ingredients}
175 g de pois cassés\par
1 branche de thym\par
100 g de lard fumé\par
1 l d'eau\par
1 verre de lait\par
20 g de margarine\par
sel, poivre\par
\end{ingredients}

\begin{infos}
Pour XX personnes\\
Préparation : XX min\\
Cuisson : 45 min\\
\end{infos}

\begin{etapes}
\item Faire fondre la margarine dans la cocotte-mine. Ajouter les lardon
\item Dès que ceux-ci sont devenus transparents, ajouter les pois cassés, l'eau, une bonne pincée de poivre, un peu de sel (attention les lardons sont déjà salés) et le thym.
\item Couvrir et laisser cuire doucement 25 min à partir du moment ou la soupape chuchote.
\item Laisser échapper la pression, batter le potage avec un fouet tout en ajoutant le lait petit à petit. Il est également possible de le passer au mixer en ayant au préalable retiré les lardons.
\item chauffer à nouveau 2 min et servir
\end{etapes}

\end{recette}
% Generated file 2018-11-25 21:43:22.235044316 +01:00
\begin{recette}{Soupe de légumes aux lentilles corail}{Soupe de légumes aux lentilles corail}

\begin{ingredients}
2 pommes de terre\par
3 petites carottes\par
1 poireau ou 1 gros oignon\par
1 poivron rouge ou vert\par
3 dl de lentilles corail\par
470g de tomates pelées et leur jus (une boîte)\par
1 L d’eau\par
1 bouillon cube aux légumes\par
1 cuillerée à café de cumin ou de curry\par
sel\par
1 feuille de laurier\par
\end{ingredients}

\begin{infos}
Pour 4 personnes\\
Préparation : 40 min\\
Cuisson : 30 min\\
\end{infos}

\begin{etapes}
\item Éplucher et couper les légumes en dés. Mettre dans une marmite avec le reste des ingrédients (ajouter le sel en fin de cuisson).
\item Laisser cuire pendant une bonne demi-heure après ébullition, ou jusqu’à ce que les légumes soient tendres.
\item Servir bien chaud, avec éventuellement un peu de crème fraîche.
\end{etapes}

\end{recette}
	\chapter{Pains}
% Generated file 2018-11-25 21:43:22.298520339 +01:00
\begin{recette}{Petits pains aux lardons}{Petits pains aux lardons}

\begin{ingredients}
500 g de farine\par
1/2 cuillère à café de sel\par
25 g de levure de boulanger\par
25 cl de lait tiède\par
50 g de beurre\par
1 oeuf\par
lardons allumettes\par
\end{ingredients}

\begin{infos}
Pour XX personnes\\
Préparation : XX min*	\\
Cuisson : 15 -- 20 min\\
\end{infos}

\begin{etapes}
\item Mélanger la farine, la levure et le sel
\item Ajouter le reste des ingrédients et pétrir
\item Laisser reposer (combien de temps ?)
\item Façonner les petits pains en faisant des boules de 40 g de pâte environ.
\item Cuire à 200 °C pendant 15 à 20 minute
\end{etapes}

\begin{conseils}
Pour faire le grand raisin, prendre 1 kg de farine pour 1 cube de levure et 2 boîtes de lardon.
\end{conseils}

\end{recette}
	\chapter{Gratins}
	\chapter{Plat unique ? (couscous choucroute)}
\section[\normalsize{Tartiflette (façon Claire T.)}]{Tartiflette (façon Claire T.)}

\begin{ingredients}
\item 1 kg de pommes de terre \`a chair ferme
\item 1 reblochon fermier ou fruit\'e
\item 200g de lardons
\item 1 gros oignon
\item 2 cuill\`eres \`a soupe de cr\`eme fra\^iche
\item vin blanc de Savoie (Apremont)
\item 1 gousse d'ail
\end{ingredients}
\begin{infos}
\item Pour 6 personnes
\item Préparation : 50 min
\item Cuisson : 10 min
\end{infos}
\begin{etapes}
\item Faire cuire les pommes de terre dans de l'eau (d\'epart \`a froid) pendant 20 \`a 25 minutes. Egoutter et \'eplucher.
\item Pr\'echauffer le four \`a 200° C.
\item Emincer finement l'oignon. Faire fondre les lardons dans une po\^ele, puis ajouter les oignons (avant que les lardons ne colorent). 
\item Laisser revenir \`a feu doux.
\item Pendant ce temps, couper les pommes de terre en rondelles dans un saladier. Frotter un plat \`a gratin avec une gousse d'ail et le beurrer g\'en\'ereusement. 
\item Gratter le reblochon des deux côt\'es, ôter la pastille de cas\'eine et le couper en deux dans l'\'epaisseur.
\item Quand les lardons et les oignons ont bien fondu (attention \`a ne pas faire trop colorer), ajouter un peu de vin blanc et faire r\'eduire. 
\item Les ajouter aux pommes de terre. Ajouter la cr\`eme et m\'elanger.
\item Verser ce m\'elange dans le plat \`a gratin, arroser de vin blanc et poser les deux moiti\'es de reblochon par dessus, croûte vers le haut.
\item Enfourner une dizaine de minutes, jusqu'\`a ce que le fromage ait bien fondue.
\end{etapes}
\begin{conseils}
\end{conseils}

\section[\normalsize{Chili con carne}]{Chili con carne}

\begin{ingredients}
\item 1 oignon
\item 1 poivron rouge
\item 500 g de viande hach\'ee
\item 3 bo\^ites de 400 g de haricots rouges
\item 2 bo\^ites de 400 g de tomates en d\'es
\item \'epices mexicaines
\item cumin
\item huile d'olive
\end{ingredients}
\begin{infos}
\item Pour 6 personnes
\item Préparation : 30 min
\item Cuisson : 30 min
\end{infos}
\begin{etapes}
\item Faire revenir l'oignon et le poivron \'eminc\'es dans un peu d'huile. 
\item Ajouter la viande hach\'ee et faire dorer.
\item Ajouter les haricots rouges \'egoutt\'es, le m\'elange d'\'epices et le cumin. Laisser cuire quelques minutes.
\item Ajouter les tomates, bien m\'elanger et laisser revenir \`a feu doux 30 minutes. Rectifier l'assaisonnement si n\'ecessaire.
\end{etapes}
\begin{conseils}
\end{conseils}


% Hamburgers											% <-- x1
% % Hamburgers											% <-- x1
% % Hamburgers											% <-- x1
% \include{./recettes/Claire/hamburgers}								% <-- x1
\section[\normalsize{Hamburgers}]{\LARGE{\textsc{Hamburgers}}}		% <-- x2


\begin{itemize}
\item Pour 4 personnes
\item Préparation : 1h + 30 min
\item Cuisson : 10 min
\end{itemize}

\subsection*{\textsc{Ingr\'edients~:}}

\begin{itemize}
\item 4 grands pains \`a hamburgers
\item 4 steaks hach\'es (500 g de viande hach\'ee)
\item 1 oeuf
\item 1 cuill\`ere \`a soupe de chapelure
\item 1 oignon frais
\item 2 cuill\`eres \`a soupe de persil
\item 1 cuill\`ere \`a soupe de parmesan
\item Sel, poivre
\item 4 feuilles de salade
\item 1 ou 2 tomates
\item 1 oignon rouge
\item 8 tranches de cheddar
\end{itemize}
Sauce :
\begin{itemize}
\item 1 yaourt nature
\item 2 cuill\`eres \`a soupe de ketchup
\item 1 \`a 2 cuill\`eres \`a soupe de moutarde
\end{itemize}


\subsection*{\textsc{Marche \`a suivre~:}}

\begin{enumerate}
\item Mettre les steaks dans un saladier, les \'ecraser rapidement \`a la fourchette. Ajouter la chapelure, l'oeuf, l'oignon et le persil hach\'es finement, le parmesan, sel et poivre, et m\'elanger. 
\item Façonner 4 boules un peu aplaties, les rouler dans un peu de farine et les d\'eposer dans une assiette. Filmer et laisser reposer au frais 1h.
\item Laver la salade et les tomates, \'eplucher l'oignon, pr\'eparer la sauce. Aplatir les steaks \`a un diam\`etre \`a peine inf\'erieur \`a celui des pains et les faire cuire \`a feu vif.
\item Pr\'echauffer le four \`a 230° C. 
\item Trancher les tomates, couper la salade en larges lani\`eres, couper l'oignon en rondelles. 
\item Toaster l\'eg\`erement la moiti\'e inf\'erieure des pains et les d\'eposer sur une plaque de cuisson. Mettre une cuill\`ere de sauce, un peu de salade, 2 rondelles de tomates et une d'oignon et une tranche de cheddar sur chaque moiti\'e de pain. 
\item D\'eposer un steaks sur la moiti\'e inf\'erieure, et recouvrir avec l'autre moiti\'e. Presser un peu les hamburgers, les piquer avec une brochette et enfourner quelques minutes le temps que le fromage fonde.
\end{enumerate}
\subsection*{\textsc{Conseil~:}}
Possibilit\'e de rajouter un peu de coriandre cisel\'ee dans la viande, ou encore 1/4 de cuill\`ere \`a caf\'e de coriandre moulue et 1/4 de cuill\`ere \`a caf\'e de cumin.

								% <-- x1
\section[\normalsize{Hamburgers}]{\LARGE{\textsc{Hamburgers}}}		% <-- x2


\begin{itemize}
\item Pour 4 personnes
\item Préparation : 1h + 30 min
\item Cuisson : 10 min
\end{itemize}

\subsection*{\textsc{Ingr\'edients~:}}

\begin{itemize}
\item 4 grands pains \`a hamburgers
\item 4 steaks hach\'es (500 g de viande hach\'ee)
\item 1 oeuf
\item 1 cuill\`ere \`a soupe de chapelure
\item 1 oignon frais
\item 2 cuill\`eres \`a soupe de persil
\item 1 cuill\`ere \`a soupe de parmesan
\item Sel, poivre
\item 4 feuilles de salade
\item 1 ou 2 tomates
\item 1 oignon rouge
\item 8 tranches de cheddar
\end{itemize}
Sauce :
\begin{itemize}
\item 1 yaourt nature
\item 2 cuill\`eres \`a soupe de ketchup
\item 1 \`a 2 cuill\`eres \`a soupe de moutarde
\end{itemize}


\subsection*{\textsc{Marche \`a suivre~:}}

\begin{enumerate}
\item Mettre les steaks dans un saladier, les \'ecraser rapidement \`a la fourchette. Ajouter la chapelure, l'oeuf, l'oignon et le persil hach\'es finement, le parmesan, sel et poivre, et m\'elanger. 
\item Façonner 4 boules un peu aplaties, les rouler dans un peu de farine et les d\'eposer dans une assiette. Filmer et laisser reposer au frais 1h.
\item Laver la salade et les tomates, \'eplucher l'oignon, pr\'eparer la sauce. Aplatir les steaks \`a un diam\`etre \`a peine inf\'erieur \`a celui des pains et les faire cuire \`a feu vif.
\item Pr\'echauffer le four \`a 230° C. 
\item Trancher les tomates, couper la salade en larges lani\`eres, couper l'oignon en rondelles. 
\item Toaster l\'eg\`erement la moiti\'e inf\'erieure des pains et les d\'eposer sur une plaque de cuisson. Mettre une cuill\`ere de sauce, un peu de salade, 2 rondelles de tomates et une d'oignon et une tranche de cheddar sur chaque moiti\'e de pain. 
\item D\'eposer un steaks sur la moiti\'e inf\'erieure, et recouvrir avec l'autre moiti\'e. Presser un peu les hamburgers, les piquer avec une brochette et enfourner quelques minutes le temps que le fromage fonde.
\end{enumerate}
\subsection*{\textsc{Conseil~:}}
Possibilit\'e de rajouter un peu de coriandre cisel\'ee dans la viande, ou encore 1/4 de cuill\`ere \`a caf\'e de coriandre moulue et 1/4 de cuill\`ere \`a caf\'e de cumin.

								% <-- x1
\section[\normalsize{Hamburgers}]{\LARGE{\textsc{Hamburgers}}}		% <-- x2


\begin{itemize}
\item Pour 4 personnes
\item Préparation : 1h + 30 min
\item Cuisson : 10 min
\end{itemize}

\subsection*{\textsc{Ingr\'edients~:}}

\begin{itemize}
\item 4 grands pains \`a hamburgers
\item 4 steaks hach\'es (500 g de viande hach\'ee)
\item 1 oeuf
\item 1 cuill\`ere \`a soupe de chapelure
\item 1 oignon frais
\item 2 cuill\`eres \`a soupe de persil
\item 1 cuill\`ere \`a soupe de parmesan
\item Sel, poivre
\item 4 feuilles de salade
\item 1 ou 2 tomates
\item 1 oignon rouge
\item 8 tranches de cheddar
\end{itemize}
Sauce :
\begin{itemize}
\item 1 yaourt nature
\item 2 cuill\`eres \`a soupe de ketchup
\item 1 \`a 2 cuill\`eres \`a soupe de moutarde
\end{itemize}


\subsection*{\textsc{Marche \`a suivre~:}}

\begin{enumerate}
\item Mettre les steaks dans un saladier, les \'ecraser rapidement \`a la fourchette. Ajouter la chapelure, l'oeuf, l'oignon et le persil hach\'es finement, le parmesan, sel et poivre, et m\'elanger. 
\item Façonner 4 boules un peu aplaties, les rouler dans un peu de farine et les d\'eposer dans une assiette. Filmer et laisser reposer au frais 1h.
\item Laver la salade et les tomates, \'eplucher l'oignon, pr\'eparer la sauce. Aplatir les steaks \`a un diam\`etre \`a peine inf\'erieur \`a celui des pains et les faire cuire \`a feu vif.
\item Pr\'echauffer le four \`a 230° C. 
\item Trancher les tomates, couper la salade en larges lani\`eres, couper l'oignon en rondelles. 
\item Toaster l\'eg\`erement la moiti\'e inf\'erieure des pains et les d\'eposer sur une plaque de cuisson. Mettre une cuill\`ere de sauce, un peu de salade, 2 rondelles de tomates et une d'oignon et une tranche de cheddar sur chaque moiti\'e de pain. 
\item D\'eposer un steaks sur la moiti\'e inf\'erieure, et recouvrir avec l'autre moiti\'e. Presser un peu les hamburgers, les piquer avec une brochette et enfourner quelques minutes le temps que le fromage fonde.
\end{enumerate}
\subsection*{\textsc{Conseil~:}}
Possibilit\'e de rajouter un peu de coriandre cisel\'ee dans la viande, ou encore 1/4 de cuill\`ere \`a caf\'e de coriandre moulue et 1/4 de cuill\`ere \`a caf\'e de cumin.


%Riz aux saucisses											% <-- x1
% %Riz aux saucisses											% <-- x1
% %Riz aux saucisses											% <-- x1
% \include{./recettes/Claire/rizauxsaucisses}								% <-- x1
\section[\normalsize{Riz aux saucisses}]{\LARGE{\textsc{Riz aux saucisses}}}		% <-- x2


\begin{itemize}
\item Pour 3 personnes
\item Préparation : 15 min
\item Cuisson : 15 min
\end{itemize}

\subsection*{\textsc{Ingr\'edients~:}}

\begin{itemize}
\item 1 bel oignon
\item 2 saucisses de toulouse
\item 3 doses de riz
\item 2 doses d'eau
\item 4 doses de vin rouge
\item fromage rap\'e ou parmesan
\item sel, poivre
\end{itemize}


\subsection*{\textsc{Marche \`a suivre~:}}

\begin{enumerate}
\item Faire revenir l'oignon \'eminc\'e dans un peu d'huile sans faire colorer. 
\item Ajouter les saucisses coup\'ees en rondelles et faire dorer. \item Verser le riz, remuer jusqu'\`a ce que les grains commencent \`a devenir translucides. 
\item Ajouter l'eau et le vin, saler, poivrer et couvrir. 
\item Laisser cuire jusqu'\`a ce que le liquide soit pratiquement absorb\'e (10 \`a 15 minutes). 
\item Ajouter le fromage rap\'e, m\'elanger et finir la cuisson encore quelques minutes.
\end{enumerate}
\subsection*{\textsc{Conseil~:}}
								% <-- x1
\section[\normalsize{Riz aux saucisses}]{\LARGE{\textsc{Riz aux saucisses}}}		% <-- x2


\begin{itemize}
\item Pour 3 personnes
\item Préparation : 15 min
\item Cuisson : 15 min
\end{itemize}

\subsection*{\textsc{Ingr\'edients~:}}

\begin{itemize}
\item 1 bel oignon
\item 2 saucisses de toulouse
\item 3 doses de riz
\item 2 doses d'eau
\item 4 doses de vin rouge
\item fromage rap\'e ou parmesan
\item sel, poivre
\end{itemize}


\subsection*{\textsc{Marche \`a suivre~:}}

\begin{enumerate}
\item Faire revenir l'oignon \'eminc\'e dans un peu d'huile sans faire colorer. 
\item Ajouter les saucisses coup\'ees en rondelles et faire dorer. \item Verser le riz, remuer jusqu'\`a ce que les grains commencent \`a devenir translucides. 
\item Ajouter l'eau et le vin, saler, poivrer et couvrir. 
\item Laisser cuire jusqu'\`a ce que le liquide soit pratiquement absorb\'e (10 \`a 15 minutes). 
\item Ajouter le fromage rap\'e, m\'elanger et finir la cuisson encore quelques minutes.
\end{enumerate}
\subsection*{\textsc{Conseil~:}}
								% <-- x1
\section[\normalsize{Riz aux saucisses}]{\LARGE{\textsc{Riz aux saucisses}}}		% <-- x2


\begin{itemize}
\item Pour 3 personnes
\item Préparation : 15 min
\item Cuisson : 15 min
\end{itemize}

\subsection*{\textsc{Ingr\'edients~:}}

\begin{itemize}
\item 1 bel oignon
\item 2 saucisses de toulouse
\item 3 doses de riz
\item 2 doses d'eau
\item 4 doses de vin rouge
\item fromage rap\'e ou parmesan
\item sel, poivre
\end{itemize}


\subsection*{\textsc{Marche \`a suivre~:}}

\begin{enumerate}
\item Faire revenir l'oignon \'eminc\'e dans un peu d'huile sans faire colorer. 
\item Ajouter les saucisses coup\'ees en rondelles et faire dorer. \item Verser le riz, remuer jusqu'\`a ce que les grains commencent \`a devenir translucides. 
\item Ajouter l'eau et le vin, saler, poivrer et couvrir. 
\item Laisser cuire jusqu'\`a ce que le liquide soit pratiquement absorb\'e (10 \`a 15 minutes). 
\item Ajouter le fromage rap\'e, m\'elanger et finir la cuisson encore quelques minutes.
\end{enumerate}
\subsection*{\textsc{Conseil~:}}
		

\part{Les desserts}
	\chapter{Tartes}
\section[\normalsize{Tarte mousseuse aux framboises}]{Tarte mousseuse aux framboises}

\begin{ingredients}
\item Pour la p\^ate :
\begin{itemize}
\item	250 g de farine
\item	10 g de levure de boulanger
\item	1 oeuf
\item	40 g de sucre
\item	10 cl de lait
\item	un quart de cuil. à caf\'e de sel
\item	100 g de beurre ramolli
\end{itemize}
\item Pour la garniture :
\begin{itemize}
\item	300 g de framboises \index{framboises}
\item	100 g de sucre
\item	3 oeufs
\item	1 cuil. à soupe de sucre glace
\item	125 g d’amandes en poudre
\end{itemize}
\end{ingredients}
\begin{infos}
\item Pour 6 personnes
\item Préparation : 45 min
\item Repos : 1 h
\item Cuisson : 40 min
\end{infos}
\begin{etapes}
\item  Dans une terrine, d\'elayez la levure dans 5 cl de lait ti\`ede et une pinc\'ee de sucre. Ajoutez 60 g de farine et m\'elangez pour obtenir une sorte de bouillie. Couvrez d’un linge et laissez lever 15 min dans un endroit ti\`ede.
\item  Versez le reste de farine sur le levain, ajoutez le reste de lait ti\`ede, le sel, le sucre et l’oeuf. P\'etrissez la p\^ate à la main ou au robot, jusqu’à ce qu’elle soit homog\`ene. Incorporez le beurre et travaillez jusqu’à ce que la p\^ate se d\'etache et ne colle plus aux mains (ou 5 min au robot).
\item Laissez reposer sous un linge 45 min dans un endroit ti\`ede. La p\^ate doit doubler de volume.
\item  P\'etrissez la p\^ate 30 secondes pour la faire retomber. Beurrez un moule rectangulaire et garnissez-le avec la p\^ate, en appuyant pour faire adh\'erer.
\item  Allumez le four th. 7 (210° C). Laissez la p\^ate au ti\`ede pendant que vous pr\'eparez la garniture.
\item  S\'eparez les blancs des jaunes d’oeufs. Fouettez les jaunes avec 50 g de sucre jusqu’à ce que le m\'elange blanchisse. Ajoutez la poudre d’amandes, m\'elangez. Battez les blancs en neige ferme avec le reste du sucre et incorporez d\'elicatement cette mousse dans la masse pr\'ec\'edente.
\end{etapes}
\begin{etapes}
\item  Etalez cette pr\'eparation sur le fond de tarte, puis r\'epartissez les framboises à la surface. Elles vont s’enfoncer dans la cr\`eme au cours de la cuisson. Enfournez la tarte à mi-hauteur et faites cuire pendant environ 40 min.
\item  Laissez refroidir 5 à 10 min dans le moule, pus d\'emoulez la tarte sur une grille. Poudrez de sucre glace avant de d\'eguster. 
\end{etapes}
\begin{conseils}
\end{conseils}



\section[\normalsize{Tarte aux pommes alsacienne (façon Claire T.)}]{Tarte aux pommes alsacienne (façon Claire T.)}


\begin{ingredients}
\item 1 p\^ate bris\'ee
\item 5 ou 6 belles pommes 
\item Pour la garniture :
\begin{itemize}
\item 2 jaunes d'oeufs
\item 1 petit pot de cr\`eme
\item Environ 75g de sucre
\item 1 c. \`a caf\'e de cannelle
\end{itemize}
\end{ingredients}
\begin{infos}
\item Pour 6 personnes
\item Préparation : 45 min
\item Cuisson : ?? min
\end{infos}
\begin{etapes}
\item \'etaler la p\^ate et la disposer dans un moule beurr\'e. Couper les pommes en tranches pas trop fines, les disposer en cercles concentriques.
\item Pr\'eparer la garniture : dans un saladier, battre les oeufs \`a la fourchette puis ajouter la cr\`eme, le sucre et la cannelle. Bien m\'elanger et verser sur les pommes.
\item Mettre au four.
\end{etapes}
\begin{conseils}
\end{conseils}

	\chapter{G\^ateaux}
% Generated file 2018-11-25 21:43:22.244903399 +01:00
\begin{recette}{Amandines à la poire}{Amandines à la poire}

\begin{ingredients}
2 poires William ou des poires au sirop \index{poire}\par
100 g chocolat noir \index{chocolat}\par
1 boite de 375 g de lait concentré sucré\par
4 oeufs\par
60 g d’amandes en poudre \index{amandes}\par
\end{ingredients}

\begin{infos}
Pour 6 personnes\\
Préparation : 15 min\\
Cuisson : 25 min\\
\end{infos}

\begin{etapes}
\item Préchauffer le four th. 5 180° C.
\item Laver et peler les poires.
\item À l’aide d’un couteau, hacher le chocolat en pépites.
\item Dans un saladier, mélanger le lait concentré sucré, les œufs, la poudre d’amandes et les pépites de chocolat.
\item Mixer les poires et ajouter-les à la préparation.
\item Répartir la préparation dans des ramequins.
\item Faites cuire 25 min.
\end{etapes}

\begin{conseils}
\end{conseils}

\end{recette}
\section[\normalsize{Brownies (façon Claire T.)}]{Brownies (façon Claire T.)}
\choco{Brownies (façon Claire T.)}

\begin{ingredients}
\item 2 oeufs
\item 50 g de sucre
\item 50 g de farine
\item 150 g de chocolat
\item 50 g de beurre
\item noix (env. 50 g)
\end{ingredients}
\begin{infos}
\item Pour 6 personnes
\item Préparation : 15 min
\item Cuisson : 15 min
\end{infos}
\begin{etapes}
\item Pr\'echauffer le four \`a 200° C.
\item Dans une terrine, battre les oeufs et le sucre. 
\item Ajouter la farine, m\'elanger.
\item Faire fondre le chocolat avec le beurre. 
\item Ajouter \`a la pr\'eparation et bien m\'elanger.
\item Ajouter les noix en morceaux.
\item Faire cuire 15 \`a 20 minutes.
\end{etapes}
\begin{conseils}
En divisant les quantit\'es par deux on obtient 6 mini brownies (dans des moules \`a muffins).
\end{conseils}
\section[\normalsize{Fondant au chocolat (façon Claire T.)}]{Fondant au chocolat (façon Claire T.)}
\choco{Fondant au chocolat (façon Claire T.)}
\begin{ingredients}
\item 200 g de chocolat
\item 150 g de beurre
\item 5 oeufs
\item 1 cuill\`ere \`a soupe de farine
\item 180 g de sucre
\end{ingredients}
\begin{infos}
\item Pour 6 personnes
\item Préparation : 25 min
\item Cuisson : 20 min
\end{infos}
\begin{etapes}
\item Pr\'echauffer le four \`a 190° C.
\item Faire fondre le chocolat et le beurre au bain marie. 
\item Ajouter au sucre et laisser refroidir un peu. 
\item Ajouter les oeufs un \`a un en remuant bien \`a chaque fois. Verser la farine, lisser le m\'elange.
\item Verser dans un moule \`a manqu\'e beurr\'e et faire cuire 20 minutes au four.
\end{etapes}
\begin{conseils}
\end{conseils}

\section[\normalsize{Gâteau au chocolat (façon Claire T.)}]{Gâteau au chocolat (façon Claire T.)}

\begin{ingredients}
\item 5 oeufs
\item 150 g de sucre
\item 250 g de chocolat
\item 100 g d'amandes en poudre
\item 200 g de beurre
\item 3 c. \`a soupe de f\'ecule de pomme de terre
\item 2 sachets de sucre vanill\'e
\item 1/2 c. \`a caf\'e de levure
\item Pour le glaçage :
\begin{itemize}
\item 125 g de chocolat \`a croquer
\item 50 g de beurre
\item 100 g de sucre glace
\end{itemize}
\end{ingredients}
\begin{infos}
\item Pour 6 personnes
\item Préparation : 45 min
\item Cuisson : 40 min
\end{infos}
\begin{etapes}
\item Fouetter les jaunes d'oeufs avec les sucres jusqu'\`a ce qu'ils blanchissent et fassent ruban.
\item Faire fondre le chocolat avec deux cuill\`eres \`a soupe d'eau au bain-marie non bouillant. L'ajouter aux jaunes, m\'elanger.
\item Ajouter la poudre d'amandes tamis\'ee avec la levure et la f\'ecule, le beurre tr\`es ramolli mais non fondu, puis les blancs d'oeufs battus en neige avec une pinc\'ee de sel.
\item Verser dans un moule de 24 cm de diam\`etre, beurr\'e et saupoudr\'e de sucre semoule, en ne remplissant qu'aux trois quarts et mettre au four th 6-7. D\'emouler ti\`ede.
\item Pour le glaçage, faire fondre le chocolat avec une cuill\`ere \`a soupe d'eau et le beurre au bain-marie. Ajouter le sucre glace tamis\'e par cuill\`eres puis enduire le g\^ateau.
\end{etapes}
\begin{conseils}
\end{conseils}

\section[\normalsize{Muffins aux pépites de chocolat (façon Claire T.)}]{\choco{Muffins aux pépites de chocolat (façon Claire T.)}}
%\choco{Muffins aux pépites de chocolat (façon Claire T.)}
%\choco{}

\begin{ingredients}
\item 300 g de farine
\item 1 sachet de levure
\item 100 g de sucre
\item 1 pinc\'ee de sel
\item 150 g de chocolat \`a p\^atisserie
\item 25 cl de lait
\item 2 oeufs
\item 75 g de beurre fondu
\item 1 cuill\`ere \`a soupe de sirop d'\'erable
\end{ingredients}
\begin{infos}
\item Pour 4 personnes
\item Préparation : 40 min
\item Cuisson : 20 min
\end{infos}
\begin{etapes}
\item Pr\'echauffer le four \`a 180° C (th. 6).
\item Couper chaque carr\'e de chocolat en 4 en se servant d'un grand couteau.
\item Dans un grand bol, m\'elanger la farine, la levure, le sucre, la pinc\'ee de sel et le chocolat.
\item Dans un pichet, casser et battre l\'eg\`erement les oeufs. Ajouter le lait, le sirop d'\'erable et le beurre fondu.
\item Verser ce liquide sur le m\'elange sec et m\'elanger juste assez pour que la farine ne soit plus visible : la p\^ate doit \^etre grumeleuse.
\item Verser dans les moules \`a muffins \`a l'aide d'une grande cuill\`ere et enfourner pour 20 minutes.
\end{etapes}
\begin{conseils}
\end{conseils}

%Scones											% <-- x1
% %Scones											% <-- x1
% %Scones											% <-- x1
% \include{./recettes/Claire/scones}								% <-- x1
\section[\normalsize{Scones}]{\LARGE{\textsc{Scones}}}		% <-- x2


\begin{itemize}
\item Pour 4 personnes
\item Préparation : 15 min
\item Cuisson : 15 min
\end{itemize}

\subsection*{\textsc{Ingr\'edients~:}}

\begin{itemize}

\item 250 g de farine
\item 40 g de sucre
\item 50 g de beurre
\item 1 jaune d'oeuf
\item 150 ml de lait
\item 50 g de raisins secs
\item 1 pinc\'ee de sel
\item 1 sachet de levure chimique
\end{itemize}


\subsection*{\textsc{Marche \`a suivre~:}}

\begin{enumerate}
\item Faire pr\'echauffer le four \`a 220° C.
\item Dans un bol, m\'elanger la farine, le sucre, le sachet de levure et la pinc\'ee de sel. Rajouter au m\'elange le beurre tr\`es mou coup\'e en petites lamelles.
\item Dans un autre bol, m\'elanger ensemble le lait avec le jaune d'oeuf.
\item Rajouter progressivement ce m\'elange \`a la farine, rajouter un peu de farine si la p\^ate colle trop.
\item Rajouter les raisins. Etaler la p\^ate sur 2cm d'\'epaisseur, d\'ecouper des cercles \`a l'emporte pi\`ece ou avec un verre, les d\'eposer sur une plaque de cuisson anti-adh\'esive. Badigeonner de lait avec un pinceau.  
\item Mettre au four pendant 15 min.
\end{enumerate}
\subsection*{\textsc{Conseil~:}}
								% <-- x1
\section[\normalsize{Scones}]{\LARGE{\textsc{Scones}}}		% <-- x2


\begin{itemize}
\item Pour 4 personnes
\item Préparation : 15 min
\item Cuisson : 15 min
\end{itemize}

\subsection*{\textsc{Ingr\'edients~:}}

\begin{itemize}

\item 250 g de farine
\item 40 g de sucre
\item 50 g de beurre
\item 1 jaune d'oeuf
\item 150 ml de lait
\item 50 g de raisins secs
\item 1 pinc\'ee de sel
\item 1 sachet de levure chimique
\end{itemize}


\subsection*{\textsc{Marche \`a suivre~:}}

\begin{enumerate}
\item Faire pr\'echauffer le four \`a 220° C.
\item Dans un bol, m\'elanger la farine, le sucre, le sachet de levure et la pinc\'ee de sel. Rajouter au m\'elange le beurre tr\`es mou coup\'e en petites lamelles.
\item Dans un autre bol, m\'elanger ensemble le lait avec le jaune d'oeuf.
\item Rajouter progressivement ce m\'elange \`a la farine, rajouter un peu de farine si la p\^ate colle trop.
\item Rajouter les raisins. Etaler la p\^ate sur 2cm d'\'epaisseur, d\'ecouper des cercles \`a l'emporte pi\`ece ou avec un verre, les d\'eposer sur une plaque de cuisson anti-adh\'esive. Badigeonner de lait avec un pinceau.  
\item Mettre au four pendant 15 min.
\end{enumerate}
\subsection*{\textsc{Conseil~:}}
								% <-- x1
\section[\normalsize{Scones}]{\LARGE{\textsc{Scones}}}		% <-- x2


\begin{itemize}
\item Pour 4 personnes
\item Préparation : 15 min
\item Cuisson : 15 min
\end{itemize}

\subsection*{\textsc{Ingr\'edients~:}}

\begin{itemize}

\item 250 g de farine
\item 40 g de sucre
\item 50 g de beurre
\item 1 jaune d'oeuf
\item 150 ml de lait
\item 50 g de raisins secs
\item 1 pinc\'ee de sel
\item 1 sachet de levure chimique
\end{itemize}


\subsection*{\textsc{Marche \`a suivre~:}}

\begin{enumerate}
\item Faire pr\'echauffer le four \`a 220° C.
\item Dans un bol, m\'elanger la farine, le sucre, le sachet de levure et la pinc\'ee de sel. Rajouter au m\'elange le beurre tr\`es mou coup\'e en petites lamelles.
\item Dans un autre bol, m\'elanger ensemble le lait avec le jaune d'oeuf.
\item Rajouter progressivement ce m\'elange \`a la farine, rajouter un peu de farine si la p\^ate colle trop.
\item Rajouter les raisins. Etaler la p\^ate sur 2cm d'\'epaisseur, d\'ecouper des cercles \`a l'emporte pi\`ece ou avec un verre, les d\'eposer sur une plaque de cuisson anti-adh\'esive. Badigeonner de lait avec un pinceau.  
\item Mettre au four pendant 15 min.
\end{enumerate}
\subsection*{\textsc{Conseil~:}}
		
	\chapter{Flans}
% Millas Bordelais											% <-- x1
% % Millas Bordelais											% <-- x1
% % Millas Bordelais											% <-- x1
% \include{./recettes/millasbordelais}								% <-- x1
\section[\normalsize{Millas Bordelais}]{\LARGE{\textsc{Millas Bordelais}}}		% <-- x2


\begin{itemize}
\item Pour 6 personnes*
\item Préparation : 15 min*
\item Cuisson : 30 min
\end{itemize}

\subsection*{\textsc{Ingr\'edients~:}}

\begin{itemize}
\item 50 cl de lait vanill\'e
\item 3 oeufs
\item 6 cs de sucre
\item 6 cs de farine
\item 50 g de beurre
\end{itemize}


\subsection*{\textsc{Marche \`a suivre~:}}

\begin{enumerate}
\item Mettre les jaunes et le sucre.

\item Battre en cr\`eme.

\item Ajouter le beurre fondu, la farine.

\item D\'elayer avec le lait chaud.

\item Battre les blancs en neige, les incorporer.

\item Beurrer le moule.

\item Mettre au four 30 min th.6-7.
\end{enumerate}
\subsection*{\textsc{Conseil~:}}

								% <-- x1
\section[\normalsize{Millas Bordelais}]{\LARGE{\textsc{Millas Bordelais}}}		% <-- x2


\begin{itemize}
\item Pour 6 personnes*
\item Préparation : 15 min*
\item Cuisson : 30 min
\end{itemize}

\subsection*{\textsc{Ingr\'edients~:}}

\begin{itemize}
\item 50 cl de lait vanill\'e
\item 3 oeufs
\item 6 cs de sucre
\item 6 cs de farine
\item 50 g de beurre
\end{itemize}


\subsection*{\textsc{Marche \`a suivre~:}}

\begin{enumerate}
\item Mettre les jaunes et le sucre.

\item Battre en cr\`eme.

\item Ajouter le beurre fondu, la farine.

\item D\'elayer avec le lait chaud.

\item Battre les blancs en neige, les incorporer.

\item Beurrer le moule.

\item Mettre au four 30 min th.6-7.
\end{enumerate}
\subsection*{\textsc{Conseil~:}}

								% <-- x1
\section[\normalsize{Millas Bordelais}]{\LARGE{\textsc{Millas Bordelais}}}		% <-- x2


\begin{itemize}
\item Pour 6 personnes*
\item Préparation : 15 min*
\item Cuisson : 30 min
\end{itemize}

\subsection*{\textsc{Ingr\'edients~:}}

\begin{itemize}
\item 50 cl de lait vanill\'e
\item 3 oeufs
\item 6 cs de sucre
\item 6 cs de farine
\item 50 g de beurre
\end{itemize}


\subsection*{\textsc{Marche \`a suivre~:}}

\begin{enumerate}
\item Mettre les jaunes et le sucre.

\item Battre en cr\`eme.

\item Ajouter le beurre fondu, la farine.

\item D\'elayer avec le lait chaud.

\item Battre les blancs en neige, les incorporer.

\item Beurrer le moule.

\item Mettre au four 30 min th.6-7.
\end{enumerate}
\subsection*{\textsc{Conseil~:}}


	\chapter{Brioches et viennoiseries}
% Kugelhof											% <-- x1
% % Kugelhof											% <-- x1
% % Kugelhof											% <-- x1
% \include{./recettes/kugelhof}								% <-- x1
\section[\normalsize{Kugelhof}]{\LARGE{\textsc{Kugelhof}}}		% <-- x2


\begin{itemize}
\item Pour 6 personnes
\item Préparation : 20 min*
\item Cuisson : 45 min
\end{itemize}

\subsection*{\textsc{Ingr\'edients~:}}

\begin{itemize}
\item 333 g de farine
\item 50 g de sucre
\item 100 g de beurre
\item 6 g de sel
\item 2 oeufs
\item 13 cl de lait
\item 1 paquet de levure de boulanger
\item 50 g  de raisins secs
\item 20 g d’amandes effil\'ees 
\end{itemize}


\subsection*{\textsc{Marche \`a suivre~:}}

\begin{enumerate}
\item Dans une terrine, verser la farine, le sucre, le sel , les oeufs, la levure et le lait ti\`ede. 

\item M\'elanger et p\'etrir pendant 10 minutes . La p\^ate doit devenir \'elastique et se d\'etacher des bords de la terrine. 

\item Ajouter le beurre mou.

\item Laisser reposer dans un endroit ti\`ede quelques heures. La p\^ate doit doubler de volume. 

\item Casser la p\^ate et ajouter les raisins tremp\'es dans l ’eau.

\item Mettre dans un moule bien beurr\'e et garni d’amandes.

\item Laisser la p\^ate monter. Elle doit doubler de volume.

\item Mettre dans un four pr\'echauff\'e : 200° C pendant 45 minutes environ.
\end{enumerate}
\subsection*{\textsc{Conseil~:}}

								% <-- x1
\section[\normalsize{Kugelhof}]{\LARGE{\textsc{Kugelhof}}}		% <-- x2


\begin{itemize}
\item Pour 6 personnes
\item Préparation : 20 min*
\item Cuisson : 45 min
\end{itemize}

\subsection*{\textsc{Ingr\'edients~:}}

\begin{itemize}
\item 333 g de farine
\item 50 g de sucre
\item 100 g de beurre
\item 6 g de sel
\item 2 oeufs
\item 13 cl de lait
\item 1 paquet de levure de boulanger
\item 50 g  de raisins secs
\item 20 g d’amandes effil\'ees 
\end{itemize}


\subsection*{\textsc{Marche \`a suivre~:}}

\begin{enumerate}
\item Dans une terrine, verser la farine, le sucre, le sel , les oeufs, la levure et le lait ti\`ede. 

\item M\'elanger et p\'etrir pendant 10 minutes . La p\^ate doit devenir \'elastique et se d\'etacher des bords de la terrine. 

\item Ajouter le beurre mou.

\item Laisser reposer dans un endroit ti\`ede quelques heures. La p\^ate doit doubler de volume. 

\item Casser la p\^ate et ajouter les raisins tremp\'es dans l ’eau.

\item Mettre dans un moule bien beurr\'e et garni d’amandes.

\item Laisser la p\^ate monter. Elle doit doubler de volume.

\item Mettre dans un four pr\'echauff\'e : 200° C pendant 45 minutes environ.
\end{enumerate}
\subsection*{\textsc{Conseil~:}}

								% <-- x1
\section[\normalsize{Kugelhof}]{\LARGE{\textsc{Kugelhof}}}		% <-- x2


\begin{itemize}
\item Pour 6 personnes
\item Préparation : 20 min*
\item Cuisson : 45 min
\end{itemize}

\subsection*{\textsc{Ingr\'edients~:}}

\begin{itemize}
\item 333 g de farine
\item 50 g de sucre
\item 100 g de beurre
\item 6 g de sel
\item 2 oeufs
\item 13 cl de lait
\item 1 paquet de levure de boulanger
\item 50 g  de raisins secs
\item 20 g d’amandes effil\'ees 
\end{itemize}


\subsection*{\textsc{Marche \`a suivre~:}}

\begin{enumerate}
\item Dans une terrine, verser la farine, le sucre, le sel , les oeufs, la levure et le lait ti\`ede. 

\item M\'elanger et p\'etrir pendant 10 minutes . La p\^ate doit devenir \'elastique et se d\'etacher des bords de la terrine. 

\item Ajouter le beurre mou.

\item Laisser reposer dans un endroit ti\`ede quelques heures. La p\^ate doit doubler de volume. 

\item Casser la p\^ate et ajouter les raisins tremp\'es dans l ’eau.

\item Mettre dans un moule bien beurr\'e et garni d’amandes.

\item Laisser la p\^ate monter. Elle doit doubler de volume.

\item Mettre dans un four pr\'echauff\'e : 200° C pendant 45 minutes environ.
\end{enumerate}
\subsection*{\textsc{Conseil~:}}


% Pain d'\'epices											% <-- x1
% % Pain d'\'epices											% <-- x1
% % Pain d'\'epices											% <-- x1
% \include{./recettes/Claire/paindepicefaconclaire}								% <-- x1
\section[\normalsize{Pain d'\'epices (fa\c con Claire T.)}]{\LARGE{\textsc{Pain d'\'epices (fa\c con Claire T.)}}}		% <-- x2


\begin{itemize}
\item Pour 8 personnes
\item Préparation : 20 min
\item Cuisson : 50 min
\end{itemize}

\subsection*{\textsc{Ingr\'edients~:}}

\begin{itemize}

\item 300 g de farine
\item 1/3 l de lait
\item 100 g de cassonnade
\item 100 g de miel
\item 1 yaourt
\item 2 c. \`a caf\'e rases de bicarbonate de soude
\item 1 cuill\`eres \`a caf\'e de cannelle
\item 1 cuill\`eres \`a caf\'e de 4 \'epices
\item 1 cuill\`ere \`a caf\'e de gingembre
\end{itemize}


\subsection*{\textsc{Marche \`a suivre~:}}

\begin{enumerate}

\item Faire pr\'echauffer le four \`a 160° C.
\item Faire fondre \`a feu tr\`es doux le sucre, le lait et le 
miel.
\item Dans une terrine, m\'elanger la farine, les \'epices et le 
yaourt. Verser le lait. 
\item Ajouter le bicarbonate de soude d\'elay\'e dans 2 cuill
\`eres \`a soupe d'eau chaude.
\item Mettre au four 25 minutes, puis 25 minutes \`a 200° C.


\end{enumerate}
\subsection*{\textsc{Conseil~:}}
Si pas de bicarbonate de soude, remplacer par 1/2 sachet de levure.
Ne pas h\'esiter \`a rajouter des \'epices si besoin.								% <-- x1
\section[\normalsize{Pain d'\'epices (fa\c con Claire T.)}]{\LARGE{\textsc{Pain d'\'epices (fa\c con Claire T.)}}}		% <-- x2


\begin{itemize}
\item Pour 8 personnes
\item Préparation : 20 min
\item Cuisson : 50 min
\end{itemize}

\subsection*{\textsc{Ingr\'edients~:}}

\begin{itemize}

\item 300 g de farine
\item 1/3 l de lait
\item 100 g de cassonnade
\item 100 g de miel
\item 1 yaourt
\item 2 c. \`a caf\'e rases de bicarbonate de soude
\item 1 cuill\`eres \`a caf\'e de cannelle
\item 1 cuill\`eres \`a caf\'e de 4 \'epices
\item 1 cuill\`ere \`a caf\'e de gingembre
\end{itemize}


\subsection*{\textsc{Marche \`a suivre~:}}

\begin{enumerate}

\item Faire pr\'echauffer le four \`a 160° C.
\item Faire fondre \`a feu tr\`es doux le sucre, le lait et le 
miel.
\item Dans une terrine, m\'elanger la farine, les \'epices et le 
yaourt. Verser le lait. 
\item Ajouter le bicarbonate de soude d\'elay\'e dans 2 cuill
\`eres \`a soupe d'eau chaude.
\item Mettre au four 25 minutes, puis 25 minutes \`a 200° C.


\end{enumerate}
\subsection*{\textsc{Conseil~:}}
Si pas de bicarbonate de soude, remplacer par 1/2 sachet de levure.
Ne pas h\'esiter \`a rajouter des \'epices si besoin.								% <-- x1
\section[\normalsize{Pain d'\'epices (fa\c con Claire T.)}]{\LARGE{\textsc{Pain d'\'epices (fa\c con Claire T.)}}}		% <-- x2


\begin{itemize}
\item Pour 8 personnes
\item Préparation : 20 min
\item Cuisson : 50 min
\end{itemize}

\subsection*{\textsc{Ingr\'edients~:}}

\begin{itemize}

\item 300 g de farine
\item 1/3 l de lait
\item 100 g de cassonnade
\item 100 g de miel
\item 1 yaourt
\item 2 c. \`a caf\'e rases de bicarbonate de soude
\item 1 cuill\`eres \`a caf\'e de cannelle
\item 1 cuill\`eres \`a caf\'e de 4 \'epices
\item 1 cuill\`ere \`a caf\'e de gingembre
\end{itemize}


\subsection*{\textsc{Marche \`a suivre~:}}

\begin{enumerate}

\item Faire pr\'echauffer le four \`a 160° C.
\item Faire fondre \`a feu tr\`es doux le sucre, le lait et le 
miel.
\item Dans une terrine, m\'elanger la farine, les \'epices et le 
yaourt. Verser le lait. 
\item Ajouter le bicarbonate de soude d\'elay\'e dans 2 cuill
\`eres \`a soupe d'eau chaude.
\item Mettre au four 25 minutes, puis 25 minutes \`a 200° C.


\end{enumerate}
\subsection*{\textsc{Conseil~:}}
Si pas de bicarbonate de soude, remplacer par 1/2 sachet de levure.
Ne pas h\'esiter \`a rajouter des \'epices si besoin.
%Briochettes \`a la pur\'ee d'amandes et au citron								% <-- x1
% %Briochettes \`a la pur\'ee d'amandes et au citron								% <-- x1
% %Briochettes \`a la pur\'ee d'amandes et au citron								% <-- x1
% \include{./recettes/Claire/briochealapureedamandesetaucitron}								% <-- x1
\section[\normalsize{Briochettes \`a la pur\'ee d'amandes et au citron}]{\LARGE{\textsc{Briochettes \`a la pur\'ee d'amandes et au citron}}}		% <-- x2


\begin{itemize}
\item Pour 6 personnes
\item Préparation : 3h + 1 nuit
\item Cuisson : 15 min
\end{itemize}

\subsection*{\textsc{Ingr\'edients~:}}

\begin{itemize}
\item 2 oeufs (1)
\item 1 grosse c \`a s de pur\'ee d'amandes (2)
\item 1 yaourt nature (3)
\item lait (1 + 2 + 3 + lait = 425 ml)
\item le zeste d'un citron
\item 80 g de sucre (moiti\'e blanc, moiti\'e roux)
\item 1 cuill\`ere \`a caf\'e de sel
\item 500 g de farine
\item 1 sachet de levure Briochin 
\end{itemize}


\subsection*{\textsc{Marche \`a suivre~:}}

\begin{enumerate}
\item Mélanger tous les ingrédient soit à la main soit au robot ou à la MAP
\item Pétrir une bonne dizaine de minutes (à la main)
\item Quand le p\'etrissage est fini, laisser lever 30 minutes puis filmer et mettre au r\'efrig\'erateur pour la nuit.
\item Le lendemain, verser la p\^ate sur le plan de travail farin\'e et la d\'ecouper au couteau en 12 portions. 
\item Pr\'elever dans chaque portion un morceau de la taille d'une grosse bille pour la t\^ete. Façonner les parts en boules, les r\'epartir dans les empreintes (\`a briochettes ou \`a muffins) l\'eg\`erement beurr\'ees et les inciser en croix sur le dessus \`a l'aide de ciseaux. \item D\'eposer la petite boule dans le creux form\'e en appuyant l\'eg\`erement (dur\'ee totale : environ 30 min). 
\item Laisser lever 1h \`a 1h30.
\item Pr\'echauffer le four \`a 180\ C. 
\item Dorer les briochettes \`a l'oeuf ou au lait, et enfourner pour 12 \`a 15 minutes.

\end{enumerate}
\subsection*{\textsc{Conseil~:}}
P\^ate trop humide au d\'epart, il a fallu rajouter plusieurs c \`a s de farine pendant le p\'etrissage. Diminuer la quantit\'e de liquide ?
Si utilisation de la MAP : Mettre les ingr\'edients dans l'ordre dans la cuve de la MAP, lancer le programme p\^ate lev\'ee. V\'erifier l'aspect de la p\^ate, rajouter de la farine si besoin. 								% <-- x1
\section[\normalsize{Briochettes \`a la pur\'ee d'amandes et au citron}]{\LARGE{\textsc{Briochettes \`a la pur\'ee d'amandes et au citron}}}		% <-- x2


\begin{itemize}
\item Pour 6 personnes
\item Préparation : 3h + 1 nuit
\item Cuisson : 15 min
\end{itemize}

\subsection*{\textsc{Ingr\'edients~:}}

\begin{itemize}
\item 2 oeufs (1)
\item 1 grosse c \`a s de pur\'ee d'amandes (2)
\item 1 yaourt nature (3)
\item lait (1 + 2 + 3 + lait = 425 ml)
\item le zeste d'un citron
\item 80 g de sucre (moiti\'e blanc, moiti\'e roux)
\item 1 cuill\`ere \`a caf\'e de sel
\item 500 g de farine
\item 1 sachet de levure Briochin 
\end{itemize}


\subsection*{\textsc{Marche \`a suivre~:}}

\begin{enumerate}
\item Mélanger tous les ingrédient soit à la main soit au robot ou à la MAP
\item Pétrir une bonne dizaine de minutes (à la main)
\item Quand le p\'etrissage est fini, laisser lever 30 minutes puis filmer et mettre au r\'efrig\'erateur pour la nuit.
\item Le lendemain, verser la p\^ate sur le plan de travail farin\'e et la d\'ecouper au couteau en 12 portions. 
\item Pr\'elever dans chaque portion un morceau de la taille d'une grosse bille pour la t\^ete. Façonner les parts en boules, les r\'epartir dans les empreintes (\`a briochettes ou \`a muffins) l\'eg\`erement beurr\'ees et les inciser en croix sur le dessus \`a l'aide de ciseaux. \item D\'eposer la petite boule dans le creux form\'e en appuyant l\'eg\`erement (dur\'ee totale : environ 30 min). 
\item Laisser lever 1h \`a 1h30.
\item Pr\'echauffer le four \`a 180\ C. 
\item Dorer les briochettes \`a l'oeuf ou au lait, et enfourner pour 12 \`a 15 minutes.

\end{enumerate}
\subsection*{\textsc{Conseil~:}}
P\^ate trop humide au d\'epart, il a fallu rajouter plusieurs c \`a s de farine pendant le p\'etrissage. Diminuer la quantit\'e de liquide ?
Si utilisation de la MAP : Mettre les ingr\'edients dans l'ordre dans la cuve de la MAP, lancer le programme p\^ate lev\'ee. V\'erifier l'aspect de la p\^ate, rajouter de la farine si besoin. 								% <-- x1
\section[\normalsize{Briochettes \`a la pur\'ee d'amandes et au citron}]{\LARGE{\textsc{Briochettes \`a la pur\'ee d'amandes et au citron}}}		% <-- x2


\begin{itemize}
\item Pour 6 personnes
\item Préparation : 3h + 1 nuit
\item Cuisson : 15 min
\end{itemize}

\subsection*{\textsc{Ingr\'edients~:}}

\begin{itemize}
\item 2 oeufs (1)
\item 1 grosse c \`a s de pur\'ee d'amandes (2)
\item 1 yaourt nature (3)
\item lait (1 + 2 + 3 + lait = 425 ml)
\item le zeste d'un citron
\item 80 g de sucre (moiti\'e blanc, moiti\'e roux)
\item 1 cuill\`ere \`a caf\'e de sel
\item 500 g de farine
\item 1 sachet de levure Briochin 
\end{itemize}


\subsection*{\textsc{Marche \`a suivre~:}}

\begin{enumerate}
\item Mélanger tous les ingrédient soit à la main soit au robot ou à la MAP
\item Pétrir une bonne dizaine de minutes (à la main)
\item Quand le p\'etrissage est fini, laisser lever 30 minutes puis filmer et mettre au r\'efrig\'erateur pour la nuit.
\item Le lendemain, verser la p\^ate sur le plan de travail farin\'e et la d\'ecouper au couteau en 12 portions. 
\item Pr\'elever dans chaque portion un morceau de la taille d'une grosse bille pour la t\^ete. Façonner les parts en boules, les r\'epartir dans les empreintes (\`a briochettes ou \`a muffins) l\'eg\`erement beurr\'ees et les inciser en croix sur le dessus \`a l'aide de ciseaux. \item D\'eposer la petite boule dans le creux form\'e en appuyant l\'eg\`erement (dur\'ee totale : environ 30 min). 
\item Laisser lever 1h \`a 1h30.
\item Pr\'echauffer le four \`a 180\ C. 
\item Dorer les briochettes \`a l'oeuf ou au lait, et enfourner pour 12 \`a 15 minutes.

\end{enumerate}
\subsection*{\textsc{Conseil~:}}
P\^ate trop humide au d\'epart, il a fallu rajouter plusieurs c \`a s de farine pendant le p\'etrissage. Diminuer la quantit\'e de liquide ?
Si utilisation de la MAP : Mettre les ingr\'edients dans l'ordre dans la cuve de la MAP, lancer le programme p\^ate lev\'ee. V\'erifier l'aspect de la p\^ate, rajouter de la farine si besoin. 
	\chapter{Mousses}
	\chapter{Cr\`emes}
% Generated file 2019-02-17 16:33:30.822080050 +01:00
\begin{recette}{Crème mic-mac}{Crème mic-mac}

\begin{ingredients}
5 œufs\par
125 g de chocolat dessert\par
50 g de farine\par
50 g de beurre\par
1 paquet de sucre vanillé\par
150 g de sucre\par
1 l de lait\par
\end{ingredients}

\begin{infos}
Pour 6 personnes\\
Préparation : 15 min\\
\end{infos}

\begin{etapes}
\item Faire bouillir le lait.
\item Faire fondre le chocolat brisé dans 2 cs d’eau.
\item Mélanger le sucre, la farine, 3 jaunes et 2 œufs entiers.
\item Délayer avec le lait chaud.
\item Faire chauffer jusqu’aux 1ers bouillons.
\item Ajouter le beurre hors du feu.
\item 1ère moitié ajouter le sucre vanillé
\item 2ème moitié le chocolat fondu.
\item Verser en même temps les 2 crèmes.
\item Servir froid.
\end{etapes}

\begin{conseils}
Remplacer le lait demi-écrémé par du lait entier pour obtenir encore plus d'onctuosité.
\end{conseils}

\end{recette}
	\chapter{Verrines}
%Tiramisu aux framboises en verrines											% <-- x1
% %Tiramisu aux framboises en verrines											% <-- x1
% %Tiramisu aux framboises en verrines											% <-- x1
% \include{./recettes/Claire/tiramisuauxfranboisesenverrines}								% <-- x1
\section[\normalsize{Tiramisu aux framboises en verrines}]{\LARGE{\textsc{Tiramisu aux framboises en verrines}}}		% <-- x2


\begin{itemize}
\item Pour 4 personnes
\item Préparation : 30 + 180 min
\end{itemize}

\subsection*{\textsc{Ingr\'edients~:}}

\begin{itemize}
\item 2 oeufs
\item 250 g de mascarpone
\item 50 g de sucre roux
\item 1 sachet de sucre vanill\'e
\item 100 g de framboises
\item 1/4 de zeste de citron vert hach\'e
\item environ 10 biscuits rose de Reims
\item amandes effil\'ees
\end{itemize}
Pour le coulis :
\begin{itemize}
\item 150 g de framboises
\item 50 g de sucre
\end{itemize}

\subsection*{\textsc{Marche \`a suivre~:}}

\begin{enumerate}
\item S\'eparer le blanc des jaunes d’oeufs. 
\item M\'elanger les jaunes avec le sucre et le sucre vanill\'e. 
\item Ajouter le mascarpone au fouet, puis le zeste de citron vert. 
\item Monter les blancs en neige et les incorporer d\'elicatement \`a la spatule au m\'elange pr\'ec\'edent.
\item Pr\'eparer le coulis en mixant 150 g de framboises avec 50 g de sucre.
\item Tapisser les verrines de biscuits. Recouvrir de coulis et de quelques framboises et \'etaler une couche de cr\`eme.
\item Alterner biscuits, coulis, framboises et cr\`eme. Terminer par une couche de cr\`eme. 
\item Garder 4 framboises pour la d\'eco.
\item Filmer les verrines et r\'efrig\'erer au moins 3 heures.
\item Au moment de servir, saupoudrer d’amandes effil\'ees et d\'eposer une framboise au centre.
\end{enumerate}
\subsection*{\textsc{Conseil~:}}
								% <-- x1
\section[\normalsize{Tiramisu aux framboises en verrines}]{\LARGE{\textsc{Tiramisu aux framboises en verrines}}}		% <-- x2


\begin{itemize}
\item Pour 4 personnes
\item Préparation : 30 + 180 min
\end{itemize}

\subsection*{\textsc{Ingr\'edients~:}}

\begin{itemize}
\item 2 oeufs
\item 250 g de mascarpone
\item 50 g de sucre roux
\item 1 sachet de sucre vanill\'e
\item 100 g de framboises
\item 1/4 de zeste de citron vert hach\'e
\item environ 10 biscuits rose de Reims
\item amandes effil\'ees
\end{itemize}
Pour le coulis :
\begin{itemize}
\item 150 g de framboises
\item 50 g de sucre
\end{itemize}

\subsection*{\textsc{Marche \`a suivre~:}}

\begin{enumerate}
\item S\'eparer le blanc des jaunes d’oeufs. 
\item M\'elanger les jaunes avec le sucre et le sucre vanill\'e. 
\item Ajouter le mascarpone au fouet, puis le zeste de citron vert. 
\item Monter les blancs en neige et les incorporer d\'elicatement \`a la spatule au m\'elange pr\'ec\'edent.
\item Pr\'eparer le coulis en mixant 150 g de framboises avec 50 g de sucre.
\item Tapisser les verrines de biscuits. Recouvrir de coulis et de quelques framboises et \'etaler une couche de cr\`eme.
\item Alterner biscuits, coulis, framboises et cr\`eme. Terminer par une couche de cr\`eme. 
\item Garder 4 framboises pour la d\'eco.
\item Filmer les verrines et r\'efrig\'erer au moins 3 heures.
\item Au moment de servir, saupoudrer d’amandes effil\'ees et d\'eposer une framboise au centre.
\end{enumerate}
\subsection*{\textsc{Conseil~:}}
								% <-- x1
\section[\normalsize{Tiramisu aux framboises en verrines}]{\LARGE{\textsc{Tiramisu aux framboises en verrines}}}		% <-- x2


\begin{itemize}
\item Pour 4 personnes
\item Préparation : 30 + 180 min
\end{itemize}

\subsection*{\textsc{Ingr\'edients~:}}

\begin{itemize}
\item 2 oeufs
\item 250 g de mascarpone
\item 50 g de sucre roux
\item 1 sachet de sucre vanill\'e
\item 100 g de framboises
\item 1/4 de zeste de citron vert hach\'e
\item environ 10 biscuits rose de Reims
\item amandes effil\'ees
\end{itemize}
Pour le coulis :
\begin{itemize}
\item 150 g de framboises
\item 50 g de sucre
\end{itemize}

\subsection*{\textsc{Marche \`a suivre~:}}

\begin{enumerate}
\item S\'eparer le blanc des jaunes d’oeufs. 
\item M\'elanger les jaunes avec le sucre et le sucre vanill\'e. 
\item Ajouter le mascarpone au fouet, puis le zeste de citron vert. 
\item Monter les blancs en neige et les incorporer d\'elicatement \`a la spatule au m\'elange pr\'ec\'edent.
\item Pr\'eparer le coulis en mixant 150 g de framboises avec 50 g de sucre.
\item Tapisser les verrines de biscuits. Recouvrir de coulis et de quelques framboises et \'etaler une couche de cr\`eme.
\item Alterner biscuits, coulis, framboises et cr\`eme. Terminer par une couche de cr\`eme. 
\item Garder 4 framboises pour la d\'eco.
\item Filmer les verrines et r\'efrig\'erer au moins 3 heures.
\item Au moment de servir, saupoudrer d’amandes effil\'ees et d\'eposer une framboise au centre.
\end{enumerate}
\subsection*{\textsc{Conseil~:}}
	
	\chapter{Biscuits}
\section[\normalsize{Biscuits de noël à la cannelle}]{Biscuits de noël à la cannelle}

\begin{ingredients}
\item 270 g de beurre
\item 500 g de farine
\item 2 + 1 oeufs
\item 150 g de poudre d'amandes
\item 15 g de cannelle
\item 1 zeste de citron rap\'e
\item 250 g de sucre 
\end{ingredients}
\begin{infos}
\item Pour 1 grande boite ($\approx$ 120 pièces)
\item Préparation : 30 min + 12~h de repos
\item Cuisson : 10 min / fournée  ($\approx$ 40 min)
\end{infos}
\begin{etapes}
\item Dans une terrine, p\'etrir le beurre ramolli avec la farine du bout des doigts. 
\item Ajouter 2 oeufs, la poudre d'amandes, la cannelle, le zeste de citron et le sucre. 
\item Travailler jusqu'\`a obtenir une p\^ate homog\`ene. 
\item Couvrir d'un torchon et laisser reposer une nuit dans un endroit frais.
\item Etaler la p\^ate (pour une épaisseur comprise entre 3 et 10~mm selon les goûts), d\'ecouper \`a l'emporte pi\`eces les biscuits et les d\'eposer sur une plaque de cuisson. 
\item Dorer \`a l'oeuf battu et faire cuire au four \`a 200° C
\item Cuisson 10 min environ selon le four
\end{etapes}
\begin{conseils}
Il est possible d'en faire une version sans cannelle.
\end{conseils}

\section[\normalsize{Cookies (façon Claire T.)}]{Cookies (façon Claire T.)}

\begin{ingredients}
\item 510 g de farine
\item 200 g de beurre
\item 170 g de sucre roux
\item 170 g de sucre semoule
\item 340 g de chocolat
\item 2 oeufs
\item 1 c. \`a caf\'e de bicarbonate de soude
\item 1 c. \`a caf\'e d'extrait de vanille
\end{ingredients}
\begin{infos}
\item Pour 12 personnes
\item Préparation : 15 min
\item Cuisson : 10 min
\end{infos}
\begin{etapes}
\item Pr\'echauffer le four \`a 180° C.
\item M\'elanger les 2 sucres et le beurre ramolli jusqu'\`a 
obtention d'un aspect cr\'emeux.
\item Ajouter l'oeuf, la vanille, 1 pinc\'ee de sel et m\'elanger 
le tout. Ajouter la farine au fur et \`a mesure, puis le chocolat 
bris\'e en morceaux.
\item Beurrer la plaque du four et y d\'eposer des tas de p\^ate 
d'environ une cuill\`ere \`a soupe (bien espacer).
\item Faire cuire 8 \`a 12 minutes environ.
\end{etapes}
\begin{conseils}
\end{conseils}

	\chapter{Fruits}
% Generated file 2018-12-02 20:50:19.412057143 +01:00
\begin{recette}{Pêches Ascona}{Pêches Ascona}

\begin{ingredients}
4 jaunes d’oeufs\par
2 grosses cuillerées à soupe de farine\par
50g de sucre semoule\par
1 sachet de sucre vanillé\par
50 cl de lait\par
4 blancs d’œufs\par
1 boîte de pêches au sirop\par
amandes effilées\par
\end{ingredients}

\begin{infos}
Pour 6 personnes\\
Préparation : 15 min\\
Cuisson : 5 min\\
\end{infos}

\begin{etapes}
\item Travailler les jaunes d’oeufs dans une casserole avec la farine, le sucre et le sucre vanillé.
\item Mouiller peu à peu de lait bouillant et faites épaissir sur feu doux.
\item Laisser tiédir, puis incorporez les blancs battus en neige très ferme.
\item Verser dans un compotier et laisser refroidir complètement.
\item Garnissez le dessus de demi pêches.
\item Faites fondre la gelée de framboise sur le feu très doux avec le sirop de la boîte de pêches.
\item Laissez tiédir avant d’en napper les pêches.
\item Mettez au frais.
\item Faites griller une poignée d’amandes effilées.
\item Parsemez-en le dessert au moment de servir.
\end{etapes}

\end{recette}
	\chapter{Glaces}
	\chapter{Autres}
% Generated file 2019-02-17 16:33:30.825398171 +01:00
\begin{recette}{Teurgoule}{Teurgoule}

\begin{ingredients}
2 litres de lait\par
130 g de riz\index{riz} rond\par
150 g de sucre + vanille + cannelle\par
\end{ingredients}

\begin{infos}
Pour XX personnes*	\\
Préparation : 15 min\\
Cuisson : 4 min\\
\end{infos}

\begin{etapes}
\item Beurrer le plat.
\item Saupoudrer de cannelle.
\item Mettre tous les ingrédients dans le plat.
\item Mettre 4 heures au four th.3-4 = 200°s C.
\end{etapes}

\end{recette}
% Nom de la recette à entrer entre les accolades {}
\section{Cigares banane-choco}

% Informations génériques
% Changer de ligne pour chaque et commencer par : \item
% Mettre une * si l'information n'est pas certaine 
\begin{itemize}
\item Pour 4 personnes*			% Nombre de personnes qu'on pourra nourrir ! :)
\item Préparation : 15 min*		% Temps de préparation (sans la cuisson)
\item Cuisson : 10 min			% Temps de cuisson
\end{itemize}

\subsection*{\textsc{Ingrédients~:}}

% Ici lister les ingrédients 
% Changer de ligne pour chaque ingrédient et commencer la ligne par : \item
% rajouter autant de ligne que d'ingrédient
\begin{itemize}
\item 3 feuilles de brick
\item 2 bananes
\item 12 carrés de chocolat noir
\item 25 g de beurre
\end{itemize}


\subsection*{\textsc{Marche à suivre~:}}

% Ici les étapes à réaliser
% Une étape par ligne, chaque ligne commence par un \item
% Pour exemple les étapes pour faire un millas ;)
\begin{enumerate}
\item Faites fondre le beurre au micro-onde à puissance moyenne. 

\item Allumez le four à 210° C, th 7.

\item Épluchez les bananes et coupez-les en trois tronçons, puis chacun en deux dans la longueur.

\item Coupez les feuilles de brick en quatre triangles. 

\item Beurrez-les au pinceau.

\item Déposez la partie la plus large vers vous. Posez un tronçon de banane et un carré de chocolat dessus. Rabattez les côtés, enroulez. Formez ainsi 12 petits cigares.
Déposez-les sur une plaque recouverte de papier cuisson. 

\item Cuisez 10 min au four.
\end{enumerate}

\subsection*{\textsc{Conseil~:}}
% Ici écrire les conseils concernant la recette 
Servez les cigares tout chauds avec une boule de glace à la vanille.

Vin : banyuls ou maury

% Generated file 2018-12-02 20:50:19.462947258 +01:00
\begin{recette}{Oeufs à la neige}{Oeufs à la neige}

\begin{ingredients}
un demi litre de lait\par
75 g de sucre\par
4 oeufs\par
\end{ingredients}

\begin{infos}
Pour XX personnes\\
Préparation : XX min\\
Cuisson : XX min\\
\end{infos}

\begin{etapes}
\item Faire chauffer le lait.
\item Pendant ce temps battre les blancs en meringue (25 g de sucre).
\item Faire pocher 1 à 2 min de chaque côté. On peut tremper une cuillère dans de l'eau bouillante pour prélever des blancs et lisser avec une autre cuillère ou spatule mouillée.
\item Égoutter.
\item Disposer sur un plat creux.
\item Faire la crème anglaise.
\item Verser cette crème dans le plat ou sont disposés les blancs. Ceux-ci surnagent.
\item On peut verser sur les blancs des amandes grillées ou du caramel brun roux.
\end{etapes}

\begin{conseils}
Variante : ÎLE FLOTTANTE
Faire un sirop de caramel dans un moule à charlotte (50g de sucre).
Faire cuire la meringue dans le moule enduit du sirop.
Cuire à four modéré au bain marie (20 à 30 min
Démouler l’île sur un plat à crème.
Verser s’y la crème anglaise.
\end{conseils}

\end{recette}
% Nom de la recette à entrer entre les accolades {}
\section{Falafels}

\begin{itemize}
\item Pour 50 falafels*		% Nombre de personnes qu'on pourra nourrir ! :)
\item Préparation : 45 min + 24h*		% Temps de préparation (sans la cuisson)
\item Cuisson : 20 min*			% Temps de cuisson
\end{itemize}

\subsection*{\textsc{Ingrédients~:}}

\begin{itemize}
\item 500g de pois chiches secs ou fèves sèches
\item 6 gousses d'ail
\item 1/2 bouquet de persil plat
\item 1/2 bouquet de coriandre (en tout pour les 2: 60g)
\item 1/2 oignon
\item 1 cuillère à café de bicarbonate de soude
\item 1 cuillère à soupe de sésame doré
\item 2 cuillères à café de coriandre en poudre
\item 2 cuillères à café de cumin en poudre
\item 1/2 cuillère à café de piment en poudre
\item sel, poivre
\item huile pour friture
\end{itemize}


\subsection*{\textsc{Marche à suivre~:}}

\begin{enumerate}
\item La veille, mettre les pois chiches à tremper pendant 12 à 24h.

\item Égoutter et sécher soigneusement les pois chiches.

\item Les verser dans le bol d'un mixeur. Ajouter les herbes lavées et bien séchées et l'oignon coupé en morceaux.

\item Mixer par à coups pour obtenir une pâte qui colle un peu, mais pas une purée. Elle doit s'amalgamer si on la tasse.

\item Mettre la pâte dans un bol puis ajouter les épices : coriandre, cumin, piment, sel, bicarbonate de soude, et le sésame. Bien mélanger.

\item Former des boulettes applaties d'environ 5 cm de diamètre.

\item Faire cuire dans l'huile bien chaude jusqu'à obtenir une couleur bien dorée à brune.
\end{enumerate}


\subsection*{\textsc{Conseil~:}}
Attention à bien sécher tous les ingrédients et ne surtout pas rajouter d'eau, sinon les falafels risquent de se déliter à la cuisson.
On peut les congeler une fois cuits et les faire réchauffer au four ensuite.



% Nom de la recette à entrer entre les accolades {}
\section{Saumon en papillottes tout simple}

\begin{itemize}
\item Pour 4 personnes		% Nombre de personnes qu'on pourra nourrir ! :)
\item Préparation : 10 min		% Temps de préparation (sans la cuisson)
\item Cuisson : 15 min			% Temps de cuisson
\end{itemize}

\subsection*{\textsc{Ingrédients~:}}

\begin{itemize}
\item 4 pavés de saumon
\item huile d'olive
\item thym, romarin (ou herbes de Provence)
\item fleur de sel, poivre
\end{itemize}


\subsection*{\textsc{Marche à suivre~:}}

\begin{enumerate}
\item Préchauffer le four à 180°C.

\item Placer les pavés sur 4 feuilles de papier sulfurisé ou d'aluminium, ou dans des papillottes en silicone.

\item Les arroser d'un petit filet d'huile d'olive.

\item Saupoudrer de thym et de romarin (briser le romarin en petits morceaux au préalable).

\item Saler, poivrer, refermer les papillottes.

\item Faire cuire 10 à 15 minutes, le saumon ne doit pas être trop sec.

\end{enumerate}


\subsection*{\textsc{Conseil~:}}
% Ici écrire les conseils concernant la recette 




\part{Les p\^ates et "Morceaux de Recettes"}
	\chapter{P\^ate à ...}
% Pâte à quiche											% <-- x1
% % Pâte à quiche											% <-- x1
% % Pâte à quiche											% <-- x1
% \include{./recettes/Claire/pateaquiche}								% <-- x1
\section[\normalsize{P\^ate \`a quiche}]{\LARGE{\textsc{P\^ate \`a quiche}}}		% <-- x2


\begin{itemize}
\item Pour 6 personnes
\item Préparation : 10 min
\end{itemize}

\subsection*{\textsc{Ingr\'edients~:}}

\begin{itemize}

\item 220 g de farine
\item 100 g de beurre
\item 1 oeuf
\item Eau
\item Sel
\end{itemize}


\subsection*{\textsc{Marche \`a suivre~:}}

\begin{enumerate}
\item Verser la farine dans une terrine. 
\item Ajouter le beurre mou coup\'e en morceaux et m\'elanger du bout des doigts jusqu'\`a ce que la farine ait absorb\'e tout le beurre (on obtient un sable grossier). 
\item Ajouter l'oeuf et un peu d'eau (1 ou 2 cuill\`eres \`a soupe) et travailler rapidement jusqu'\`a former une boule de p\^ate souple et non collante. 
\item Filmer et laisser reposer au r\'efrig\'erateur une demie heure au moins.
\end{enumerate}
\subsection*{\textsc{Conseil~:}}
								% <-- x1
\section[\normalsize{P\^ate \`a quiche}]{\LARGE{\textsc{P\^ate \`a quiche}}}		% <-- x2


\begin{itemize}
\item Pour 6 personnes
\item Préparation : 10 min
\end{itemize}

\subsection*{\textsc{Ingr\'edients~:}}

\begin{itemize}

\item 220 g de farine
\item 100 g de beurre
\item 1 oeuf
\item Eau
\item Sel
\end{itemize}


\subsection*{\textsc{Marche \`a suivre~:}}

\begin{enumerate}
\item Verser la farine dans une terrine. 
\item Ajouter le beurre mou coup\'e en morceaux et m\'elanger du bout des doigts jusqu'\`a ce que la farine ait absorb\'e tout le beurre (on obtient un sable grossier). 
\item Ajouter l'oeuf et un peu d'eau (1 ou 2 cuill\`eres \`a soupe) et travailler rapidement jusqu'\`a former une boule de p\^ate souple et non collante. 
\item Filmer et laisser reposer au r\'efrig\'erateur une demie heure au moins.
\end{enumerate}
\subsection*{\textsc{Conseil~:}}
								% <-- x1
\section[\normalsize{P\^ate \`a quiche}]{\LARGE{\textsc{P\^ate \`a quiche}}}		% <-- x2


\begin{itemize}
\item Pour 6 personnes
\item Préparation : 10 min
\end{itemize}

\subsection*{\textsc{Ingr\'edients~:}}

\begin{itemize}

\item 220 g de farine
\item 100 g de beurre
\item 1 oeuf
\item Eau
\item Sel
\end{itemize}


\subsection*{\textsc{Marche \`a suivre~:}}

\begin{enumerate}
\item Verser la farine dans une terrine. 
\item Ajouter le beurre mou coup\'e en morceaux et m\'elanger du bout des doigts jusqu'\`a ce que la farine ait absorb\'e tout le beurre (on obtient un sable grossier). 
\item Ajouter l'oeuf et un peu d'eau (1 ou 2 cuill\`eres \`a soupe) et travailler rapidement jusqu'\`a former une boule de p\^ate souple et non collante. 
\item Filmer et laisser reposer au r\'efrig\'erateur une demie heure au moins.
\end{enumerate}
\subsection*{\textsc{Conseil~:}}

% Pâte brisée (façon Claire T.)											% <-- x1
% % Pâte brisée (façon Claire T.)											% <-- x1
% % Pâte brisée (façon Claire T.)											% <-- x1
% \include{./recettes/Claire/patebriseefaconclaire}								% <-- x1
\section[\normalsize{P\^ate bris\'ee (fa\c con Claire T.)}]{\LARGE{\textsc{P\^ate bris\'ee (fa\c con Claire T.)}}}		% <-- x2


\begin{itemize}
\item Pour 6 personnes
\item Préparation : 10 min
\end{itemize}

\subsection*{\textsc{Ingr\'edients~:}}

\begin{itemize}
\item 250 g de farine
\item 1 oeuf
\item 100 \`a 125 g de beurre
\item Eau
\item Sel
\end{itemize}


\subsection*{\textsc{Marche \`a suivre~:}}

\begin{enumerate}
\item Versez la farine dans une terrine et faites un puit. 
\item Ajoutez l'oeuf, une pinc\'ee de sel et le beurre coup\'e en morceaux. 
\item M\'elangez en ajoutant de l'eau peu \`a peu jusqu'\`a obtention d'une boule de p\^ate souple et pas trop collante.
\item Laissez reposer avant utilisation.

\end{enumerate}
\subsection*{\textsc{Conseil~:}}
								% <-- x1
\section[\normalsize{P\^ate bris\'ee (fa\c con Claire T.)}]{\LARGE{\textsc{P\^ate bris\'ee (fa\c con Claire T.)}}}		% <-- x2


\begin{itemize}
\item Pour 6 personnes
\item Préparation : 10 min
\end{itemize}

\subsection*{\textsc{Ingr\'edients~:}}

\begin{itemize}
\item 250 g de farine
\item 1 oeuf
\item 100 \`a 125 g de beurre
\item Eau
\item Sel
\end{itemize}


\subsection*{\textsc{Marche \`a suivre~:}}

\begin{enumerate}
\item Versez la farine dans une terrine et faites un puit. 
\item Ajoutez l'oeuf, une pinc\'ee de sel et le beurre coup\'e en morceaux. 
\item M\'elangez en ajoutant de l'eau peu \`a peu jusqu'\`a obtention d'une boule de p\^ate souple et pas trop collante.
\item Laissez reposer avant utilisation.

\end{enumerate}
\subsection*{\textsc{Conseil~:}}
								% <-- x1
\section[\normalsize{P\^ate bris\'ee (fa\c con Claire T.)}]{\LARGE{\textsc{P\^ate bris\'ee (fa\c con Claire T.)}}}		% <-- x2


\begin{itemize}
\item Pour 6 personnes
\item Préparation : 10 min
\end{itemize}

\subsection*{\textsc{Ingr\'edients~:}}

\begin{itemize}
\item 250 g de farine
\item 1 oeuf
\item 100 \`a 125 g de beurre
\item Eau
\item Sel
\end{itemize}


\subsection*{\textsc{Marche \`a suivre~:}}

\begin{enumerate}
\item Versez la farine dans une terrine et faites un puit. 
\item Ajoutez l'oeuf, une pinc\'ee de sel et le beurre coup\'e en morceaux. 
\item M\'elangez en ajoutant de l'eau peu \`a peu jusqu'\`a obtention d'une boule de p\^ate souple et pas trop collante.
\item Laissez reposer avant utilisation.

\end{enumerate}
\subsection*{\textsc{Conseil~:}}

\section[\normalsize{Pancakes}]{Pancakes}

\begin{ingredients}
\item 250 g de farine
\item 30 g de sucre
\item 2 oeufs
\item 1 sachet de levure traditionnelle 
\item 65 g de beurre 
\item 1 pinc\'ee de sel 
\item 30 cl de lait
\end{ingredients}
\begin{infos}
\item Pour 4 personnes
\item Préparation : 10 min + 1h
\end{infos}
\begin{etapes}
\item Mettre la farine, la levure, le sel et le sucre dans un saladier. Rajouter ensuite les oeufs entiers et m\'elanger.
\item Ajouter ensuite le beurre fondu, puis d\'elayer progressivement le m\'elange avec le lait afin d'\'eviter les 
grumeaux.
\item Laisser reposer au minimum 1h au r\'efrig\'erateur avant de faire cuire.
\end{etapes}
\begin{conseils}
\end{conseils}

\section[\normalsize{P\^ate \`a cr\^epes (fa\c con Claire T.)}]{P\^ate \`a cr\^epes (fa\c con Claire T.)}

\begin{ingredients}
\item 250 g de farine
\item 1/2 l de lait
\item 3 oeufs
\item 1 cuill\`ere d'huile
\item 1 pinc\'ee de sel
\item 1 ou 2 cuill\`eres d'eau
\end{ingredients}
\begin{infos}
\item Pour 2 --- 4 personnes
\item Préparation : 15 min
\end{infos}
\begin{etapes}
\item Mettre la farine dans une terrine. Faire un puits, et y casser les oeufs entiers.
\item Ajouter l'huile, le sel et un peu de lait ; travailler \'energiquement la p\^ate avec une cuill\`ere (en bois, de pr\'ef\'erence), pour la rendre l\'eg\`ere.
\item Mouiller progressivement avec le lait, jusqu'\`a ce que la p\^ate devienne homog\`ene. On peut, \`a ce moment l\`a, ajouter de l'extrait de fleur d'oranger, un peu de jus de citron, de la vanille, etc...
\item Ajouter ensuite 1 \`a 2 cuill\`eres d'eau. Puis, passer dans une passoire 1 \`a 2 fois la p\^ate pour enlever les grumeaux.
\item Laisser reposer la p\^ate pendant 1 h, recouverte d'une serviette ou d'un chiffon propre.
\end{etapes}
\begin{conseils}
\end{conseils}

	\chapter{P\^at\'e}
	\chapter{Sauces}
\section[\normalsize{Pesto rouge}]{Pesto rouge}

\begin{ingredients}
\item 25 g de pignons
\item 50 g de tomates s\'ech\'ees
\item 25 g de parmesan rap\'e
\item 1 gousse d’ail
\item 7,5 cl d’huile d’olive
\end{ingredients}
\begin{infos}
\item Pour 6 personnes
\item Préparation : 15 min
\end{infos}
\begin{etapes}
\item Faire griller les pignons dans une po\^ele \`a sec, sans 
mati\`ere grasse.
\item Couper les tomates et la gousse d’ail en morceaux.
\item Passer tous les ingr\'edients au mixer.
\end{etapes}
\begin{conseils}
\end{conseils}

\part{Quelle recette aujourd'hui ?}
%	\chapter{Rapide et simple}
%	\chapter{Aujourd'hui on reçoit}
%	\chapter{Recette d'hiver}
%	\chapter{Recette d'\'et\'e}

\addcontentsline{toc}{chapter}{\listchoconame}
\listofchoco

\printindex

\backmatter

\end{document}