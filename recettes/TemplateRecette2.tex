% Nom de la recette à entrer entre les accolades {}
\section{Nom de la recette}

% Informations génériques
% Changer de ligne pour chaque et commencer par : \item
% Mettre une * si l'information n'est pas certaine 
\begin{itemize}
\item Pour XX personnes*		% Nombre de personnes qu'on pourra nourrir ! :)
\item Préparation : XX min*		% Temps de préparation (sans la cuisson)
\item Cuisson : XX min			% Temps de cuisson
\end{itemize}

\subsection*{\bsc{Ingrédients~:}}

% Ici lister les ingrédients 
% Changer de ligne pour chaque ingrédient et commencer la ligne par : \item
% rajouter autant de ligne que d'ingrédient
\begin{itemize}
\item ingredient 1
\item ingredient 2
\item ingredient 3
\end{itemize}


\subsection*{\textsc{Marche à suivre~:}}

% Ici les étapes à réaliser
% Une étape par ligne, chaque ligne commence par un \item
% Pour exemple les étapes pour faire un millas ;)
\begin{enumerate}
\item Mettre les jaunes et le sucre.

\item Battre en cr\`eme.

\item Ajouter le beurre fondu, la farine.

\item D\'elayer avec le lait chaud.

\item Battre les blancs en neige, les incorporer.

\item Beurrer le moule.

\item Mettre au four 30 min th.6-7.
\end{enumerate}


\subsection*{\textsc{Conseil~:}}
% Ici écrire les conseils concernant la recette 


