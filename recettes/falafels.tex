% Generated file 2019-02-17 16:33:30.939890960 +01:00
\begin{recette}{Falafels}{Falafels}

\begin{ingredients}
500g de pois chiches secs ou fèves sèches\par
6 gousses d'ail\par
1/2 bouquet de persil plat\par
1/2 bouquet de coriandre (en tout pour les 2: 60g)\par
1/2 oignon\par
1 cuillère à café de bicarbonate de soude\par
1 cuillère à soupe de sésame doré\par
2 cuillères à café de coriandre en poudre\par
2 cuillères à café de cuminn poudre\par
1/2 cuillère à café de piment en poudre\par
sel, poivre\par
huile pour friture\par
\end{ingredients}

\begin{infos}
Pour 50 falafels\\
Préparation : 45 min 24h\\
Cuisson : 20 min\\
\end{infos}

\begin{etapes}
\item La veille, mettre les pois chiches à tremper pendant 12 à 24h.
\item Égoutter et sécher soigneusement les pois chiches.
\item Les verser dans le bol d'un mixeur. Ajouter les herbes lavées et bien séchées et l'oignon coupé en morceaux.
\item Mixer par à coups pour obtenir une pâte qui colle un peu, mais pas une purée. Elle doit s'amalgamer si on la tasse.
\item Mettre la pâte dans un bol puis ajouter les épices : coriandre, cuminpiment, sel, bicarbonate de soude, et le sésame. Bien mélanger.
\item Former des boulettes applaties d'environ 5 cm de diamètre.
\item Faire cuire dans l'huile bien chaude jusqu'à obtenir une couleur bien dorée à brune.
\end{etapes}

\begin{conseils}
Attention à bien sécher tous les ingrédients et ne surtout pas rajouter d'eau, sinon les falafels risquent de se déliter à la cuisson.
On peut les congeler une fois cuits et les faire réchauffer au four ensuite.
\end{conseils}

\end{recette}