% Nom de la recette à entrer entre les accolades {}
\section{Soupe au lard et aux poids cassés}

\begin{ingredients}
% Ici lister les ingrédients 
% Changer de ligne pour chaque ingrédient et commencer la ligne par : \item
% rajouter autant de ligne que d'ingrédient
\item 175 g de pois cassés
\item 1 branche de thym
\item 100 g de lard fumé
\item 1 l d'eau
\item 1 verre de lait
\item 20 g de margarine
\item sel, poivre
\end{ingredients}
\begin{infos}
% Informations génériques
% Changer de ligne pour chaque et commencer par : \item
% Mettre une * si l'information n'est pas certaine 
\item Pour XX personnes*		% Nombre de personnes qu'on pourra nourrir ! :)
\item Préparation : XX min*		% Temps de préparation (sans la cuisson)
\item Cuisson : 45 min			% Temps de cuisson
\end{infos}
\begin{etapes}
% Ici les étapes à réaliser
% Une étape par ligne, chaque ligne commence par un \item
% Pour exemple les étapes pour faire un millas ;)
\item Faire fondre la margarine dans la cocotte-minute. Ajouter les lardon
\item Dès que ceux-ci sont devenus transparents, ajouter les pois cassés, l'eau, une bonne pincée de poivre, un peu de sel (attention les lardons sont déjà salés) et le thym.
\item Couvrir et laisser cuire doucement 25 minutes à partir du moment ou la soupape chuchote.
\item Laisser échapper la pression, batter le potage avec un fouet tout en ajoutant le lait petit à petit. Il est également possible de le passer au mixer en ayant au préalable retiré les lardons.
\item chauffer à nouveau 2 minutes et servir
\end{etapes}
\begin{conseils}
% Ici écrire les conseils concernant la recette 
\end{conseils}
