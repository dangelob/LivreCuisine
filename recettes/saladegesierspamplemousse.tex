% Nom de la recette à entrer entre les accolades {}
\begin{recette}{Salade de gésiers au pamplemousse}{Salade de gésiers au pamplemousse}

\begin{ingredients}
% Ici lister les ingrédients 
% Changer de ligne pour chaque ingrédient et commencer la ligne par : \item
% rajouter autant de ligne que d'ingrédient
 2 poignées de mesclun (salade mélangée)\par
 200 g de gésiers de volaille confits\par
 1 pamplemousse rose\par
 1 pamplemousse jaune\par
 50 g de pignons de pin\par
 2-3 pincées de piment doux\par
 1 cuillère à soupe de vinaigre de cidre\par
 3 cuillères à soupe d'huile\par
 Quelques brins de cerfeuil, sel, poivre
\end{ingredients}
\begin{infos}
% Informations génériques
% Changer de ligne pour chaque et commencer par : \item
% Mettre une * si l'information n'est pas certaine 
  4 personnes	\\	% Nombre de personnes qu'on pourra nourrir ! :)
 Préparation : 20 min		% Temps de préparation (sans la cuisson)
%\item Cuisson : 3 min			% Temps de cuisson
\end{infos}
\begin{etapes}
% Ici les étapes à réaliser
% Une étape par ligne, chaque ligne commence par un \item
% Pour exemple les étapes pour faire un millas ;)
\item Laver la salade. Peler à vif les 2 pamplemousses. Découper la chair en morceaux.
\item Dans une poêle, faire dorer les pignons de pin dans une cuillère à soupe d'huile bien chaude et les réserver.
\item Dans la même poêle, faites revenir les gésiers.
\item Émulsionner deux cuillères à soupe d'huile et le vinaigre avec le piment doux.
\item Mélanger l'ensemble des ingrédients délicatement en rectifiant l'assaisonnement si nécessaire. Parsemer de cerfeuil avant de servir.
\end{etapes}

\end{recette}