% Generated file 2018-12-02 20:50:18.957775640 +01:00
\begin{recette}{Soupe de navets, lard croustillant et croûtons}{Soupe de navets, lard croustillant et croûtons}

\begin{ingredients}
800 g de navets (nouveaux)\par
260 g de pommes de terre\par
1 oignon\par
1/2 l de bouillon de volaille\par
10 à 15 cl de lait\par
10 cl de crème fraîche épaisse\par
sel, poivre\par
4 fines tranches de pancetta\par
2 tranches de pain de campagne (rassis)\par
1 noisette de beurre\par
\end{ingredients}

\begin{infos}
Pour 4 personnes\\
Préparation : 25 min\\
Cuisson : 35 min\\
\end{infos}

\begin{etapes}
\item Peler les navets et les pommes de terre, les couper en morceaux.
\item Éminr l'oignon finement.
\item Faire revenir l'oignon dans un peu de beurre, à feu doux, sans le faire colorer.
\item Ajouter les pommes de terre et les navets, puis mouiller avec le bouillon de volaille. Faire cuire à feu doux et à couvert pendant 30 à 35 min jusqu'à ce que les légumes soient tendres.
\item Mixer le tout, ajouter le lait (la quantité peut varier en fonction de la consistance désirée), la crème fraîche, rectifier en sel et poivre, réserver au chaud.
\item Découper le pain de campagne en petits dés et la pancetta en morceaux. Faire chauffer une poêle, y déposer le pain (sans huile), laisser dessécher 3 minà feu vif, en remuant constamment.
\item Ajouter alors la pancetta, la faire dorer à feu vif pour qu'elle devienne bien croustillante, tout en continuant à remuer pour que les croûtons s'imprègnent de la graisse de cuisson.
\item Servir la soupe dans des petits bols ou des assiettes creuses et répartir les croûtons et la pancetta sur le dessus.
\end{etapes}

\end{recette}