% Nom de la recette à entrer entre les accolades {}
\section{Soupe de potiron}

\begin{ingredients}
% Ici lister les ingrédients 
% Changer de ligne pour chaque ingrédient et commencer la ligne par : \item
% rajouter autant de ligne que d'ingrédient
\item 1,5 kg de potiron
\item 2 grosses pommes de terre (bintje)
\item 1 blanc de poireau
\item 1 oignon
\item 1,2 de bouillon de volaille (2 tablettes de concentré)
\item 25 g de beurre
\item 200 g de crème
\item 1 pincée de noix de muscade râpée
\item 1 morceau de sucre
\item 1 bouquet garni
\item 10 brins de ciboulette
\item sel
\end{ingredients}
\begin{infos}
% Informations génériques
% Changer de ligne pour chaque et commencer par : \item
% Mettre une * si l'information n'est pas certaine 
\item Pour 6 personnes		% Nombre de personnes qu'on pourra nourrir ! :)
\item Préparation : 20 min		% Temps de préparation (sans la cuisson)
\item Cuisson : 40 min			% Temps de cuisson
\end{infos}
\begin{etapes}
% Ici les étapes à réaliser
% Une étape par ligne, chaque ligne commence par un \item
% Pour exemple les étapes pour faire un millas ;)
\item Éplucher et épépiner le quartier de potiron. Couper la chair en morceaux. Fendre le blanc de poireau.Le rincer et l'émincer finement. Éplucher les pommes de terre et les couper en gros dés. Peler et hacher l’oignon.  
\item Chauffer le beurre dans un faitout. Faites-y revenir l’oignon et le poireau 5 minutes à feu très doux sans laisser colorer. Mouiller avec le bouillon. Ajouter le potiron, les pommes de terre, le bouquet garni et le sucre. Saler légèrement. Laisser cuire 30 minutes.
\item Retirer  alors le bouquet garni. Passez la soupe au robot-mixer ou au moulin à légumes (grille fine).
\item Reverser la soupe dans le faitout propre. Rectifier l’assaisonnement , parfumer de noix de muscade râpée et porter à nouveau à ébullition.
\item Au moment de servir ajouter la crème et la ciboulette ciselée.
\end{etapes}
\begin{conseils}
% Ici écrire les conseils concernant la recette 
\end{conseils}
