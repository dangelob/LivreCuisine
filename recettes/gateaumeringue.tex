% Nom de la recette à entrer entre les accolades {}
\section{Gâteau meringué}

\begin{ingredients}
% Ici lister les ingrédients 
% Changer de ligne pour chaque ingrédient et commencer la ligne par : \item
% rajouter autant de ligne que d'ingrédient
\item 350 g de sucre semoule
\item 240 g de farine
\item 200 g de chocolat à croquer
\item 2,5 dl de lait
\item 3 oeufs
\item 50 g de beurre mou
\item 1 cuillère à soupe de levure chimique
\item 1 cuillère à café de vanille en poudre
\end{ingredients}
\begin{infos}
% Informations génériques
% Changer de ligne pour chaque et commencer par : \item
% Mettre une * si l'information n'est pas certaine 
\item Pour 8 personnes		% Nombre de personnes qu'on pourra nourrir ! :)
\item Préparation : 30 min*		% Temps de préparation (sans la cuisson)
\item Cuisson : 60 min			% Temps de cuisson
\end{infos}
\begin{etapes}
% Ici les étapes à réaliser
% Une étape par ligne, chaque ligne commence par un \item
% Pour exemple les étapes pour faire un millas ;)
\item Préchauffer le four à thermostat 5 (200°C) et beurrer le moule (rectangulaire 33x22 cm).
Casser les oeufs en séparant les blancs des jaunes; mettez les blancs dans un saladier et les jaunes dans une tasse.
Tamiser la farine avec la levure.
\item Casser le chocolat en petits morceaux, faites fondre ceux-ci dans une grande jatte posée au dessus d'une casserole remplie d'eau frémissante.
\item Passer une terrine à l'eau chaude, l'essuyer, y mettre le beurre et le travailler avec une cuillère pour le rendre crémeux.
Y verser ensuite 250 g de sucre en pluie en continuant de tourner.
Ajouter les jaunes d'oeufs un à un, en mélangeant entre chaque addition.
Incorporer alors le chocolat et la vanille.
\item Verser un tiers du lait, mélanger, ajouter un tiers de la farine, mélanger et continuer ainsi jusqu'à épuisement des ingrédients.
\item Ajouter 2 pincées de sel aux blancs d'oeufs et les battre en neige ; lorsqu'ils commencent à prendre, verser progressivement le sucre restant tout en continuant à battre jusqu'à obtenir la consistance d'une meringue.
Incorporer ces blancs battus à la pâte en soulevant la masse de bas en haut en tournant à l'aide d'une spatule.
\item Remplir le moule de pâte et faire cuire 1~h au four. À la fin de la cuisson, le gâteau doit être recouvert d'une croûte mince et satinée. Laisser tiédir pendant 5 minutes, puis démouler le gâteau et laisser le refroidir complètement avant de le servir.
\end{etapes}
\begin{conseils}
% Ici écrire les conseils concernant la recette
\end{conseils}
