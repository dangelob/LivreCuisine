% Generated file 2018-12-02 20:50:19.225370704 +01:00
\begin{recette}{Gâteau meringué}{Gâteau meringué}

\begin{ingredients}
300 g de sucre semoule\par
240 g de farine\par
200 g de chocolat à croquer\par
250 ml de lait\par
3 oeufs\par
50 g de beurre mou\par
1 cuillère à soupe de levure chimique\par
1 cuillère à café de vanille en poudre\par
\end{ingredients}

\begin{infos}
Pour 8 personnes\\
Préparation : 30 min\\
Cuisson : 50-60 min\\
\end{infos}

\begin{etapes}
\item Préchauffer le four à thermostat 5 (200°C) et beurrer le moule (rectangulaire 33x22 cm).
\item Casser les oeufs en séparant les blancs des jaunes; mettez les blancs dans un saladier et les jaunes dans une tasse.
\item Tamiser la farine avec la levure.
\item Casser le chocolat en petits morceaux, faites fondre ceux-ci dans une grande jatte posée au dessus d'une casserole remplie d'eau frémissante.
\item Passer une terrine à l'eau chaude, l'essuyer, y mettre le beurre et le travailler avec une cuillère pour le rendre crémeux.
\item Y verser ensuite 250 g de sucre en pluie en continuant de tourner.
\item Ajouter les jaunes d'oeufs un à un, en mélangeant entre chaque addition.
\item Incorporer alors le chocolat et la vanille.
\item Verser un tiers du lait, mélanger, ajouter un tiers de la farine, mélanger et continuer ainsi jusqu'à épuisement des ingrédients.
\item Ajouter 2 pincées de sel aux blancs d'oeufs et les battre en neige ; lorsqu'ils commencent à prendre, verser progressivement le sucre restant tout en continuant à battre jusqu'à obtenir la consistance d'une meringue.
\item Incorporer ces blancs battus à la pâte en soulevant la masse de bas en haut en tournant à l'aide d'une spatule.
\item Remplir le moule de pâte et faire cuire 1~h au four. À la fin de la cuisson, le gâteau doit être recouvert d'une croûte min et satinée. Laisser tiédir pendant 5 min, puis démouler le gâteau et laisser le refroidir complètement avant de le servir.
\end{etapes}

\end{recette}