% Generated file 2018-12-02 20:50:19.276427424 +01:00
\begin{recette}{Briochettes à la purée d'amandes et au citron}{Briochettes à la purée d'amandes et au citron}

\begin{ingredients}
2 oeufs (1)\par
1 grosse c à s de purée d'amandes (2)\par
1 yaourt nature (3)\par
lait (1 + 2 + 3 + lait = 425 ml)\par
le zeste d'un citron\par
80 g de sucre (moitié blanc, moitié roux)\par
1 cuillère à café de sel\par
500 g de farine\par
1 sachet de levure Briochin\par
\end{ingredients}

\begin{infos}
Pour 6 personnes\\
Préparation : 3h + 1 nuit\\
Cuisson : 15 min\\
\end{infos}

\begin{etapes}
\item Mélanger tous les ingrédient soit à la main soit au robot ou à la MAP
\item Pétrir une bonne dizaine de min (à la main)
\item Quand le pétrissage est fini, laisser lever 30 min puis filmer et mettre au réfrigérateur pour la nuit.
\item Le lendemain, verser la pâte sur le plan de travail fariné et la découper au couteau en 12 portions.
\item Prélever dans chaque portion un morceau de la taille d'une grosse bille pour la tête. Façonner les parts en boules, les répartir dans les empreintes (à briochettes ou à muffins) légèrement beurrées et les inciser en croix sur le dessus à l'aide de ciseaux. * Déposer la petite boule dans le creux formé en appuyant légèrement (durée totale : environ 30 min
\item Laisser lever 1h à 1h30.
\item Préchauffer le four à 180\ C.
\item Dorer les briochettes à l'oeuf ou au lait, et enfourner pour 12 à 15 min.
\end{etapes}

\begin{conseils}
Pâte trop humide au départ, il a fallu rajouter plusieurs c à s de farine pendant le pétrissage. Diminr la quantité de liquide ?
Si utilisation de la MAP : Mettre les ingrédients dans l'ordre dans la cuve de la MAP, lancer le programme pâte levée. Vérifier l'aspect de la pâte, rajouter de la farine si besoin.
\end{conseils}

\end{recette}