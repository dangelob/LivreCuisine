% Nom de la recette à entrer entre les accolades {}
\section{La fontaine de légumes}

\begin{ingredients}
% Ici lister les ingrédients 
% Changer de ligne pour chaque ingrédient et commencer la ligne par : \item
% rajouter autant de ligne que d'ingrédient
\item 1 barquette de tomates cerises
\item 3 oignons nouveaux
\item 400 g de pointes d'asperges vertes surgelées
\item 2 gousses d'ail
\item 200 g de comté
\item 3 tranches de pain de mie
\item 3 cuillère à soupe de câpres
\item 4 oeufs + 1 jaune
\item 10 cl d'huile
\item 1 cuillère à soupe de moutarde
\item 30 g de beurre
\item sel, poivre
\end{ingredients}
\begin{infos}
% Informations génériques
% Changer de ligne pour chaque et commencer par : \item
% Mettre une * si l'information n'est pas certaine 
\item Pour 6 personnes		% Nombre de personnes qu'on pourra nourrir ! :)
\item Préparation : 30 min		% Temps de préparation (sans la cuisson)
\item Cuisson : 10 min			% Temps de cuisson
\end{infos}
\begin{etapes}
% Ici les étapes à réaliser
% Une étape par ligne, chaque ligne commence par un \item
% Pour exemple les étapes pour faire un millas ;)
\item Cuire les asperges à l'eau bouillante salée. Rafraîchir, égoutter et couper en tronçons. Plonger les oeufs dans une casserole d'eau bouillante et les cuire dur (10 minutes). Écaler les oeufs tièdes et les couper en tranches.
\item Couper le pain de mie en dés et les dorer à la poêle dans le beurre chaud. Nettoyer les oignons et les émincer. Rincer les tomates et les couper en deux. Détailler le comté en cubes.
\item Peler et hacher finement l'ail.
Dans un bol, préparer la mayonnaise avec le jaune d'oeuf, la moutarde, le sel et le poivre. Émulsionner avec l'huile en filet.
Ajouter l'ail haché et mélanger.
\item Dans un saladier profond, transparent, disposer la moitié des tomates, les oeufs en tranches, les câpres, les oignons, les dés de comté, le reste de tomates les tronçons d'asperges et terminer par les croûtons.
Servir avec la mayonnaise à part.
\end{etapes}
\begin{conseils}
% Ici écrire les conseils concernant la recette
Le bon vin : sylvaner à 8--10°C
\end{conseils}
