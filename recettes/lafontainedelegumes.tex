% Generated file 2019-02-17 16:33:30.755126962 +01:00
\begin{recette}{La fontaine de légumes}{La fontaine de légumes}

\begin{ingredients}
1 barquette de tomates cerises\par
3 oignons nouveaux\par
400 g de pointes d'asperges vertes surgelées\par
2 gousses d'ail\par
200 g de comté\par
3 tranches de pain de mie\par
3 cuillère à soupe de câpres\par
4 oeufs + 1 jaune\par
10 cl d'huile\par
1 cuillère à soupe de moutarde\par
30 g de beurre\par
sel, poivre\par
\end{ingredients}

\begin{infos}
6 personnes\\
Préparation : 30 min\\
Cuisson : 10 min\\
\end{infos}

\begin{etapes}
\item Cuire les asperges à l'eau bouillante salée. Rafraîchir, égoutter et couper en tronçons. Plonger les oeufs dans une casserole d'eau bouillante et les cuire dur (10 min). Écaler les oeufs tièdes et les couper en tranches.
\item Couper le pain de mie en dés et les dorer à la poêle dans le beurre chaud. Nettoyer les oignons et les éminr. Rincer les tomates et les couper en deux. Détailler le comté en cubes.
\item Peler et hacher finement l'ail.
\item Dans un bol, préparer la mayonnaise avec le jaune d'oeuf, la moutarde, le sel et le poivre. Émulsionner avec l'huile en filet.
\item Ajouter l'ail haché et mélanger.
\item Dans un saladier profond, transparent, disposer la moitié des tomates, les oeufs en tranches, les câpres, les oignons, les dés de comté, le reste de tomates les tronçons d'asperges et termin par les croûtons.
\item Servir avec la mayonnaise à part.
\end{etapes}

\begin{conseils}
Le bon vin : sylvaner à 8--10°C
	
\end{conseils}

\end{recette}