% Nom de la recette à entrer entre les accolades {}
\begin{recette}{Salade Romaine sauce roquefort}{Salade Romaine sauce roquefort}
\begin{ingredients}
% Ici lister les ingrédients 
% Changer de ligne pour chaque ingrédient et commencer la ligne par : \item
% rajouter autant de ligne que d'ingrédient
 1 grosse laitue romaine \par
 120 g de roquefort ou St Agurs \par
 150 g de lardons nature \par
 2 grandes tranches de pain de mie sans croûte \par
 10 cl de crème liquide \par
 1 cuillère à soupe de vinaigre de xéres \par
 2 cuillères à soupe d'huile d'olive \par
 1 gousse d'ail, poivre 
\end{ingredients}
\begin{infos}
% Informations génériques
% Changer de ligne pour chaque et commencer par : \item
% Mettre une * si l'information n'est pas certaine 
 6 personnes	\\	% Nombre de personnes qu'on pourra nourrir ! :)
 Préparation : 20 min \\		% Temps de préparation (sans la cuisson)
 Cuisson : 5 min			% Temps de cuisson
\end{infos}
\begin{etapes}
% Ici les étapes à réaliser
% Une étape par ligne, chaque ligne commence par un \item
% Pour exemple les étapes pour faire un millas ;)
\item Rincer puis essorer les feuilles de romaine. Écraser finement la moitié du roquefort à la fourchette et mélangez-y la crème et le vinaigre. Poivrer cette sauce.
\item Couper le reste du roquefort en morceaux. Taillez les tranches de pain de mie en dés. Dans une poêle avec la gousse d'ail écrasée et l'huile, laisser dorer les dés de pain de mie 1 à 2 minutes en les tournant. Réserver. Les remplacer par les lardons. les faire revenir 4 à 5 minutes à feu doux jusqu'à ce qu'ils colorent légèrement.
\item Répartisser les feuilles de romaine dans des coupelles, les arroser de sauce. Garnir avec les petits morceaux de roquefort, les croûtons et les lardons encore chauds. Servir immédiatement.
\end{etapes}
\end{recette}