% Generated file 2019-02-17 16:33:30.931494136 +01:00
\begin{recette}{Tartiflette (façon Claire T.)}{Tartiflette (façon Claire T.)}

\begin{ingredients}
1 kg de pommes de terre à chair ferme\par
1 reblochon fermier ou fruité\par
200g de lardons\par
1 gros oignon\par
2 cuillères à soupe de crème fraîche\par
vin blanc de Savoie (Apremont)\par
1 gousse d'ail\par
\end{ingredients}

\begin{infos}
Pour 6 personnes\\
Préparation : 50 min\\
Cuisson : 10 min\\
\end{infos}

\begin{etapes}
\item Faire cuire les pommes de terre dans de l'eau (départ à froid) pendant 20 à 25 min. Egoutter et éplucher.
\item Préchauffer le four à 200° C.
\item Eminr finement l'oignon. Faire fondre les lardons dans une poêle, puis ajouter les oignons (avant que les lardons ne colorent).
\item Laisser revenir à feu doux.
\item Pendant ce temps, couper les pommes de terre en rondelles dans un saladier. Frotter un plat à gratin avec une gousse d'ail et le beurrer généreusement.
\item Gratter le reblochon des deux côtés, ôter la pastille de caséine et le couper en deux dans l'épaisseur.
\item Quand les lardons et les oignons ont bien fondu (attention à ne pas faire trop colorer), ajouter un peu de vin blanc et faire réduire.
\item Les ajouter aux pommes de terre. Ajouter la crème et mélanger.
\item Verser ce mélange dans le plat à gratin, arroser de vin blanc et poser les deux moitiés de reblochon par dessus, croûte vers le haut.
\item Enfourner une dizaine de min, jusqu'à ce que le fromage ait bien fondue.
\end{etapes}

\end{recette}