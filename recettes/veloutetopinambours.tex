% Generated file 2018-12-02 20:50:19.311981852 +01:00
\begin{recette}{Velouté de topinambours au chèvre frais}{Velouté de topinambours au chèvre frais}

\begin{ingredients}
800 g de topinambours\par
1 petit oignon\par
bouillon de légumes\par
200 ml de lait\par
beurre\par
sel, poivre\par
50 g de chèvre frais\par
crème fraîche (selon l'envie)\par
1 bonne poignée de noisettes\par
\end{ingredients}

\begin{infos}
Pour 4 personnes\\
Préparation : 40 min\\
Cuisson : 15 min\\
\end{infos}

\begin{etapes}
\item Laver et peler les topinambours, les couper en petits morceaux. Éminr l'oignon.
\item Dans une casserole, faire chauffer une noix de beurre. Y faire revenir l'oignon et le topinambour pendant 1 à 2 minsur feux doux. Mouiller avec le bouillon de légumes (recouvrir) et laisser mijoter pendant environ 15 minsur feux doux, pour que le topinambour soit cuit (piquer au couteau, il ne doit pas rester sur la lame). Rajouter un peu de bouillon pendant la cuisson si besoin.
\item Pendant ce temps, concasser les noisettes et les torréfier à sec dans une poêle.
\item Réserver le bouillon. Mixer le topinambour, ajouter le lait, puis progressivement un peu de bouillon jusqu'à obtenir la consistance désirée. Ajouter le chèvre, la crème et bien mixer. Rectifier l'assaisonnement au besoin.
\item Servir parsemé de noisettes concassées avec éventuellement un filet d'huile de noisette.
\end{etapes}

\begin{conseils}
Ne pas hésiter à préparer ce plat à l'avance, comme tous les plats mijotés c'est encore meilleur réchauffé.
\end{conseils}

\end{recette}