% Generated file 2018-11-25 21:43:22.382497412 +01:00
\begin{recette}{Chop Suey de porc}{Chop Suey de porc}

\begin{ingredients}
300 g de filet de porc\par
200 g d'oignons\par
200 g de pousses de soja\par
200 g de pousses de bambou en conserve\par
6 champignons parfumés\par
Quelques brins de coriandre\par
40 g de vermicelles de soja\par
7 cuillères à soupe d'huile\par
1 cuillère à soupe de vin chinois (ou xerès)\par
2 cuillères à soupe de sauce soja\par
sel\par
\end{ingredients}

\begin{infos}
Pour 4 personnes\\
Préparation : 30 min\\
Cuisson : ?? min\\
\end{infos}

\begin{etapes}
\item Faire gonfler dans de l'eau chaude les champignons.
\item Peler et hacher les oignons, couper le porc en lanières. Verser 4 cuillères à soupe d'huile dans un wok. Quand l'huile est chaude,
\item faire revenir les oignons 5 minutes. Augmenter le feu et ajouter la viande. Saler légèrement et faire cuire 5 minutes en remuant. Réserver.
\item Nettoyer et rincer le soja, égoutter et rincer le bambou, rincer et ciseler la coriandre.
\item Faire cuire les vermicelles comme indiqué sur le paquet. Rincer à l'eau froide.
\item Eponger les champignons et les couper en lamelles en éliminant les pieds. Verser l'huile restante dans le wok, faire chauffer et
\item mettre les vermicelles, le soja, le bambou et les champignons. Saler légèrement, cuire 4 à 5 minutes en remuant sans arrêt.
\item Ajouter la viande pour la faire réchauffer, puis le vin chinois et la sauce soja. Bien mélanger. Parsemer de coriandre et servir.
\end{etapes}

\end{recette}