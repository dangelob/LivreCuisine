% Generated file 2018-11-25 21:43:22.190154115 +01:00
\begin{recette}{Lapin aux pruneaux}{Lapin aux pruneaux}

\begin{ingredients}
1 beau lapin\index{lapin} coupé en 8 morceaux\par
50 cl litre de vin rouge\par
1 oignon piqué de 2 clous de girofle\par
1 bouquet garni\par
16 pruneaux dénoyautés\par
1 cuillerée à soupe d’huile\par
1 gousse d’ail\par
1 cuillerée à soupe de gelée de groseille\par
sel, poivre\par
1 cuillerée à soupe de crème\par
1 cuillerée à soupe rase de Maizena\par
80 g de margarine\par
1 petite boîte de champignons de Paris\par
\end{ingredients}

\begin{infos}
Pour 6 personnes\\
Préparation : 30 min\\
Cuisson : 50 min\\
\end{infos}

\begin{etapes}
\item La veille, faites mariner le lapin au frais avec la gousse d’ail, le vin rouge, l’huile, l’oignon et le bouquet garni.
\item Faites chauffer la margarine dans la cocotte-minute SEB : mettez-y à dorer les morceaux de lapin bien égouttés, versez par-dessus la marinade et la Ma\"\i zena mélangée dans très peu d’eau froide, tourner.
\item Puis ajouter les champignons, salez, poivrez et faites cuire 20 minutes à feu doux à partir du moment où la soupape chuchote.
\item Ajoutez alors les pruneaux, faites cuire 5 minutes sous pression.
\item Mélangez la crème avec la gelée de groseille. Ouvrez la cocotte-minute, mettez les morceaux de lapin dans le plat de service, versez dans la sauce le mélange crème-gelée et nappez le lapin.
\end{etapes}

\begin{conseils}
Une purée maison sera la bienvenue comme garniture.
\end{conseils}

\end{recette}