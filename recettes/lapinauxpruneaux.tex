\section[\normalsize{Lapin aux pruneaux}]{Lapin aux pruneaux}

\begin{ingredients}
\item	1 beau lapin\index{lapin} coup\'e en 8 morceaux
\item	50 cl litre de vin rouge
\item	1 oignon piqu\'e de 2 clous de girofle
\item	1 bouquet garni
\item	16 pruneaux d\'enoyaut\'es 
\item	1 cuiller\'ee \`a soupe d’huile
\item	1 gousse d’ail
\item	1 cuiller\'ee \`a soupe de gel\'ee de groseille
\item	sel, poivre
\item	1 cuiller\'ee \`a soupe de cr\`eme
\item	1 cuiller\'ee \`a soupe rase de Ma\"\i zena
\item	80 g de margarine
\item	1 petite boîte de champignons de Paris
\end{ingredients}
\begin{infos}
\item Pour 6 personnes*
\item Préparation : 30 min*
\item Cuisson : 50 min*
\end{infos}
\begin{etapes}
\item La veille, faites mariner le lapin au frais avec la gousse d’ail, le vin rouge, l’huile, l’oignon et le bouquet garni.
\item Faites chauffer la margarine dans la cocotte-minute SEB : mettez-y \`a dorer les morceaux de lapin bien \'egoutt\'es, versez par-dessus la marinade et la Ma\"\i zena m\'elang\'ee dans tr\`es peu d’eau froide, tourner. 
\item Puis ajouter les champignons, salez, poivrez et faites cuire 20 minutes \`a feu doux \`a partir du moment où la soupape chuchote.
\item Ajoutez alors les pruneaux, faites cuire 5 minutes sous pression.
\item M\'elangez la cr\`eme avec la gel\'ee de groseille. Ouvrez la cocotte-minute, mettez les morceaux de lapin dans le plat de service, versez dans la sauce le m\'elange cr\`eme-gel\'ee et nappez le lapin.
\end{etapes}
\begin{conseils}
Une pur\'ee maison sera la bienvenue comme garniture.
\end{conseils}