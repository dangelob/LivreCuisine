% Generated file 2019-02-17 16:33:30.945603416 +01:00
\begin{recette}{Minestrone d'hiver}{Minestrone d'hiver}

\begin{ingredients}
3 carottes\par
2 branches de céleri\par
2 navets\par
2 pommes de terre\par
1 poireau\par
1 gros oignon\par
2 gousses d'ail\par
1 petite boîte de tomates concassées\par
250 g de mélange spécial minestrone (épeautre, haricots secs, lentilles)\par
petits pois (quantité ?)\par
1 bouquet garni (thym, laurier)\par
sel, poivre\par
petites pâtes (facultatif)\par
\end{ingredients}

\begin{infos}
Pour 6 personnes\\
Préparation : trempage 12h et 30 min\\
Cuisson : mini 1h30\\
\end{infos}

\begin{etapes}
\item La veille faire tremper le mélange minestrone pendant 12h dans une grande quantité d'eau.
\item Faire blondir l'oignon émincé et l'ail hachés dans de l'huile d'olive. Ajouter le poireau émincé finement, laisser fondre un peu. Ajouter les carottes, les navets, le céleri et les pommes de terre préalablemment coupés en dés puis mélanger. Après quelques instants, ajouter le mélange minestrone, les tomates, le bouquet garni.
\item Couvrir d'eau et porter à ébullition. Laisser cuire à feu doux à couvert au moins 1h30.
\item 15 min avant la fin, ajouter les petits pois. En fin de cuisson assaisonner et éventuellement ajouter des petites pâtes (environ 1 poignée pour 2 personnes).
\end{etapes}

\begin{conseils}
Généralement on utilise un mélange minestrone bio "Moulin des moines".
Idées de variantes : mettre du chou frisé - ajouter du pesto à la fin (ou du basilic)
\end{conseils}

\end{recette}