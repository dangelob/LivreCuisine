% Tarte mousseuse aux framboises											% <-- x1
% % Tarte mousseuse aux framboises											% <-- x1
% % Tarte mousseuse aux framboises											% <-- x1
% % Tarte mousseuse aux framboises											% <-- x1
% \include{./recettes/tartemousseuseauxframboises}								% <-- x1
\section[\normalsize{Tarte mousseuse aux framboises}]{\LARGE{\textsc{Tarte mousseuse aux framboises}}}		% <-- x2


\begin{itemize}
\item Pour 6 personnes
\item Préparation : 45 min
\item Repos : 1 h
\item Cuisson : 40 min
\end{itemize}

\subsection*{\textsc{Ingr\'edients~:}}

Pour la p\^ate :
\begin{itemize}
\item	250 g de farine
\item	10 g de levure de boulanger
\item	1 oeuf
\item	40 g de sucre
\item	10 cl de lait
\item	un quart de cuil. à caf\'e de sel
\item	100 g de beurre ramolli
\end{itemize}
Pour la garniture :
\begin{itemize}
\item	300 g de framboises \index{framboises}
\item	100 g de sucre
\item	3 oeufs
\item	1 cuil. à soupe de sucre glace
\item	125 g d’amandes en poudre
\end{itemize}


\subsection*{\textsc{Marche \`a suivre~:}}

\begin{enumerate}
\item  Dans une terrine, d\'elayez la levure dans 5 cl de lait ti\`ede et une pinc\'ee de sucre. Ajoutez 60 g de farine et m\'elangez pour obtenir une sorte de bouillie. Couvrez d’un linge et laissez lever 15 min dans un endroit ti\`ede.

\item  Versez le reste de farine sur le levain, ajoutez le reste de lait ti\`ede, le sel, le sucre et l’oeuf. P\'etrissez la p\^ate à la main ou au robot, jusqu’à ce qu’elle soit homog\`ene. Incorporez le beurre et travaillez jusqu’à ce que la p\^ate se d\'etache et ne colle plus aux mains (ou 5 min au robot).

\item Laissez reposer sous un linge 45 min dans un endroit ti\`ede. La p\^ate doit doubler de volume.

\item  P\'etrissez la p\^ate 30 secondes pour la faire retomber. Beurrez un moule rectangulaire et garnissez-le avec la p\^ate, en appuyant pour faire adh\'erer.

\item  Allumez le four th. 7 (210° C). Laissez la p\^ate au ti\`ede pendant que vous pr\'eparez la garniture.

\item  S\'eparez les blancs des jaunes d’oeufs. Fouettez les jaunes avec 50 g de sucre jusqu’à ce que le m\'elange blanchisse. Ajoutez la poudre d’amandes, m\'elangez. Battez les blancs en neige ferme avec le reste du sucre et incorporez d\'elicatement cette mousse dans la masse pr\'ec\'edente.

\item  Etalez cette pr\'eparation sur le fond de tarte, puis r\'epartissez les framboises à la surface. Elles vont s’enfoncer dans la cr\`eme au cours de la cuisson. Enfournez la tarte à mi-hauteur et faites cuire pendant environ 40 min.

\item  Laissez refroidir 5 à 10 min dans le moule, pus d\'emoulez la tarte sur une grille. Poudrez de sucre glace avant de d\'eguster. 
\end{enumerate}
\subsection*{\textsc{Conseil~:}}

								% <-- x1
\section[\normalsize{Tarte mousseuse aux framboises}]{\LARGE{\textsc{Tarte mousseuse aux framboises}}}		% <-- x2


\begin{itemize}
\item Pour 6 personnes
\item Préparation : 45 min
\item Repos : 1 h
\item Cuisson : 40 min
\end{itemize}

\subsection*{\textsc{Ingr\'edients~:}}

Pour la p\^ate :
\begin{itemize}
\item	250 g de farine
\item	10 g de levure de boulanger
\item	1 oeuf
\item	40 g de sucre
\item	10 cl de lait
\item	un quart de cuil. à caf\'e de sel
\item	100 g de beurre ramolli
\end{itemize}
Pour la garniture :
\begin{itemize}
\item	300 g de framboises \index{framboises}
\item	100 g de sucre
\item	3 oeufs
\item	1 cuil. à soupe de sucre glace
\item	125 g d’amandes en poudre
\end{itemize}


\subsection*{\textsc{Marche \`a suivre~:}}

\begin{enumerate}
\item  Dans une terrine, d\'elayez la levure dans 5 cl de lait ti\`ede et une pinc\'ee de sucre. Ajoutez 60 g de farine et m\'elangez pour obtenir une sorte de bouillie. Couvrez d’un linge et laissez lever 15 min dans un endroit ti\`ede.

\item  Versez le reste de farine sur le levain, ajoutez le reste de lait ti\`ede, le sel, le sucre et l’oeuf. P\'etrissez la p\^ate à la main ou au robot, jusqu’à ce qu’elle soit homog\`ene. Incorporez le beurre et travaillez jusqu’à ce que la p\^ate se d\'etache et ne colle plus aux mains (ou 5 min au robot).

\item Laissez reposer sous un linge 45 min dans un endroit ti\`ede. La p\^ate doit doubler de volume.

\item  P\'etrissez la p\^ate 30 secondes pour la faire retomber. Beurrez un moule rectangulaire et garnissez-le avec la p\^ate, en appuyant pour faire adh\'erer.

\item  Allumez le four th. 7 (210° C). Laissez la p\^ate au ti\`ede pendant que vous pr\'eparez la garniture.

\item  S\'eparez les blancs des jaunes d’oeufs. Fouettez les jaunes avec 50 g de sucre jusqu’à ce que le m\'elange blanchisse. Ajoutez la poudre d’amandes, m\'elangez. Battez les blancs en neige ferme avec le reste du sucre et incorporez d\'elicatement cette mousse dans la masse pr\'ec\'edente.

\item  Etalez cette pr\'eparation sur le fond de tarte, puis r\'epartissez les framboises à la surface. Elles vont s’enfoncer dans la cr\`eme au cours de la cuisson. Enfournez la tarte à mi-hauteur et faites cuire pendant environ 40 min.

\item  Laissez refroidir 5 à 10 min dans le moule, pus d\'emoulez la tarte sur une grille. Poudrez de sucre glace avant de d\'eguster. 
\end{enumerate}
\subsection*{\textsc{Conseil~:}}

								% <-- x1
\section[\normalsize{Tarte mousseuse aux framboises}]{\LARGE{\textsc{Tarte mousseuse aux framboises}}}		% <-- x2


\begin{itemize}
\item Pour 6 personnes
\item Préparation : 45 min
\item Repos : 1 h
\item Cuisson : 40 min
\end{itemize}

\subsection*{\textsc{Ingr\'edients~:}}

Pour la p\^ate :
\begin{itemize}
\item	250 g de farine
\item	10 g de levure de boulanger
\item	1 oeuf
\item	40 g de sucre
\item	10 cl de lait
\item	un quart de cuil. à caf\'e de sel
\item	100 g de beurre ramolli
\end{itemize}
Pour la garniture :
\begin{itemize}
\item	300 g de framboises \index{framboises}
\item	100 g de sucre
\item	3 oeufs
\item	1 cuil. à soupe de sucre glace
\item	125 g d’amandes en poudre
\end{itemize}


\subsection*{\textsc{Marche \`a suivre~:}}

\begin{enumerate}
\item  Dans une terrine, d\'elayez la levure dans 5 cl de lait ti\`ede et une pinc\'ee de sucre. Ajoutez 60 g de farine et m\'elangez pour obtenir une sorte de bouillie. Couvrez d’un linge et laissez lever 15 min dans un endroit ti\`ede.

\item  Versez le reste de farine sur le levain, ajoutez le reste de lait ti\`ede, le sel, le sucre et l’oeuf. P\'etrissez la p\^ate à la main ou au robot, jusqu’à ce qu’elle soit homog\`ene. Incorporez le beurre et travaillez jusqu’à ce que la p\^ate se d\'etache et ne colle plus aux mains (ou 5 min au robot).

\item Laissez reposer sous un linge 45 min dans un endroit ti\`ede. La p\^ate doit doubler de volume.

\item  P\'etrissez la p\^ate 30 secondes pour la faire retomber. Beurrez un moule rectangulaire et garnissez-le avec la p\^ate, en appuyant pour faire adh\'erer.

\item  Allumez le four th. 7 (210° C). Laissez la p\^ate au ti\`ede pendant que vous pr\'eparez la garniture.

\item  S\'eparez les blancs des jaunes d’oeufs. Fouettez les jaunes avec 50 g de sucre jusqu’à ce que le m\'elange blanchisse. Ajoutez la poudre d’amandes, m\'elangez. Battez les blancs en neige ferme avec le reste du sucre et incorporez d\'elicatement cette mousse dans la masse pr\'ec\'edente.

\item  Etalez cette pr\'eparation sur le fond de tarte, puis r\'epartissez les framboises à la surface. Elles vont s’enfoncer dans la cr\`eme au cours de la cuisson. Enfournez la tarte à mi-hauteur et faites cuire pendant environ 40 min.

\item  Laissez refroidir 5 à 10 min dans le moule, pus d\'emoulez la tarte sur une grille. Poudrez de sucre glace avant de d\'eguster. 
\end{enumerate}
\subsection*{\textsc{Conseil~:}}

								% <-- x1
\section[\normalsize{Tarte mousseuse aux framboises}]{\LARGE{\textsc{Tarte mousseuse aux framboises}}}		% <-- x2


\begin{itemize}
\item Pour 6 personnes
\item Préparation : 45 min
\item Repos : 1 h
\item Cuisson : 40 min
\end{itemize}

\subsection*{\textsc{Ingr\'edients~:}}

Pour la p\^ate :
\begin{itemize}
\item	250 g de farine
\item	10 g de levure de boulanger
\item	1 oeuf
\item	40 g de sucre
\item	10 cl de lait
\item	un quart de cuil. à caf\'e de sel
\item	100 g de beurre ramolli
\end{itemize}
Pour la garniture :
\begin{itemize}
\item	300 g de framboises \index{framboises}
\item	100 g de sucre
\item	3 oeufs
\item	1 cuil. à soupe de sucre glace
\item	125 g d’amandes en poudre
\end{itemize}


\subsection*{\textsc{Marche \`a suivre~:}}

\begin{enumerate}
\item  Dans une terrine, d\'elayez la levure dans 5 cl de lait ti\`ede et une pinc\'ee de sucre. Ajoutez 60 g de farine et m\'elangez pour obtenir une sorte de bouillie. Couvrez d’un linge et laissez lever 15 min dans un endroit ti\`ede.

\item  Versez le reste de farine sur le levain, ajoutez le reste de lait ti\`ede, le sel, le sucre et l’oeuf. P\'etrissez la p\^ate à la main ou au robot, jusqu’à ce qu’elle soit homog\`ene. Incorporez le beurre et travaillez jusqu’à ce que la p\^ate se d\'etache et ne colle plus aux mains (ou 5 min au robot).

\item Laissez reposer sous un linge 45 min dans un endroit ti\`ede. La p\^ate doit doubler de volume.

\item  P\'etrissez la p\^ate 30 secondes pour la faire retomber. Beurrez un moule rectangulaire et garnissez-le avec la p\^ate, en appuyant pour faire adh\'erer.

\item  Allumez le four th. 7 (210° C). Laissez la p\^ate au ti\`ede pendant que vous pr\'eparez la garniture.

\item  S\'eparez les blancs des jaunes d’oeufs. Fouettez les jaunes avec 50 g de sucre jusqu’à ce que le m\'elange blanchisse. Ajoutez la poudre d’amandes, m\'elangez. Battez les blancs en neige ferme avec le reste du sucre et incorporez d\'elicatement cette mousse dans la masse pr\'ec\'edente.

\item  Etalez cette pr\'eparation sur le fond de tarte, puis r\'epartissez les framboises à la surface. Elles vont s’enfoncer dans la cr\`eme au cours de la cuisson. Enfournez la tarte à mi-hauteur et faites cuire pendant environ 40 min.

\item  Laissez refroidir 5 à 10 min dans le moule, pus d\'emoulez la tarte sur une grille. Poudrez de sucre glace avant de d\'eguster. 
\end{enumerate}
\subsection*{\textsc{Conseil~:}}

