% Nom de la recette à entrer entre les accolades {}
\section{Galette au sucre}

\begin{ingredients}
% Ici lister les ingrédients 
% Changer de ligne pour chaque ingrédient et commencer la ligne par : \item
% rajouter autant de ligne que d'ingrédient
\item 250 g de farine
\item 1 cuillère à soupe de sucre
\item 2 oeufs
\item 1/2 verre de lait tiède
\item 75 g de beurre
\item 20 g de levure (ou 1 sachet)
\end{ingredients}
\begin{infos}
% Informations génériques
% Changer de ligne pour chaque et commencer par : \item
% Mettre une * si l'information n'est pas certaine 
\item Pour 6 personnes		% Nombre de personnes qu'on pourra nourrir ! :)
\item Préparation : 30 min*		% Temps de préparation (sans la cuisson)
\item Cuisson : 20 min			% Temps de cuisson
\end{infos}
\begin{etapes}
% Ici les étapes à réaliser
% Une étape par ligne, chaque ligne commence par un \item
% Pour exemple les étapes pour faire un millas ;)
\item Préchauffez votre four à 180 °C (thermostat 6).  
\item Mettre la farine, le sucre, les oeufs, la levure, le lait et le sel. Pétrir jusqu'à ce que la pâte se détache des parois.
\item Ajouter le beurre mou. Mélanger et laisser lever (combien de temps ?) dans un endroit tiède.
\item Mettre la pâte dans un moule à tarte beurré. Saupoudrer de sucre et parsemer de noisettes de beurre.
\item Faire cuire au four 20 minutes à 180°C
\end{etapes}
\begin{conseils}
% Ici écrire les conseils concernant la recette 
\end{conseils}
