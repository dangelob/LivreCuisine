\section[\normalsize{R\^oti de porc au miel}]{\LARGE{\textsc{R\^oti de porc au miel}}}


\begin{itemize}
\item Pour 6 personnes
\item Préparation : 15 min	
\item Cuisson : 50 min
\end{itemize}
\subsection*{\textsc{Ingrédients~:}}

\begin{itemize}
\item 1 rôti de porc\index{porc} d’environ 1 kg
\item 2 cuill\`ere \`a soupe de miel\index{miel} liquide
\item	30 g de beurre
\item	15 cl de Noilly Prat
\item 1 brin de thym
\item Noix de muscade
\item Sel, poivre
\end{itemize}


\subsection*{\textsc{Marche \`a suivre~:}}

\begin{enumerate}
\item Pr\'echauffez le four sur th.6 (180° C). Enduisez le r\^oti avec le beurre ramolli, salez-le, poivrez-le, puis parsemez-le de thym effeuill\'e et d’un peu de muscade fra\^ichement  r\^ap\'ee.

\item D\'eposez le r\^oti dans un plat, enfournez et cuisez pendant 35 min environ en l’arrosant régulièrement de son jus. A mi-cuisson, ajoutez le Noilly Prat.

\item Ouvrez le four, nappez le r\^oti avec le miel et continuez de cuire 15 min en l’arrosant deux fois avec son jus. Présentez-le d\'eficel\'e et coup\'e en tranches.
\end{enumerate}


\subsection*{\textsc{Conseil~:}}

Faites cuire votre r\^oti dans un plat compatible avec une cuisson au four juste assez grand pour le contenir ; vous \'eviterez ainsi au jus de br\^uler.

Le bon vin : un \emph{gewurztraminer}.
