% Nom de la recette à entrer entre les accolades {}
\section{Salade croquante de fenouil, pomme et pécorino}

\begin{ingredients}
% Ici lister les ingrédients 
% Changer de ligne pour chaque ingrédient et commencer la ligne par : \item
% rajouter autant de ligne que d'ingrédient
\item 2 gros bulbes de fenouil
\item 2 belles pommes
\item 130 g de pécorino 
\item 3 cuillères à soupe d'hule d'olive
\item 60 g de pignons
\item sel
\end{ingredients}
\begin{infos}
% Informations génériques
% Changer de ligne pour chaque et commencer par : \item
% Mettre une * si l'information n'est pas certaine 
\item Pour 4/6 personnes		% Nombre de personnes qu'on pourra nourrir ! :)
\item Préparation : 20 min		% Temps de préparation (sans la cuisson)
\item Cuisson : 3 min			% Temps de cuisson
\end{infos}
\begin{etapes}
% Ici les étapes à réaliser
% Une étape par ligne, chaque ligne commence par un \item
% Pour exemple les étapes pour faire un millas ;)
\item Faire dorer les pignons quelques minutes dans le four, à 180°C, puis les laisser refroidir.
\item Laver les fenouils. Couper les tiges vertes ainsi que la base. Émincer finement les 2 bulbes de façon à obtenir des petits morceaux. Lavez les pommes et les détailler de la même manière.
\item Retirer la croûte du pécorino et le couper en bâtonnets. Réunir les morceaux de fenouil, pomme et fromage dans un saladier.
\item Mélanger, dans un ramequin, 3 cuillères à soupe d'huile d'olive avec 2 cuillères à soupe de jus de citron. Saler un peu. Verser la sauce sur la salade et remuez bien.
\item Parsemer les pignons dorés sur le dessus et déguster.
\end{etapes}
\begin{conseils}
% Ici écrire les conseils concernant la recette 
Cette salade fraîche et croquante peut se préparer à l'avance avec sa vinaigrette (le jus de citron évitera que le fenouil et la pomme ne noircissent). Dans ce cas, conservez-là au frais dans une boite hermétique et n'ajoutez que le fromage et les pignons qu'au dernier moment.

Choisissez plutôt des pommes rouges et ne les pelez pas. Leur peau apportera de la couleur à la salade. Si vous aimez les saveurs plus acidulées, vous pouvez aussi utiliser des pommes vertes.
\end{conseils}
