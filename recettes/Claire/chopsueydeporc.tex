%Chop Suey de porc											% <-- x1
% %Chop Suey de porc											% <-- x1
% %Chop Suey de porc											% <-- x1
% %Chop Suey de porc											% <-- x1
% \include{./recettes/Claire/chopsueydeporc}								% <-- x1
\section[\normalsize{Chop Suey de porc}]{\LARGE{\textsc{Chop Suey de porc}}}		% <-- x2


\begin{itemize}
\item Pour 4 personnes
\item Préparation : 30 min
\item Cuisson : ?? min
\end{itemize}

\subsection*{\textsc{Ingr\'edients~:}}

\begin{itemize}
\item 300 g de filet de porc
\item 200 g d'oignons
\item 200 g de pousses de soja
\item 200 g de pousses de bambou en conserve
\item 6 champignons parfum\'es
\item Quelques brins de coriandre
\item 40 g de vermicelles de soja
\item 7 cuill\`eres \`a soupe d'huile
\item 1 cuill\`ere \`a soupe de vin chinois (ou xer\`es)
\item 2 cuill\`eres \`a soupe de sauce soja
\item sel
\end{itemize}


\subsection*{\textsc{Marche \`a suivre~:}}

\begin{enumerate}
\item Faire gonfler dans de l'eau chaude les champignons.
\item Peler et hacher les oignons, couper le porc en lani\`eres. Verser 4 cuill\`eres \`a soupe d'huile dans un wok. Quand l'huile est chaude, 
\item faire revenir les oignons 5 minutes. Augmenter le feu et ajouter la viande. Saler l\'eg\`erement et faire cuire 5 minutes en remuant. R\'eserver.
\item Nettoyer et rincer le soja, \'egoutter et rincer le bambou, rincer et ciseler la coriandre.
\item Faire cuire les vermicelles comme indiqu\'e sur le paquet. Rincer \`a l'eau froide.
\item Eponger les champignons et les couper en lamelles en \'eliminant les pieds. Verser l'huile restante dans le wok, faire chauffer et
\item mettre les vermicelles, le soja, le bambou et les champignons. Saler l\'eg\`erement, cuire 4 \`a 5 minutes en remuant sans arr\^et.
\item Ajouter la viande pour la faire r\'echauffer, puis le vin chinois et la sauce soja. Bien m\'elanger. Parsemer de coriandre et servir.
\end{enumerate}
\subsection*{\textsc{Conseil~:}}

								% <-- x1
\section[\normalsize{Chop Suey de porc}]{\LARGE{\textsc{Chop Suey de porc}}}		% <-- x2


\begin{itemize}
\item Pour 4 personnes
\item Préparation : 30 min
\item Cuisson : ?? min
\end{itemize}

\subsection*{\textsc{Ingr\'edients~:}}

\begin{itemize}
\item 300 g de filet de porc
\item 200 g d'oignons
\item 200 g de pousses de soja
\item 200 g de pousses de bambou en conserve
\item 6 champignons parfum\'es
\item Quelques brins de coriandre
\item 40 g de vermicelles de soja
\item 7 cuill\`eres \`a soupe d'huile
\item 1 cuill\`ere \`a soupe de vin chinois (ou xer\`es)
\item 2 cuill\`eres \`a soupe de sauce soja
\item sel
\end{itemize}


\subsection*{\textsc{Marche \`a suivre~:}}

\begin{enumerate}
\item Faire gonfler dans de l'eau chaude les champignons.
\item Peler et hacher les oignons, couper le porc en lani\`eres. Verser 4 cuill\`eres \`a soupe d'huile dans un wok. Quand l'huile est chaude, 
\item faire revenir les oignons 5 minutes. Augmenter le feu et ajouter la viande. Saler l\'eg\`erement et faire cuire 5 minutes en remuant. R\'eserver.
\item Nettoyer et rincer le soja, \'egoutter et rincer le bambou, rincer et ciseler la coriandre.
\item Faire cuire les vermicelles comme indiqu\'e sur le paquet. Rincer \`a l'eau froide.
\item Eponger les champignons et les couper en lamelles en \'eliminant les pieds. Verser l'huile restante dans le wok, faire chauffer et
\item mettre les vermicelles, le soja, le bambou et les champignons. Saler l\'eg\`erement, cuire 4 \`a 5 minutes en remuant sans arr\^et.
\item Ajouter la viande pour la faire r\'echauffer, puis le vin chinois et la sauce soja. Bien m\'elanger. Parsemer de coriandre et servir.
\end{enumerate}
\subsection*{\textsc{Conseil~:}}

								% <-- x1
\section[\normalsize{Chop Suey de porc}]{\LARGE{\textsc{Chop Suey de porc}}}		% <-- x2


\begin{itemize}
\item Pour 4 personnes
\item Préparation : 30 min
\item Cuisson : ?? min
\end{itemize}

\subsection*{\textsc{Ingr\'edients~:}}

\begin{itemize}
\item 300 g de filet de porc
\item 200 g d'oignons
\item 200 g de pousses de soja
\item 200 g de pousses de bambou en conserve
\item 6 champignons parfum\'es
\item Quelques brins de coriandre
\item 40 g de vermicelles de soja
\item 7 cuill\`eres \`a soupe d'huile
\item 1 cuill\`ere \`a soupe de vin chinois (ou xer\`es)
\item 2 cuill\`eres \`a soupe de sauce soja
\item sel
\end{itemize}


\subsection*{\textsc{Marche \`a suivre~:}}

\begin{enumerate}
\item Faire gonfler dans de l'eau chaude les champignons.
\item Peler et hacher les oignons, couper le porc en lani\`eres. Verser 4 cuill\`eres \`a soupe d'huile dans un wok. Quand l'huile est chaude, 
\item faire revenir les oignons 5 minutes. Augmenter le feu et ajouter la viande. Saler l\'eg\`erement et faire cuire 5 minutes en remuant. R\'eserver.
\item Nettoyer et rincer le soja, \'egoutter et rincer le bambou, rincer et ciseler la coriandre.
\item Faire cuire les vermicelles comme indiqu\'e sur le paquet. Rincer \`a l'eau froide.
\item Eponger les champignons et les couper en lamelles en \'eliminant les pieds. Verser l'huile restante dans le wok, faire chauffer et
\item mettre les vermicelles, le soja, le bambou et les champignons. Saler l\'eg\`erement, cuire 4 \`a 5 minutes en remuant sans arr\^et.
\item Ajouter la viande pour la faire r\'echauffer, puis le vin chinois et la sauce soja. Bien m\'elanger. Parsemer de coriandre et servir.
\end{enumerate}
\subsection*{\textsc{Conseil~:}}

								% <-- x1
\section[\normalsize{Chop Suey de porc}]{\LARGE{\textsc{Chop Suey de porc}}}		% <-- x2


\begin{itemize}
\item Pour 4 personnes
\item Préparation : 30 min
\item Cuisson : ?? min
\end{itemize}

\subsection*{\textsc{Ingr\'edients~:}}

\begin{itemize}
\item 300 g de filet de porc
\item 200 g d'oignons
\item 200 g de pousses de soja
\item 200 g de pousses de bambou en conserve
\item 6 champignons parfum\'es
\item Quelques brins de coriandre
\item 40 g de vermicelles de soja
\item 7 cuill\`eres \`a soupe d'huile
\item 1 cuill\`ere \`a soupe de vin chinois (ou xer\`es)
\item 2 cuill\`eres \`a soupe de sauce soja
\item sel
\end{itemize}


\subsection*{\textsc{Marche \`a suivre~:}}

\begin{enumerate}
\item Faire gonfler dans de l'eau chaude les champignons.
\item Peler et hacher les oignons, couper le porc en lani\`eres. Verser 4 cuill\`eres \`a soupe d'huile dans un wok. Quand l'huile est chaude, 
\item faire revenir les oignons 5 minutes. Augmenter le feu et ajouter la viande. Saler l\'eg\`erement et faire cuire 5 minutes en remuant. R\'eserver.
\item Nettoyer et rincer le soja, \'egoutter et rincer le bambou, rincer et ciseler la coriandre.
\item Faire cuire les vermicelles comme indiqu\'e sur le paquet. Rincer \`a l'eau froide.
\item Eponger les champignons et les couper en lamelles en \'eliminant les pieds. Verser l'huile restante dans le wok, faire chauffer et
\item mettre les vermicelles, le soja, le bambou et les champignons. Saler l\'eg\`erement, cuire 4 \`a 5 minutes en remuant sans arr\^et.
\item Ajouter la viande pour la faire r\'echauffer, puis le vin chinois et la sauce soja. Bien m\'elanger. Parsemer de coriandre et servir.
\end{enumerate}
\subsection*{\textsc{Conseil~:}}

