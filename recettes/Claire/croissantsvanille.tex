% Nom de la recette à entrer entre les accolades {}
\section{Croissants à la vanille (Vanillekipferl)}

\begin{ingredients}
% Ici lister les ingrédients 
% Changer de ligne pour chaque ingrédient et commencer la ligne par : \item
% rajouter autant de ligne que d'ingrédient
%\begin{itemize}
\item 250 g de farine
\item 180 g de beurre
\item 100 g de poudre d'amandes
\item 65 g de sucre
\item 1 cuillère à café d'extrait de vanille
\item 1 pincée de sel
\item sucre vanillé (2 à 3 sachets) et sucre glace pour la finition
%\end{itemize}
\end{ingredients}
\begin{infos}
% Informations génériques
% Changer de ligne pour chaque et commencer par : \item
% Mettre une * si l'information n'est pas certaine 
\item Pour 40 à 50 pièces		% Nombre de personnes qu'on pourra nourrir ! :)
\item Préparation : 30 min + 1h de repos		% Temps de préparation (sans la cuisson)
\item Cuisson : 20 à 30 min		% Temps de cuisson
\end{infos}
\begin{etapes}
% Ici les étapes à réaliser
% Une étape par ligne, chaque ligne commence par un \item
% Pour exemple les étapes pour faire un millas ;)
\item Pétrir à la main la farine, le beurre coupé en dés, le sucre, les amandes, la vanille et le sel de manière à obtenir une pâte homogène. Ajouter un petit peu d'eau si la pâte est trop friable.
\item Rouler en boule et mettre au réfrigérateur pendant 1 heure au moins.
\item Préchauffer le four à 170°C.
\item Prélever de petits morceaux de pâte, les rouler en boudin en faisant les extrémités plus fines et les façonner en forme de petits croissants. Les déposer sur une plaque en les espaçant un peu.
\item Faire cuire 10 à 12 min jusqu'à ce qu'ils soient très légèrement dorés. Les déposer sur une grille à la sortie du four. Attention ils cassent facilement !
\item Mélanger dans une assiette creuse du sucre glace et du sucre vanillé (environ 2-3 cuillères à soupe pour un sachet). Passer les croissants encore tièdes dans ce mélange, seulement sur le dessus ou entièrement selon qu'on les aime plus ou moins sucrés.
\item Finir de laisser refroidir et ranger dans une boîte en métal.
\end{etapes}
\begin{conseils}
% Ici écrire les conseils concernant la recette 
\end{conseils}
