% Poulet basquaise (façon Claire T.)											% <-- x1
% % Poulet basquaise (façon Claire T.)											% <-- x1
% % Poulet basquaise (façon Claire T.)											% <-- x1
% % Poulet basquaise (façon Claire T.)											% <-- x1
% \include{./recettes/Claire/pouletbasquaisefaconclaire}						% <-- x1
\section[\normalsize{Poulet basquaise (fa\c con Claire T.)}]{\LARGE{\textsc{(fa\c con Claire T.)}}}		% <-- x2


\begin{itemize}
\item Pour 6 personnes
\item Préparation : 35 min
\item Cuisson : 55 min
\end{itemize}

\subsection*{\textsc{Ingr\'edients~:}}

\begin{itemize}
\item 6 morceaux de poulet
\item 1 kg de tomates
\item 700 g de poivrons (verts et rouges)
\item 3 oignons \'eminc\'es
\item 3 gousses d'ail
\item 1 verre de vin blanc
\item 1 bouquet garni, 
\item huile d'olive, 
\item poivre, sel
\end{itemize}


\subsection*{\textsc{Marche \`a suivre~:}}

\begin{enumerate}
\item Dans une cocotte, faire dorer dans l'huile d'olive les morceaux de poulet sal\'es et poivr\'es. R\'eserver.
\item Faire chauffer 4 cuill\`eres \`a soupe d'huile, y faire dorer les oignons, l'ail press\'e, les poivrons taill\'es en lani\`eres. Laisser cuire 5 min.
\item Laver, \'eplucher et couper les tomates en morceaux, les ajouter \`a la cocotte, sel, poivre. Couvrir et laisser mijoter 20 min.
\item Ajouter le poulet aux l\'egumes, ajouter le bouquet garni et le vin blanc, couvrir et laisser cuire \`a feu tr\`es doux 35 min. 
\end{enumerate}
\subsection*{\textsc{Conseil~:}}
Attention \`a la cuisson, ça accroche vite...
						% <-- x1
\section[\normalsize{Poulet basquaise (fa\c con Claire T.)}]{\LARGE{\textsc{(fa\c con Claire T.)}}}		% <-- x2


\begin{itemize}
\item Pour 6 personnes
\item Préparation : 35 min
\item Cuisson : 55 min
\end{itemize}

\subsection*{\textsc{Ingr\'edients~:}}

\begin{itemize}
\item 6 morceaux de poulet
\item 1 kg de tomates
\item 700 g de poivrons (verts et rouges)
\item 3 oignons \'eminc\'es
\item 3 gousses d'ail
\item 1 verre de vin blanc
\item 1 bouquet garni, 
\item huile d'olive, 
\item poivre, sel
\end{itemize}


\subsection*{\textsc{Marche \`a suivre~:}}

\begin{enumerate}
\item Dans une cocotte, faire dorer dans l'huile d'olive les morceaux de poulet sal\'es et poivr\'es. R\'eserver.
\item Faire chauffer 4 cuill\`eres \`a soupe d'huile, y faire dorer les oignons, l'ail press\'e, les poivrons taill\'es en lani\`eres. Laisser cuire 5 min.
\item Laver, \'eplucher et couper les tomates en morceaux, les ajouter \`a la cocotte, sel, poivre. Couvrir et laisser mijoter 20 min.
\item Ajouter le poulet aux l\'egumes, ajouter le bouquet garni et le vin blanc, couvrir et laisser cuire \`a feu tr\`es doux 35 min. 
\end{enumerate}
\subsection*{\textsc{Conseil~:}}
Attention \`a la cuisson, ça accroche vite...
						% <-- x1
\section[\normalsize{Poulet basquaise (fa\c con Claire T.)}]{\LARGE{\textsc{(fa\c con Claire T.)}}}		% <-- x2


\begin{itemize}
\item Pour 6 personnes
\item Préparation : 35 min
\item Cuisson : 55 min
\end{itemize}

\subsection*{\textsc{Ingr\'edients~:}}

\begin{itemize}
\item 6 morceaux de poulet
\item 1 kg de tomates
\item 700 g de poivrons (verts et rouges)
\item 3 oignons \'eminc\'es
\item 3 gousses d'ail
\item 1 verre de vin blanc
\item 1 bouquet garni, 
\item huile d'olive, 
\item poivre, sel
\end{itemize}


\subsection*{\textsc{Marche \`a suivre~:}}

\begin{enumerate}
\item Dans une cocotte, faire dorer dans l'huile d'olive les morceaux de poulet sal\'es et poivr\'es. R\'eserver.
\item Faire chauffer 4 cuill\`eres \`a soupe d'huile, y faire dorer les oignons, l'ail press\'e, les poivrons taill\'es en lani\`eres. Laisser cuire 5 min.
\item Laver, \'eplucher et couper les tomates en morceaux, les ajouter \`a la cocotte, sel, poivre. Couvrir et laisser mijoter 20 min.
\item Ajouter le poulet aux l\'egumes, ajouter le bouquet garni et le vin blanc, couvrir et laisser cuire \`a feu tr\`es doux 35 min. 
\end{enumerate}
\subsection*{\textsc{Conseil~:}}
Attention \`a la cuisson, ça accroche vite...
						% <-- x1
\section[\normalsize{Poulet basquaise (fa\c con Claire T.)}]{\LARGE{\textsc{(fa\c con Claire T.)}}}		% <-- x2


\begin{itemize}
\item Pour 6 personnes
\item Préparation : 35 min
\item Cuisson : 55 min
\end{itemize}

\subsection*{\textsc{Ingr\'edients~:}}

\begin{itemize}
\item 6 morceaux de poulet \index{poulet}
\item 1 kg de tomates \index{tomate}
\item 700 g de poivrons (verts et rouges) \index{poivron}
\item 3 oignons \'eminc\'es \index{oignon}
\item 3 gousses d'ail 
\item 1 verre de vin blanc
\item 1 bouquet garni, 
\item huile d'olive, 
\item poivre, sel
\end{itemize}


\subsection*{\textsc{Marche \`a suivre~:}}

\begin{enumerate}
\item Dans une cocotte, faire dorer dans l'huile d'olive les morceaux de poulet sal\'es et poivr\'es. R\'eserver.
\item Faire chauffer 4 cuill\`eres \`a soupe d'huile, y faire dorer les oignons, l'ail press\'e, les poivrons taill\'es en lani\`eres. Laisser cuire 5 min.
\item Laver, \'eplucher et couper les tomates en morceaux, les ajouter \`a la cocotte, sel, poivre. Couvrir et laisser mijoter 20 min.
\item Ajouter le poulet aux l\'egumes, ajouter le bouquet garni et le vin blanc, couvrir et laisser cuire \`a feu tr\`es doux 35 min. 
\end{enumerate}
\subsection*{\textsc{Conseil~:}}
Attention \`a la cuisson, ça accroche vite...
