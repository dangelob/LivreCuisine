% Nom de la recette à entrer entre les accolades {}
\section{Velouté de topinambours au chèvre frais}

\begin{ingredients}
\item 800 g de topinambours
\item 1 petit oignon
\item bouillon de légumes
\item 200 ml de lait
\item beurre
\item sel, poivre
\item 50 g de chèvre frais
\item crème fraîche (selon l'envie)
\item 1 bonne poignée de noisettes
\end{ingredients}
\begin{infos}
\item Pour 4 personnes		% Nombre de personnes qu'on pourra nourrir ! :)
\item Préparation : 40 min		% Temps de préparation (sans la cuisson)
\item Cuisson : 15 min			% Temps de cuisson
\end{infos}
\begin{etapes}
\item Laver et peler les topinambours, les couper en petits morceaux. Émincer l'oignon.
\item Dans une casserole, faire chauffer une noix de beurre. Y faire revenir l'oignon et le topinambour pendant 1 à 2 minutes sur feux doux. Mouiller avec le bouillon de légumes (recouvrir) et laisser mijoter pendant environ 15 minutes sur feux doux, pour que le topinambour soit cuit (piquer au couteau, il ne doit pas rester sur la lame). Rajouter un peu de bouillon pendant la cuisson si besoin.
\item Pendant ce temps, concasser les noisettes et les torréfier à sec dans une poêle.
\item Réserver le bouillon. Mixer le topinambour, ajouter le lait, puis progressivement un peu de bouillon jusqu'à obtenir la consistance désirée. Ajouter le chèvre, la crème et bien mixer. Rectifier l'assaisonnement au besoin.
\item Servir parsemé de noisettes concassées avec éventuellement un filet d'huile de noisette.
\end{etapes}
\begin{conseils}
Ne pas hésiter à préparer ce plat à l'avance, comme tous les plats mijotés c'est encore meilleur réchauffé.
\end{conseils}




