\section[\normalsize{Soupe de navets, lard croustillant et croûtons}]{Soupe de navets, lard croustillant et croûtons}

\begin{ingredients}
\item 800 g de navets (nouveaux)
\item 260 g de pommes de terre
\item 1 oignon
\item 1/2 l de bouillon de volaille
\item 10 \`a 15 cl de lait
\item 10 cl de cr\`eme fra\^iche \'epaisse
\item sel, poivre
\item 4 fines tranches de pancetta
\item 2 tranches de pain de campagne (rassis)
\item 1 noisette de beurre
\end{ingredients}
\begin{infos}
\item Pour 4 personnes
\item Préparation : 25 min
\item Cuisson : 35 min
\end{infos}
\begin{etapes}
\item Peler les navets et les pommes de terre, les couper en morceaux. 
\item Émincer l'oignon finement. 
\item Faire revenir l'oignon dans un peu de beurre, \`a feu doux, sans le faire colorer.
\item Ajouter les pommes de terre et les navets, puis mouiller avec le bouillon de volaille. Faire cuire \`a feu doux et \`a couvert pendant 30 \`a 35 minutes, jusqu'\`a ce que les l\'egumes soient tendres.
\item Mixer le tout, ajouter le lait (la quantit\'e peut varier en fonction de la consistance d\'esir\'ee), la cr\`eme fra\^iche, rectifier en sel et poivre, r\'eserver au chaud.
\item D\'ecouper le pain de campagne en petits d\'es et la pancetta en morceaux. Faire chauffer une po\^ele, y d\'eposer le pain (sans huile), laisser dess\'echer 3 minutes \`a feu vif, en remuant constamment. 
\item Ajouter alors la pancetta, la faire dorer \`a feu vif pour qu'elle devienne bien croustillante, tout en continuant \`a remuer pour que les croûtons s'impr\`egnent de la graisse de cuisson.
\item Servir la soupe dans des petits bols ou des assiettes creuses et r\'epartir les croûtons et la pancetta sur le dessus.
\end{etapes}
\begin{conseils}
\end{conseils}
