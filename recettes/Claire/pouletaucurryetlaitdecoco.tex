\section[\normalsize{Poulet au curry et lait de coco}]{Poulet au curry et lait de coco}

\begin{ingredients}
\item 4 blancs de poulet \index{poulet}
\item 1 oignon \index{oignon}
\item 4 gousses d'ail
\item une bo\^ite de lait de coco\index{lait de coco}
\item curry en poudre
\item 1 c. \`a soupe de gingembre hach\'e (facultatif)
\item le jus d'1 citron vert
\item huile
\item sel, poivre
\end{ingredients}
\begin{infos}
\item Pour 4 personnes
\item Préparation : 45 min
\item Cuisson : ?? min
\end{infos}
\begin{etapes}
\item Pr\'eparer les ingr\'edients : couper les blancs de poulet en lani\`eres, \'emincer l'oignon, \'eplucher les gousses d'ail (les hacher si on ne dispose pas d'un presse ail).
\item Faire chauffer un peu d'huile dans le wok. Faire revenir la viande en plusieurs fois ; elle doit \^etre saisie et presque cuite. R\'eserver. 
\item Remettre un peu d'huile et faire revenir l'oignon. Lorsqu'il est est tendre, ajouter le poulet et les gousses d'ail hach\'ees. Bien m\'elanger, saler, poivrer.
\item Ajouter environ quatre cuill\`eres \`a caf\'e de curry (en mettre selon son goût), le gingembre et m\'elanger, puis verser les 3/4 du jus de citron vert environ (ne pas tout mettre afin de pouvoir rectifier l'assaisonnement si n\'ecessaire). Lorsque la viande est cuite, ajouter le lait de coco (ne pas tout mettre non 
plus).
\item Goûter et rectifier si besoin en ajoutant du jus de citron, du lait de coco ou du curry. Laisser mijoter un peu et servir bien chaud avec du riz cuit \`a la vapeur.
\end{etapes}
\begin{conseils}
\end{conseils}

