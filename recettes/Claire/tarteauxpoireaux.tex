% Tarte aux poireaux										% <-- x1
% % Tarte aux poireaux										% <-- x1
% % Tarte aux poireaux										% <-- x1
% % Tarte aux poireaux										% <-- x1
% \include{./recettes/Claire/tarteauxpoireaux}								% <-- x1
\section[\normalsize{Tarte aux poireaux}]{\LARGE{\textsc{Tarte aux poireaux}}}		% <-- x2


\begin{itemize}
\item Pour 6 personnes
\item Préparation : 20 min
\item Cuisson : 40 min
\end{itemize}

\subsection*{\textsc{Ingr\'edients~:}}

\begin{itemize}
\item 1 p\^ate bris\'ee
\item 2 \`a 3 poireaux
\item 100 g de lardons
\item 1 petite cuill\`ere de farine
\item 4 oeufs
\item 200 g de cr\`eme fra\^iche
\item poivre 
\end{itemize}


\subsection*{\textsc{Marche \`a suivre~:}}

\begin{enumerate}
\item Pr\'echauffer le four th 6/7.

\item Laver et \'emincer les poireaux. Faire revenir les lardons dans une pôele. Lorsqu'ils commencent \`a fondre, ajouter les poireaux. Quand ils commencent \`a colorer, lier avec une petite cuill\`ere de farine. Laisser revenir \`a feu doux.

\item Battre dans un bol les oeufs, la cr\`eme fra\^iche et du poivre.

\item Chemiser le moule \`a tarte, ajouter la fondue de poireaux. Verser l'appareil oeufs et cr\`eme par dessus. 

\item Enfourner et faire cuire 40 minutes environ.
\end{enumerate}
\subsection*{\textsc{Conseil~:}}

								% <-- x1
\section[\normalsize{Tarte aux poireaux}]{\LARGE{\textsc{Tarte aux poireaux}}}		% <-- x2


\begin{itemize}
\item Pour 6 personnes
\item Préparation : 20 min
\item Cuisson : 40 min
\end{itemize}

\subsection*{\textsc{Ingr\'edients~:}}

\begin{itemize}
\item 1 p\^ate bris\'ee
\item 2 \`a 3 poireaux
\item 100 g de lardons
\item 1 petite cuill\`ere de farine
\item 4 oeufs
\item 200 g de cr\`eme fra\^iche
\item poivre 
\end{itemize}


\subsection*{\textsc{Marche \`a suivre~:}}

\begin{enumerate}
\item Pr\'echauffer le four th 6/7.

\item Laver et \'emincer les poireaux. Faire revenir les lardons dans une pôele. Lorsqu'ils commencent \`a fondre, ajouter les poireaux. Quand ils commencent \`a colorer, lier avec une petite cuill\`ere de farine. Laisser revenir \`a feu doux.

\item Battre dans un bol les oeufs, la cr\`eme fra\^iche et du poivre.

\item Chemiser le moule \`a tarte, ajouter la fondue de poireaux. Verser l'appareil oeufs et cr\`eme par dessus. 

\item Enfourner et faire cuire 40 minutes environ.
\end{enumerate}
\subsection*{\textsc{Conseil~:}}

								% <-- x1
\section[\normalsize{Tarte aux poireaux}]{\LARGE{\textsc{Tarte aux poireaux}}}		% <-- x2


\begin{itemize}
\item Pour 6 personnes
\item Préparation : 20 min
\item Cuisson : 40 min
\end{itemize}

\subsection*{\textsc{Ingr\'edients~:}}

\begin{itemize}
\item 1 p\^ate bris\'ee
\item 2 \`a 3 poireaux
\item 100 g de lardons
\item 1 petite cuill\`ere de farine
\item 4 oeufs
\item 200 g de cr\`eme fra\^iche
\item poivre 
\end{itemize}


\subsection*{\textsc{Marche \`a suivre~:}}

\begin{enumerate}
\item Pr\'echauffer le four th 6/7.

\item Laver et \'emincer les poireaux. Faire revenir les lardons dans une pôele. Lorsqu'ils commencent \`a fondre, ajouter les poireaux. Quand ils commencent \`a colorer, lier avec une petite cuill\`ere de farine. Laisser revenir \`a feu doux.

\item Battre dans un bol les oeufs, la cr\`eme fra\^iche et du poivre.

\item Chemiser le moule \`a tarte, ajouter la fondue de poireaux. Verser l'appareil oeufs et cr\`eme par dessus. 

\item Enfourner et faire cuire 40 minutes environ.
\end{enumerate}
\subsection*{\textsc{Conseil~:}}

								% <-- x1
\section[\normalsize{Tarte aux poireaux}]{\LARGE{\textsc{Tarte aux poireaux}}}		% <-- x2


\begin{itemize}
\item Pour 6 personnes
\item Préparation : 20 min
\item Cuisson : 40 min
\end{itemize}

\subsection*{\textsc{Ingr\'edients~:}}

\begin{itemize}
\item 1 p\^ate bris\'ee
\item 2 \`a 3 poireaux
\item 100 g de lardons
\item 1 petite cuill\`ere de farine
\item 4 oeufs
\item 200 g de cr\`eme fra\^iche
\item poivre 
\end{itemize}


\subsection*{\textsc{Marche \`a suivre~:}}

\begin{enumerate}
\item Pr\'echauffer le four th 6/7.

\item Laver et \'emincer les poireaux. Faire revenir les lardons dans une pôele. Lorsqu'ils commencent \`a fondre, ajouter les poireaux. Quand ils commencent \`a colorer, lier avec une petite cuill\`ere de farine. Laisser revenir \`a feu doux.

\item Battre dans un bol les oeufs, la cr\`eme fra\^iche et du poivre.

\item Chemiser le moule \`a tarte, ajouter la fondue de poireaux. Verser l'appareil oeufs et cr\`eme par dessus. 

\item Enfourner et faire cuire 40 minutes environ.
\end{enumerate}
\subsection*{\textsc{Conseil~:}}

