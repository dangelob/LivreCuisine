\section[\normalsize{Tartiflette (façon Claire T.)}]{Tartiflette (façon Claire T.)}

\begin{ingredients}
\item 1 kg de pommes de terre \`a chair ferme
\item 1 reblochon fermier ou fruit\'e
\item 200g de lardons
\item 1 gros oignon
\item 2 cuill\`eres \`a soupe de cr\`eme fra\^iche
\item vin blanc de Savoie (Apremont)
\item 1 gousse d'ail
\end{ingredients}
\begin{infos}
\item Pour 6 personnes
\item Préparation : 50 min
\item Cuisson : 10 min
\end{infos}
\begin{etapes}
\item Faire cuire les pommes de terre dans de l'eau (d\'epart \`a froid) pendant 20 \`a 25 minutes. Egoutter et \'eplucher.
\item Pr\'echauffer le four \`a 200° C.
\item Emincer finement l'oignon. Faire fondre les lardons dans une po\^ele, puis ajouter les oignons (avant que les lardons ne colorent). 
\item Laisser revenir \`a feu doux.
\item Pendant ce temps, couper les pommes de terre en rondelles dans un saladier. Frotter un plat \`a gratin avec une gousse d'ail et le beurrer g\'en\'ereusement. 
\item Gratter le reblochon des deux côt\'es, ôter la pastille de cas\'eine et le couper en deux dans l'\'epaisseur.
\item Quand les lardons et les oignons ont bien fondu (attention \`a ne pas faire trop colorer), ajouter un peu de vin blanc et faire r\'eduire. 
\item Les ajouter aux pommes de terre. Ajouter la cr\`eme et m\'elanger.
\item Verser ce m\'elange dans le plat \`a gratin, arroser de vin blanc et poser les deux moiti\'es de reblochon par dessus, croûte vers le haut.
\item Enfourner une dizaine de minutes, jusqu'\`a ce que le fromage ait bien fondue.
\end{etapes}
\begin{conseils}
\end{conseils}
