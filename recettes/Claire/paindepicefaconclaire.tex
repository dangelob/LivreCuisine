% Pain d'\'epices											% <-- x1
% % Pain d'\'epices											% <-- x1
% % Pain d'\'epices											% <-- x1
% % Pain d'\'epices											% <-- x1
% \include{./recettes/Claire/paindepicefaconclaire}								% <-- x1
\section[\normalsize{Pain d'\'epices (fa\c con Claire T.)}]{\LARGE{\textsc{Pain d'\'epices (fa\c con Claire T.)}}}		% <-- x2


\begin{itemize}
\item Pour 8 personnes
\item Préparation : 20 min
\item Cuisson : 50 min
\end{itemize}

\subsection*{\textsc{Ingr\'edients~:}}

\begin{itemize}

\item 300 g de farine
\item 1/3 l de lait
\item 100 g de cassonnade
\item 100 g de miel
\item 1 yaourt
\item 2 c. \`a caf\'e rases de bicarbonate de soude
\item 1 cuill\`eres \`a caf\'e de cannelle
\item 1 cuill\`eres \`a caf\'e de 4 \'epices
\item 1 cuill\`ere \`a caf\'e de gingembre
\end{itemize}


\subsection*{\textsc{Marche \`a suivre~:}}

\begin{enumerate}

\item Faire pr\'echauffer le four \`a 160° C.
\item Faire fondre \`a feu tr\`es doux le sucre, le lait et le 
miel.
\item Dans une terrine, m\'elanger la farine, les \'epices et le 
yaourt. Verser le lait. 
\item Ajouter le bicarbonate de soude d\'elay\'e dans 2 cuill
\`eres \`a soupe d'eau chaude.
\item Mettre au four 25 minutes, puis 25 minutes \`a 200° C.


\end{enumerate}
\subsection*{\textsc{Conseil~:}}
Si pas de bicarbonate de soude, remplacer par 1/2 sachet de levure.
Ne pas h\'esiter \`a rajouter des \'epices si besoin.								% <-- x1
\section[\normalsize{Pain d'\'epices (fa\c con Claire T.)}]{\LARGE{\textsc{Pain d'\'epices (fa\c con Claire T.)}}}		% <-- x2


\begin{itemize}
\item Pour 8 personnes
\item Préparation : 20 min
\item Cuisson : 50 min
\end{itemize}

\subsection*{\textsc{Ingr\'edients~:}}

\begin{itemize}

\item 300 g de farine
\item 1/3 l de lait
\item 100 g de cassonnade
\item 100 g de miel
\item 1 yaourt
\item 2 c. \`a caf\'e rases de bicarbonate de soude
\item 1 cuill\`eres \`a caf\'e de cannelle
\item 1 cuill\`eres \`a caf\'e de 4 \'epices
\item 1 cuill\`ere \`a caf\'e de gingembre
\end{itemize}


\subsection*{\textsc{Marche \`a suivre~:}}

\begin{enumerate}

\item Faire pr\'echauffer le four \`a 160° C.
\item Faire fondre \`a feu tr\`es doux le sucre, le lait et le 
miel.
\item Dans une terrine, m\'elanger la farine, les \'epices et le 
yaourt. Verser le lait. 
\item Ajouter le bicarbonate de soude d\'elay\'e dans 2 cuill
\`eres \`a soupe d'eau chaude.
\item Mettre au four 25 minutes, puis 25 minutes \`a 200° C.


\end{enumerate}
\subsection*{\textsc{Conseil~:}}
Si pas de bicarbonate de soude, remplacer par 1/2 sachet de levure.
Ne pas h\'esiter \`a rajouter des \'epices si besoin.								% <-- x1
\section[\normalsize{Pain d'\'epices (fa\c con Claire T.)}]{\LARGE{\textsc{Pain d'\'epices (fa\c con Claire T.)}}}		% <-- x2


\begin{itemize}
\item Pour 8 personnes
\item Préparation : 20 min
\item Cuisson : 50 min
\end{itemize}

\subsection*{\textsc{Ingr\'edients~:}}

\begin{itemize}

\item 300 g de farine
\item 1/3 l de lait
\item 100 g de cassonnade
\item 100 g de miel
\item 1 yaourt
\item 2 c. \`a caf\'e rases de bicarbonate de soude
\item 1 cuill\`eres \`a caf\'e de cannelle
\item 1 cuill\`eres \`a caf\'e de 4 \'epices
\item 1 cuill\`ere \`a caf\'e de gingembre
\end{itemize}


\subsection*{\textsc{Marche \`a suivre~:}}

\begin{enumerate}

\item Faire pr\'echauffer le four \`a 160° C.
\item Faire fondre \`a feu tr\`es doux le sucre, le lait et le 
miel.
\item Dans une terrine, m\'elanger la farine, les \'epices et le 
yaourt. Verser le lait. 
\item Ajouter le bicarbonate de soude d\'elay\'e dans 2 cuill
\`eres \`a soupe d'eau chaude.
\item Mettre au four 25 minutes, puis 25 minutes \`a 200° C.


\end{enumerate}
\subsection*{\textsc{Conseil~:}}
Si pas de bicarbonate de soude, remplacer par 1/2 sachet de levure.
Ne pas h\'esiter \`a rajouter des \'epices si besoin.								% <-- x1
\section[\normalsize{Pain d'\'epices (fa\c con Claire T.)}]{\LARGE{\textsc{Pain d'\'epices (fa\c con Claire T.)}}}		% <-- x2


\begin{itemize}
\item Pour 8 personnes
\item Préparation : 20 min
\item Cuisson : 50 min
\end{itemize}

\subsection*{\textsc{Ingr\'edients~:}}

\begin{itemize}

\item 300 g de farine
\item 1/3 l de lait
\item 100 g de cassonnade
\item 100 g de miel
\item 1 yaourt
\item 2 c. \`a caf\'e rases de bicarbonate de soude
\item 1 cuill\`eres \`a caf\'e de cannelle
\item 1 cuill\`eres \`a caf\'e de 4 \'epices
\item 1 cuill\`ere \`a caf\'e de gingembre
\end{itemize}


\subsection*{\textsc{Marche \`a suivre~:}}

\begin{enumerate}

\item Faire pr\'echauffer le four \`a 160° C.
\item Faire fondre \`a feu tr\`es doux le sucre, le lait et le 
miel.
\item Dans une terrine, m\'elanger la farine, les \'epices et le 
yaourt. Verser le lait. 
\item Ajouter le bicarbonate de soude d\'elay\'e dans 2 cuill
\`eres \`a soupe d'eau chaude.
\item Mettre au four 25 minutes, puis 25 minutes \`a 200° C.


\end{enumerate}
\subsection*{\textsc{Conseil~:}}
Si pas de bicarbonate de soude, remplacer par 1/2 sachet de levure.
Ne pas h\'esiter \`a rajouter des \'epices si besoin.