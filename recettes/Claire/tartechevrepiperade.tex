% Tarte chèvre-piperade											% <-- x1
% \section[\normalsize{Tarte ch\`evre-piperade}]{Tarte ch\`evre-piperade}

\begin{ingredients}
\item 1 p\^ate bris\'ee
\item 300 g de poivrons en petits d\'es (surgel\'es)
\item 200 g de chair de tomates pel\'ees
\item 2 oeufs
\item 1 belle gousse d'ail
\item 1 petit oignon
\item 50 g de pancetta
\item 10 cl de cr\`eme
\item 1/2 bûche de ch\`evre
\item 1 cuill\`ere \`a caf\'e de piment d'Espelette 
\item quelques feuilles de basilic frais
\item 2 cuill\`eres \`a soupe d'huile d'olive
\end{ingredients}
\begin{infos}
\item Pour 6 personnes
\item Préparation : 45 min
\item Cuisson : 35 min
\end{infos}
\begin{etapes}
\item Pr\'echauffer le four \`a 200° C. 
\item Faire revenir l'oignon et l'ail dans 2 cuill\`eres \`a soupe d'huile d'olive. 
\item Ajouter les d\'es de poivron, la tomate et le piment d'Espelette, et laisser mijoter 15 minutes environ jusqu'\`a ce que les poivrons deviennent tendres. R\'eserver. 
\item Dans une autre po\^ele, faire revenir, \`a sec et \`a feu tr\`es vif, la pancetta coup\'ee en petis morceaux, jusqu'\`a ce qu'elle soit bien dor\'ee. 
\item Egoutter pour ôter l'exc\`es de gras et r\'eserver.
\item M\'elanger les l\'egumes et la pancetta. Tapisser un moule \`a tarte avec la p\^ate bris\'ee et la piquer. R\'epartir les l\'egumes dessus.
\item Battre les oeufs avec la cr\`eme dans un saladier. Saler l\'eg\`erement et verser sur la tarte. Garnir de rondelles de ch\`evre et de feuilles de basilic.
\item Baisser la temp\'erature du four \`a 180° C et enfourner pour 30 \`a 40 minutes. Servir ti\`ede ou froid.
\end{etapes}
\begin{conseils}
\end{conseils}
								% <-- x1
\section[\normalsize{Tarte ch\`evre-piperade}]{\LARGE{\textsc{Tarte ch\`evre-piperade}}}		% <-- x2


\begin{itemize}
\item Pour 6 personnes
\item Préparation : 45 min
\item Cuisson : 35 min
\end{itemize}

\subsection*{\textsc{Ingr\'edients~:}}

\begin{itemize}
\item 1 p\^ate bris\'ee
\item 300 g de poivrons en petits d\'es (surgel\'es)
\item 200 g de chair de tomates pel\'ees
\item 2 oeufs
\item 1 belle gousse d'ail
\item 1 petit oignon
\item 50 g de pancetta
\item 10 cl de cr\`eme
\item 1/2 bûche de ch\`evre
\item 1 cuill\`ere \`a caf\'e de piment d'Espelette 
\item quelques feuilles de basilic frais
\item 2 cuill\`eres \`a soupe d'huile d'olive
\end{itemize}


\subsection*{\textsc{Marche \`a suivre~:}}

\begin{enumerate}
\item Pr\'echauffer le four \`a 200° C. 

\item Faire revenir l'oignon et l'ail dans 2 cuill\`eres \`a soupe d'huile d'olive. 

\item Ajouter les d\'es de poivron, la tomate et le piment d'Espelette, et laisser mijoter 15 minutes environ jusqu'\`a ce que les poivrons deviennent tendres. R\'eserver. 

\item Dans une autre po\^ele, faire revenir, \`a sec et \`a feu tr\`es vif, la pancetta coup\'ee en petis morceaux, jusqu'\`a ce qu'elle soit bien dor\'ee. 

\item Egoutter pour ôter l'exc\`es de gras et r\'eserver.

\item M\'elanger les l\'egumes et la pancetta. Tapisser un moule \`a tarte avec la p\^ate bris\'ee et la piquer. R\'epartir les l\'egumes dessus.

\item Battre les oeufs avec la cr\`eme dans un saladier. Saler l\'eg\`erement et verser sur la tarte. Garnir de rondelles de ch\`evre et de feuilles de basilic.

\item Baisser la temp\'erature du four \`a 180° C et enfourner pour 30 \`a 40 minutes. Servir ti\`ede ou froid.
\end{enumerate}
\subsection*{\textsc{Conseil~:}}

