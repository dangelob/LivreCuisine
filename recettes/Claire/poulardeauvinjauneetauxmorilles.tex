% Poularde au vin jaune et aux morilles							% <-- x1
% % Poularde au vin jaune et aux morilles							% <-- x1
% % Poularde au vin jaune et aux morilles							% <-- x1
% % Poularde au vin jaune et aux morilles							% <-- x1
% \include{./recettes/Claire/poulardeauvinjauneetauxmorilles}				% <-- x1
\section[\normalsize{Poularde au vin jaune et aux morilles}]{\LARGE{\textsc{Poularde au vin jaune et aux morilles}}}		% <-- x2


\begin{itemize}
\item Pour 6 personnes
\item Préparation : 60 min*
\item Cuisson : 50 min
\end{itemize}

\subsection*{\textsc{Ingr\'edients~:}}

\begin{itemize}
\item 1 poularde de 2 kg coup\'ee en morceaux
\item 50 g de morielles s\'ech\'ees
\item 25 cl de vin jaune
\item 25 g de beurre
\item 50 cl de cr\`eme liquide
\item 1 cuill\`ere \`a soupe de farine
\item Sel, poivre
\end{itemize}


\subsection*{\textsc{Marche \`a suivre~:}}

\begin{enumerate}
\item Laisser tremper les morilles 30 \`a 60 minutes dans une jatte d'eau ti\`ede pour les r\'ehydrater.
\item Pendant ce temps, pr\'echauffer le four \`a 180° C. Saler, poivrer et fariner les morceaux de poularde. Les faire dorer dans une cocotte allant au four avec le beurre. 
\item Couvrir, cuire 25 min. au four.
\item Retirer la poularde de la cocotte, jeter la graisse. Verser le vin jaune, le faire bouillir et r\'eduire 3 minutes. Ajouter la cr\`eme.
\item Les morilles \'egoutt\'ees, replacer la poularde et cuire 20 minutes \`a feu doux sans couvrir. 
\item Rectifier l'assaisonnement en fin de cuisson, servir chaud. 
\end{enumerate}
\subsection*{\textsc{Conseil~:}}

Peut n\'ecessiter un temps de cuisson plus long. Pour corser la sauce, m\'elanger l'eau de trempage des morilles filtr\'ee avec la cr\`eme et verser dans le vin r\'eduit.				% <-- x1
\section[\normalsize{Poularde au vin jaune et aux morilles}]{\LARGE{\textsc{Poularde au vin jaune et aux morilles}}}		% <-- x2


\begin{itemize}
\item Pour 6 personnes
\item Préparation : 60 min*
\item Cuisson : 50 min
\end{itemize}

\subsection*{\textsc{Ingr\'edients~:}}

\begin{itemize}
\item 1 poularde de 2 kg coup\'ee en morceaux
\item 50 g de morielles s\'ech\'ees
\item 25 cl de vin jaune
\item 25 g de beurre
\item 50 cl de cr\`eme liquide
\item 1 cuill\`ere \`a soupe de farine
\item Sel, poivre
\end{itemize}


\subsection*{\textsc{Marche \`a suivre~:}}

\begin{enumerate}
\item Laisser tremper les morilles 30 \`a 60 minutes dans une jatte d'eau ti\`ede pour les r\'ehydrater.
\item Pendant ce temps, pr\'echauffer le four \`a 180° C. Saler, poivrer et fariner les morceaux de poularde. Les faire dorer dans une cocotte allant au four avec le beurre. 
\item Couvrir, cuire 25 min. au four.
\item Retirer la poularde de la cocotte, jeter la graisse. Verser le vin jaune, le faire bouillir et r\'eduire 3 minutes. Ajouter la cr\`eme.
\item Les morilles \'egoutt\'ees, replacer la poularde et cuire 20 minutes \`a feu doux sans couvrir. 
\item Rectifier l'assaisonnement en fin de cuisson, servir chaud. 
\end{enumerate}
\subsection*{\textsc{Conseil~:}}

Peut n\'ecessiter un temps de cuisson plus long. Pour corser la sauce, m\'elanger l'eau de trempage des morilles filtr\'ee avec la cr\`eme et verser dans le vin r\'eduit.				% <-- x1
\section[\normalsize{Poularde au vin jaune et aux morilles}]{\LARGE{\textsc{Poularde au vin jaune et aux morilles}}}		% <-- x2


\begin{itemize}
\item Pour 6 personnes
\item Préparation : 60 min*
\item Cuisson : 50 min
\end{itemize}

\subsection*{\textsc{Ingr\'edients~:}}

\begin{itemize}
\item 1 poularde de 2 kg coup\'ee en morceaux
\item 50 g de morielles s\'ech\'ees
\item 25 cl de vin jaune
\item 25 g de beurre
\item 50 cl de cr\`eme liquide
\item 1 cuill\`ere \`a soupe de farine
\item Sel, poivre
\end{itemize}


\subsection*{\textsc{Marche \`a suivre~:}}

\begin{enumerate}
\item Laisser tremper les morilles 30 \`a 60 minutes dans une jatte d'eau ti\`ede pour les r\'ehydrater.
\item Pendant ce temps, pr\'echauffer le four \`a 180° C. Saler, poivrer et fariner les morceaux de poularde. Les faire dorer dans une cocotte allant au four avec le beurre. 
\item Couvrir, cuire 25 min. au four.
\item Retirer la poularde de la cocotte, jeter la graisse. Verser le vin jaune, le faire bouillir et r\'eduire 3 minutes. Ajouter la cr\`eme.
\item Les morilles \'egoutt\'ees, replacer la poularde et cuire 20 minutes \`a feu doux sans couvrir. 
\item Rectifier l'assaisonnement en fin de cuisson, servir chaud. 
\end{enumerate}
\subsection*{\textsc{Conseil~:}}

Peut n\'ecessiter un temps de cuisson plus long. Pour corser la sauce, m\'elanger l'eau de trempage des morilles filtr\'ee avec la cr\`eme et verser dans le vin r\'eduit.				% <-- x1
\section[\normalsize{Poularde au vin jaune et aux morilles}]{\LARGE{\textsc{Poularde au vin jaune et aux morilles}}}		% <-- x2


\begin{itemize}
\item Pour 6 personnes
\item Préparation : 60 min*
\item Cuisson : 50 min
\end{itemize}

\subsection*{\textsc{Ingr\'edients~:}}

\begin{itemize}
\item 1 poularde de 2 kg coup\'ee en morceaux
\item 50 g de morielles s\'ech\'ees
\item 25 cl de vin jaune
\item 25 g de beurre
\item 50 cl de cr\`eme liquide
\item 1 cuill\`ere \`a soupe de farine
\item Sel, poivre
\end{itemize}


\subsection*{\textsc{Marche \`a suivre~:}}

\begin{enumerate}
\item Laisser tremper les morilles 30 \`a 60 minutes dans une jatte d'eau ti\`ede pour les r\'ehydrater.
\item Pendant ce temps, pr\'echauffer le four \`a 180° C. Saler, poivrer et fariner les morceaux de poularde. Les faire dorer dans une cocotte allant au four avec le beurre. 
\item Couvrir, cuire 25 min. au four.
\item Retirer la poularde de la cocotte, jeter la graisse. Verser le vin jaune, le faire bouillir et r\'eduire 3 minutes. Ajouter la cr\`eme.
\item Les morilles \'egoutt\'ees, replacer la poularde et cuire 20 minutes \`a feu doux sans couvrir. 
\item Rectifier l'assaisonnement en fin de cuisson, servir chaud. 
\end{enumerate}
\subsection*{\textsc{Conseil~:}}

Peut n\'ecessiter un temps de cuisson plus long. Pour corser la sauce, m\'elanger l'eau de trempage des morilles filtr\'ee avec la cr\`eme et verser dans le vin r\'eduit.