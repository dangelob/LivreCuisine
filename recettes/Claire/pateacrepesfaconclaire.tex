% Pâte à crêpes (façon Claire T.)										% <-- x1
% % Pâte à crêpes (façon Claire T.)										% <-- x1
% % Pâte à crêpes (façon Claire T.)										% <-- x1
% % Pâte à crêpes (façon Claire T.)										% <-- x1
% \include{./recettes/Claire/pateacrepesfaconclaire}								% <-- x1
\section[\normalsize{P\^ate \`a cr\^epes (fa\c con Claire T.)}]{\LARGE{\textsc{P\^ate \`a cr\^epes (fa\c con Claire T.)}}}		% <-- x2


\begin{itemize}
\item Pour 2 --- 4 personnes
\item Préparation : 15 min
\end{itemize}

\subsection*{\textsc{Ingr\'edients~:}}

\begin{itemize}
\item 250 g de farine
\item 1/2 l de lait
\item 3 oeufs
\item 1 cuill\`ere d'huile
\item 1 pinc\'ee de sel
\item 1 ou 2 cuill\`eres d'eau
\end{itemize}


\subsection*{\textsc{Marche \`a suivre~:}}

\begin{enumerate}
\item Mettre la farine dans une terrine. Faire un puits, et y casser les oeufs entiers.
\item Ajouter l'huile, le sel et un peu de lait ; travailler \'energiquement la p\^ate avec une cuill\`ere (en bois, de pr\'ef\'erence), pour la rendre l\'eg\`ere.
\item Mouiller progressivement avec le lait, jusqu'\`a ce que la p\^ate devienne homog\`ene. On peut, \`a ce moment l\`a, ajouter de l'extrait de fleur d'oranger, un peu de jus de citron, de la vanille, etc...
\item Ajouter ensuite 1 \`a 2 cuill\`eres d'eau. Puis, passer dans une passoire 1 \`a 2 fois la p\^ate pour enlever les grumeaux.
\item Laisser reposer la p\^ate pendant 1 h, recouverte d'une serviette ou d'un chiffon propre.

\end{enumerate}
\subsection*{\textsc{Conseil~:}}
								% <-- x1
\section[\normalsize{P\^ate \`a cr\^epes (fa\c con Claire T.)}]{\LARGE{\textsc{P\^ate \`a cr\^epes (fa\c con Claire T.)}}}		% <-- x2


\begin{itemize}
\item Pour 2 --- 4 personnes
\item Préparation : 15 min
\end{itemize}

\subsection*{\textsc{Ingr\'edients~:}}

\begin{itemize}
\item 250 g de farine
\item 1/2 l de lait
\item 3 oeufs
\item 1 cuill\`ere d'huile
\item 1 pinc\'ee de sel
\item 1 ou 2 cuill\`eres d'eau
\end{itemize}


\subsection*{\textsc{Marche \`a suivre~:}}

\begin{enumerate}
\item Mettre la farine dans une terrine. Faire un puits, et y casser les oeufs entiers.
\item Ajouter l'huile, le sel et un peu de lait ; travailler \'energiquement la p\^ate avec une cuill\`ere (en bois, de pr\'ef\'erence), pour la rendre l\'eg\`ere.
\item Mouiller progressivement avec le lait, jusqu'\`a ce que la p\^ate devienne homog\`ene. On peut, \`a ce moment l\`a, ajouter de l'extrait de fleur d'oranger, un peu de jus de citron, de la vanille, etc...
\item Ajouter ensuite 1 \`a 2 cuill\`eres d'eau. Puis, passer dans une passoire 1 \`a 2 fois la p\^ate pour enlever les grumeaux.
\item Laisser reposer la p\^ate pendant 1 h, recouverte d'une serviette ou d'un chiffon propre.

\end{enumerate}
\subsection*{\textsc{Conseil~:}}
								% <-- x1
\section[\normalsize{P\^ate \`a cr\^epes (fa\c con Claire T.)}]{\LARGE{\textsc{P\^ate \`a cr\^epes (fa\c con Claire T.)}}}		% <-- x2


\begin{itemize}
\item Pour 2 --- 4 personnes
\item Préparation : 15 min
\end{itemize}

\subsection*{\textsc{Ingr\'edients~:}}

\begin{itemize}
\item 250 g de farine
\item 1/2 l de lait
\item 3 oeufs
\item 1 cuill\`ere d'huile
\item 1 pinc\'ee de sel
\item 1 ou 2 cuill\`eres d'eau
\end{itemize}


\subsection*{\textsc{Marche \`a suivre~:}}

\begin{enumerate}
\item Mettre la farine dans une terrine. Faire un puits, et y casser les oeufs entiers.
\item Ajouter l'huile, le sel et un peu de lait ; travailler \'energiquement la p\^ate avec une cuill\`ere (en bois, de pr\'ef\'erence), pour la rendre l\'eg\`ere.
\item Mouiller progressivement avec le lait, jusqu'\`a ce que la p\^ate devienne homog\`ene. On peut, \`a ce moment l\`a, ajouter de l'extrait de fleur d'oranger, un peu de jus de citron, de la vanille, etc...
\item Ajouter ensuite 1 \`a 2 cuill\`eres d'eau. Puis, passer dans une passoire 1 \`a 2 fois la p\^ate pour enlever les grumeaux.
\item Laisser reposer la p\^ate pendant 1 h, recouverte d'une serviette ou d'un chiffon propre.

\end{enumerate}
\subsection*{\textsc{Conseil~:}}
								% <-- x1
\section[\normalsize{P\^ate \`a cr\^epes (fa\c con Claire T.)}]{\LARGE{\textsc{P\^ate \`a cr\^epes (fa\c con Claire T.)}}}		% <-- x2


\begin{itemize}
\item Pour 2 --- 4 personnes
\item Préparation : 15 min
\end{itemize}

\subsection*{\textsc{Ingr\'edients~:}}

\begin{itemize}
\item 250 g de farine
\item 1/2 l de lait
\item 3 oeufs
\item 1 cuill\`ere d'huile
\item 1 pinc\'ee de sel
\item 1 ou 2 cuill\`eres d'eau
\end{itemize}


\subsection*{\textsc{Marche \`a suivre~:}}

\begin{enumerate}
\item Mettre la farine dans une terrine. Faire un puits, et y casser les oeufs entiers.
\item Ajouter l'huile, le sel et un peu de lait ; travailler \'energiquement la p\^ate avec une cuill\`ere (en bois, de pr\'ef\'erence), pour la rendre l\'eg\`ere.
\item Mouiller progressivement avec le lait, jusqu'\`a ce que la p\^ate devienne homog\`ene. On peut, \`a ce moment l\`a, ajouter de l'extrait de fleur d'oranger, un peu de jus de citron, de la vanille, etc...
\item Ajouter ensuite 1 \`a 2 cuill\`eres d'eau. Puis, passer dans une passoire 1 \`a 2 fois la p\^ate pour enlever les grumeaux.
\item Laisser reposer la p\^ate pendant 1 h, recouverte d'une serviette ou d'un chiffon propre.

\end{enumerate}
\subsection*{\textsc{Conseil~:}}
