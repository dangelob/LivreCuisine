\section[\normalsize{Gâteau au chocolat (façon Claire T.)}]{Gâteau au chocolat (façon Claire T.)}

\begin{ingredients}
\item 5 oeufs
\item 150 g de sucre
\item 250 g de chocolat
\item 100 g d'amandes en poudre
\item 200 g de beurre
\item 3 c. \`a soupe de f\'ecule de pomme de terre
\item 2 sachets de sucre vanill\'e
\item 1/2 c. \`a caf\'e de levure
\item Pour le glaçage :
\begin{itemize}
\item 125 g de chocolat \`a croquer
\item 50 g de beurre
\item 100 g de sucre glace
\end{itemize}
\end{ingredients}
\begin{infos}
\item Pour 6 personnes
\item Préparation : 45 min
\item Cuisson : 40 min
\end{infos}
\begin{etapes}
\item Fouetter les jaunes d'oeufs avec les sucres jusqu'\`a ce qu'ils blanchissent et fassent ruban.
\item Faire fondre le chocolat avec deux cuill\`eres \`a soupe d'eau au bain-marie non bouillant. L'ajouter aux jaunes, m\'elanger.
\item Ajouter la poudre d'amandes tamis\'ee avec la levure et la f\'ecule, le beurre tr\`es ramolli mais non fondu, puis les blancs d'oeufs battus en neige avec une pinc\'ee de sel.
\item Verser dans un moule de 24 cm de diam\`etre, beurr\'e et saupoudr\'e de sucre semoule, en ne remplissant qu'aux trois quarts et mettre au four th 6-7. D\'emouler ti\`ede.
\item Pour le glaçage, faire fondre le chocolat avec une cuill\`ere \`a soupe d'eau et le beurre au bain-marie. Ajouter le sucre glace tamis\'e par cuill\`eres puis enduire le g\^ateau.
\end{etapes}
\begin{conseils}
\end{conseils}
