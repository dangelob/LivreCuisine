\section{Bœuf en cocotte aux carottes}

\begin{ingredients}
\item 1 kg de viande de boeuf (joue, paleron, gîte, macreuse...)
\item 1,5 kg d carottes
\item 1 L de vin rouge
\item 4 ou 5 échalotes
\item 1 oignon
\item 1 bouquet garni (thym et laurier)
\item 4 clous de girofle
\item quelques grains de poivre
\item 2 cubes de bouillon de boeuf
\item 2 cuillères à soupe de farine
\end{ingredients}
\begin{infos}
\item Pour 6 personnes		% Nombre de personnes qu'on pourra nourrir ! :)
\item Préparation : 30 min + repos 12 h		% Temps de préparation (sans la cuisson)
\item Cuisson : 3 h			% Temps de cuisson
\end{infos}
\begin{etapes}
\item La veille, détailler la viande en gros cubes de 2cm de côté environ. Couper l'oignon en 2 et y planter les clous de girofles. Placer le tout dans un grand saladier avec le bouquet garni et les grains de poivre. Recouvrir avec le vin rouge. Placer au frais au moins 12 heures.
\item Le jour même, émincer les échalotes finement. Egoutter la viande, la mettre dans un autre récipient. L'éponger grossièrement avec du papier absorbant. 
\item Dans une cocotte en fonte faire revenir les cubes de boeuf dans un peu d'huile. Attendre quelques minutes puis avec une écumoire, retirer la viande. Récupérer le vin rendu par la viande et l'ajouter à celui de la marinade. Remettre 2 cuillères à soupe d'huile dans la cocotte et la viande. Faire cuire les morceaux de tous les côtés, saler et poivrer. Retirer de nouveau les cubes de boeuf, réserver. 
\item Ajouter une cuillère d'huile puis les échalotes dans la cocotte. Les faire rissoler jusqu'à ce qu'elles deviennent translucides puis saupoudrer avec la farine. Faire cuire ainsi quelques minutes. 
\item Mouiller avec le vin de la marinade en prenant soin de le passer à travers un chinois. Ajouter les cubes de bouillon. Remettre la viande dans la cocotte. Laisser mijoter à couvert à feu doux.
\item Peler et couper les carottes en fines rondelles. Les ajouter dans la cocotte environ une heure avant la fin de la cuisson.
\item Si besoin, lier la sauce à l'aide de maïzena.
\end{etapes}
\begin{conseils}
Ne pas hésiter à préparer ce plat à l'avance, comme tous les plats mijotés c'est encore meilleur réchauffé.
\end{conseils}
