% Muffins aux pépites de chocolat façon Claire T.					% <-- x1
% \section[\normalsize{Muffins aux pépites de chocolat (façon Claire T.)}]{\choco{Muffins aux pépites de chocolat (façon Claire T.)}}
%\choco{Muffins aux pépites de chocolat (façon Claire T.)}
%\choco{}

\begin{ingredients}
\item 300 g de farine
\item 1 sachet de levure
\item 100 g de sucre
\item 1 pinc\'ee de sel
\item 150 g de chocolat \`a p\^atisserie
\item 25 cl de lait
\item 2 oeufs
\item 75 g de beurre fondu
\item 1 cuill\`ere \`a soupe de sirop d'\'erable
\end{ingredients}
\begin{infos}
\item Pour 4 personnes
\item Préparation : 40 min
\item Cuisson : 20 min
\end{infos}
\begin{etapes}
\item Pr\'echauffer le four \`a 180° C (th. 6).
\item Couper chaque carr\'e de chocolat en 4 en se servant d'un grand couteau.
\item Dans un grand bol, m\'elanger la farine, la levure, le sucre, la pinc\'ee de sel et le chocolat.
\item Dans un pichet, casser et battre l\'eg\`erement les oeufs. Ajouter le lait, le sirop d'\'erable et le beurre fondu.
\item Verser ce liquide sur le m\'elange sec et m\'elanger juste assez pour que la farine ne soit plus visible : la p\^ate doit \^etre grumeleuse.
\item Verser dans les moules \`a muffins \`a l'aide d'une grande cuill\`ere et enfourner pour 20 minutes.
\end{etapes}
\begin{conseils}
\end{conseils}
								% <-- x1
\section[\normalsize{Muffins aux p\'epites de chocolat (fa\c con Claire T.)}]{\LARGE{\textsc{Muffins aux p\'epites de chocolat (fa\c con Claire T.)}}}		% <-- x2


\begin{itemize}
\item Pour 4 personnes
\item Préparation : 40 min
\item Cuisson : 20 min
\end{itemize}

\subsection*{\textsc{Ingr\'edients~:}}

\begin{itemize}
\item 300 g de farine
\item 1 sachet de levure
\item 100 g de sucre
\item 1 pinc\'ee de sel
\item 150 g de chocolat \`a p\^atisserie
\item 25 cl de lait
\item 2 oeufs
\item 75 g de beurre fondu
\item 1 cuill\`ere \`a soupe de sirop d'\'erable
\end{itemize}


\subsection*{\textsc{Marche \`a suivre~:}}

\begin{enumerate}
\item Pr\'echauffer le four \`a 180° C (th. 6).
\item Couper chaque carr\'e de chocolat en 4 en se servant d'un grand couteau.
\item Dans un grand bol, m\'elanger la farine, la levure, le sucre, la pinc\'ee de sel et le chocolat.
\item Dans un pichet, casser et battre l\'eg\`erement les oeufs. Ajouter le lait, le sirop d'\'erable et le beurre fondu.
\item Verser ce liquide sur le m\'elange sec et m\'elanger juste assez pour que la farine ne soit plus visible : la p\^ate doit \^etre grumeleuse.
\item Verser dans les moules \`a muffins \`a l'aide d'une grande cuill\`ere et enfourner pour 20 minutes.
\end{enumerate}
\subsection*{\textsc{Conseil~:}}
