% Nom de la recette à entrer entre les accolades {}
\section{Poêlée de petit épeautre aux échalotes et pesto de roquette}

\begin{ingredients}
% Ici lister les ingrédients 
% Changer de ligne pour chaque ingrédient et commencer la ligne par : \item
% rajouter autant de ligne que d'ingrédient
\item Pour le pesto :
\begin{itemize}
\item 60 g de roquette
\item 40 g de pignons (ou noisettes, noix, amandes...)
\item 20 g de parmesan
\item 3 cuillères à soupe d'huile d'olive
\item 1 cuillère à soupe de jus de citron
\end{itemize}
\item Pour le petit épeautre :
\begin{itemize}
\item 200 g de petit épeautre
\item 8 à 12 échalotes (selon la taille)
\item 3 belles poignées de roquette (facultatif)
\item huile d'olive
\end{itemize}
\end{ingredients}
\begin{infos}
% Informations génériques
% Changer de ligne pour chaque et commencer par : \item
% Mettre une * si l'information n'est pas certaine 
\item Pour 4 personnes*		% Nombre de personnes qu'on pourra nourrir ! :)
\item Préparation : 20 min*		% Temps de préparation (sans la cuisson)
\item Cuisson : 75 min		% Temps de cuisson
\end{infos}
\begin{etapes}
% Ici les étapes à réaliser
% Une étape par ligne, chaque ligne commence par un \item
% Pour exemple les étapes pour faire un millas ;)
\item Mesurer le volume de petit épeautre. Si possible, le faire tremper pendant environ douze heures. Égoutter et rincer. 
\item Verser le petit épeautre dans une grande casserole avec deux fois son volume d'eau froide. Porter à frémissements. Ajouter des aromates (romarin, laurier, bouillon cube...). Couvrir et laisser cuire à feu doux 45 min (si trempage) à 1 heure. Couper le feu et laisser gonfler 5 à 10 minutes. Égoutter.
\item Pendant ce temps, préparer le pesto en mixant ensemble tous les ingrédients. Réserver.
\item Peler les échalotes. Les couper en deux ou en quatre (si elles sont grosses) dans la longueur.
\item Faire chauffer 2 cuillères à soupe d’huile dans une sauteuse et faire suer les échalotes sans trop les défaire. Ajouter le petit-épeautre et le pesto. Mélanger pendant 1 min. Ajouter la roquette restante, mélanger et servir.
\end{etapes}
\begin{conseils}
% Ici écrire les conseils concernant la recette 
\end{conseils}
