% Nom de la recette à entrer entre les accolades {}
\section{Damiers vanille et chocolat (Wiss un schwarzi butter bredele)}

\begin{ingredients}
% Ici lister les ingrédients 
% Changer de ligne pour chaque ingrédient et commencer la ligne par : \item
% rajouter autant de ligne que d'ingrédient
%\begin{itemize}
\item 250 g de farine
\item 150 g de beurre pommade
\item 100 g de sucre
\item 1 sachet de sucre vanillé
\item 1 cuillère à café de levure chimique
\item 1 cuillère à soupe d'eau de vie (ou d'eau)
\item 30 g de cacao non sucré en poudre
%\end{itemize}
\end{ingredients}
\begin{infos}
% Informations génériques
% Changer de ligne pour chaque et commencer par : \item
% Mettre une * si l'information n'est pas certaine 
\item Pour environ 75 pièces		% Nombre de personnes qu'on pourra nourrir ! :)
\item Préparation : 1h + 1h de repos		% Temps de préparation (sans la cuisson)
\item Cuisson : 20 min		% Temps de cuisson
\end{infos}
\begin{etapes}
% Ici les étapes à réaliser
% Une étape par ligne, chaque ligne commence par un \item
% Pour exemple les étapes pour faire un millas ;)
\item Dans un bol mélanger la farine et la levure chimique, rajouter le sucre et le sucre vanillé. Incorporer l’eau de vie et le beurre pommade et pétrir le tout pour former une boule de pâte. Ajouter un peu d'eau si nécessaire.
\item Partager la pâte en deux, et incorporer le cacao dans une moitié.
\item Rouler chaque partie en forme de petits boudins de taille égale, d'environ 1 cm d'épaisseur. Disposer un boudin vanille et un boudin chocolat côte à côte puis déposer par dessus deux autres boudins en inversant les couleurs de façon à faire un damier carré. Appuyer légèrement de tous les côtés pour les souder.
\item Laisser durcir au frais pendant 1h.
\item Préchauffer le four à 160°C.
\item Découper la pâte en tranches d'un bon centimètre avec un couteau. Déposer les damiers sur une plaque en les espaçant bien et faire cuire 8 à 10 min.
\item Laisser refroidir sur une grille et ranger dans une boîte en métal.
\end{etapes}
\begin{conseils}
% Ici écrire les conseils concernant la recette 
On peut aussi façonner les biscuits en spirales : étaler les pâtes sur 2 mm d'épaisseur, les découper à la même taille, les superposer en appuyant un peu et rouler en cylindre dans le sens de la longueur. Après repos au frais, couper des tranches de la même façon que pour les damiers.
\end{conseils}
