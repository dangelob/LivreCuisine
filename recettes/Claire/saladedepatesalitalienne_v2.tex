\clearpage
\newgeometry{inner=2.5cm,outer=9cm,bottom=2.5cm, marginparwidth=7cm,marginparsep=1.5cm}
\section{Salade de pâtes à l'italienne}		% <-- x2


\marginnote{
	\begin{itemize}
	\item[] 300 g de p\^ates
	\item[] 1 melon
	\item[] 4 belles tomates
	\item[] 150 g de mozzarella ou de feta
	\item[] 4 tranches de jambon de parme
	\item[] 12 feuilles de basilic
	\item[] 8 olives noires d\'enoyaut\'ees
	\item[] Le jus d'1 citron
	\item[] 6 cuill\`eres \`a soupe d'huile d'olive
	\item[] Sel, poivre
	\end{itemize}
}[0cm]

\begin{flushright}
	\textbf{4 portions} \\
	\textbf{20 min}
\end{flushright}

\vspace{.5cm}

\begingroup
\noindent
Couper le melon en deux et détailler la chair en billes à l'aide d'une cuillère parisienne. Couper les tomates en morceaux, la mozzarella en dés et le jambon en lanières. Détailler les olives en petits morceaux. \par
~\\
\noindent
Dans un saladier, mélanger le jus d'un citron, l'huile d'olive et 6 feuilles de basilic ciselées. Saler et poivrer.
\par ~\\
\noindent
Ajouter les pâtes dans le saladier. Mélanger. Ajouter les billes de melon, les tomates, la mozzarella, le jambon et les olives. Parsemer du reste de basilic.

\endgroup

\begin{center}
	\par\noindent\rule{.5\textwidth}{0.4pt} 
\end{center}
%\par\noindent\centerline{\rule{.5\textwidth}{0.4pt} }

Servir frais, le melon n'est pas indispensable.

\vspace{1cm}

