\section[\normalsize{Stir fry de porc aux nouilles}]{Stir fry de porc aux nouilles}

\begin{ingredients}
\item 30ml d'huile
\item 500g de filet de porc en fines lamelles 
\item 250g de poireau en rondelles de 6 mm 
\item 2 gousses d'ail \'ecras\'ees 
\item 2,5 cm de gigembre pel\'e et finement h\^ach\'e 
\item 200g de nouilles Chinoises 
\item 250g de champignons Huitres ou Chinois 
\item 200g de chou chinois 
\item 40ml de sauce Soja
\item 15ml de miel liquide 
\item 10ml d'huile de s\'esame 
\end{ingredients}
\begin{infos}
\item Pour 4 personnes
\item Préparation : 35 min
\item Cuisson : 10 min
\end{infos}
\begin{etapes}
\item Faire chauffer l'huile dans un wok ou une grande poêle. 
\item Ajouter le porc, faire cuire pendant 3 minutes en remuant sans cesse. 
\item Ajouter les poireaux et cuire pendant 2 minutes. Retirer la viande et les poireaux du feu, r\'eserver.
\item Essuyer le wok avec un Sopalin. Faire chauffer le reste de l'huile. Ajouter le gingembre et l'ail et sauter 1 minute.
\item Pendant ce temps faire cuire les nouilles suivant les instructions sur le paquet.
\item Ajouter les champignons et le chou et continuer la cuisson pendant 3 minutes. 
\item Remettre le porc et les poireaux dans la poêle. Remuer. 
\item Ajouter la sauce soja ou haricots noirs, le miel et 50ml d'eau. Cuire pendant 2 minutes ou jusqu'\`a ce que la sauce soit tr\`es chaude.
\item Égoutter les nouilles. Arroser de l'huile de s\'esame et bien remuer. Servir de suite avec le porc.
\end{etapes}
\begin{conseils}
\end{conseils}
