% Cake aux tomates séchées, feta et olives						% <-- x1
% \section[\normalsize{Cake aux tomates s\'ech\'ees, feta et olives}]{Cake aux tomates s\'ech\'ees, feta et olives}

\begin{ingredients}
\item 200 g de farine
\item 4 oeufs
\item 1 sachet de levure
\item 10 cl de lait
\item 2 cuill\`eres \`a soupe d'huile d'olive
\item 15 tomates s\'ech\'ees
\item 200 de feta
\item 10 olives vertes et 5 noires d\'enoyaut\'ees
\item 5 feuilles de basilic
\item Poivre
\end{ingredients}
\begin{infos}
\item Pour 6 personnes
\item Préparation : 20 min
\item Cuisson : 40 min
\end{infos}
\begin{etapes}
\item Pr\'echauffer le four \`a 180° C
\item M\'elanger ensemble la farine, les oeufs et la levure.
\item Ajouter l'huile d'olive, et le lait froid.
\item Bien m\'elanger, en soulevant avec la cuill\`ere, afin d'a\'erer la p\^ate.
\item Ajouter les tomates s\'ech\'ees coup\'ee en morceaux, la feta, les olives vertes (en petits morceaux) et le basilic cisel\'e. 
\item poivrer.
\item Verser la pr\'eparation dans un moule \`a cake.
\item Faire cuire pendant 40 min. V\'erifier la cuisson avec la lame d'un couteau, elle doit ressortir s\`eche.
\end{etapes}
\begin{conseils}
\end{conseils}					% <-- x1
\section[\normalsize{Cake aux tomates s\'ech\'ees, feta et olives}]{\LARGE{\textsc{Cake aux tomates s\'ech\'ees, feta et olives}}}		% <-- x2


\begin{itemize}
\item Pour 6 personnes
\item Préparation : 20 min
\item Cuisson : 40 min
\end{itemize}

\subsection*{\textsc{Ingr\'edients~:}}

\begin{itemize}
\item 200 g de farine
\item 4 oeufs
\item 1 sachet de levure
\item 10 cl de lait
\item 2 cuill\`eres \`a soupe d'huile d'olive
\item 15 tomates s\'ech\'ees
\item 200 de feta
\item 10 olives vertes et 5 noires d\'enoyaut\'ees
\item 5 feuilles de basilic
\item Poivre
\end{itemize}


\subsection*{\textsc{Marche \`a suivre~:}}

\begin{enumerate}
\item Pr\'echauffer le four \`a 180° C

\item M\'elanger ensemble la farine, les oeufs et la levure.

\item Ajouter l'huile d'olive, et le lait froid.

\item Bien m\'elanger, en soulevant avec la cuill\`ere, afin d'a\'erer la p\^ate.

\item Ajouter les tomates s\'ech\'ees coup\'ee en morceaux, la feta, les olives vertes (en petits morceaux) et le basilic cisel\'e. 
\item poivrer.
\item Verser la pr\'eparation dans un moule \`a cake.
\item Faire cuire pendant 40 min. V\'erifier la cuisson avec la lame d'un couteau, elle doit ressortir s\`eche.

\end{enumerate}
\subsection*{\textsc{Conseil~:}}

