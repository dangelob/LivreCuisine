%Tiramisu aux framboises en verrines											% <-- x1
% %Tiramisu aux framboises en verrines											% <-- x1
% %Tiramisu aux framboises en verrines											% <-- x1
% %Tiramisu aux framboises en verrines											% <-- x1
% \include{./recettes/Claire/tiramisuauxfranboisesenverrines}								% <-- x1
\section[\normalsize{Tiramisu aux framboises en verrines}]{\LARGE{\textsc{Tiramisu aux framboises en verrines}}}		% <-- x2


\begin{itemize}
\item Pour 4 personnes
\item Préparation : 30 + 180 min
\end{itemize}

\subsection*{\textsc{Ingr\'edients~:}}

\begin{itemize}
\item 2 oeufs
\item 250 g de mascarpone
\item 50 g de sucre roux
\item 1 sachet de sucre vanill\'e
\item 100 g de framboises
\item 1/4 de zeste de citron vert hach\'e
\item environ 10 biscuits rose de Reims
\item amandes effil\'ees
\end{itemize}
Pour le coulis :
\begin{itemize}
\item 150 g de framboises
\item 50 g de sucre
\end{itemize}

\subsection*{\textsc{Marche \`a suivre~:}}

\begin{enumerate}
\item S\'eparer le blanc des jaunes d’oeufs. 
\item M\'elanger les jaunes avec le sucre et le sucre vanill\'e. 
\item Ajouter le mascarpone au fouet, puis le zeste de citron vert. 
\item Monter les blancs en neige et les incorporer d\'elicatement \`a la spatule au m\'elange pr\'ec\'edent.
\item Pr\'eparer le coulis en mixant 150 g de framboises avec 50 g de sucre.
\item Tapisser les verrines de biscuits. Recouvrir de coulis et de quelques framboises et \'etaler une couche de cr\`eme.
\item Alterner biscuits, coulis, framboises et cr\`eme. Terminer par une couche de cr\`eme. 
\item Garder 4 framboises pour la d\'eco.
\item Filmer les verrines et r\'efrig\'erer au moins 3 heures.
\item Au moment de servir, saupoudrer d’amandes effil\'ees et d\'eposer une framboise au centre.
\end{enumerate}
\subsection*{\textsc{Conseil~:}}
								% <-- x1
\section[\normalsize{Tiramisu aux framboises en verrines}]{\LARGE{\textsc{Tiramisu aux framboises en verrines}}}		% <-- x2


\begin{itemize}
\item Pour 4 personnes
\item Préparation : 30 + 180 min
\end{itemize}

\subsection*{\textsc{Ingr\'edients~:}}

\begin{itemize}
\item 2 oeufs
\item 250 g de mascarpone
\item 50 g de sucre roux
\item 1 sachet de sucre vanill\'e
\item 100 g de framboises
\item 1/4 de zeste de citron vert hach\'e
\item environ 10 biscuits rose de Reims
\item amandes effil\'ees
\end{itemize}
Pour le coulis :
\begin{itemize}
\item 150 g de framboises
\item 50 g de sucre
\end{itemize}

\subsection*{\textsc{Marche \`a suivre~:}}

\begin{enumerate}
\item S\'eparer le blanc des jaunes d’oeufs. 
\item M\'elanger les jaunes avec le sucre et le sucre vanill\'e. 
\item Ajouter le mascarpone au fouet, puis le zeste de citron vert. 
\item Monter les blancs en neige et les incorporer d\'elicatement \`a la spatule au m\'elange pr\'ec\'edent.
\item Pr\'eparer le coulis en mixant 150 g de framboises avec 50 g de sucre.
\item Tapisser les verrines de biscuits. Recouvrir de coulis et de quelques framboises et \'etaler une couche de cr\`eme.
\item Alterner biscuits, coulis, framboises et cr\`eme. Terminer par une couche de cr\`eme. 
\item Garder 4 framboises pour la d\'eco.
\item Filmer les verrines et r\'efrig\'erer au moins 3 heures.
\item Au moment de servir, saupoudrer d’amandes effil\'ees et d\'eposer une framboise au centre.
\end{enumerate}
\subsection*{\textsc{Conseil~:}}
								% <-- x1
\section[\normalsize{Tiramisu aux framboises en verrines}]{\LARGE{\textsc{Tiramisu aux framboises en verrines}}}		% <-- x2


\begin{itemize}
\item Pour 4 personnes
\item Préparation : 30 + 180 min
\end{itemize}

\subsection*{\textsc{Ingr\'edients~:}}

\begin{itemize}
\item 2 oeufs
\item 250 g de mascarpone
\item 50 g de sucre roux
\item 1 sachet de sucre vanill\'e
\item 100 g de framboises
\item 1/4 de zeste de citron vert hach\'e
\item environ 10 biscuits rose de Reims
\item amandes effil\'ees
\end{itemize}
Pour le coulis :
\begin{itemize}
\item 150 g de framboises
\item 50 g de sucre
\end{itemize}

\subsection*{\textsc{Marche \`a suivre~:}}

\begin{enumerate}
\item S\'eparer le blanc des jaunes d’oeufs. 
\item M\'elanger les jaunes avec le sucre et le sucre vanill\'e. 
\item Ajouter le mascarpone au fouet, puis le zeste de citron vert. 
\item Monter les blancs en neige et les incorporer d\'elicatement \`a la spatule au m\'elange pr\'ec\'edent.
\item Pr\'eparer le coulis en mixant 150 g de framboises avec 50 g de sucre.
\item Tapisser les verrines de biscuits. Recouvrir de coulis et de quelques framboises et \'etaler une couche de cr\`eme.
\item Alterner biscuits, coulis, framboises et cr\`eme. Terminer par une couche de cr\`eme. 
\item Garder 4 framboises pour la d\'eco.
\item Filmer les verrines et r\'efrig\'erer au moins 3 heures.
\item Au moment de servir, saupoudrer d’amandes effil\'ees et d\'eposer une framboise au centre.
\end{enumerate}
\subsection*{\textsc{Conseil~:}}
								% <-- x1
\section[\normalsize{Tiramisu aux framboises en verrines}]{\LARGE{\textsc{Tiramisu aux framboises en verrines}}}		% <-- x2


\begin{itemize}
\item Pour 4 personnes
\item Préparation : 30 + 180 min
\end{itemize}

\subsection*{\textsc{Ingr\'edients~:}}

\begin{itemize}
\item 2 oeufs
\item 250 g de mascarpone
\item 50 g de sucre roux
\item 1 sachet de sucre vanill\'e
\item 100 g de framboises
\item 1/4 de zeste de citron vert hach\'e
\item environ 10 biscuits rose de Reims
\item amandes effil\'ees
\end{itemize}
Pour le coulis :
\begin{itemize}
\item 150 g de framboises
\item 50 g de sucre
\end{itemize}

\subsection*{\textsc{Marche \`a suivre~:}}

\begin{enumerate}
\item S\'eparer le blanc des jaunes d’oeufs. 
\item M\'elanger les jaunes avec le sucre et le sucre vanill\'e. 
\item Ajouter le mascarpone au fouet, puis le zeste de citron vert. 
\item Monter les blancs en neige et les incorporer d\'elicatement \`a la spatule au m\'elange pr\'ec\'edent.
\item Pr\'eparer le coulis en mixant 150 g de framboises avec 50 g de sucre.
\item Tapisser les verrines de biscuits. Recouvrir de coulis et de quelques framboises et \'etaler une couche de cr\`eme.
\item Alterner biscuits, coulis, framboises et cr\`eme. Terminer par une couche de cr\`eme. 
\item Garder 4 framboises pour la d\'eco.
\item Filmer les verrines et r\'efrig\'erer au moins 3 heures.
\item Au moment de servir, saupoudrer d’amandes effil\'ees et d\'eposer une framboise au centre.
\end{enumerate}
\subsection*{\textsc{Conseil~:}}
