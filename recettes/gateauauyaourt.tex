% Nom de la recette à entrer entre les accolades {}
\section{Gâteau au yaourt}

\begin{ingredients}
% Ici lister les ingrédients 
% Changer de ligne pour chaque ingrédient et commencer la ligne par : \item
% rajouter autant de ligne que d'ingrédient
\item 1 pot de yaourt nature
\item 2 pot de sucre
\item 3 pot de farine
\item 1 pot d'huile
\item 3 oeufs
\item 1 sachet de levure chimique
\item 15 g de beurre
\end{ingredients}
\begin{infos}
% Informations génériques
% Changer de ligne pour chaque et commencer par : \item
% Mettre une * si l'information n'est pas certaine 
\item Pour 4 personnes		% Nombre de personnes qu'on pourra nourrir ! :)
\item Préparation : 15 min*		% Temps de préparation (sans la cuisson)
\item Cuisson : 35 min			% Temps de cuisson
\end{infos}
\begin{etapes}
% Ici les étapes à réaliser
% Une étape par ligne, chaque ligne commence par un \item
% Pour exemple les étapes pour faire un millas ;)
\item Préchauffer le four à 180°C.
\item Verser le yaourt dans un saladier puis rincer le pot pour pouvoir l'utiliser comme doseur.
\item Ajouter le sucre, les oeufs et mélanger le tout pour obtenir un mélange mousseux.
\item Ajouter la farine et la levure. Mélanger et ajouter l'huile.
\item Beurrer le moule, verser la préparation et laisser cuire 35 minutes.
\end{etapes}
\begin{conseils}
% Ici écrire les conseils concernant la recette
Il est possible de remplacer le pot d'huile par un pot de beurre ou de crème fraîche et le pot de farine par un pot d'amandes en poudre. 
\end{conseils}
