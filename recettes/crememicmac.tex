% Nom de la recette à entrer entre les accolades {}
\section{Crème mic-mac}

% Informations génériques
% Changer de ligne pour chaque et commencer par : \item
% Mettre une * si l'information n'est pas certaine 
\begin{itemize}
\item Pour 6 personnes*		% Nombre de personnes qu'on pourra nourrir ! :)
\item Préparation : 15 min*		% Temps de préparation (sans la cuisson)
\end{itemize}

\subsection*{\textsc{Ingrédients~:}}

% Ici lister les ingrédients 
% Changer de ligne pour chaque ingrédient et commencer la ligne par : \item
% rajouter autant de ligne que d'ingrédient
\begin{itemize}
\item 5 œufs
\item 125 g de chocolat dessert
\item 50 g de farine
\item 50 g de beurre
\item 1 paquet de sucre vanillé
\item 150 g de sucre
\item 1 l de lait
\end{itemize}


\subsection*{\textsc{Marche à suivre~:}}

% Ici les étapes à réaliser
% Une étape par ligne, chaque ligne commence par un \item
% Pour exemple les étapes pour faire un millas ;)
\begin{enumerate}
\item Faire bouillir le lait.

\item Faire fondre le chocolat brisé dans 2 cs d’eau.

\item Mélanger le sucre, la farine, 3 jaunes et 2 œufs entiers.

\item Délayer avec le lait chaud.

\item Faire chauffer jusqu’aux 1ers bouillons.

\item Ajouter le beurre hors du feu.
\item 1ère moitié ajouter le sucre vanillé
\item 2ème moitié le chocolat fondu.
\item Verser en même temps les 2 crèmes.
\item Servir froid.
\end{enumerate}


\subsection*{\textsc{Conseil~:}}
% Ici écrire les conseils concernant la recette 
Remplacer le lait demi-écrémé par du lait entier pour obtenir encore plus d'onctuosité.

