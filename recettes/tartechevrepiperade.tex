% Generated file 2018-11-25 21:43:22.354383084 +01:00
\begin{recette}{Tarte chèvre-piperade}{Tarte chèvre-piperade}

\begin{ingredients}
1 pâte brisée\par
300 g de poivrons en petits dés (surgelés)\par
200 g de chair de tomates pelées\par
2 oeufs\par
1 belle gousse d'ail\par
1 petit oignon\par
50 g de pancetta\par
10 cl de crème\par
1/2 bûche de chèvre\par
1 cuillère à café de piment d'Espelette\par
quelques feuilles de basilic frais\par
2 cuillères à soupe d'huile d'olive\par
\end{ingredients}

\begin{infos}
Pour 6 personnes\\
Préparation : 45 min\\
Cuisson : 35 min\\
\end{infos}

\begin{etapes}
\item Préchauffer le four à 200° C.
\item Faire revenir l'oignon et l'ail dans 2 cuillères à soupe d'huile d'olive.
\item Ajouter les dés de poivron, la tomate et le piment d'Espelette, et laisser mijoter 15 minutes environ jusqu'à ce que les poivrons deviennent tendres. Réserver.
\item Dans une autre poêle, faire revenir, à sec et à feu très vif, la pancetta coupée en petis morceaux, jusqu'à ce qu'elle soit bien dorée.
\item Egoutter pour ôter l'excès de gras et réserver.
\item Mélanger les légumes et la pancetta. Tapisser un moule à tarte avec la pâte brisée et la piquer. Répartir les légumes dessus.
\item Battre les oeufs avec la crème dans un saladier. Saler légèrement et verser sur la tarte. Garnir de rondelles de chèvre et de feuilles de basilic.
\item Baisser la température du four à 180° C et enfourner pour 30 à 40 minutes. Servir tiède ou froid.
\end{etapes}

\end{recette}