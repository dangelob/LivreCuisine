% Nom de la recette à entrer entre les accolades {}
\section{Crème renversée}

\begin{ingredients}
% Ici lister les ingrédients 
% Changer de ligne pour chaque ingrédient et commencer la ligne par : \item
% rajouter autant de ligne que d'ingrédient
\item 1/2 litre de lait
\item 3 oeufs
\item 100 g de sucre
\item une gousse de vanille
\item 1 pincée de sel
\end{ingredients}
\begin{infos}
% Informations génériques
% Changer de ligne pour chaque et commencer par : \item
% Mettre une * si l'information n'est pas certaine 
\item Pour 6 personnes		% Nombre de personnes qu'on pourra nourrir ! :)
\item Préparation : 15 min*		% Temps de préparation (sans la cuisson)
%\item Cuisson : 20 min			% Temps de cuisson
\end{infos}
\begin{etapes}
% Ici les étapes à réaliser
% Une étape par ligne, chaque ligne commence par un \item
% Pour exemple les étapes pour faire un millas ;)
\item Faire bouillir le lait avec le sucre, le sel et la gousse de vanille.
\item Hors du feu ajouter le lait peu à peu aux oeufs battus en fouettant sans arrêt.
\item Verser dans un moule à soufflé caramélisé.
\item Verser 2 cm d'eau dans la cocotte minute. Y déposer le moule à soufflé et le couvrir d'une assiette.
\item Fermer la cocotte et placer là sur un feu vif. Dès le chuchotement, réduire le feu et laisser cuire 7 minutes. Aussitôt la cuisson terminée, ôter la soupape. Quand la vapeur s'est échappée, ouvrir la cocotte minute.
\end{etapes}
\begin{conseils}
% Ici écrire les conseils concernant la recette 
Servir très froid.
\end{conseils}
