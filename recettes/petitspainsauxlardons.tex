% Nom de la recette à entrer entre les accolades {}
\section{Petits pains aux lardons}

\begin{ingredients}
% Ici lister les ingrédients 
% Changer de ligne pour chaque ingrédient et commencer la ligne par : \item
% rajouter autant de ligne que d'ingrédient
\item 500 g de farine
\item 1/2 cuillère à café de sel
\item 25 g de levure de boulanger
\item 25 cl de lait tiède
\item 50 g de beurre
\item 1 oeuf
\item lardons allumettes
\end{ingredients}
\begin{infos}
% Informations génériques
% Changer de ligne pour chaque et commencer par : \item
% Mettre une * si l'information n'est pas certaine 
\item Pour XX personnes*		% Nombre de personnes qu'on pourra nourrir ! :)
\item Préparation : XX min*		% Temps de préparation (sans la cuisson)
\item Cuisson : 15 -- 20 min			% Temps de cuisson
\end{infos}
\begin{etapes}
% Ici les étapes à réaliser
% Une étape par ligne, chaque ligne commence par un \item
% Pour exemple les étapes pour faire un millas ;)
\item Mélanger la farine, la levure et le sel  
\item Ajouter le reste des ingrédients et pétrir
\item Laisser reposer (combien de temps ?)
\item Façonner les petits pains en faisant des boules de 40 g de pâte environ.
\item Cuire à 200 °C pendant 15 à 20 minute
\end{etapes}
\begin{conseils}
% Ici écrire les conseils concernant la recette 
Pour faire le grand raisin, prendre 1 kg de farine pour 1 cube de levure et 2 boîtes de lardon.
\end{conseils}
