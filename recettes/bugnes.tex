% Generated file 2019-02-17 16:33:31.024261347 +01:00
\begin{recette}{Bugnes}{Bugnes}

\begin{ingredients}
1 kg de farine\par
1 sachet de levure chimique\par
10 g de sel\par
200 g de sucre\par
1 sachet de sucre vanillé ou 2 c. à soupe d'eau de fleur d'oranger\par
200 g de beurre\par
8 œufs\par
sucre glace\par
grande friture\par
\end{ingredients}

\begin{infos}
Pour une grande corbeille\\
Préparation : 20 min + 30 min de repos\\
Cuisson : 3 à 4 min par fournée\\
\end{infos}

\begin{etapes}
\item Disposer la farine en fontaine ; au centre mettre le sel, le sucre, le sucre vanillé ou la fleur d'oranger, la levure, le beurre juste ramolli et les œufs entiers. Mélanger du bout des doigts, puis pétrir avec les mains jusqu'à ce que le mélange soit parfait. Mettre en boule, couvrir d'un torchon, laisser reposer 30 min.
\item Détacher de la masse des morceaux gros comme une orange. Les étendre au rouleau sur quelque millimètres d'épaisseur.Tailler des bandes de 4 cm de large, les couper en longs losanges puis fendre le milieu pour y passer une des pointes (tirer jusqu'au bout pour la retourner complètement). Préparer toutes les bugnes avant d'entreprendre la cuisson, car elles cuisent très vite et il faut les surveiller pour qu'elles ne brunissent pas.
\item Faire cuire dans la grande friture chaude non fumante, égoutter sur du papier absorbant et servir saupoudré de sucre glace.
\end{etapes}

\begin{conseils}
Les bugnes se conservent bien dans une boîte en métal.
\end{conseils}

\end{recette}