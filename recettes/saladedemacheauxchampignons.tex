% Nom de la recette à entrer entre les accolades {}
\section{Salade de mâche aux champignons}

\begin{ingredients}
% Ici lister les ingrédients 
% Changer de ligne pour chaque ingrédient et commencer la ligne par : \item
% rajouter autant de ligne que d'ingrédient
\item 150 g de mâche
\item 150 g de champignons de Paris
\item 100 g de chèvre frais
\item 1/2 citron
\item sauce :
\begin{itemize}
\item 1 cuillère à soupe de jus de citron
\item 3 cuillères à soupe d'huile d'olive
\item 1 oeuf dur
\item 2 cuillère à café de câpres
\item 6 olives vertes
\end{itemize}
\end{ingredients}
\begin{infos}
% Informations génériques
% Changer de ligne pour chaque et commencer par : \item
% Mettre une * si l'information n'est pas certaine 
\item Pour 4 personnes		% Nombre de personnes qu'on pourra nourrir ! :)
\item Préparation : 15 min		% Temps de préparation (sans la cuisson)
%\item Cuisson : 45 min			% Temps de cuisson
\end{infos}
\begin{etapes}
% Ici les étapes à réaliser
% Une étape par ligne, chaque ligne commence par un \item
% Pour exemple les étapes pour faire un millas ;)
\item \textbf{Sauce :} Fouetter une cuillère à soupe de jus de citron avec du sel et du poivre. Incorporer 3 cuillères à soupe d'huile d'olive. Ajouter un œuf dur, 2 cuillères à café de câpres et 6 olives vertes. Hacher le tout.
\item Couper la base des champignons. Les rouler entre vos mains sous un filet d'eau froide pour éliminer le sable. Les essuyer dans un linge. Les émincer et les arroser de jus de citron.
\item Laver et essorer la mâche puis la répartir avec les champignons dans un saladier.
\item Émietter le fromage dessus.
\item Server avec la sauce aux olives et aux câpres.
\end{etapes}
\begin{conseils}
% Ici écrire les conseils concernant la recette 
\end{conseils}
