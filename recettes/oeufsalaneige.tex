% Nom de la recette à entrer entre les accolades {}
\section{Oeufs à la neige}

\begin{ingredients}
\item un demi litre de lait
\item 75 g de sucre
\item 4 oeufs
\end{ingredients}
\begin{infos}
\item Pour XX personnes*		% Nombre de personnes qu'on pourra nourrir ! :)
\item Préparation : XX min*		% Temps de préparation (sans la cuisson)
\item Cuisson : XX min			% Temps de cuisson
\end{infos}
\begin{etapes}
\item Faire chauffer le lait.
\item Pendant ce temps battre les blancs en meringue (25 g de sucre).
\item Faire pocher 1 à 2 minutes de chaque côté. On peut tremper une cuillère dans de l'eau bouillante pour prélever des blancs et lisser avec une autre cuillère ou spatule mouillée.
\item Égoutter.
\item Disposer sur un plat creux.
\item Faire la crème anglaise.
\item Verser cette crème dans le plat ou sont disposés les blancs. Ceux-ci surnagent.
\item On peut verser sur les blancs des amandes grillées ou du caramel brun roux.
\end{etapes}
\begin{conseils}
Variante : ÎLE FLOTTANTE
\begin{itemize}
\item Faire un sirop de caramel dans un moule à charlotte (50g de sucre).
\item Faire cuire la meringue dans le moule enduit du sirop.
\item Cuire à four modéré au bain marie (20 à 30 min).
\item Démouler l’île sur un plat à crème.
\item Verser s’y la crème anglaise.
\end{itemize}
\end{conseils}
