% Nom de la recette à entrer entre les accolades {}
\section{Oeufs à la neige}

% Informations génériques
% Changer de ligne pour chaque et commencer par : \item
% Mettre une * si l'information n'est pas certaine 
\begin{itemize}
\item Pour XX personnes*		% Nombre de personnes qu'on pourra nourrir ! :)
\item Préparation : XX min*		% Temps de préparation (sans la cuisson)
\item Cuisson : XX min			% Temps de cuisson
\end{itemize}

\subsection*{\textsc{Ingrédients~:}}

% Ici lister les ingrédients 
% Changer de ligne pour chaque ingrédient et commencer la ligne par : \item
% rajouter autant de ligne que d'ingrédient
\begin{itemize}
\item un demi litre de lait
\item 75 g de sucre
\item 4 oeufs
\end{itemize}


\subsection*{\textsc{Marche à suivre~:}}

% Ici les étapes à réaliser
% Une étape par ligne, chaque ligne commence par un \item
% Pour exemple les étapes pour faire un millas ;)
\begin{enumerate}
\item Faire chauffer le lait.

\item Pendant ce temps battre les blancs en meringue (25 g de sucre).

\item Faire pocher 1 à 2 minutes de chaque côté. On peut tremper une cuillère dans de l'eau bouillante pour prélever des blancs et lisser avec une autre cuillère ou spatule mouillée.

\item Egoutter.

\item Disposer sur un plat creux.

\item Faire la crème anglaise.

\item Verser cette crème dans le plat ou sont disposés les blancs. Ceux-ci surnagent.

\item On peut verser sur les blancs des amandes grillées ou du caramel brun roux.

\end{enumerate}

Variantes :

ILE FLOTTANTE

- Faire un sirop de caramel dans un moule à charlotte (50g de sucre).
- Faire cuire la meringue dans le moule enduit du sirop.
- Cuire à four modéré au bain marie (20 à 30 min).
- Démouler l’île sur un plat à crème.
- Verser s’y la crème anglaise.


\subsection*{\textsc{Conseil~:}}
% Ici écrire les conseils concernant la recette 


