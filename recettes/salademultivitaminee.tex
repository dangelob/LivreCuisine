% Nom de la recette à entrer entre les accolades {}
\section{Salade multivitaminée}

\begin{ingredients}
% Ici lister les ingrédients 
% Changer de ligne pour chaque ingrédient et commencer la ligne par : \item
% rajouter autant de ligne que d'ingrédient
\item 1 fenouil
\item 1 pamplemousse rose
\item 1 avocat
\item 3 sucrines (coeur de laitue)
\item 1/2 botte de radis
\item 1 grosse poignée de roquette
\item 1 petit citron
\item 4 cuillères à soupe d'huile d'olive
\item sel, poivre
\end{ingredients}
\begin{infos}
% Informations génériques
% Changer de ligne pour chaque et commencer par : \item
% Mettre une * si l'information n'est pas certaine 
\item Pour 4 personnes		% Nombre de personnes qu'on pourra nourrir ! :)
\item Préparation : 15 min		% Temps de préparation (sans la cuisson)
%\item Cuisson : 3 min			% Temps de cuisson
\end{infos}
\begin{etapes}
% Ici les étapes à réaliser
% Une étape par ligne, chaque ligne commence par un \item
% Pour exemple les étapes pour faire un millas ;)
\item  Faites une vinaigrette avec le jus de citron, l'huile d'olive, le sel et le poivre. Réserver
\item Prélever les quartiers de pamplemousse.
\item Laver tous les légumes. Émincer le fenouil et les radis.
\item Dans un saladier, disposer la roquette, les feuilles de sucrine, le fenouil, les radis, et les quartiers de pamplemousse.
\item Au dernier moment, ajouter l'avocat coupé en quartiers et assaissonner avec la vinaigrette.
\item Pour un repas complet, ajouter une escalope de poulet grillé émincée ou 300 g de crevettes roses décortiquées.
\end{etapes}
\begin{conseils}
% Ici écrire les conseils concernant la recette 
\end{conseils}
