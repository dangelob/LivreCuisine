% Generated file 2019-02-17 16:33:30.964629060 +01:00
\begin{recette}{Rôti de porc aux pruneaux}{Rôti de porc aux pruneaux}

\begin{ingredients}
1 rôti de porc (800 g env.)\par
1 ou 2 oignon\par
200 à 300 g de pruneaux\par
600 g de pommes de terre\par
1 petite courge butternut\par
1 petit verre de vin blanc\par
2 c. à soupe d'huile\par
2 branches de thym\par
1 feuille de laurier\par
sel, poivre\par
\end{ingredients}

\begin{infos}
Pour 4 personnes\\
Préparation : 15 min\\
Cuisson : 15 min\\
\end{infos}

\begin{etapes}
\item Faire chauffer l'huile dans une cocotte, y faire dorer le rôti sur toutes ses faces et réserver. Émincer l'oignon et le faire revenir dans la cocotte. Remettre le rôti, ajouter le vin blanc et un verre d'eau, le thym, le laurier, saler et poivrer.
\item Couvrir et laisser cuire à feu doux 1h. Ajouter de l'eau en cours de cuisson si nécessaire.
\item Laver et éplucher les pommes de terre, les couper en deux si elles sont grosses. Les disposer autour du rôti 30 à 40 min avant la fin de la cuisson. Laver et éplucher la courge, la couper en deux. Ôter les graines et la couper en gros morceaux. L'ajouter dans la cocotte avec les pruneaux 20 min avant la fin de la cuisson.
\item Servir le rôti découpé en tranches avec sa garniture.
\end{etapes}

\begin{conseils}
Ajuster la quantité de pruneaux selon les goûts
\end{conseils}

\end{recette}