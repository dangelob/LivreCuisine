% Nom de la recette à entrer entre les accolades {}
\section{Cigares banane-choco}

% Informations génériques
% Changer de ligne pour chaque et commencer par : \item
% Mettre une * si l'information n'est pas certaine 
\begin{itemize}
\item Pour 4 personnes*			% Nombre de personnes qu'on pourra nourrir ! :)
\item Préparation : 15 min*		% Temps de préparation (sans la cuisson)
\item Cuisson : 10 min			% Temps de cuisson
\end{itemize}

\subsection*{\textsc{Ingrédients~:}}

% Ici lister les ingrédients 
% Changer de ligne pour chaque ingrédient et commencer la ligne par : \item
% rajouter autant de ligne que d'ingrédient
\begin{itemize}
\item 3 feuilles de brick
\item 2 bananes
\item 12 carrés de chocolat noir
\item 25 g de beurre
\end{itemize}


\subsection*{\textsc{Marche à suivre~:}}

% Ici les étapes à réaliser
% Une étape par ligne, chaque ligne commence par un \item
% Pour exemple les étapes pour faire un millas ;)
\begin{enumerate}
\item Faites fondre le beurre au micro-onde à puissance moyenne. 

\item Allumez le four à 210° C, th 7.

\item Épluchez les bananes et coupez-les en trois tronçons, puis chacun en deux dans la longueur.

\item Coupez les feuilles de brick en quatre triangles. 

\item Beurrez-les au pinceau.

\item Déposez la partie la plus large vers vous. Posez un tronçon de banane et un carré de chocolat dessus. Rabattez les côtés, enroulez. Formez ainsi 12 petits cigares.
Déposez-les sur une plaque recouverte de papier cuisson. 

\item Cuisez 10 min au four.
\end{enumerate}

\subsection*{\textsc{Conseil~:}}
% Ici écrire les conseils concernant la recette 
Servez les cigares tout chauds avec une boule de glace à la vanille.

Vin : banyuls ou maury
